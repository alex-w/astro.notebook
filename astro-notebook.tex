\documentclass[10pt, a5paper, twoside]{article}

%%% Русский шрифт
\usepackage[utf8]{inputenc}
\usepackage[russian]{babel}
\usepackage[OT1]{fontenc}

%%% Математика
% Шрифты для математики
\usepackage{amsmath}
\usepackage{amsfonts}
\usepackage{amssymb}
% Операторы
\DeclareMathOperator{\const}{const}
% Астрономические символы
\usepackage{wasysym}
\usepackage{starfont}	% Значок Цереры

%%% Поля и разметка страницы
\usepackage[inner=0.9cm, outer=1.5cm, top=1.5cm, bottom=1.2 cm, headsep=4mm]{geometry}

%%% Набор текста
% Разрывы страниц в формулах
\allowdisplaybreaks	
% Последняя строка абзаца
\widowpenalty10000
\clubpenalty10000

%%% Иллюстрации
\usepackage{graphicx}
\usepackage{subcaption}
\usepackage{wrapfig}
\graphicspath{{./img/}}
% Графики
\usepackage{pgfplots}
\pgfplotsset{	width	=	14	cm,
					x label style={
        					font = {\small\sffamily},
        					yshift = .5pc
      				},
      				tick label style={
      					font = {\scriptsize},
      				},
      				y label style={
      					font = {\small\sffamily},
      					yshift = -.8pc
      				},
      				every tick/.style	=	{
						black, 
						line width 	= 	.5	pt
					},
					axis line style 	= 	{
						line width 	= 	.5	pt
					},
					grid style	=	{
						gray,
						dotted
					},
					minor x tick num = 1,
 					minor y tick num = 1,
 					no markers,
 					grid = major,
 					every axis/.append style	=	{
 						line width	=	.7	pt
 					}
				}
%Подписи
\usepackage		[margin		= 10	pt,
					font		= footnotesize, 
					labelfont	= bf, 
					labelsep	= endash, 
					labelfont	= bf,
					textfont	= sl,
					margin		= 0 	pt,  
					aboveskip 	= 4		pt, 
					belowskip 	= -6	pt]	{caption}
\usepackage		[margin		= 10	pt,
					font		= footnotesize, 
					labelfont	= bf, 
					labelsep	= endash, 
					labelfont	= bf,
					textfont	= sl,
					margin		= 0 	pt,  
					aboveskip 	= 4		pt, 
					belowskip 	= 6	pt]	{subcaption}


%%% Insert pdf pages
\usepackage[final]{pdfpages}

%%% Таблицы
\usepackage{tabularx}
\newcolumntype{L}[1]{>{\hsize=#1\hsize\raggedright\arraybackslash}X}%
\newcolumntype{R}[1]{>{\hsize=#1\hsize\raggedleft\arraybackslash}X}%
\newcolumntype{C}[1]{>{\hsize=#1\hsize\centering\arraybackslash}X}

%%% Нумерованные списки
\usepackage{enumitem}
\setlist[enumerate]{itemsep=1pt, label={\arabic*.}, leftmargin=1pc}

%%% Новые команды
\newcommand{\term}[1]{{\sffamily\bfseries#1}}
\newcommand{\moon}{\!\!\rightmoon}
\newcommand{\au}{\text{а.\,е.}}
\newcommand{\tw}{\textwidth}
\newcommand{\imp}[1]{{\itshape#1}}
\renewcommand{\vec}[1]{\mathbf{#1}}

%%% Cчетчики
\renewcommand{\theequation}{\thesection.\arabic{equation}}
\numberwithin{equation}{section}
	
%%% Размеры
\setlength    {\parskip}        { 4pt }
\setlength    {\parindent}      { 0pt }
\renewcommand {\baselinestretch}{ 1.02 }

%%% Гиперссылки
\usepackage{hyperref}
\hypersetup	{
    colorlinks=false,    
    linktoc=all
}

%%% Колонтитулы
\usepackage{fancyhdr}
\pagestyle{fancy}
\fancyhf{}
\fancyhead[CO]{\slshape \small \leftmark}
\fancyhead[LE, RO]{\slshape \small \thepage}
\fancyhead[CE]{\slshape \hyperref[toc]{Астрадь}}

%%% main
\begin{document}\includepdf[pages=-]{sys/empty-page.pdf}
\setcounter{page}{1}
\thispagestyle{empty}
\vspace*{1cm}
{\hspace{1pc} {\bfseries А.\,С.~Шепелев, Д.\,А.~Долгов, С.\,Д.~Молчанов, С.\,Б.~Борисов.} Астрадь~--- краткий сборник теории по астрономии. 2018~---~60~с.}
\vskip4cm
\begin{center}
	{\slshape 2-е издание\\
	переработанное}\\[5mm]
	\small  Тираж 20 экз.
\end{center}
\vskip6cm
\begin{center}
Астрономический кружок им.~Е.\,П.~Левитана\\[3mm]
г.\,Жуковский\\[1cm]
	2018
\end{center}
\newpage
\newpage
\thispagestyle{empty}
\begin{center}
\includegraphics[width=0.95\tw]{sys/cover.pdf}\\[1pc]
{\scshape Жуковский \\ \year}	
\end{center}

\setcounter{tocdepth}{2}
{
	\small 
	\renewcommand{\baselinestretch}{ 0.95 } 
	\tableofcontents 
	\label{toc}
	\renewcommand {\baselinestretch}{ 1.02 }
}
\section{Небесная механика}
\section{Небесная механика}
\subsection{Расстояние и размеры}
\term{Астрономическая единица}~--- единица измерения расстояния в астрономии, равная большой полуоси орбиты Земли. \begin{equation}
	1~\au = 149\:597\:870\:700~\text{м} \simeq 1.5 \times 10^{11}~\text{м}.
\end{equation}

\term{Годичный параллакс}\footnote{Важно отметить, здесь x$\pi$~--- лишь обозначение, ничего общего с числом $\pi$ не имеющее.} ($\pi$) объекта~--- это угол, под которым видно 
орбиту Земли из окрестностей данного объекта. Применяется к объектам вне 
Солнечной системы. \begin{equation}
	\tg \pi = \frac{a_\oplus}{r},
	\label{eq:parallax-sin}	
\end{equation}
где $a_\oplus$~--- большая полуось орбиты Земли и $r$~--- расстояние до объекта 
имеют одинаковые единицы измерений. Учитывая малость угла $\pi$, можно считать $\tg\pi \simeq \pi$ в \eqref{eq:parallax-sin}, тогда
\begin{equation}
	\pi = \frac{a_\oplus}{r}.
	\label{eq:parallax}
\end{equation} 
\begin{figure}[h!]
	\centering
	\vspace{-1pc}
	\includegraphics[width = 0.7\tw]{parallax.pdf}
	\caption{Схема годичного параллакса}
\end{figure}

Расстояние $r$, с которого большая полуось орбиты Земли $a_\oplus$ видна под углом $\pi = 1''$ называется \term{1 парсеком}. Так как \begin{equation}
	1~\text{рад} = \frac{180^\circ}{\pi} \simeq  3 438' \simeq 206265'' 
\quad \Longrightarrow \quad \mathsf{1~\text{\sffamily пк} = 
206265~\text{\sffamily а.\,е.}},
\end{equation} 
следовательно, записывая большую полуось орбиты Земли в \au, а расстояние до звезды в парсеках, получаем параллакс в секундах. Таким образом,
\begin{equation}
	r_\text{пк} = \frac{1~\au}{\pi''}.
\end{equation}

\term{Угловой размер объекта}~--- это угол, под которым видно объект. Для сферически симметричных объектов с радиусом $R$, угловой размер (диаметр) при наблюдении с расстояния $r$ определяется как
\begin{equation}
\rho = 2 \arcsin \frac{R}{r}.
\end{equation}
В случае, когда $r\gg R$, можно считать, что $\sin \rho \simeq \rho$, тогда
\begin{equation}
	\rho \simeq \frac{2 R}{r}.
\end{equation}

\vspace{-1.5pc}
\begin{figure}[h!]
	\begin{minipage}[b]{0.5\tw}
		\begin{flushleft}
			\includegraphics[width = 0.93\tw]{angle-size}
			\captionof{figure}{Угловой размер}
		\end{flushleft}
	\end{minipage}
	\begin{minipage}[b]{0.5\tw}
		\centering
		\includegraphics[width = \tw]{parallax-horiz}
		\captionof{figure}{Горизонтальный параллакс}
	\end{minipage}
\end{figure}

\term{Горизонтальный параллакс}~$(p)$~--- это угловой радиус Земли при наблюдении с объекта:
\begin{equation}
\sin p=\frac{R_\oplus}{r}.
\end{equation}

\term{Правило Тициуса-Боде} --- эмпирическая формула, приблизительно описывающая 
радиусы орбит планет в Солнечной системе:
\begin{equation}r=\frac{3\cdot 2^n+4}{10}, \quad n=-\infty, 0, 1, 2...
\end{equation}


\subsection{Закон всемирного тяготения}
Согласно \imp{закону всемирного тяготения}, сила притяжения
между двумя точечными телами с массами $M$ и $m$,
находящимися на расстоянии $r$, равна
\begin{equation}
	F=\frac{GMm}{r^2}, \label{eq:grav-law}
\end{equation}\nopagebreak где $G\simeq 6.67\cdot 10^{-11}~\text{м}^3 /
\left( \text{кг} \cdot \text{с}^2 \right)$~---
\term{гравитационная постоянная}.

\term{Гравитационный потенциал} поля точечной (или сферически
симметричной) массы $M$ на расстоянии $r$ от нее равен
работе, которую необходимо затратить, чтобы принести
единичную массу с бесконечности в данную точку. Так как
гравитационные силы между двумя массами --- это силы
притяжения, то эта работа отрицательна. Данная
величина также является \term{потенциальной энергией} точечной
массы на расстоянии $r$ от массы $M$, а выражение для нее имеет
следующий вид:
\begin{equation}
	U=-\frac{GM}{r}.
\end{equation}

Напряженность гравитационного поля $dU/dr$ часто называют
\term{ускорением свободного падения} $g$, она вычисляется по формуле
\begin{equation}
	g = \frac{GM}{r^2}.
	\label{eq:g}
\end{equation}
Тогда (\ref{eq:grav-law}) можно записать как
\begin{equation}
	F = mg.
\end{equation}

\input{sections/cel-mech.energy-conserv.tex}
\input{sections/cel-mech.kepler-laws.tex}
\subsection{Движение по орбите}
\begin{figure}[t]
	\centering
	\begin{tikzpicture}
		\footnotesize
		
		%	\foreach \x in {0, .1,...,5} {
		%		\draw [line width=.1pt] (\x, -3) -- (\x, 3);
		%	};
		%
		%	\foreach \y in {-3, -2.9,...,3} {
		%		\draw [line width=.1pt] (0, \y) -- (5, \y);
		%	};
		
		\draw [thick] (0, 0) .. controls (2, 4) and (3, -1) .. (5, -1);
		\draw [-latex] (0, -2) -- (1.25, 1.5);
		\draw [-latex] (0, -2) -- (3.3, .1);
		\draw [-latex] (1.25, 1.5) -- (3.3, .1);
		
		\draw (.3, -1.8) arc(31:70:0.36);
		
		\draw (.6, -.3) node [anchor = east] {$\vec{r}(t)$};
		\draw (1.6, -.9) node [anchor = north west] {$\vec{r}(t + dt)$};
		\draw (2.3, 0.9) node [anchor = north east] {$d\vec{r}$};
		\draw (0.2, -1.5) node [anchor = south west] {$\boldsymbol{\omega} \,dt$};
		
		\draw[fill=white] (1.25, 1.5) circle (0.03);
		\draw[fill=white] (3.3, .1) circle (0.03);
		\draw[fill=white] (0, -2) circle (0.03);
		
	\end{tikzpicture}
	\caption{}
\end{figure}

Рассмотрим такую физическую величину, как \term{секториальная скорость}~--- это векторная величина, описывающая ориентированную площадь, заметаемую радиус вектором тела за единицу времени. Пусть в момент времени $t$ тело находилось в точке $\vec{r}(t)$, а через промежуток времени $dt$~--- в точке $\vec{r}(t + dt)$. Обозначим перемещение тела за этот промежуток времени как $d\vec{r}$. Его можно выразить через скорость тела в момент времени $t$, считая ее постоянной на промежутке от $t$ до $t + dt$: $d\vec{r} = \vec{v} \, dt$. Площадь, которую заметает радиус-вектор тело $\vec{r}(t)$ равна половине параллелограмма, построенного на векторах $\vec{r}(t)$ и $d\vec{r}$. Поэтому можно записать
\begin{equation*}
	\vec{s} = \frac{1}{2} [\vec{r} \times \vec{v} dt],
\end{equation*}
следовательно секториальная скорость равна
\begin{equation*}
	\boldsymbol{\sigma} = \frac{d \vec{s}}{dt} = \frac{1}{2} [\vec{r} \times \vec{v}] = \frac{\vec{l}}{2} = \frac{\vec{L}}{2m},
\end{equation*}
где $\vec{l}$~--- удельный момент импульса (на единицу массы). Полученное выражение доказывает \imp{второй закон Кеплера}.

С другой стороны, перемещение $d\vec{r}$ можно выразить через угловую скорость $\boldsymbol{\omega}$, как $d \vec{r} = [\vec{r} \times \boldsymbol{\omega}\,dt]$. Тогда
\begin{equation*}
	\boldsymbol{\sigma}
	= \frac{1}{2} \big[ \vec{r} \times [\vec{r} \times \boldsymbol{\omega} ]\big]
	= \vec{r} \underbrace{(\vec{r}, \boldsymbol{\omega})}_0 - \boldsymbol{\omega} ( \vec{r}, \vec{r} )
	= r^2 \boldsymbol{\omega}.
\end{equation*}
Получим \imp{третий закон Кеплера}, заметив, что модуль секториальной скорости можно записать, как
\begin{gather*}
	\sigma
	= \frac{S_\text{эл}}{T}
	= \frac{\pi a b}{T}
	= \frac{L}{2m},\\
	\frac{\pi a^2 \sqrt{1 - e^2}}{T}
	= \frac{m \sqrt{\dfrac{GM}{a} \cdot \dfrac{1 + e}{1 - e}} \cdot a(1-e)}{2m},\\
	\frac{4\pi^2 a^4 (1 - e^2)}{T^2}
	= a^2(1-e)^2 \cdot \frac{GM}{a} \cdot \frac{1 + e}{1-e}
\end{gather*}
\begin{equation}
	\frac{T^2}{a^3} = \frac{4\pi^2}{GM}.чч
\end{equation}

Получим еще одно важное соотношение~--- \term{интеграл энергии}~--- формулу для скорости тела на орбите с большой полуосью $a$ в точке, удалённой на расстояние~$r$ от центрального тела с массой $M$. Для этого рассмотрим  сначала точку перицентра ($q$, <<п>>) и апоцентра ($Q$, <<a>>) данной орбиты, запишем для них закон сохранения энергии и закон сохранения момента импульса:
\begin{gather*}
	-\frac{GMm}{q} + \frac{m v^2_\text{п}}{2} = -\frac{GMm}{Q} + \frac{m v^2_\text{а}}{2},\\
	mv_\text{п}q = mv_\text{a}Q.
\end{gather*}
Из ЗСМИ и выражений для перицентрического~$q$ и апоцентрического~$Q$ расстояний через большую полуось $a$ и эксцентриситет $e$ имеем:
\begin{equation*}
	\frac{v_\text{а}}{v_\text{п}} = \frac{1 - e}{1 + e}.
\end{equation*}
Использую это соотношения, преобразуем ЗСЭ:
\begin{gather}
	\frac{v_\text{п}^2}{2} \left( 1 - \frac{(1 -e)^2}{(1 + e)^2} \right) = GM \left( \frac{1}{a(1-e)} - \frac{1}{a(1+e)} \right),\\
	\frac{v_\text{п}^2}{2} \cdot \frac{ 1 + 2e + e^2 - 1 + 2e - e^2}{(1+e)^2} = \frac{GM}{a} \cdot \frac{1 + e - 1 +  e}{(1+e)(1-e)},\\
	v_\text{п} = \sqrt{\frac{GM}{a}}\sqrt{\frac{1+e}{1-e}}, \quad \quad v_\text{a} = \sqrt{\frac{GM}{a}}\sqrt{\frac{1-e}{1+e}}.
\end{gather}
Запишем теперь ЗСЭ для перицентра и произвольной точки орбиты на расстоянии $r$:
\begin{gather*}
	-\frac{GMm}{q} + \frac{m v^2_\text{п}}{2} = -\frac{GMm}{r} + \frac{m v^2}{2},\\
	-\frac{GMm}{q} + \frac{GMm}{2a} \cdot \frac{1+e}{1-e} = -\frac{GMm}{r} + \frac{m v^2}{2},\\
	v^2 = GM \left( \frac{2}{r} - \frac{2}{a(1 - e)} + \frac{1+e}{a (1-e) }\right) = GM \left( \frac{2}{r} - \frac{1}{a} \right),
\end{gather*}
\begin{equation}
	v = \sqrt{ GM \left( \frac{2}{r} - \frac{1}{a} \right)}.
	\label{eq:int-energy}
\end{equation}
Полученное выражение и называется интегралом энергии. Согласно \eqref{eq:int-energy} и \eqref{eq:ellipse-pol-eq} для скорости тела в произвольной точке орбиты также справедливо выражение
\begin{equation}
	v = \sqrt{\frac{GM}{p}\cdot(1 + 2 e \cos \nu + e^2)},
\end{equation}
где $\nu$~--- истинная аномалия, а $p$~--- фокальный параметр.

Найдем величину момента импульса пробной массы $m$ на эллиптической орбите. В силу постоянства данной величины, можно выбрать любую точку орбиты для её поиска. Проще всего рассмотреть перицентр или апоцентр, рассмотрим первый.
\begin{multline*}
	L
	= m v_q q
	= m \sqrt{\frac{GM}{a} \frac{1+e}{1-e}} \cdot a(1-e) =\\
	= m \sqrt{GMa (1 + e)(1-e)}
	= m \sqrt{GMa(1-e^2)}
	= m \sqrt{GMp}.
\end{multline*}

Для параболической также рассмотрим точку перицентра:
\begin{multline*}
	L
	= m v_q q
	= m v_2(q) q
	= m \sqrt{\frac{2GM}{q}} \cdot q =\\
	= m \sqrt{2GMq}
	= m \sqrt{2GM \cdot \frac{p}{2}}
	= m \sqrt{GMp}.
\end{multline*}


\input{sections/cel-mech.orbit-elem.tex}
\subsection{Точки Лагранжа}

\term{Точки Лагранжа}~--- точки, во вращающейся системе из двух массивных тел,
\begin{wrapfigure}[14]{l}{0.48\tw}
	\centering
	\vspace{-.5pc}
	\includegraphics[width = .48\tw]{lagr-points}
	\captionof{figure}{Точки Лагранжа}
	\label{pic:larg-points}	
\end{wrapfigure}
в которых третье тело с пренебрежимо 
малой массой, не испытывающее воздействие никаких 
других сил, кроме гравитационных, со стороны двух 
первых тел, может оставаться неподвижным относительно 
этих тел. В этих точках гравитационные силы, 
действующие на малое тело, уравновешиваются силами инерции.

Точки $L_1$, $L_2$ и $L_3$ лежат на одной прямой, 
соединяющей два массивных тела. Точки $L_4$ и $L_5$ 
образуют равносторнние треугольники с массивными 
телами.

Для расстояний до точек $L_1$, $L_2$ и $L_3$ от 
центра масс системы справедливы следующие выражения:
\begin{equation}r_1=R\left(1-\sqrt[3]{\frac{\alpha}
{3}}\right), \quad r_2=R\left(1+\sqrt[3]{\frac{\alpha}
{3}}\right), \quad r_3=R\left(1+\frac{5}{12}\alpha\right),
\end{equation}
где $\alpha=M_2 / (M_1 + M_2)$, $R$~--- расстояние между 
телами, $M_1$ --- масса более массивного тела, $M_2$
 --- масса второго тела.

Если $M_2 \ll M_1$, то точки $L_1$ и $L_2$ находятся 
примерно на одинаковом расстоянии от тела $M_2$, равном
\begin{equation}
r\approx R\sqrt[3]{\frac{M_2}{3M_1}}.
\end{equation}

Расстояния от центра масс системы до точек $L_4$ и $L_5$ в координатной системе с центром координат в центре масс системы рассчитываются по  формулам
\begin{equation}
	 r_4 = \left ( \frac{R}{2} \cdot \frac{M_1-M_2}{M_1+M_2} ,   \frac{\sqrt{3}R}{2} \right ), \quad r_5 = \left ( \frac{R}{2} \cdot \frac{M_1-M_2}{M_1+M_2} ,   -\frac{\sqrt{3}R}{2} \right ). 
\end{equation}
\subsection{Приливы и отливы}

\term{Приливы и отливы}~--- периодические вертикальные колебания уровня океана, являющиеся результатом изменения положения Луны и Солнца. Хотя силы тяготения Солнца почти в 200 раз больше, чем силы тяготения Луны, приливные силы, порождаемые Луной, почти вдвое больше порождаемых Солнцем. Это происходит из-за того, что приливные силы зависят не от величины гравитационного поля, а от степени его неоднородности. Высота приливов зависит от взаимного расположения Луны и Солнца: наибольший~---  силы от Луны и от Солнца действуют вдоль одного направления, а наименьший~--- под прямым углом друг к другу.

\begin{minipage}{.24\tw}
Ускорение в центре Земли ($T$) определяется формулой \eqref{eq:g}:
\begin{equation*}
	a_T=\frac{G M}{r^2},
\end{equation*}
$M$~--- масса возмущающего тела,
\end{minipage}
\hfill
\begin{minipage}{0.74\tw}
	\vspace{-.5pc}
	\includegraphics[width = \tw]{Ebb_flow}
	\captionof{figure}{К объяснению приливных сил}\label{Ebb_flow}
\end{minipage}\\[-0.5pc]

$r$~--- расстояние между центрами Земли и данного тела. Аналогично, ускорения в точках $A$ и $B$ равны соответственно
\begin{equation}
	a_A = \frac{G M}{(r - R)^2} \quad \text{и} \quad a_B = \frac{GM}{(r + R)^2},
\end{equation}
где $R$~--- радиус Земли или иного тела, подверженного воздействию приливных сил. Ускорение в точке $A$ относительно точки $T$ равно
\begin{equation}
	a_A - a_T = a_T \cdot \frac{2 r R - R^2}{(r - R)^2} = \frac{GM \left(2 r R - R^2 \right)}{r^2 (r - R)^2} \xrightarrow{R \ll r} \frac{2 G M R}{r^3}.
	\label{eq:ebb-force}
\end{equation}

Под действием лунного притяжения водная оболочка Земли принимает форму 
эллипсоида, который вытянут по направлению к Луне. Близ точек $A$ и $B$ будет 
прилив, а в точках $F$ и $D$ --- отлив (см.~Рис.\,\ref{Ebb_flow}).
\nopagebreak
\subsection{Затмения}
Диаметр тени спутника при полном центральном затмении (когда центры трёх тел лежат на одной прямой) с большой точностью равен 
\begin{equation}
d_\text{тени} = 2 \cdot \frac{R_{\moon}(a_\oplus - R_\oplus) - R_\odot \left( a_{\moonч} - R_\oplus \right)}{a_\oplus - a_{\moon}}.
\end{equation}
Среднее значение  этой величины около 200~км, максимальное~--- около 215~км. При нецентральном затмении максимальный диаметр тени Луны на поверхности Земли может достигать 270~км. Это даёт оценку на продолжительность, равную 7.5 минутам. Большинство полных затмений длятся 2\,--\,4~минуты.

\begin{figure}[h!]
\centering
\vspace{-.5pc}
\includegraphics[width = 10cm]{full_eclipse}
\caption{Полное солнечное затмение со стороны северного полюса эклиптики}
\label{fig:eclipses-full-solar-eslipse}
\end{figure}
При \term{кольцеобразном солнечном затмении} Луна так расположена относительно Земли, что конус её тени не достаёт до поверхности планеты, и вокруг Луны можно наблюдать яркое кольцо незакрытой части солнечного диска.

\begin{figure}[h!]
	\centering
	\includegraphics[width = 10cm]{partly-eclipse}
	\caption{Кольцеобразное солнечное затмение со стороны северного полюса эклиптики}
	\label{fig:eclipses-circle-solar-eslipse}
\end{figure}
При особом расположении Луны и Земли возможны \term{гибридные} затмения, когда в разных пунктах Земли наблюдаются либо \imp{кольцеобразное}, либо \imp{полное} затмение. Причиной такого явления является шарообразность Земли.

\vspace{-1pc}
\begin{figure}[h!]
	\centering
	\includegraphics[width=10cm]{moon-eclipse}
	\caption{Схема лунного затмения со стороны северного полюса эклиптики}
	\label{fig:moon-eclipse-scheme}
\end{figure}
\term{Лунное затмение}, в отличие от солнечного, видно со всего ночного полушария Земли. Диаметр земной тени на расстоянии Луны превышает размер последней примерно в 2.5\,--\,3 раза. Бывают \term{частные}, когда лишь часть Луны попадает в земную тень, \term{полные}~--- Луна полностью погружается в тень Земли, и \term{полутеневые}~--- Луна проходит через полутень Земли, не затрагивая конус тени.

\term{Синодический месяц}~--- промежуток времени между одинаковыми фазами Луны, равен 29.53 суток.

\term{Драконический месяц}~--- промежуток времени между двумя последовательными прохождениями Луны через один и тот же узел орбиты,~--- 27.21 суток.

\term{Сарос}~--- промежуток  времени, по прошествии которого солнечные и лунные затмения повторяются в прежнем порядке. Происходит это из-за того, что каждый сарос Луна, орбита Луны и Солнце возвращаются в прежнее положение относительно далёких звёзд. Сарос длится ровно 242 драконических месяца, или 223 синодических месяца, или 18 лет 11 дней 8 часов.

\begin{wrapfigure}[8]{r}{.42\tw}
	\centering
	\vspace{-1pc}
	\includegraphics[width = 0.2\textwidth]{phases}
	\hfill
	\includegraphics[width = 0.2\textwidth]{phases-2}
	\caption{Частное и полное затмение}
	\label{fig:part-eclipses-scheme}
\end{wrapfigure}
Важной характеристикой любого затмения является его \term{фаза}~--- для \imp{частных} и \imp{кольцеобразных} затмений: отношение закрытой части $x$ диаметра\footnote{Здесь имеется в виду \imp{угловой} диаметр} затмеваемого тела, проходящего через центр затмевающего тела, ко всему диаметру затмеваемого тела $D$; для \imp{полного}: единица плюс отношение расстояния\footnote{Расстояние между окружностями $l_1$ и $l_2$~--- это $\min |L_1L_2|$ по всем $L_1 \in l_1$ и $L_2 \in l_2$.} между краями дисков затмеваемого и затмевающего тел к диаметру затмеваемого тела $D$.
\begin{equation}
\Phi_{\text{част}} = \frac{x}{D} < 1, \quad \quad \quad \Phi_{\text{полн}} =  1 + \frac{\min\{d_1, d_2\}}{D} > 1.
\end{equation}
Иногда вводят такое понятие, как \term{площадная фаза затмения}, т.\,е. отношение площади закрытой части диска затмеваемого тела к полной площади его диска. Чаще всего  площадную фазу используют применительно к двойным звёздам, когда считают падение блеска при затмении одной звезды другой.

\input{sections/cel-mech.planet-config.tex}
\input{sections/cel-mech.phase-angle.tex}
\input{sections/cel-mech.synod-period.tex}
\subsection{Собственное движение звёзд}
\term{Собственным движением} $(\mu)$ называется изменение координат звёзд на небесной сфере, вызванное относительным движением звёзд и Солнца, обычно измеряется в mas/год.
\begin{equation}
	\mu = \frac{V_\tau}{D},
\end{equation}
где $V_\tau$~--- тангенциальная относительная скорость звезды, $D$~--- расстояние до неё.

\change{Разделяют также собственное движение по склонению~--- $\mu_\delta$ и собственное движение по прямому восхождению~--- $\mu_\alpha$, которые определяются следующими выражениями:}
\begin{equation}
  \mu_\delta = \frac{\delta(t_2) - \delta(t_1)}{t_2 - t_1}, \quad \quad \mu_\alpha = \frac{\alpha(t_2) - \alpha(t_1)}{t_2 - t_1}.
\end{equation}
\change{
\begin{wrapfigure}{r}{.4\tw}
\begin{flushright}
	\vspace{-1pc}
	\begin{tikzpicture}
	\footnotesize
	\draw [dashes] (0, 4) arc(90:0:3 and 4);
	\draw [dashes] (0, 4) arc(90:0:2 and 4); 
	%
	\draw [dashes] (3.47, 2) arc(0:-70:3.47 and 1.16);	
	\draw [dashes] (2.64, 3) arc(0:-70:2.64 and 0.88);
	%
	\draw [thick, -latex] (2.3, 2.55) arc(-34:-56:2.64 and 0.88);
	\draw [thick, -latex] (2.3, 2.55) arc(53:29:2 and 4);
	\draw [thick, -latex] (2.3, 2.55) .. controls (2.3, 1.9) and (2.1, 1.4) .. (1.93, 1.03);
	%
	\draw (.9, 2.2) node [anchor=south] {$\delta(t_1)$};
	\draw (1.2, .9) node [anchor=south] {$\delta(t_2)$};
	%
	\draw (2, 0) node [anchor=north] {$\alpha(t_2)$};
	\draw (3, 0) node [anchor=north] {$\alpha(t_1)$};
	%
	\draw [fill=white] (2.3, 2.55) circle (0.03);
	\draw [fill=white] (1.93, 1.03) circle (0.03);
	\draw [fill=white] (0, 4) circle (0.03);
	%
	\draw (0, 4) node [anchor=north] {$P$};
	%
	\draw (1.9, 2.4) node [anchor=south] {$\mu_\alpha$};
	\draw (2.6, 2.05) node [anchor=west] {$\mu_\delta$};
	\draw (2.06, 1.65) node [anchor=south] {$\mu$};
	%
\end{tikzpicture}
\end{flushright}
\end{wrapfigure}
 Как отсюда видно, $\mu_\alpha$ является угловой скоростью по малому кругу, а значит, зависит от $\delta$. Следовательно, полное собственное движение $\mu$ можно найти, как
\begin{equation}
	\mu = \sqrt{\mu_\delta^2 + \mu_\alpha^2 \cos^2 \delta},
\end{equation}
потому что радиус малого круга, состоящего из точек со склонением~$\delta$, равен $R \cos \delta$, где $R$~--- радиус сферы, содержащей этот круг.
}

\begin{figure}[h!]
\begin{subfigure}[b]{0.47\tw}
	\begin{tikzpicture}[scale=1.05]
	\footnotesize
	
%	\foreach \x in {0, .1,...,4} {
%		\draw [line width=.1pt] (\x - 1, 0) -- (\x - 1, 4);
%	};
%	
%	\foreach \x in {0, 1,...,4} {
%		\draw [line width=.4pt] (\x - 1, 0) -- (\x - 1, 4);
%	};
%	
%	\foreach \y in {0, .1,...,4} {
%		\draw [line width=.1pt] (-1, \y) -- (4, \y);
%	};
%	
%	\foreach \y in {0, 1,...,4} {
%		\draw [line width=.4pt] (-1, \y) -- (4, \y);
%	};
	
	\draw [double] (.21, .21) arc (45:104:.3);
	\draw (-.93, 3.71) arc (-76:-35:.3);
	
	\draw (0, 0) -- (-1, 4);
	\draw (0, 0) -- (2, 2);
	\draw (-1, 4) -- (2.6, 1.6);
	
	\draw [thick, -latex] (-1, 4) -- (0, 4.25);
	\draw [thick, -latex] (-1, 4) -- (-.6, 2.4);
	
	\draw [fill=white] (-1, 4) circle (.03);
	\draw [fill=white] (0, 0) circle (.03);
	\draw [fill=white] (2, 2) circle (.03);
	
	\draw (1, 1) node [anchor=north west] {$R$};
	\draw (-.45, 2.1) node [anchor=north east] {$R_0$};
	\draw (.5, 2.95) node [anchor=south west] {$V \Delta t$};
	\draw (0, 0) node [anchor=north] {Солнце};
	\draw (-1, 4) node [anchor=south east] {Звезда};
	
	\draw (.1, .3) node [anchor=south] {$\xi$};
	\draw (-.9, 3.75) node [anchor=north west] {$\gamma$};
	
	\draw (-.5, 4.15) node [anchor=south] {$V_\tau$};
	\draw (-.75, 3.1) node [anchor=east] {$V_r$};
\end{tikzpicture}
\caption{}
\label{pic:phase-angle-1}
\end{subfigure}
\hfill
\begin{subfigure}[b]{0.47\tw}
\begin{tikzpicture}[scale=0.9]
	\footnotesize
	
	\draw (.2, 4.86) arc (-45:-135:0.28);
	\draw [double] (-1.65, 1.51) arc (5:80:0.25);
	
	\draw (0, 5) .. controls (-1.5, 4) and (-2, 2) .. (-2, 0);
	\draw (0, 5) .. controls (1.5, 4) and (2, 2) .. (2, 0);
	\draw (-2, 0) .. controls (-1, -.5) and (1, -.5) .. (2, 0);
	\draw (-1.9, 1.5) .. controls (-1, 1.5) and (1, 2) .. (1.5, 3);
	
	\draw [fill=white] (0, 5) circle (.03);
	\draw [fill=white] (-2, 0) circle (.03);
	\draw [fill=white] (2, 0) circle (.03);
	\draw [fill=white] (-1.9, 1.5) circle (.03);
	\draw [fill=white] (1.5, 3) circle (.03);
	
	\draw (-2, .2) -- (-1.8, .11) -- (-1.8, -.09);
	\draw (2, .2) -- (1.8, .11) -- (1.8, -.09);
	
	\draw (0, 5) node [anchor=south] {$P$};
	\draw (0, 1.9) node [anchor=north] {$\xi$};
	\draw (0, -.4) node [anchor=south] {$\Delta \alpha$};
	\draw (0, 4.8) node [anchor=north] {$\Delta \alpha$};
	\draw (-1, 4) node [anchor=east] {$90^\circ - \delta$};
	\draw (.9, 4.2) node [anchor=west] {$90^\circ - (\delta + \Delta \delta)$};
	\draw (0, -.4) node [anchor=north] {Небесный экватор};
\end{tikzpicture}
\caption{}
%\label{pic:phase-angle-2}
\end{subfigure}
\caption{}
\end{figure}


\change{Получим выражение для координат звезды, имеющей собственное движение $\mu = (\mu_\alpha, \mu_\delta)$, лучевую скорость $V_r$ и параллакс в начальный момент времени $\pi_0$. Найдем сначала тангенциальную скорость:
\begin{equation*}
	V_\tau = R_0 \sqrt{ \mu_\delta^2 + \mu_\alpha^2 \cos^2 \delta} = \frac{\sqrt{ \mu_\delta^2 + \mu_\alpha^2 \cos^2 \delta}}{\pi_0}.
\end{equation*}
Определим теперь угол между лучем зрения и полной скоростью звезды:
\begin{equation*}
	\gamma = \arctan \frac{V_\tau}{V_r}.
\end{equation*}
При этом полная скорость равна
\begin{equation*}
	V_0 = \sqrt{V_\tau^2 + V_r^2}.
\end{equation*}
Из теоремы косинусов можно найти расстояние для звезды через промежуток времени $\Delta t$:
\begin{equation*}
	R = \sqrt{R_0^2 + (V_0 \Delta t)^2 - 2 R_0 V_0 \Delta t \cos \gamma}.
\end{equation*}
Тогда угловое перемещение звезды равно
\begin{equation*}
	\sin \xi = \frac{V_0 \Delta t \sin \alpha}{R}.
\end{equation*}
Через компоненты собственного движения нетрудно получить угол между направлением на полюс и вектором полного собственного движения в начальный момент:
\begin{equation*}
	\tg \psi =  \frac{\mu_a \cos \delta}{\mu_\delta}.
\end{equation*}
Теперь с помощью сферической теоремы косинусов можно определить склонение звезды через время $\Delta t$:
\begin{equation*}
	\sin (\delta - \Delta \delta) = \cos \xi \sin \delta + \sin \xi \cos \delta \cos \psi.
\end{equation*}
Далее из сферической теоремы синусов получаем выражение для изменения прямого восхождения за время $\Delta t$~---
\begin{equation*}
	\sin \Delta \alpha = \frac{\sin \psi \sin \xi}{\cos (\delta - \Delta \delta)}.
\end{equation*}
}






\input{sections/cel-mech.precession.tex}
\subsection{Закон всемирного тяготения}
Согласно \imp{закону всемирного тяготения}, сила притяжения 
между двумя точечными телами с массами $M$ и $m$,
находящимися на расстоянии $r$ равна
\begin{equation}
	F=\frac{GMm}{r^2}, \label{eq:grav-law}
\end{equation}\nopagebreak где $G\simeq 6.67\cdot 10^{-11}~\text{м}^3 / 
\left( \text{кг} \cdot \text{с}^2 \right)$~--- 
\term{гравитационная постоянная}.

\term{Гравитационный потенциал} поля точечной (или сферически 
симметричной) массы $M$ на расстоянии $r$ от нее равен
работе, которую необходимо затратить, чтобы принести
единичную массу с бесконечности в данную точку. Так как
гравитационные силы между двумя массами --- это силы 
притяжения, то эта работа отрицательна. Данная
величина также является \term{потенциальной энергией} точечной
массы на расстоянии $r$ от массы $M$, а выражение для нее имеет 
следующий вид:\begin{equation}
U=-\frac{GM}{r}
\end{equation}

Напряженность гравитационного поля $dU/dr$ часто называют 
\term{ускорением свободного падения} $g$, где
\begin{equation}
	g = \frac{GM}{r^2}
	\label{eq:g}
\end{equation}
Тогда (\ref{eq:grav-law}) можно записать, как \begin{equation}
	F = mg
\end{equation}
%\begin{table}[h!]
%\centering
%\begin{tabular}{|c|c|c|c|}
%\hline 
%{\bfseries Планета} & $\mathbf{g}$, 
%{\bfseries м/$\text{\bfseries c}^2$} 
%& {\bfseries Планета} & $\mathbf{g}$, 
%{\bfseries м/$\text{\bfseries c}^2$}\\
%\hline
%Солнце & 276. & Марс & 3.73\\
%\hline
%Меркурий & 3.73 & Юпитер & 25.9\\
%\hline
%Венера & 8.87 & Сатурн & 11.2\\
%\hline
%Земля & 9.82 & Уран & 9.01\\
%\hline
%Луна & 1.63 & Нептун & 11.3\\
%\hline
%\end{tabular}
%\caption{Ускорение свободного падения на поверхности тел 
%солнечной системы}
%\end{table}

\subsection{Закон сохранения энергии и типы орбит}
Для движения тела c массой $m$ в гравитационном  в поле тела 
с массой \linebreak $M\gg m$ со скорость $v$ на расстоянии $r$ от 
гравитационного центра справедливо следующее соотношение: 
\begin{equation}
\frac{m v^2}{2}-\frac{GM m }{r}=E_0,
\end{equation}
где $E_0$ --- постоянная величина, если на тело не действуют
внешние силы кроме силы притяжения к центральному телу, 
равная сумме кинетической и потенциальной энергии тела. Данное равенство принято называть \term{законом сохранения энергии} для тела, движущегося в поле консервативных (потенциальных) сил.

\begin{wrapfigure}[10]{l}{.5\tw}
	\centering
	\vspace{-1pc}
	\includegraphics[width = 0.45\tw]{speeds}
	\caption{Возможные траектории тела \label{pic:orbits}}
\end{wrapfigure}
Если $E_0>0$, то траектория тела~--- \imp{гипербола}, 
ветви которой асимптотически приближаются к двум прямым. Стоит заметить,
что на бесконечно большом удалении тела с массой $m$ от массивного тела
его скорость остается положительной, так как суммарная энергия $E_0$ 
больше нуля.

Если $E_0=0$, то траектория тела~--- \imp{парабола}. При стремлении
расстояния $r$ между телами к бесконечности, скорость тела с стремится к нулю.

Отсюда становится очевидно, что на параболической и гиперболический
 траекториях движение тела не ограничено (инфинитно). 

Если $E_0<0$, то траектория тела~--- \imp{эллипс}. При 
эллиптической траектории движение ограничено (финитно), так как малое тело
не может бесконечно удалять с неотрицательной скоростью по причине того,
что суммарная энергия меньше нуля.

На Рис.\,\ref{pic:orbits} представлены примеры возможных траекторий тела 
относительно центрального (точка C). При $v_0 > v_{2}$ --- тело движется 
по гиперболе, при $v_0 = v_{2}$ --- по параболе, 
а при $v_0 < v_{2}$ --- по эллипсу.

\term{Первая космическая скорость} --- минимальная скорость, необходимая для 
того, чтобы маломассивное тело стало искусственным спутником центрального тела.
\begin{equation}v_1=\sqrt{\frac{GM}{R}},
\end{equation}
где $M$ --- масса центрального тела, а $R$~--- радиус орбиты. Отсюда несложно получить выражение для
скорости искусственного небесного тела на высоте~$h$ над~поверхностью тела радиуса $r_0$:
\begin{equation}
v_h=\sqrt{\frac{GM}{r_0+h}}=\sqrt{\frac{g r_0^2}{r_0+h}}.
\end{equation}

\term{Параболическая} или \term{вторая космическая скорость} --- 
минимальная скорость, необходимая для того, чтобы маломассивное тело преодолело гравитационное притяжение центрального тела, стартуя с расстояния $r$ от его центра масс, и покинуло замкнутую орбиту вокруг 
последнего. Выражение для которой имеет следующий вид:\begin{equation}
v_{2}=\sqrt{\frac{2GM}{r}}.
\end{equation}
Для стабильной системы, частный случай~--- тело на круговой орбите, справедлива 
\term{теорема о вириале}:
\begin{equation}
2 \langle T\rangle 
= -\sum _{{k=1}}^{N}\langle (\vec{F}_{k}, \vec{r}_{k})\rangle 
= -\langle \Pi \rangle,
\end{equation}
где $\langle T\rangle$ --- средняя полная кинетическая энергия, $\vec{F}_k$~--- сила, 
действующая на $k$-ю частицу, $\vec{r}_k$~--- радиус-вектор $k$-й частицы. Другими словами, удвоенная средняя полная 
кинетическая энергия $T$ равна средней полной потенциальной энергии $\Pi$ со знаком минус. Применяя теорему о вириале для тела, обращающегося по круговой орбите можно 
получить выражения для первой космической скорости.
\subsection{Законы Кеплера}
\term{I-ый закон:} Все планеты движутся по 
эллиптическим орбитам, в одном из фокусов которых 
находится Солнце.\\

\term{ II-ой закон:} Радиус-вектор планеты за 
равные промежутки времени заметает равные площади:
\begin{equation}
\frac{dS}{dt}=\const = \frac{S_\text{элл}}{T} = \frac{\pi a b }{T}.
\end{equation}
\term{III-ий закон:} Квадраты периодов обращения планет 
относятся, как кубы больших полуосей их орбит.
\begin{equation}
\frac{T^2_1}{T^2_2}=\frac{a^3_1}{a^3_2},
\end{equation}
где $a$ --- большая полуось, $T$ --- период обращения.
\begin{figure}[h!]
\begin{minipage}[b]{0.5\textwidth}
\centering
\includegraphics[width = 0.84\textwidth]{first-kepler}
\caption{Первый закон Кеплера}
\end{minipage}
\begin{minipage}[b]{0.5\textwidth}
\centering
\includegraphics[width = 0.972\textwidth]{second-kepler}
\caption {Второй закон Кеплера}
\end{minipage}
\end{figure}

\term{Обобщённый} Ньютоном \term{III-ий закон Кеплера} имеет следующий вид:
\begin{equation}
\frac{T^2_1( M_1 + m_1)}{T^2_2( M_2 + m_2 )}=\frac{a^3_1}{a^3_2} \quad \Longleftrightarrow \quad 
	\frac{T^2 ( M + m )}{a^3} = \frac{4 \pi^2}{G},
\end{equation}
где $M_1$ и $M_2$~--- массы центральных тел, $m_1$ и 
$m_2$~--- массы обращающихся вокруг них тел. Так как массы планет 
$m$ много меньше массы звезды $M$, то $M + m \simeq M$.

\section{Небесная механика}
\subsection{Расстояние и размеры}
\term{Астрономическая единица}~--- единица измерения расстояния в астрономии, равная большой полуоси орбиты Земли. \begin{equation}
	1~\au = 149\:597\:870\:700~\text{м} \simeq 1.5 \times 10^{11}~\text{м}.
\end{equation}

\term{Годичный параллакс}\footnote{Важно отметить, здесь x$\pi$~--- лишь обозначение, ничего общего с числом $\pi$ не имеющее.} ($\pi$) объекта~--- это угол, под которым видно 
орбиту Земли из окрестностей данного объекта. Применяется к объектам вне 
Солнечной системы. \begin{equation}
	\tg \pi = \frac{a_\oplus}{r},
	\label{eq:parallax-sin}	
\end{equation}
где $a_\oplus$~--- большая полуось орбиты Земли и $r$~--- расстояние до объекта 
имеют одинаковые единицы измерений. Учитывая малость угла $\pi$, можно считать $\tg\pi \simeq \pi$ в \eqref{eq:parallax-sin}, тогда
\begin{equation}
	\pi = \frac{a_\oplus}{r}.
	\label{eq:parallax}
\end{equation} 
\begin{figure}[h!]
	\centering
	\vspace{-1pc}
	\includegraphics[width = 0.7\tw]{parallax.pdf}
	\caption{Схема годичного параллакса}
\end{figure}

Расстояние $r$, с которого большая полуось орбиты Земли $a_\oplus$ видна под углом $\pi = 1''$ называется \term{1 парсеком}. Так как \begin{equation}
	1~\text{рад} = \frac{180^\circ}{\pi} \simeq  3 438' \simeq 206265'' 
\quad \Longrightarrow \quad \mathsf{1~\text{\sffamily пк} = 
206265~\text{\sffamily а.\,е.}},
\end{equation} 
следовательно, записывая большую полуось орбиты Земли в \au, а расстояние до звезды в парсеках, получаем параллакс в секундах. Таким образом,
\begin{equation}
	r_\text{пк} = \frac{1~\au}{\pi''}.
\end{equation}

\term{Угловой размер объекта}~--- это угол, под которым видно объект. Для сферически симметричных объектов с радиусом $R$, угловой размер (диаметр) при наблюдении с расстояния $r$ определяется как
\begin{equation}
\rho = 2 \arcsin \frac{R}{r}.
\end{equation}
В случае, когда $r\gg R$, можно считать, что $\sin \rho \simeq \rho$, тогда
\begin{equation}
	\rho \simeq \frac{2 R}{r}.
\end{equation}

\vspace{-1.5pc}
\begin{figure}[h!]
	\begin{minipage}[b]{0.5\tw}
		\begin{flushleft}
			\includegraphics[width = 0.93\tw]{angle-size}
			\captionof{figure}{Угловой размер}
		\end{flushleft}
	\end{minipage}
	\begin{minipage}[b]{0.5\tw}
		\centering
		\includegraphics[width = \tw]{parallax-horiz}
		\captionof{figure}{Горизонтальный параллакс}
	\end{minipage}
\end{figure}

\term{Горизонтальный параллакс}~$(p)$~--- это угловой радиус Земли при наблюдении с объекта:
\begin{equation}
\sin p=\frac{R_\oplus}{r}.
\end{equation}

\term{Правило Тициуса-Боде} --- эмпирическая формула, приблизительно описывающая 
радиусы орбит планет в Солнечной системе:
\begin{equation}r=\frac{3\cdot 2^n+4}{10}, \quad n=-\infty, 0, 1, 2...
\end{equation}


\subsection{Закон всемирного тяготения}
Согласно \imp{закону всемирного тяготения}, сила притяжения
между двумя точечными телами с массами $M$ и $m$,
находящимися на расстоянии $r$, равна
\begin{equation}
	F=\frac{GMm}{r^2}, \label{eq:grav-law}
\end{equation}\nopagebreak где $G\simeq 6.67\cdot 10^{-11}~\text{м}^3 /
\left( \text{кг} \cdot \text{с}^2 \right)$~---
\term{гравитационная постоянная}.

\term{Гравитационный потенциал} поля точечной (или сферически
симметричной) массы $M$ на расстоянии $r$ от нее равен
работе, которую необходимо затратить, чтобы принести
единичную массу с бесконечности в данную точку. Так как
гравитационные силы между двумя массами --- это силы
притяжения, то эта работа отрицательна. Данная
величина также является \term{потенциальной энергией} точечной
массы на расстоянии $r$ от массы $M$, а выражение для нее имеет
следующий вид:
\begin{equation}
	U=-\frac{GM}{r}.
\end{equation}

Напряженность гравитационного поля $dU/dr$ часто называют
\term{ускорением свободного падения} $g$, она вычисляется по формуле
\begin{equation}
	g = \frac{GM}{r^2}.
	\label{eq:g}
\end{equation}
Тогда (\ref{eq:grav-law}) можно записать как
\begin{equation}
	F = mg.
\end{equation}

\input{sections/cel-mech.energy-conserv.tex}
\input{sections/cel-mech.kepler-laws.tex}
\subsection{Движение по орбите}
\begin{figure}[t]
	\centering
	\begin{tikzpicture}
		\footnotesize
		
		%	\foreach \x in {0, .1,...,5} {
		%		\draw [line width=.1pt] (\x, -3) -- (\x, 3);
		%	};
		%
		%	\foreach \y in {-3, -2.9,...,3} {
		%		\draw [line width=.1pt] (0, \y) -- (5, \y);
		%	};
		
		\draw [thick] (0, 0) .. controls (2, 4) and (3, -1) .. (5, -1);
		\draw [-latex] (0, -2) -- (1.25, 1.5);
		\draw [-latex] (0, -2) -- (3.3, .1);
		\draw [-latex] (1.25, 1.5) -- (3.3, .1);
		
		\draw (.3, -1.8) arc(31:70:0.36);
		
		\draw (.6, -.3) node [anchor = east] {$\vec{r}(t)$};
		\draw (1.6, -.9) node [anchor = north west] {$\vec{r}(t + dt)$};
		\draw (2.3, 0.9) node [anchor = north east] {$d\vec{r}$};
		\draw (0.2, -1.5) node [anchor = south west] {$\boldsymbol{\omega} \,dt$};
		
		\draw[fill=white] (1.25, 1.5) circle (0.03);
		\draw[fill=white] (3.3, .1) circle (0.03);
		\draw[fill=white] (0, -2) circle (0.03);
		
	\end{tikzpicture}
	\caption{}
\end{figure}

Рассмотрим такую физическую величину, как \term{секториальная скорость}~--- это векторная величина, описывающая ориентированную площадь, заметаемую радиус вектором тела за единицу времени. Пусть в момент времени $t$ тело находилось в точке $\vec{r}(t)$, а через промежуток времени $dt$~--- в точке $\vec{r}(t + dt)$. Обозначим перемещение тела за этот промежуток времени как $d\vec{r}$. Его можно выразить через скорость тела в момент времени $t$, считая ее постоянной на промежутке от $t$ до $t + dt$: $d\vec{r} = \vec{v} \, dt$. Площадь, которую заметает радиус-вектор тело $\vec{r}(t)$ равна половине параллелограмма, построенного на векторах $\vec{r}(t)$ и $d\vec{r}$. Поэтому можно записать
\begin{equation*}
	\vec{s} = \frac{1}{2} [\vec{r} \times \vec{v} dt],
\end{equation*}
следовательно секториальная скорость равна
\begin{equation*}
	\boldsymbol{\sigma} = \frac{d \vec{s}}{dt} = \frac{1}{2} [\vec{r} \times \vec{v}] = \frac{\vec{l}}{2} = \frac{\vec{L}}{2m},
\end{equation*}
где $\vec{l}$~--- удельный момент импульса (на единицу массы). Полученное выражение доказывает \imp{второй закон Кеплера}.

С другой стороны, перемещение $d\vec{r}$ можно выразить через угловую скорость $\boldsymbol{\omega}$, как $d \vec{r} = [\vec{r} \times \boldsymbol{\omega}\,dt]$. Тогда
\begin{equation*}
	\boldsymbol{\sigma}
	= \frac{1}{2} \big[ \vec{r} \times [\vec{r} \times \boldsymbol{\omega} ]\big]
	= \vec{r} \underbrace{(\vec{r}, \boldsymbol{\omega})}_0 - \boldsymbol{\omega} ( \vec{r}, \vec{r} )
	= r^2 \boldsymbol{\omega}.
\end{equation*}
Получим \imp{третий закон Кеплера}, заметив, что модуль секториальной скорости можно записать, как
\begin{gather*}
	\sigma
	= \frac{S_\text{эл}}{T}
	= \frac{\pi a b}{T}
	= \frac{L}{2m},\\
	\frac{\pi a^2 \sqrt{1 - e^2}}{T}
	= \frac{m \sqrt{\dfrac{GM}{a} \cdot \dfrac{1 + e}{1 - e}} \cdot a(1-e)}{2m},\\
	\frac{4\pi^2 a^4 (1 - e^2)}{T^2}
	= a^2(1-e)^2 \cdot \frac{GM}{a} \cdot \frac{1 + e}{1-e}
\end{gather*}
\begin{equation}
	\frac{T^2}{a^3} = \frac{4\pi^2}{GM}.чч
\end{equation}

Получим еще одно важное соотношение~--- \term{интеграл энергии}~--- формулу для скорости тела на орбите с большой полуосью $a$ в точке, удалённой на расстояние~$r$ от центрального тела с массой $M$. Для этого рассмотрим  сначала точку перицентра ($q$, <<п>>) и апоцентра ($Q$, <<a>>) данной орбиты, запишем для них закон сохранения энергии и закон сохранения момента импульса:
\begin{gather*}
	-\frac{GMm}{q} + \frac{m v^2_\text{п}}{2} = -\frac{GMm}{Q} + \frac{m v^2_\text{а}}{2},\\
	mv_\text{п}q = mv_\text{a}Q.
\end{gather*}
Из ЗСМИ и выражений для перицентрического~$q$ и апоцентрического~$Q$ расстояний через большую полуось $a$ и эксцентриситет $e$ имеем:
\begin{equation*}
	\frac{v_\text{а}}{v_\text{п}} = \frac{1 - e}{1 + e}.
\end{equation*}
Использую это соотношения, преобразуем ЗСЭ:
\begin{gather}
	\frac{v_\text{п}^2}{2} \left( 1 - \frac{(1 -e)^2}{(1 + e)^2} \right) = GM \left( \frac{1}{a(1-e)} - \frac{1}{a(1+e)} \right),\\
	\frac{v_\text{п}^2}{2} \cdot \frac{ 1 + 2e + e^2 - 1 + 2e - e^2}{(1+e)^2} = \frac{GM}{a} \cdot \frac{1 + e - 1 +  e}{(1+e)(1-e)},\\
	v_\text{п} = \sqrt{\frac{GM}{a}}\sqrt{\frac{1+e}{1-e}}, \quad \quad v_\text{a} = \sqrt{\frac{GM}{a}}\sqrt{\frac{1-e}{1+e}}.
\end{gather}
Запишем теперь ЗСЭ для перицентра и произвольной точки орбиты на расстоянии $r$:
\begin{gather*}
	-\frac{GMm}{q} + \frac{m v^2_\text{п}}{2} = -\frac{GMm}{r} + \frac{m v^2}{2},\\
	-\frac{GMm}{q} + \frac{GMm}{2a} \cdot \frac{1+e}{1-e} = -\frac{GMm}{r} + \frac{m v^2}{2},\\
	v^2 = GM \left( \frac{2}{r} - \frac{2}{a(1 - e)} + \frac{1+e}{a (1-e) }\right) = GM \left( \frac{2}{r} - \frac{1}{a} \right),
\end{gather*}
\begin{equation}
	v = \sqrt{ GM \left( \frac{2}{r} - \frac{1}{a} \right)}.
	\label{eq:int-energy}
\end{equation}
Полученное выражение и называется интегралом энергии. Согласно \eqref{eq:int-energy} и \eqref{eq:ellipse-pol-eq} для скорости тела в произвольной точке орбиты также справедливо выражение
\begin{equation}
	v = \sqrt{\frac{GM}{p}\cdot(1 + 2 e \cos \nu + e^2)},
\end{equation}
где $\nu$~--- истинная аномалия, а $p$~--- фокальный параметр.

Найдем величину момента импульса пробной массы $m$ на эллиптической орбите. В силу постоянства данной величины, можно выбрать любую точку орбиты для её поиска. Проще всего рассмотреть перицентр или апоцентр, рассмотрим первый.
\begin{multline*}
	L
	= m v_q q
	= m \sqrt{\frac{GM}{a} \frac{1+e}{1-e}} \cdot a(1-e) =\\
	= m \sqrt{GMa (1 + e)(1-e)}
	= m \sqrt{GMa(1-e^2)}
	= m \sqrt{GMp}.
\end{multline*}

Для параболической также рассмотрим точку перицентра:
\begin{multline*}
	L
	= m v_q q
	= m v_2(q) q
	= m \sqrt{\frac{2GM}{q}} \cdot q =\\
	= m \sqrt{2GMq}
	= m \sqrt{2GM \cdot \frac{p}{2}}
	= m \sqrt{GMp}.
\end{multline*}


\input{sections/cel-mech.orbit-elem.tex}
\subsection{Точки Лагранжа}

\term{Точки Лагранжа}~--- точки, во вращающейся системе из двух массивных тел,
\begin{wrapfigure}[14]{l}{0.48\tw}
	\centering
	\vspace{-.5pc}
	\includegraphics[width = .48\tw]{lagr-points}
	\captionof{figure}{Точки Лагранжа}
	\label{pic:larg-points}	
\end{wrapfigure}
в которых третье тело с пренебрежимо 
малой массой, не испытывающее воздействие никаких 
других сил, кроме гравитационных, со стороны двух 
первых тел, может оставаться неподвижным относительно 
этих тел. В этих точках гравитационные силы, 
действующие на малое тело, уравновешиваются силами инерции.

Точки $L_1$, $L_2$ и $L_3$ лежат на одной прямой, 
соединяющей два массивных тела. Точки $L_4$ и $L_5$ 
образуют равносторнние треугольники с массивными 
телами.

Для расстояний до точек $L_1$, $L_2$ и $L_3$ от 
центра масс системы справедливы следующие выражения:
\begin{equation}r_1=R\left(1-\sqrt[3]{\frac{\alpha}
{3}}\right), \quad r_2=R\left(1+\sqrt[3]{\frac{\alpha}
{3}}\right), \quad r_3=R\left(1+\frac{5}{12}\alpha\right),
\end{equation}
где $\alpha=M_2 / (M_1 + M_2)$, $R$~--- расстояние между 
телами, $M_1$ --- масса более массивного тела, $M_2$
 --- масса второго тела.

Если $M_2 \ll M_1$, то точки $L_1$ и $L_2$ находятся 
примерно на одинаковом расстоянии от тела $M_2$, равном
\begin{equation}
r\approx R\sqrt[3]{\frac{M_2}{3M_1}}.
\end{equation}

Расстояния от центра масс системы до точек $L_4$ и $L_5$ в координатной системе с центром координат в центре масс системы рассчитываются по  формулам
\begin{equation}
	 r_4 = \left ( \frac{R}{2} \cdot \frac{M_1-M_2}{M_1+M_2} ,   \frac{\sqrt{3}R}{2} \right ), \quad r_5 = \left ( \frac{R}{2} \cdot \frac{M_1-M_2}{M_1+M_2} ,   -\frac{\sqrt{3}R}{2} \right ). 
\end{equation}
\subsection{Приливы и отливы}

\term{Приливы и отливы}~--- периодические вертикальные колебания уровня океана, являющиеся результатом изменения положения Луны и Солнца. Хотя силы тяготения Солнца почти в 200 раз больше, чем силы тяготения Луны, приливные силы, порождаемые Луной, почти вдвое больше порождаемых Солнцем. Это происходит из-за того, что приливные силы зависят не от величины гравитационного поля, а от степени его неоднородности. Высота приливов зависит от взаимного расположения Луны и Солнца: наибольший~---  силы от Луны и от Солнца действуют вдоль одного направления, а наименьший~--- под прямым углом друг к другу.

\begin{minipage}{.24\tw}
Ускорение в центре Земли ($T$) определяется формулой \eqref{eq:g}:
\begin{equation*}
	a_T=\frac{G M}{r^2},
\end{equation*}
$M$~--- масса возмущающего тела,
\end{minipage}
\hfill
\begin{minipage}{0.74\tw}
	\vspace{-.5pc}
	\includegraphics[width = \tw]{Ebb_flow}
	\captionof{figure}{К объяснению приливных сил}\label{Ebb_flow}
\end{minipage}\\[-0.5pc]

$r$~--- расстояние между центрами Земли и данного тела. Аналогично, ускорения в точках $A$ и $B$ равны соответственно
\begin{equation}
	a_A = \frac{G M}{(r - R)^2} \quad \text{и} \quad a_B = \frac{GM}{(r + R)^2},
\end{equation}
где $R$~--- радиус Земли или иного тела, подверженного воздействию приливных сил. Ускорение в точке $A$ относительно точки $T$ равно
\begin{equation}
	a_A - a_T = a_T \cdot \frac{2 r R - R^2}{(r - R)^2} = \frac{GM \left(2 r R - R^2 \right)}{r^2 (r - R)^2} \xrightarrow{R \ll r} \frac{2 G M R}{r^3}.
	\label{eq:ebb-force}
\end{equation}

Под действием лунного притяжения водная оболочка Земли принимает форму 
эллипсоида, который вытянут по направлению к Луне. Близ точек $A$ и $B$ будет 
прилив, а в точках $F$ и $D$ --- отлив (см.~Рис.\,\ref{Ebb_flow}).
\nopagebreak
\subsection{Затмения}
Диаметр тени спутника при полном центральном затмении (когда центры трёх тел лежат на одной прямой) с большой точностью равен 
\begin{equation}
d_\text{тени} = 2 \cdot \frac{R_{\moon}(a_\oplus - R_\oplus) - R_\odot \left( a_{\moonч} - R_\oplus \right)}{a_\oplus - a_{\moon}}.
\end{equation}
Среднее значение  этой величины около 200~км, максимальное~--- около 215~км. При нецентральном затмении максимальный диаметр тени Луны на поверхности Земли может достигать 270~км. Это даёт оценку на продолжительность, равную 7.5 минутам. Большинство полных затмений длятся 2\,--\,4~минуты.

\begin{figure}[h!]
\centering
\vspace{-.5pc}
\includegraphics[width = 10cm]{full_eclipse}
\caption{Полное солнечное затмение со стороны северного полюса эклиптики}
\label{fig:eclipses-full-solar-eslipse}
\end{figure}
При \term{кольцеобразном солнечном затмении} Луна так расположена относительно Земли, что конус её тени не достаёт до поверхности планеты, и вокруг Луны можно наблюдать яркое кольцо незакрытой части солнечного диска.

\begin{figure}[h!]
	\centering
	\includegraphics[width = 10cm]{partly-eclipse}
	\caption{Кольцеобразное солнечное затмение со стороны северного полюса эклиптики}
	\label{fig:eclipses-circle-solar-eslipse}
\end{figure}
При особом расположении Луны и Земли возможны \term{гибридные} затмения, когда в разных пунктах Земли наблюдаются либо \imp{кольцеобразное}, либо \imp{полное} затмение. Причиной такого явления является шарообразность Земли.

\vspace{-1pc}
\begin{figure}[h!]
	\centering
	\includegraphics[width=10cm]{moon-eclipse}
	\caption{Схема лунного затмения со стороны северного полюса эклиптики}
	\label{fig:moon-eclipse-scheme}
\end{figure}
\term{Лунное затмение}, в отличие от солнечного, видно со всего ночного полушария Земли. Диаметр земной тени на расстоянии Луны превышает размер последней примерно в 2.5\,--\,3 раза. Бывают \term{частные}, когда лишь часть Луны попадает в земную тень, \term{полные}~--- Луна полностью погружается в тень Земли, и \term{полутеневые}~--- Луна проходит через полутень Земли, не затрагивая конус тени.

\term{Синодический месяц}~--- промежуток времени между одинаковыми фазами Луны, равен 29.53 суток.

\term{Драконический месяц}~--- промежуток времени между двумя последовательными прохождениями Луны через один и тот же узел орбиты,~--- 27.21 суток.

\term{Сарос}~--- промежуток  времени, по прошествии которого солнечные и лунные затмения повторяются в прежнем порядке. Происходит это из-за того, что каждый сарос Луна, орбита Луны и Солнце возвращаются в прежнее положение относительно далёких звёзд. Сарос длится ровно 242 драконических месяца, или 223 синодических месяца, или 18 лет 11 дней 8 часов.

\begin{wrapfigure}[8]{r}{.42\tw}
	\centering
	\vspace{-1pc}
	\includegraphics[width = 0.2\textwidth]{phases}
	\hfill
	\includegraphics[width = 0.2\textwidth]{phases-2}
	\caption{Частное и полное затмение}
	\label{fig:part-eclipses-scheme}
\end{wrapfigure}
Важной характеристикой любого затмения является его \term{фаза}~--- для \imp{частных} и \imp{кольцеобразных} затмений: отношение закрытой части $x$ диаметра\footnote{Здесь имеется в виду \imp{угловой} диаметр} затмеваемого тела, проходящего через центр затмевающего тела, ко всему диаметру затмеваемого тела $D$; для \imp{полного}: единица плюс отношение расстояния\footnote{Расстояние между окружностями $l_1$ и $l_2$~--- это $\min |L_1L_2|$ по всем $L_1 \in l_1$ и $L_2 \in l_2$.} между краями дисков затмеваемого и затмевающего тел к диаметру затмеваемого тела $D$.
\begin{equation}
\Phi_{\text{част}} = \frac{x}{D} < 1, \quad \quad \quad \Phi_{\text{полн}} =  1 + \frac{\min\{d_1, d_2\}}{D} > 1.
\end{equation}
Иногда вводят такое понятие, как \term{площадная фаза затмения}, т.\,е. отношение площади закрытой части диска затмеваемого тела к полной площади его диска. Чаще всего  площадную фазу используют применительно к двойным звёздам, когда считают падение блеска при затмении одной звезды другой.

\input{sections/cel-mech.planet-config.tex}
\input{sections/cel-mech.phase-angle.tex}
\input{sections/cel-mech.synod-period.tex}
\subsection{Собственное движение звёзд}
\term{Собственным движением} $(\mu)$ называется изменение координат звёзд на небесной сфере, вызванное относительным движением звёзд и Солнца, обычно измеряется в mas/год.
\begin{equation}
	\mu = \frac{V_\tau}{D},
\end{equation}
где $V_\tau$~--- тангенциальная относительная скорость звезды, $D$~--- расстояние до неё.

\change{Разделяют также собственное движение по склонению~--- $\mu_\delta$ и собственное движение по прямому восхождению~--- $\mu_\alpha$, которые определяются следующими выражениями:}
\begin{equation}
  \mu_\delta = \frac{\delta(t_2) - \delta(t_1)}{t_2 - t_1}, \quad \quad \mu_\alpha = \frac{\alpha(t_2) - \alpha(t_1)}{t_2 - t_1}.
\end{equation}
\change{
\begin{wrapfigure}{r}{.4\tw}
\begin{flushright}
	\vspace{-1pc}
	\begin{tikzpicture}
	\footnotesize
	\draw [dashes] (0, 4) arc(90:0:3 and 4);
	\draw [dashes] (0, 4) arc(90:0:2 and 4); 
	%
	\draw [dashes] (3.47, 2) arc(0:-70:3.47 and 1.16);	
	\draw [dashes] (2.64, 3) arc(0:-70:2.64 and 0.88);
	%
	\draw [thick, -latex] (2.3, 2.55) arc(-34:-56:2.64 and 0.88);
	\draw [thick, -latex] (2.3, 2.55) arc(53:29:2 and 4);
	\draw [thick, -latex] (2.3, 2.55) .. controls (2.3, 1.9) and (2.1, 1.4) .. (1.93, 1.03);
	%
	\draw (.9, 2.2) node [anchor=south] {$\delta(t_1)$};
	\draw (1.2, .9) node [anchor=south] {$\delta(t_2)$};
	%
	\draw (2, 0) node [anchor=north] {$\alpha(t_2)$};
	\draw (3, 0) node [anchor=north] {$\alpha(t_1)$};
	%
	\draw [fill=white] (2.3, 2.55) circle (0.03);
	\draw [fill=white] (1.93, 1.03) circle (0.03);
	\draw [fill=white] (0, 4) circle (0.03);
	%
	\draw (0, 4) node [anchor=north] {$P$};
	%
	\draw (1.9, 2.4) node [anchor=south] {$\mu_\alpha$};
	\draw (2.6, 2.05) node [anchor=west] {$\mu_\delta$};
	\draw (2.06, 1.65) node [anchor=south] {$\mu$};
	%
\end{tikzpicture}
\end{flushright}
\end{wrapfigure}
 Как отсюда видно, $\mu_\alpha$ является угловой скоростью по малому кругу, а значит, зависит от $\delta$. Следовательно, полное собственное движение $\mu$ можно найти, как
\begin{equation}
	\mu = \sqrt{\mu_\delta^2 + \mu_\alpha^2 \cos^2 \delta},
\end{equation}
потому что радиус малого круга, состоящего из точек со склонением~$\delta$, равен $R \cos \delta$, где $R$~--- радиус сферы, содержащей этот круг.
}

\begin{figure}[h!]
\begin{subfigure}[b]{0.47\tw}
	\begin{tikzpicture}[scale=1.05]
	\footnotesize
	
%	\foreach \x in {0, .1,...,4} {
%		\draw [line width=.1pt] (\x - 1, 0) -- (\x - 1, 4);
%	};
%	
%	\foreach \x in {0, 1,...,4} {
%		\draw [line width=.4pt] (\x - 1, 0) -- (\x - 1, 4);
%	};
%	
%	\foreach \y in {0, .1,...,4} {
%		\draw [line width=.1pt] (-1, \y) -- (4, \y);
%	};
%	
%	\foreach \y in {0, 1,...,4} {
%		\draw [line width=.4pt] (-1, \y) -- (4, \y);
%	};
	
	\draw [double] (.21, .21) arc (45:104:.3);
	\draw (-.93, 3.71) arc (-76:-35:.3);
	
	\draw (0, 0) -- (-1, 4);
	\draw (0, 0) -- (2, 2);
	\draw (-1, 4) -- (2.6, 1.6);
	
	\draw [thick, -latex] (-1, 4) -- (0, 4.25);
	\draw [thick, -latex] (-1, 4) -- (-.6, 2.4);
	
	\draw [fill=white] (-1, 4) circle (.03);
	\draw [fill=white] (0, 0) circle (.03);
	\draw [fill=white] (2, 2) circle (.03);
	
	\draw (1, 1) node [anchor=north west] {$R$};
	\draw (-.45, 2.1) node [anchor=north east] {$R_0$};
	\draw (.5, 2.95) node [anchor=south west] {$V \Delta t$};
	\draw (0, 0) node [anchor=north] {Солнце};
	\draw (-1, 4) node [anchor=south east] {Звезда};
	
	\draw (.1, .3) node [anchor=south] {$\xi$};
	\draw (-.9, 3.75) node [anchor=north west] {$\gamma$};
	
	\draw (-.5, 4.15) node [anchor=south] {$V_\tau$};
	\draw (-.75, 3.1) node [anchor=east] {$V_r$};
\end{tikzpicture}
\caption{}
\label{pic:phase-angle-1}
\end{subfigure}
\hfill
\begin{subfigure}[b]{0.47\tw}
\begin{tikzpicture}[scale=0.9]
	\footnotesize
	
	\draw (.2, 4.86) arc (-45:-135:0.28);
	\draw [double] (-1.65, 1.51) arc (5:80:0.25);
	
	\draw (0, 5) .. controls (-1.5, 4) and (-2, 2) .. (-2, 0);
	\draw (0, 5) .. controls (1.5, 4) and (2, 2) .. (2, 0);
	\draw (-2, 0) .. controls (-1, -.5) and (1, -.5) .. (2, 0);
	\draw (-1.9, 1.5) .. controls (-1, 1.5) and (1, 2) .. (1.5, 3);
	
	\draw [fill=white] (0, 5) circle (.03);
	\draw [fill=white] (-2, 0) circle (.03);
	\draw [fill=white] (2, 0) circle (.03);
	\draw [fill=white] (-1.9, 1.5) circle (.03);
	\draw [fill=white] (1.5, 3) circle (.03);
	
	\draw (-2, .2) -- (-1.8, .11) -- (-1.8, -.09);
	\draw (2, .2) -- (1.8, .11) -- (1.8, -.09);
	
	\draw (0, 5) node [anchor=south] {$P$};
	\draw (0, 1.9) node [anchor=north] {$\xi$};
	\draw (0, -.4) node [anchor=south] {$\Delta \alpha$};
	\draw (0, 4.8) node [anchor=north] {$\Delta \alpha$};
	\draw (-1, 4) node [anchor=east] {$90^\circ - \delta$};
	\draw (.9, 4.2) node [anchor=west] {$90^\circ - (\delta + \Delta \delta)$};
	\draw (0, -.4) node [anchor=north] {Небесный экватор};
\end{tikzpicture}
\caption{}
%\label{pic:phase-angle-2}
\end{subfigure}
\caption{}
\end{figure}


\change{Получим выражение для координат звезды, имеющей собственное движение $\mu = (\mu_\alpha, \mu_\delta)$, лучевую скорость $V_r$ и параллакс в начальный момент времени $\pi_0$. Найдем сначала тангенциальную скорость:
\begin{equation*}
	V_\tau = R_0 \sqrt{ \mu_\delta^2 + \mu_\alpha^2 \cos^2 \delta} = \frac{\sqrt{ \mu_\delta^2 + \mu_\alpha^2 \cos^2 \delta}}{\pi_0}.
\end{equation*}
Определим теперь угол между лучем зрения и полной скоростью звезды:
\begin{equation*}
	\gamma = \arctan \frac{V_\tau}{V_r}.
\end{equation*}
При этом полная скорость равна
\begin{equation*}
	V_0 = \sqrt{V_\tau^2 + V_r^2}.
\end{equation*}
Из теоремы косинусов можно найти расстояние для звезды через промежуток времени $\Delta t$:
\begin{equation*}
	R = \sqrt{R_0^2 + (V_0 \Delta t)^2 - 2 R_0 V_0 \Delta t \cos \gamma}.
\end{equation*}
Тогда угловое перемещение звезды равно
\begin{equation*}
	\sin \xi = \frac{V_0 \Delta t \sin \alpha}{R}.
\end{equation*}
Через компоненты собственного движения нетрудно получить угол между направлением на полюс и вектором полного собственного движения в начальный момент:
\begin{equation*}
	\tg \psi =  \frac{\mu_a \cos \delta}{\mu_\delta}.
\end{equation*}
Теперь с помощью сферической теоремы косинусов можно определить склонение звезды через время $\Delta t$:
\begin{equation*}
	\sin (\delta - \Delta \delta) = \cos \xi \sin \delta + \sin \xi \cos \delta \cos \psi.
\end{equation*}
Далее из сферической теоремы синусов получаем выражение для изменения прямого восхождения за время $\Delta t$~---
\begin{equation*}
	\sin \Delta \alpha = \frac{\sin \psi \sin \xi}{\cos (\delta - \Delta \delta)}.
\end{equation*}
}






\input{sections/cel-mech.precession.tex}
\subsection{Кеплеровы элементы орбиты}

\term{Кеплеровы элементы} --- шесть элементов орбиты, определяющие положение
\begin{wrapfigure}[14]{r}{0.45\tw}
	\centering
	\vspace{-1pc}
	\includegraphics[width=.45\tw]{orbit-elem}
	\captionof{figure}{Кеплеровы элементы орбиты}
	\label{fig:orb-elem}
\end{wrapfigure}
небесного тела в пространстве в задаче двух тел: \imp{большая полуось} ($a$), \imp{эксцентриситет} ($e$), \imp{наклонение} ($i$), \imp{аргумент перицентра} ($\omega$), \imp{долгота восходящего узла} ($\Omega$), \imp{средняя аномалия} ($M_0$). Первые два определяют форму орбиты, третий, четвёртый и пятый~--- ориентацию плоскости орбиты по отношению к базовой системе координат, связанной с эклиптикой, последний~--- положение тела на орбите~(см.~Рис.\,\ref{fig:orb-elem}).

\term{Наклонение}~--- это угол между плоскостью орбиты небесного тела и плоскостью эклиптики.

\term{Аргумент перицентра}~--- угол между направлениями на восходящий узел орбиты и на перицентр при наблюдении из притягивающего центра.

\term{Долгота восходящего узла}~--- угол в плоскости эклиптики между направлением на точку весеннего равноденствия и восходящий узел орбиты. Отсчитывается против часовой стрелки от направления на точку весеннего равноденствия.

\term{Средняя аномалия} для тела, движущегося по невозмущённой орбите~--- произведение его среднего движения и интервала времени после прохождения перицентра.

\term{Узлы орбиты}~--- точки пересечения орбиты и плоскости эклиптики. \imp{Восходящий узел}~--- точка, в которой тело пересекает плоскость эклиптики при движении в северноим направлении, а \imp{нисходящий}~--- в южном.

\term{Истинная аномалия}~($\nu$)~--- угол между радиус вектором и направлением на перицентр.
\subsection{Точки Лагранжа}

\term{Точки Лагранжа}~--- точки, во вращающейся системе из двух массивных тел,
\begin{wrapfigure}[14]{l}{0.48\tw}
	\centering
	\vspace{-.5pc}
	\includegraphics[width = .48\tw]{lagr-points}
	\captionof{figure}{Точки Лагранжа}
	\label{pic:larg-points}	
\end{wrapfigure}
в которых третье тело с пренебрежимо 
малой массой, не испытывающее воздействие никаких 
других сил, кроме гравитационных, со стороны двух 
первых тел, может оставаться неподвижным относительно 
этих тел. В этих точках гравитационные силы, 
действующие на малое тело, уравновешиваются силами инерции.

Точки $L_1$, $L_2$ и $L_3$ лежат на одной прямой, 
соединяющей два массивных тела. Точки $L_4$ и $L_5$ 
образуют равносторнние треугольники с массивными 
телами.

Для расстояний до точек $L_1$, $L_2$ и $L_3$ от 
центра масс системы справедливы следующие выражения:
\begin{equation}r_1=R\left(1-\sqrt[3]{\frac{\alpha}
{3}}\right), \quad r_2=R\left(1+\sqrt[3]{\frac{\alpha}
{3}}\right), \quad r_3=\left(1+\frac{5}{12}\alpha\right),
\end{equation}
где $\alpha=M_2 / (M_1 + M_2)$, $R$~--- расстояние между 
телами, $M_1$ --- масса более массивного тела, $M_2$
 --- масса второго тела.

Если $M_2 \ll M_1$, то точки $L_1$ и $L_2$ находятся 
примерно на одинаковом расстоянии от тела $M_2$, равном
\begin{equation}
r\approx R\sqrt[3]{\frac{M_2}{3M_1}}
\end{equation}

\subsection{Приливы и отливы}

\term{Приливы и отливы}~--- периодические вертикальные колебания уровня океана, являющиеся результатом как изменения положения Луны, так Солнца. Хотя силы тяготения Солнца почти в 200 раз больше, чем силы тяготения Луны, приливные силы, порождаемые Луной, почти вдвое больше порождаемых Солнцем. Это происходит из-за того, что приливные силы зависят не от величины гравитационного поля, а от степени его неоднородности. Высота приливов зависит от взаимного расположения Луны и Солнца: наибольший прилив, когда приливообразующие силы от Луны и от Солнца действуют вдоль одного направления, а наименьший прилив~--- под прямым углом друг к другу.

\begin{figure}[h!]
	\centering
	\includegraphics[width = 0.8\tw]{Ebb_flow}
	\caption{К объяснению приливных сил}\label{Ebb_flow}
\end{figure}
Ускорение в центре Земли ($T$) определяется формулой \eqref{eq:g}:
\begin{equation}
	a_T=\frac{G M}{r^2},
\end{equation}
где $M$~--- масса возмущающего тела, $r$~--- расстояние между центрами Земли и данного тела. Аналогично, ускорения в точках $A$ и $B$ равны соответственно
\begin{equation}
	a_A = \frac{G M}{(r - R)^2} \quad \text{и} \quad a_B = \frac{GM}{(r + R)^2},
\end{equation}
где $R$~--- радиус Земли или иного тела, подверженного воздействию приливных сил. Ускорение в точке $A$ относительно точки $T$ равно
\begin{equation}
	a_A - a_T = a_T \cdot \frac{2 r R - R^2}{(r - R)^2} = \frac{GM \left(2 r R - R^2 \right)}{r^2 (r - R)^2}.
	\label{eq:ebb-force}
\end{equation}
При условии $R\ll r$, выражение \eqref{eq:ebb-force} принимает вид \begin{equation}
a_A - a_T = a_T \cdot \frac{2 R}{r} = \frac{2 G M R}{r}.
\end{equation}

Под действием лунного притяжения водная оболочка Земли принимает форму 
эллипсоида, который вытянут по направлению к Луне. Близ точек $A$ и $B$ будет 
прилив, а в точках $F$ и $D$ --- отлив (см.~Рис.\,\ref{Ebb_flow}).
\nopagebreak
\subsection{Солнечные и лунные затмения}
Диаметр тени спутника при полном центральном затмении (когда центры трёх тел лежат на одной прямой), с большой точностью равен: 
\begin{equation}
d_\text{тени} = 2 \cdot \frac{R_{\moon}(a_\oplus - R_\oplus) - R_\odot \left( a_\oplus - R_\oplus \right)}{a_\oplus - a_{\moon}}.
\end{equation}
Среднее значение  этой величины около 200 км, максимальное около 215 км. При нецентральном затмении максимальный диаметр тени Луны на поверхности Земли может достигать 270~км. Что дает оценку на продолжительность~--- 7.5~минут. Большинство же полных затмений длятся 2\,--\,4~минуты.

\begin{figure}[h!]
\centering
\vspace{-.5pc}
\includegraphics[width = 10cm]{full_eclipse}
\caption{Полное солнечное затмение со стороны северного полюса эклиптики}
\label{fig:eclipses-full-solar-eslipse}
\end{figure}
При \term{кольцеобразном солнечном затмении} Луна относительно Земли расположена так, что конус её тени не достаёт до поверхности планеты, и вокруг Луны можно наблюдать яркое кольцо незакрытой части солнечного диска.

\begin{figure}[h!]
	\centering
	\includegraphics[width = 10cm]{partly-eclipse}
	\caption{Кольцеобразное солнечное затмение со стороны северного полюса эклиптики}
	\label{fig:eclipses-circle-solar-eslipse}
\end{figure}
При особом расположении Луны и Земли возможны \term{гибридные} затмения, когда в разных пунктах Земли наблюдаются \imp{кольцеобразное} и \imp{полное} затмение. Причиной такого явления является шарообразность Земли.

\vspace{-1pc}
\begin{figure}[h!]
	\centering
	\includegraphics[width=10cm]{moon-eclipse}
	\caption{Схема лунного затмения со стороны северного полюса эклиптики}
	\label{fig:moon-eclipse-scheme}
\end{figure}
\term{Лунное затмение} в отличие от солнечного, видно со всего ночного полушария Земли. Диаметр земной тени на расстоянии Луны превышает размер последней примерно в 2.5\,--\,3 раза. Бывают \term{частные}, когда лишь части Луны попадает в земную тень, \term{полные}~--- Луна полностью погружается в тень Земли, и \term{полутеневые}~--- Луна проходит через полутень Земли, не затрагивая конус тени.

\term{Синодический месяц}~--- промежуток времени между одинаковыми фазами Луны, равен 29.53 суток.

\term{Драконический месяц}~--- промежуток времени между двумя последовательными прохождениями Луны через один и тот же узел орбиты, равен 27.21 суток.

\term{Сарос}~--- промежуток  времени, по прошествии которого солнечные и 
лунные затмения повторяются в прежнем порядке. Сарос длится ровно 242 драконических месяца или 223 синодических месяца. Таким образом, его продолжительность  составляет примерно 18 лет 11 дней 8 часов.

\begin{wrapfigure}[8]{r}{.42\tw}
	\centering
	\vspace{-.5pc}
	\includegraphics[width = 0.2\textwidth]{phases}
	\hfill
	\includegraphics[width = 0.2\textwidth]{phases-2}
	\caption{Частное и полное затмение}
	\label{fig:part-eclipses-scheme}
\end{wrapfigure}
Важной характеристикой любого затмения является его \term{фаза}~--- отношение закрытой части диаметра затмеваемого тела, проходящей через центр затмевающего тела, к полному диаметру затмеваемого тела. Для полного затмения эта величина рассчитывается немного иначе. Для Луны затмевающим <<телом>> является тень Земли. Пусть $D$~--- диаметр затмеваемого тела, тогда фазы \imp{частного} и \imp{полного} затмений
\begin{equation}
\Phi_{\text{част}} = \frac{x}{D}, \quad \quad \quad \Phi_{\text{полн}} = 1 + \frac{d}{D}.
\end{equation}

Иногда вводят такое понятие, как \term{площадная фаза затмения}, т.\,е. отношение площади закрытой части диска затмеваемого диска к полной площади его диска. Чаще всего  площадную фазу используют применительно к двойным звёздам, когда считают падение блеска при затмении одной звезды другой.

\section{Небесная механика}
\subsection{Расстояние и размеры}
\term{Астрономическая единица}~--- единица измерения расстояния в астрономии, равная большой полуоси орбиты Земли. \begin{equation}
	1~\au = 149\:597\:870\:700~\text{м} \simeq 1.5 \times 10^{11}~\text{м}.
\end{equation}

\term{Годичный параллакс}\footnote{Важно отметить, здесь x$\pi$~--- лишь обозначение, ничего общего с числом $\pi$ не имеющее.} ($\pi$) объекта~--- это угол, под которым видно 
орбиту Земли из окрестностей данного объекта. Применяется к объектам вне 
Солнечной системы. \begin{equation}
	\tg \pi = \frac{a_\oplus}{r},
	\label{eq:parallax-sin}	
\end{equation}
где $a_\oplus$~--- большая полуось орбиты Земли и $r$~--- расстояние до объекта 
имеют одинаковые единицы измерений. Учитывая малость угла $\pi$, можно считать $\tg\pi \simeq \pi$ в \eqref{eq:parallax-sin}, тогда
\begin{equation}
	\pi = \frac{a_\oplus}{r}.
	\label{eq:parallax}
\end{equation} 
\begin{figure}[h!]
	\centering
	\vspace{-1pc}
	\includegraphics[width = 0.7\tw]{parallax.pdf}
	\caption{Схема годичного параллакса}
\end{figure}

Расстояние $r$, с которого большая полуось орбиты Земли $a_\oplus$ видна под углом $\pi = 1''$ называется \term{1 парсеком}. Так как \begin{equation}
	1~\text{рад} = \frac{180^\circ}{\pi} \simeq  3 438' \simeq 206265'' 
\quad \Longrightarrow \quad \mathsf{1~\text{\sffamily пк} = 
206265~\text{\sffamily а.\,е.}},
\end{equation} 
следовательно, записывая большую полуось орбиты Земли в \au, а расстояние до звезды в парсеках, получаем параллакс в секундах. Таким образом,
\begin{equation}
	r_\text{пк} = \frac{1~\au}{\pi''}.
\end{equation}

\term{Угловой размер объекта}~--- это угол, под которым видно объект. Для сферически симметричных объектов с радиусом $R$, угловой размер (диаметр) при наблюдении с расстояния $r$ определяется как
\begin{equation}
\rho = 2 \arcsin \frac{R}{r}.
\end{equation}
В случае, когда $r\gg R$, можно считать, что $\sin \rho \simeq \rho$, тогда
\begin{equation}
	\rho \simeq \frac{2 R}{r}.
\end{equation}

\vspace{-1.5pc}
\begin{figure}[h!]
	\begin{minipage}[b]{0.5\tw}
		\begin{flushleft}
			\includegraphics[width = 0.93\tw]{angle-size}
			\captionof{figure}{Угловой размер}
		\end{flushleft}
	\end{minipage}
	\begin{minipage}[b]{0.5\tw}
		\centering
		\includegraphics[width = \tw]{parallax-horiz}
		\captionof{figure}{Горизонтальный параллакс}
	\end{minipage}
\end{figure}

\term{Горизонтальный параллакс}~$(p)$~--- это угловой радиус Земли при наблюдении с объекта:
\begin{equation}
\sin p=\frac{R_\oplus}{r}.
\end{equation}

\term{Правило Тициуса-Боде} --- эмпирическая формула, приблизительно описывающая 
радиусы орбит планет в Солнечной системе:
\begin{equation}r=\frac{3\cdot 2^n+4}{10}, \quad n=-\infty, 0, 1, 2...
\end{equation}


\subsection{Закон всемирного тяготения}
Согласно \imp{закону всемирного тяготения}, сила притяжения
между двумя точечными телами с массами $M$ и $m$,
находящимися на расстоянии $r$, равна
\begin{equation}
	F=\frac{GMm}{r^2}, \label{eq:grav-law}
\end{equation}\nopagebreak где $G\simeq 6.67\cdot 10^{-11}~\text{м}^3 /
\left( \text{кг} \cdot \text{с}^2 \right)$~---
\term{гравитационная постоянная}.

\term{Гравитационный потенциал} поля точечной (или сферически
симметричной) массы $M$ на расстоянии $r$ от нее равен
работе, которую необходимо затратить, чтобы принести
единичную массу с бесконечности в данную точку. Так как
гравитационные силы между двумя массами --- это силы
притяжения, то эта работа отрицательна. Данная
величина также является \term{потенциальной энергией} точечной
массы на расстоянии $r$ от массы $M$, а выражение для нее имеет
следующий вид:
\begin{equation}
	U=-\frac{GM}{r}.
\end{equation}

Напряженность гравитационного поля $dU/dr$ часто называют
\term{ускорением свободного падения} $g$, она вычисляется по формуле
\begin{equation}
	g = \frac{GM}{r^2}.
	\label{eq:g}
\end{equation}
Тогда (\ref{eq:grav-law}) можно записать как
\begin{equation}
	F = mg.
\end{equation}

\input{sections/cel-mech.energy-conserv.tex}
\input{sections/cel-mech.kepler-laws.tex}
\subsection{Движение по орбите}
\begin{figure}[t]
	\centering
	\begin{tikzpicture}
		\footnotesize
		
		%	\foreach \x in {0, .1,...,5} {
		%		\draw [line width=.1pt] (\x, -3) -- (\x, 3);
		%	};
		%
		%	\foreach \y in {-3, -2.9,...,3} {
		%		\draw [line width=.1pt] (0, \y) -- (5, \y);
		%	};
		
		\draw [thick] (0, 0) .. controls (2, 4) and (3, -1) .. (5, -1);
		\draw [-latex] (0, -2) -- (1.25, 1.5);
		\draw [-latex] (0, -2) -- (3.3, .1);
		\draw [-latex] (1.25, 1.5) -- (3.3, .1);
		
		\draw (.3, -1.8) arc(31:70:0.36);
		
		\draw (.6, -.3) node [anchor = east] {$\vec{r}(t)$};
		\draw (1.6, -.9) node [anchor = north west] {$\vec{r}(t + dt)$};
		\draw (2.3, 0.9) node [anchor = north east] {$d\vec{r}$};
		\draw (0.2, -1.5) node [anchor = south west] {$\boldsymbol{\omega} \,dt$};
		
		\draw[fill=white] (1.25, 1.5) circle (0.03);
		\draw[fill=white] (3.3, .1) circle (0.03);
		\draw[fill=white] (0, -2) circle (0.03);
		
	\end{tikzpicture}
	\caption{}
\end{figure}

Рассмотрим такую физическую величину, как \term{секториальная скорость}~--- это векторная величина, описывающая ориентированную площадь, заметаемую радиус вектором тела за единицу времени. Пусть в момент времени $t$ тело находилось в точке $\vec{r}(t)$, а через промежуток времени $dt$~--- в точке $\vec{r}(t + dt)$. Обозначим перемещение тела за этот промежуток времени как $d\vec{r}$. Его можно выразить через скорость тела в момент времени $t$, считая ее постоянной на промежутке от $t$ до $t + dt$: $d\vec{r} = \vec{v} \, dt$. Площадь, которую заметает радиус-вектор тело $\vec{r}(t)$ равна половине параллелограмма, построенного на векторах $\vec{r}(t)$ и $d\vec{r}$. Поэтому можно записать
\begin{equation*}
	\vec{s} = \frac{1}{2} [\vec{r} \times \vec{v} dt],
\end{equation*}
следовательно секториальная скорость равна
\begin{equation*}
	\boldsymbol{\sigma} = \frac{d \vec{s}}{dt} = \frac{1}{2} [\vec{r} \times \vec{v}] = \frac{\vec{l}}{2} = \frac{\vec{L}}{2m},
\end{equation*}
где $\vec{l}$~--- удельный момент импульса (на единицу массы). Полученное выражение доказывает \imp{второй закон Кеплера}.

С другой стороны, перемещение $d\vec{r}$ можно выразить через угловую скорость $\boldsymbol{\omega}$, как $d \vec{r} = [\vec{r} \times \boldsymbol{\omega}\,dt]$. Тогда
\begin{equation*}
	\boldsymbol{\sigma}
	= \frac{1}{2} \big[ \vec{r} \times [\vec{r} \times \boldsymbol{\omega} ]\big]
	= \vec{r} \underbrace{(\vec{r}, \boldsymbol{\omega})}_0 - \boldsymbol{\omega} ( \vec{r}, \vec{r} )
	= r^2 \boldsymbol{\omega}.
\end{equation*}
Получим \imp{третий закон Кеплера}, заметив, что модуль секториальной скорости можно записать, как
\begin{gather*}
	\sigma
	= \frac{S_\text{эл}}{T}
	= \frac{\pi a b}{T}
	= \frac{L}{2m},\\
	\frac{\pi a^2 \sqrt{1 - e^2}}{T}
	= \frac{m \sqrt{\dfrac{GM}{a} \cdot \dfrac{1 + e}{1 - e}} \cdot a(1-e)}{2m},\\
	\frac{4\pi^2 a^4 (1 - e^2)}{T^2}
	= a^2(1-e)^2 \cdot \frac{GM}{a} \cdot \frac{1 + e}{1-e}
\end{gather*}
\begin{equation}
	\frac{T^2}{a^3} = \frac{4\pi^2}{GM}.чч
\end{equation}

Получим еще одно важное соотношение~--- \term{интеграл энергии}~--- формулу для скорости тела на орбите с большой полуосью $a$ в точке, удалённой на расстояние~$r$ от центрального тела с массой $M$. Для этого рассмотрим  сначала точку перицентра ($q$, <<п>>) и апоцентра ($Q$, <<a>>) данной орбиты, запишем для них закон сохранения энергии и закон сохранения момента импульса:
\begin{gather*}
	-\frac{GMm}{q} + \frac{m v^2_\text{п}}{2} = -\frac{GMm}{Q} + \frac{m v^2_\text{а}}{2},\\
	mv_\text{п}q = mv_\text{a}Q.
\end{gather*}
Из ЗСМИ и выражений для перицентрического~$q$ и апоцентрического~$Q$ расстояний через большую полуось $a$ и эксцентриситет $e$ имеем:
\begin{equation*}
	\frac{v_\text{а}}{v_\text{п}} = \frac{1 - e}{1 + e}.
\end{equation*}
Использую это соотношения, преобразуем ЗСЭ:
\begin{gather}
	\frac{v_\text{п}^2}{2} \left( 1 - \frac{(1 -e)^2}{(1 + e)^2} \right) = GM \left( \frac{1}{a(1-e)} - \frac{1}{a(1+e)} \right),\\
	\frac{v_\text{п}^2}{2} \cdot \frac{ 1 + 2e + e^2 - 1 + 2e - e^2}{(1+e)^2} = \frac{GM}{a} \cdot \frac{1 + e - 1 +  e}{(1+e)(1-e)},\\
	v_\text{п} = \sqrt{\frac{GM}{a}}\sqrt{\frac{1+e}{1-e}}, \quad \quad v_\text{a} = \sqrt{\frac{GM}{a}}\sqrt{\frac{1-e}{1+e}}.
\end{gather}
Запишем теперь ЗСЭ для перицентра и произвольной точки орбиты на расстоянии $r$:
\begin{gather*}
	-\frac{GMm}{q} + \frac{m v^2_\text{п}}{2} = -\frac{GMm}{r} + \frac{m v^2}{2},\\
	-\frac{GMm}{q} + \frac{GMm}{2a} \cdot \frac{1+e}{1-e} = -\frac{GMm}{r} + \frac{m v^2}{2},\\
	v^2 = GM \left( \frac{2}{r} - \frac{2}{a(1 - e)} + \frac{1+e}{a (1-e) }\right) = GM \left( \frac{2}{r} - \frac{1}{a} \right),
\end{gather*}
\begin{equation}
	v = \sqrt{ GM \left( \frac{2}{r} - \frac{1}{a} \right)}.
	\label{eq:int-energy}
\end{equation}
Полученное выражение и называется интегралом энергии. Согласно \eqref{eq:int-energy} и \eqref{eq:ellipse-pol-eq} для скорости тела в произвольной точке орбиты также справедливо выражение
\begin{equation}
	v = \sqrt{\frac{GM}{p}\cdot(1 + 2 e \cos \nu + e^2)},
\end{equation}
где $\nu$~--- истинная аномалия, а $p$~--- фокальный параметр.

Найдем величину момента импульса пробной массы $m$ на эллиптической орбите. В силу постоянства данной величины, можно выбрать любую точку орбиты для её поиска. Проще всего рассмотреть перицентр или апоцентр, рассмотрим первый.
\begin{multline*}
	L
	= m v_q q
	= m \sqrt{\frac{GM}{a} \frac{1+e}{1-e}} \cdot a(1-e) =\\
	= m \sqrt{GMa (1 + e)(1-e)}
	= m \sqrt{GMa(1-e^2)}
	= m \sqrt{GMp}.
\end{multline*}

Для параболической также рассмотрим точку перицентра:
\begin{multline*}
	L
	= m v_q q
	= m v_2(q) q
	= m \sqrt{\frac{2GM}{q}} \cdot q =\\
	= m \sqrt{2GMq}
	= m \sqrt{2GM \cdot \frac{p}{2}}
	= m \sqrt{GMp}.
\end{multline*}


\input{sections/cel-mech.orbit-elem.tex}
\subsection{Точки Лагранжа}

\term{Точки Лагранжа}~--- точки, во вращающейся системе из двух массивных тел,
\begin{wrapfigure}[14]{l}{0.48\tw}
	\centering
	\vspace{-.5pc}
	\includegraphics[width = .48\tw]{lagr-points}
	\captionof{figure}{Точки Лагранжа}
	\label{pic:larg-points}	
\end{wrapfigure}
в которых третье тело с пренебрежимо 
малой массой, не испытывающее воздействие никаких 
других сил, кроме гравитационных, со стороны двух 
первых тел, может оставаться неподвижным относительно 
этих тел. В этих точках гравитационные силы, 
действующие на малое тело, уравновешиваются силами инерции.

Точки $L_1$, $L_2$ и $L_3$ лежат на одной прямой, 
соединяющей два массивных тела. Точки $L_4$ и $L_5$ 
образуют равносторнние треугольники с массивными 
телами.

Для расстояний до точек $L_1$, $L_2$ и $L_3$ от 
центра масс системы справедливы следующие выражения:
\begin{equation}r_1=R\left(1-\sqrt[3]{\frac{\alpha}
{3}}\right), \quad r_2=R\left(1+\sqrt[3]{\frac{\alpha}
{3}}\right), \quad r_3=R\left(1+\frac{5}{12}\alpha\right),
\end{equation}
где $\alpha=M_2 / (M_1 + M_2)$, $R$~--- расстояние между 
телами, $M_1$ --- масса более массивного тела, $M_2$
 --- масса второго тела.

Если $M_2 \ll M_1$, то точки $L_1$ и $L_2$ находятся 
примерно на одинаковом расстоянии от тела $M_2$, равном
\begin{equation}
r\approx R\sqrt[3]{\frac{M_2}{3M_1}}.
\end{equation}

Расстояния от центра масс системы до точек $L_4$ и $L_5$ в координатной системе с центром координат в центре масс системы рассчитываются по  формулам
\begin{equation}
	 r_4 = \left ( \frac{R}{2} \cdot \frac{M_1-M_2}{M_1+M_2} ,   \frac{\sqrt{3}R}{2} \right ), \quad r_5 = \left ( \frac{R}{2} \cdot \frac{M_1-M_2}{M_1+M_2} ,   -\frac{\sqrt{3}R}{2} \right ). 
\end{equation}
\subsection{Приливы и отливы}

\term{Приливы и отливы}~--- периодические вертикальные колебания уровня океана, являющиеся результатом изменения положения Луны и Солнца. Хотя силы тяготения Солнца почти в 200 раз больше, чем силы тяготения Луны, приливные силы, порождаемые Луной, почти вдвое больше порождаемых Солнцем. Это происходит из-за того, что приливные силы зависят не от величины гравитационного поля, а от степени его неоднородности. Высота приливов зависит от взаимного расположения Луны и Солнца: наибольший~---  силы от Луны и от Солнца действуют вдоль одного направления, а наименьший~--- под прямым углом друг к другу.

\begin{minipage}{.24\tw}
Ускорение в центре Земли ($T$) определяется формулой \eqref{eq:g}:
\begin{equation*}
	a_T=\frac{G M}{r^2},
\end{equation*}
$M$~--- масса возмущающего тела,
\end{minipage}
\hfill
\begin{minipage}{0.74\tw}
	\vspace{-.5pc}
	\includegraphics[width = \tw]{Ebb_flow}
	\captionof{figure}{К объяснению приливных сил}\label{Ebb_flow}
\end{minipage}\\[-0.5pc]

$r$~--- расстояние между центрами Земли и данного тела. Аналогично, ускорения в точках $A$ и $B$ равны соответственно
\begin{equation}
	a_A = \frac{G M}{(r - R)^2} \quad \text{и} \quad a_B = \frac{GM}{(r + R)^2},
\end{equation}
где $R$~--- радиус Земли или иного тела, подверженного воздействию приливных сил. Ускорение в точке $A$ относительно точки $T$ равно
\begin{equation}
	a_A - a_T = a_T \cdot \frac{2 r R - R^2}{(r - R)^2} = \frac{GM \left(2 r R - R^2 \right)}{r^2 (r - R)^2} \xrightarrow{R \ll r} \frac{2 G M R}{r^3}.
	\label{eq:ebb-force}
\end{equation}

Под действием лунного притяжения водная оболочка Земли принимает форму 
эллипсоида, который вытянут по направлению к Луне. Близ точек $A$ и $B$ будет 
прилив, а в точках $F$ и $D$ --- отлив (см.~Рис.\,\ref{Ebb_flow}).
\nopagebreak
\subsection{Затмения}
Диаметр тени спутника при полном центральном затмении (когда центры трёх тел лежат на одной прямой) с большой точностью равен 
\begin{equation}
d_\text{тени} = 2 \cdot \frac{R_{\moon}(a_\oplus - R_\oplus) - R_\odot \left( a_{\moonч} - R_\oplus \right)}{a_\oplus - a_{\moon}}.
\end{equation}
Среднее значение  этой величины около 200~км, максимальное~--- около 215~км. При нецентральном затмении максимальный диаметр тени Луны на поверхности Земли может достигать 270~км. Это даёт оценку на продолжительность, равную 7.5 минутам. Большинство полных затмений длятся 2\,--\,4~минуты.

\begin{figure}[h!]
\centering
\vspace{-.5pc}
\includegraphics[width = 10cm]{full_eclipse}
\caption{Полное солнечное затмение со стороны северного полюса эклиптики}
\label{fig:eclipses-full-solar-eslipse}
\end{figure}
При \term{кольцеобразном солнечном затмении} Луна так расположена относительно Земли, что конус её тени не достаёт до поверхности планеты, и вокруг Луны можно наблюдать яркое кольцо незакрытой части солнечного диска.

\begin{figure}[h!]
	\centering
	\includegraphics[width = 10cm]{partly-eclipse}
	\caption{Кольцеобразное солнечное затмение со стороны северного полюса эклиптики}
	\label{fig:eclipses-circle-solar-eslipse}
\end{figure}
При особом расположении Луны и Земли возможны \term{гибридные} затмения, когда в разных пунктах Земли наблюдаются либо \imp{кольцеобразное}, либо \imp{полное} затмение. Причиной такого явления является шарообразность Земли.

\vspace{-1pc}
\begin{figure}[h!]
	\centering
	\includegraphics[width=10cm]{moon-eclipse}
	\caption{Схема лунного затмения со стороны северного полюса эклиптики}
	\label{fig:moon-eclipse-scheme}
\end{figure}
\term{Лунное затмение}, в отличие от солнечного, видно со всего ночного полушария Земли. Диаметр земной тени на расстоянии Луны превышает размер последней примерно в 2.5\,--\,3 раза. Бывают \term{частные}, когда лишь часть Луны попадает в земную тень, \term{полные}~--- Луна полностью погружается в тень Земли, и \term{полутеневые}~--- Луна проходит через полутень Земли, не затрагивая конус тени.

\term{Синодический месяц}~--- промежуток времени между одинаковыми фазами Луны, равен 29.53 суток.

\term{Драконический месяц}~--- промежуток времени между двумя последовательными прохождениями Луны через один и тот же узел орбиты,~--- 27.21 суток.

\term{Сарос}~--- промежуток  времени, по прошествии которого солнечные и лунные затмения повторяются в прежнем порядке. Происходит это из-за того, что каждый сарос Луна, орбита Луны и Солнце возвращаются в прежнее положение относительно далёких звёзд. Сарос длится ровно 242 драконических месяца, или 223 синодических месяца, или 18 лет 11 дней 8 часов.

\begin{wrapfigure}[8]{r}{.42\tw}
	\centering
	\vspace{-1pc}
	\includegraphics[width = 0.2\textwidth]{phases}
	\hfill
	\includegraphics[width = 0.2\textwidth]{phases-2}
	\caption{Частное и полное затмение}
	\label{fig:part-eclipses-scheme}
\end{wrapfigure}
Важной характеристикой любого затмения является его \term{фаза}~--- для \imp{частных} и \imp{кольцеобразных} затмений: отношение закрытой части $x$ диаметра\footnote{Здесь имеется в виду \imp{угловой} диаметр} затмеваемого тела, проходящего через центр затмевающего тела, ко всему диаметру затмеваемого тела $D$; для \imp{полного}: единица плюс отношение расстояния\footnote{Расстояние между окружностями $l_1$ и $l_2$~--- это $\min |L_1L_2|$ по всем $L_1 \in l_1$ и $L_2 \in l_2$.} между краями дисков затмеваемого и затмевающего тел к диаметру затмеваемого тела $D$.
\begin{equation}
\Phi_{\text{част}} = \frac{x}{D} < 1, \quad \quad \quad \Phi_{\text{полн}} =  1 + \frac{\min\{d_1, d_2\}}{D} > 1.
\end{equation}
Иногда вводят такое понятие, как \term{площадная фаза затмения}, т.\,е. отношение площади закрытой части диска затмеваемого тела к полной площади его диска. Чаще всего  площадную фазу используют применительно к двойным звёздам, когда считают падение блеска при затмении одной звезды другой.

\input{sections/cel-mech.planet-config.tex}
\input{sections/cel-mech.phase-angle.tex}
\input{sections/cel-mech.synod-period.tex}
\subsection{Собственное движение звёзд}
\term{Собственным движением} $(\mu)$ называется изменение координат звёзд на небесной сфере, вызванное относительным движением звёзд и Солнца, обычно измеряется в mas/год.
\begin{equation}
	\mu = \frac{V_\tau}{D},
\end{equation}
где $V_\tau$~--- тангенциальная относительная скорость звезды, $D$~--- расстояние до неё.

\change{Разделяют также собственное движение по склонению~--- $\mu_\delta$ и собственное движение по прямому восхождению~--- $\mu_\alpha$, которые определяются следующими выражениями:}
\begin{equation}
  \mu_\delta = \frac{\delta(t_2) - \delta(t_1)}{t_2 - t_1}, \quad \quad \mu_\alpha = \frac{\alpha(t_2) - \alpha(t_1)}{t_2 - t_1}.
\end{equation}
\change{
\begin{wrapfigure}{r}{.4\tw}
\begin{flushright}
	\vspace{-1pc}
	\begin{tikzpicture}
	\footnotesize
	\draw [dashes] (0, 4) arc(90:0:3 and 4);
	\draw [dashes] (0, 4) arc(90:0:2 and 4); 
	%
	\draw [dashes] (3.47, 2) arc(0:-70:3.47 and 1.16);	
	\draw [dashes] (2.64, 3) arc(0:-70:2.64 and 0.88);
	%
	\draw [thick, -latex] (2.3, 2.55) arc(-34:-56:2.64 and 0.88);
	\draw [thick, -latex] (2.3, 2.55) arc(53:29:2 and 4);
	\draw [thick, -latex] (2.3, 2.55) .. controls (2.3, 1.9) and (2.1, 1.4) .. (1.93, 1.03);
	%
	\draw (.9, 2.2) node [anchor=south] {$\delta(t_1)$};
	\draw (1.2, .9) node [anchor=south] {$\delta(t_2)$};
	%
	\draw (2, 0) node [anchor=north] {$\alpha(t_2)$};
	\draw (3, 0) node [anchor=north] {$\alpha(t_1)$};
	%
	\draw [fill=white] (2.3, 2.55) circle (0.03);
	\draw [fill=white] (1.93, 1.03) circle (0.03);
	\draw [fill=white] (0, 4) circle (0.03);
	%
	\draw (0, 4) node [anchor=north] {$P$};
	%
	\draw (1.9, 2.4) node [anchor=south] {$\mu_\alpha$};
	\draw (2.6, 2.05) node [anchor=west] {$\mu_\delta$};
	\draw (2.06, 1.65) node [anchor=south] {$\mu$};
	%
\end{tikzpicture}
\end{flushright}
\end{wrapfigure}
 Как отсюда видно, $\mu_\alpha$ является угловой скоростью по малому кругу, а значит, зависит от $\delta$. Следовательно, полное собственное движение $\mu$ можно найти, как
\begin{equation}
	\mu = \sqrt{\mu_\delta^2 + \mu_\alpha^2 \cos^2 \delta},
\end{equation}
потому что радиус малого круга, состоящего из точек со склонением~$\delta$, равен $R \cos \delta$, где $R$~--- радиус сферы, содержащей этот круг.
}

\begin{figure}[h!]
\begin{subfigure}[b]{0.47\tw}
	\begin{tikzpicture}[scale=1.05]
	\footnotesize
	
%	\foreach \x in {0, .1,...,4} {
%		\draw [line width=.1pt] (\x - 1, 0) -- (\x - 1, 4);
%	};
%	
%	\foreach \x in {0, 1,...,4} {
%		\draw [line width=.4pt] (\x - 1, 0) -- (\x - 1, 4);
%	};
%	
%	\foreach \y in {0, .1,...,4} {
%		\draw [line width=.1pt] (-1, \y) -- (4, \y);
%	};
%	
%	\foreach \y in {0, 1,...,4} {
%		\draw [line width=.4pt] (-1, \y) -- (4, \y);
%	};
	
	\draw [double] (.21, .21) arc (45:104:.3);
	\draw (-.93, 3.71) arc (-76:-35:.3);
	
	\draw (0, 0) -- (-1, 4);
	\draw (0, 0) -- (2, 2);
	\draw (-1, 4) -- (2.6, 1.6);
	
	\draw [thick, -latex] (-1, 4) -- (0, 4.25);
	\draw [thick, -latex] (-1, 4) -- (-.6, 2.4);
	
	\draw [fill=white] (-1, 4) circle (.03);
	\draw [fill=white] (0, 0) circle (.03);
	\draw [fill=white] (2, 2) circle (.03);
	
	\draw (1, 1) node [anchor=north west] {$R$};
	\draw (-.45, 2.1) node [anchor=north east] {$R_0$};
	\draw (.5, 2.95) node [anchor=south west] {$V \Delta t$};
	\draw (0, 0) node [anchor=north] {Солнце};
	\draw (-1, 4) node [anchor=south east] {Звезда};
	
	\draw (.1, .3) node [anchor=south] {$\xi$};
	\draw (-.9, 3.75) node [anchor=north west] {$\gamma$};
	
	\draw (-.5, 4.15) node [anchor=south] {$V_\tau$};
	\draw (-.75, 3.1) node [anchor=east] {$V_r$};
\end{tikzpicture}
\caption{}
\label{pic:phase-angle-1}
\end{subfigure}
\hfill
\begin{subfigure}[b]{0.47\tw}
\begin{tikzpicture}[scale=0.9]
	\footnotesize
	
	\draw (.2, 4.86) arc (-45:-135:0.28);
	\draw [double] (-1.65, 1.51) arc (5:80:0.25);
	
	\draw (0, 5) .. controls (-1.5, 4) and (-2, 2) .. (-2, 0);
	\draw (0, 5) .. controls (1.5, 4) and (2, 2) .. (2, 0);
	\draw (-2, 0) .. controls (-1, -.5) and (1, -.5) .. (2, 0);
	\draw (-1.9, 1.5) .. controls (-1, 1.5) and (1, 2) .. (1.5, 3);
	
	\draw [fill=white] (0, 5) circle (.03);
	\draw [fill=white] (-2, 0) circle (.03);
	\draw [fill=white] (2, 0) circle (.03);
	\draw [fill=white] (-1.9, 1.5) circle (.03);
	\draw [fill=white] (1.5, 3) circle (.03);
	
	\draw (-2, .2) -- (-1.8, .11) -- (-1.8, -.09);
	\draw (2, .2) -- (1.8, .11) -- (1.8, -.09);
	
	\draw (0, 5) node [anchor=south] {$P$};
	\draw (0, 1.9) node [anchor=north] {$\xi$};
	\draw (0, -.4) node [anchor=south] {$\Delta \alpha$};
	\draw (0, 4.8) node [anchor=north] {$\Delta \alpha$};
	\draw (-1, 4) node [anchor=east] {$90^\circ - \delta$};
	\draw (.9, 4.2) node [anchor=west] {$90^\circ - (\delta + \Delta \delta)$};
	\draw (0, -.4) node [anchor=north] {Небесный экватор};
\end{tikzpicture}
\caption{}
%\label{pic:phase-angle-2}
\end{subfigure}
\caption{}
\end{figure}


\change{Получим выражение для координат звезды, имеющей собственное движение $\mu = (\mu_\alpha, \mu_\delta)$, лучевую скорость $V_r$ и параллакс в начальный момент времени $\pi_0$. Найдем сначала тангенциальную скорость:
\begin{equation*}
	V_\tau = R_0 \sqrt{ \mu_\delta^2 + \mu_\alpha^2 \cos^2 \delta} = \frac{\sqrt{ \mu_\delta^2 + \mu_\alpha^2 \cos^2 \delta}}{\pi_0}.
\end{equation*}
Определим теперь угол между лучем зрения и полной скоростью звезды:
\begin{equation*}
	\gamma = \arctan \frac{V_\tau}{V_r}.
\end{equation*}
При этом полная скорость равна
\begin{equation*}
	V_0 = \sqrt{V_\tau^2 + V_r^2}.
\end{equation*}
Из теоремы косинусов можно найти расстояние для звезды через промежуток времени $\Delta t$:
\begin{equation*}
	R = \sqrt{R_0^2 + (V_0 \Delta t)^2 - 2 R_0 V_0 \Delta t \cos \gamma}.
\end{equation*}
Тогда угловое перемещение звезды равно
\begin{equation*}
	\sin \xi = \frac{V_0 \Delta t \sin \alpha}{R}.
\end{equation*}
Через компоненты собственного движения нетрудно получить угол между направлением на полюс и вектором полного собственного движения в начальный момент:
\begin{equation*}
	\tg \psi =  \frac{\mu_a \cos \delta}{\mu_\delta}.
\end{equation*}
Теперь с помощью сферической теоремы косинусов можно определить склонение звезды через время $\Delta t$:
\begin{equation*}
	\sin (\delta - \Delta \delta) = \cos \xi \sin \delta + \sin \xi \cos \delta \cos \psi.
\end{equation*}
Далее из сферической теоремы синусов получаем выражение для изменения прямого восхождения за время $\Delta t$~---
\begin{equation*}
	\sin \Delta \alpha = \frac{\sin \psi \sin \xi}{\cos (\delta - \Delta \delta)}.
\end{equation*}
}






\input{sections/cel-mech.precession.tex}
\subsection{Синодический период}

\term{Синодический период} (период смены фаз)~--- время, прошедшее между двумя последовательными одноимёнными конфигурациями одного тела при наблюдении с другого.

\imp{Относительная угловая скорость} планет равна 
разности скоростей углового перемещения одной планеты ($2\pi/T_1$) и другой ($2\pi/T_2 $) по орбите. Из определения относительной угловой скорости вытекает общая формула для продолжительности синодического периода: 
\begin{equation}
\frac1S=\left| \frac1T_1-\frac1T_2 \right|.
\end{equation}
Для внешних и внутренних планет соответственно выражения принимает следующий вид: 
\begin{equation} \frac{1}{S} = \frac{1}{T_\oplus} - \frac{1}{T_\text{пл}} \quad \text{и} \quad \frac{1}{S} = \frac{1}{T_\text{пл}} - \frac{1}{T_\oplus},
\end{equation}
где $S$~--- синодический период, $T_\text{пл}$~--- сидерический период планеты, $T_\oplus$~--- сидерический период обращения Земли.

В случае, если тела обращаются в противоположные стороны, то связь 
их синодического периода с сидерическими очевидным образом принимает вид:
\begin{equation}
\frac1S=\frac1T_1+\frac1T_2.
\end{equation}
\subsection{Фазы планет и спутников}

\term{Фаза} планеты (спутника)~--- отношение площади освещённой  части видимого диска ко всей его площади.
Фаза рассчитывается по формуле
\begin{equation}
\Phi = \frac{1 + \cos \phi}{2} = \cos^2 \frac{\phi}{2},
\end{equation}
\begin{minipage}{0.67\tw}
где $\phi$~--- \term{фазовый угол} --- угол между лучом света, падающим от Солнца на планету, и лучом, отразившимся от неё в сторону наблюдателя (см.~Рис.\,\ref{fig:phase-angel-scheme}). Фаза объекта может принимать значения от 0 до 1.

Видимая границы между освещенной и неосвещенной частями поверхности объекта называется \term{терминатором}. В зоне терминатора для наблюдателя на объекте источник пересекает горизонт.
\end{minipage}
\hfill
\begin{minipage}{0.31\tw}
	\hfill
	\vspace{-.5pc}
	\includegraphics[width = \tw]{phase-angle}
	\captionof{figure}{Фазовый угол}
	\label{fig:phase-angel-scheme}
\end{minipage}


\section{Небесная механика}
\subsection{Расстояние и размеры}
\term{Астрономическая единица}~--- единица измерения расстояния в астрономии, равная большой полуоси орбиты Земли. \begin{equation}
	1~\au = 149\:597\:870\:700~\text{м} \simeq 1.5 \times 10^{11}~\text{м}.
\end{equation}

\term{Годичный параллакс}\footnote{Важно отметить, здесь x$\pi$~--- лишь обозначение, ничего общего с числом $\pi$ не имеющее.} ($\pi$) объекта~--- это угол, под которым видно 
орбиту Земли из окрестностей данного объекта. Применяется к объектам вне 
Солнечной системы. \begin{equation}
	\tg \pi = \frac{a_\oplus}{r},
	\label{eq:parallax-sin}	
\end{equation}
где $a_\oplus$~--- большая полуось орбиты Земли и $r$~--- расстояние до объекта 
имеют одинаковые единицы измерений. Учитывая малость угла $\pi$, можно считать $\tg\pi \simeq \pi$ в \eqref{eq:parallax-sin}, тогда
\begin{equation}
	\pi = \frac{a_\oplus}{r}.
	\label{eq:parallax}
\end{equation} 
\begin{figure}[h!]
	\centering
	\vspace{-1pc}
	\includegraphics[width = 0.7\tw]{parallax.pdf}
	\caption{Схема годичного параллакса}
\end{figure}

Расстояние $r$, с которого большая полуось орбиты Земли $a_\oplus$ видна под углом $\pi = 1''$ называется \term{1 парсеком}. Так как \begin{equation}
	1~\text{рад} = \frac{180^\circ}{\pi} \simeq  3 438' \simeq 206265'' 
\quad \Longrightarrow \quad \mathsf{1~\text{\sffamily пк} = 
206265~\text{\sffamily а.\,е.}},
\end{equation} 
следовательно, записывая большую полуось орбиты Земли в \au, а расстояние до звезды в парсеках, получаем параллакс в секундах. Таким образом,
\begin{equation}
	r_\text{пк} = \frac{1~\au}{\pi''}.
\end{equation}

\term{Угловой размер объекта}~--- это угол, под которым видно объект. Для сферически симметричных объектов с радиусом $R$, угловой размер (диаметр) при наблюдении с расстояния $r$ определяется как
\begin{equation}
\rho = 2 \arcsin \frac{R}{r}.
\end{equation}
В случае, когда $r\gg R$, можно считать, что $\sin \rho \simeq \rho$, тогда
\begin{equation}
	\rho \simeq \frac{2 R}{r}.
\end{equation}

\vspace{-1.5pc}
\begin{figure}[h!]
	\begin{minipage}[b]{0.5\tw}
		\begin{flushleft}
			\includegraphics[width = 0.93\tw]{angle-size}
			\captionof{figure}{Угловой размер}
		\end{flushleft}
	\end{minipage}
	\begin{minipage}[b]{0.5\tw}
		\centering
		\includegraphics[width = \tw]{parallax-horiz}
		\captionof{figure}{Горизонтальный параллакс}
	\end{minipage}
\end{figure}

\term{Горизонтальный параллакс}~$(p)$~--- это угловой радиус Земли при наблюдении с объекта:
\begin{equation}
\sin p=\frac{R_\oplus}{r}.
\end{equation}

\term{Правило Тициуса-Боде} --- эмпирическая формула, приблизительно описывающая 
радиусы орбит планет в Солнечной системе:
\begin{equation}r=\frac{3\cdot 2^n+4}{10}, \quad n=-\infty, 0, 1, 2...
\end{equation}


\subsection{Закон всемирного тяготения}
Согласно \imp{закону всемирного тяготения}, сила притяжения
между двумя точечными телами с массами $M$ и $m$,
находящимися на расстоянии $r$, равна
\begin{equation}
	F=\frac{GMm}{r^2}, \label{eq:grav-law}
\end{equation}\nopagebreak где $G\simeq 6.67\cdot 10^{-11}~\text{м}^3 /
\left( \text{кг} \cdot \text{с}^2 \right)$~---
\term{гравитационная постоянная}.

\term{Гравитационный потенциал} поля точечной (или сферически
симметричной) массы $M$ на расстоянии $r$ от нее равен
работе, которую необходимо затратить, чтобы принести
единичную массу с бесконечности в данную точку. Так как
гравитационные силы между двумя массами --- это силы
притяжения, то эта работа отрицательна. Данная
величина также является \term{потенциальной энергией} точечной
массы на расстоянии $r$ от массы $M$, а выражение для нее имеет
следующий вид:
\begin{equation}
	U=-\frac{GM}{r}.
\end{equation}

Напряженность гравитационного поля $dU/dr$ часто называют
\term{ускорением свободного падения} $g$, она вычисляется по формуле
\begin{equation}
	g = \frac{GM}{r^2}.
	\label{eq:g}
\end{equation}
Тогда (\ref{eq:grav-law}) можно записать как
\begin{equation}
	F = mg.
\end{equation}

\input{sections/cel-mech.energy-conserv.tex}
\input{sections/cel-mech.kepler-laws.tex}
\subsection{Движение по орбите}
\begin{figure}[t]
	\centering
	\begin{tikzpicture}
		\footnotesize
		
		%	\foreach \x in {0, .1,...,5} {
		%		\draw [line width=.1pt] (\x, -3) -- (\x, 3);
		%	};
		%
		%	\foreach \y in {-3, -2.9,...,3} {
		%		\draw [line width=.1pt] (0, \y) -- (5, \y);
		%	};
		
		\draw [thick] (0, 0) .. controls (2, 4) and (3, -1) .. (5, -1);
		\draw [-latex] (0, -2) -- (1.25, 1.5);
		\draw [-latex] (0, -2) -- (3.3, .1);
		\draw [-latex] (1.25, 1.5) -- (3.3, .1);
		
		\draw (.3, -1.8) arc(31:70:0.36);
		
		\draw (.6, -.3) node [anchor = east] {$\vec{r}(t)$};
		\draw (1.6, -.9) node [anchor = north west] {$\vec{r}(t + dt)$};
		\draw (2.3, 0.9) node [anchor = north east] {$d\vec{r}$};
		\draw (0.2, -1.5) node [anchor = south west] {$\boldsymbol{\omega} \,dt$};
		
		\draw[fill=white] (1.25, 1.5) circle (0.03);
		\draw[fill=white] (3.3, .1) circle (0.03);
		\draw[fill=white] (0, -2) circle (0.03);
		
	\end{tikzpicture}
	\caption{}
\end{figure}

Рассмотрим такую физическую величину, как \term{секториальная скорость}~--- это векторная величина, описывающая ориентированную площадь, заметаемую радиус вектором тела за единицу времени. Пусть в момент времени $t$ тело находилось в точке $\vec{r}(t)$, а через промежуток времени $dt$~--- в точке $\vec{r}(t + dt)$. Обозначим перемещение тела за этот промежуток времени как $d\vec{r}$. Его можно выразить через скорость тела в момент времени $t$, считая ее постоянной на промежутке от $t$ до $t + dt$: $d\vec{r} = \vec{v} \, dt$. Площадь, которую заметает радиус-вектор тело $\vec{r}(t)$ равна половине параллелограмма, построенного на векторах $\vec{r}(t)$ и $d\vec{r}$. Поэтому можно записать
\begin{equation*}
	\vec{s} = \frac{1}{2} [\vec{r} \times \vec{v} dt],
\end{equation*}
следовательно секториальная скорость равна
\begin{equation*}
	\boldsymbol{\sigma} = \frac{d \vec{s}}{dt} = \frac{1}{2} [\vec{r} \times \vec{v}] = \frac{\vec{l}}{2} = \frac{\vec{L}}{2m},
\end{equation*}
где $\vec{l}$~--- удельный момент импульса (на единицу массы). Полученное выражение доказывает \imp{второй закон Кеплера}.

С другой стороны, перемещение $d\vec{r}$ можно выразить через угловую скорость $\boldsymbol{\omega}$, как $d \vec{r} = [\vec{r} \times \boldsymbol{\omega}\,dt]$. Тогда
\begin{equation*}
	\boldsymbol{\sigma}
	= \frac{1}{2} \big[ \vec{r} \times [\vec{r} \times \boldsymbol{\omega} ]\big]
	= \vec{r} \underbrace{(\vec{r}, \boldsymbol{\omega})}_0 - \boldsymbol{\omega} ( \vec{r}, \vec{r} )
	= r^2 \boldsymbol{\omega}.
\end{equation*}
Получим \imp{третий закон Кеплера}, заметив, что модуль секториальной скорости можно записать, как
\begin{gather*}
	\sigma
	= \frac{S_\text{эл}}{T}
	= \frac{\pi a b}{T}
	= \frac{L}{2m},\\
	\frac{\pi a^2 \sqrt{1 - e^2}}{T}
	= \frac{m \sqrt{\dfrac{GM}{a} \cdot \dfrac{1 + e}{1 - e}} \cdot a(1-e)}{2m},\\
	\frac{4\pi^2 a^4 (1 - e^2)}{T^2}
	= a^2(1-e)^2 \cdot \frac{GM}{a} \cdot \frac{1 + e}{1-e}
\end{gather*}
\begin{equation}
	\frac{T^2}{a^3} = \frac{4\pi^2}{GM}.чч
\end{equation}

Получим еще одно важное соотношение~--- \term{интеграл энергии}~--- формулу для скорости тела на орбите с большой полуосью $a$ в точке, удалённой на расстояние~$r$ от центрального тела с массой $M$. Для этого рассмотрим  сначала точку перицентра ($q$, <<п>>) и апоцентра ($Q$, <<a>>) данной орбиты, запишем для них закон сохранения энергии и закон сохранения момента импульса:
\begin{gather*}
	-\frac{GMm}{q} + \frac{m v^2_\text{п}}{2} = -\frac{GMm}{Q} + \frac{m v^2_\text{а}}{2},\\
	mv_\text{п}q = mv_\text{a}Q.
\end{gather*}
Из ЗСМИ и выражений для перицентрического~$q$ и апоцентрического~$Q$ расстояний через большую полуось $a$ и эксцентриситет $e$ имеем:
\begin{equation*}
	\frac{v_\text{а}}{v_\text{п}} = \frac{1 - e}{1 + e}.
\end{equation*}
Использую это соотношения, преобразуем ЗСЭ:
\begin{gather}
	\frac{v_\text{п}^2}{2} \left( 1 - \frac{(1 -e)^2}{(1 + e)^2} \right) = GM \left( \frac{1}{a(1-e)} - \frac{1}{a(1+e)} \right),\\
	\frac{v_\text{п}^2}{2} \cdot \frac{ 1 + 2e + e^2 - 1 + 2e - e^2}{(1+e)^2} = \frac{GM}{a} \cdot \frac{1 + e - 1 +  e}{(1+e)(1-e)},\\
	v_\text{п} = \sqrt{\frac{GM}{a}}\sqrt{\frac{1+e}{1-e}}, \quad \quad v_\text{a} = \sqrt{\frac{GM}{a}}\sqrt{\frac{1-e}{1+e}}.
\end{gather}
Запишем теперь ЗСЭ для перицентра и произвольной точки орбиты на расстоянии $r$:
\begin{gather*}
	-\frac{GMm}{q} + \frac{m v^2_\text{п}}{2} = -\frac{GMm}{r} + \frac{m v^2}{2},\\
	-\frac{GMm}{q} + \frac{GMm}{2a} \cdot \frac{1+e}{1-e} = -\frac{GMm}{r} + \frac{m v^2}{2},\\
	v^2 = GM \left( \frac{2}{r} - \frac{2}{a(1 - e)} + \frac{1+e}{a (1-e) }\right) = GM \left( \frac{2}{r} - \frac{1}{a} \right),
\end{gather*}
\begin{equation}
	v = \sqrt{ GM \left( \frac{2}{r} - \frac{1}{a} \right)}.
	\label{eq:int-energy}
\end{equation}
Полученное выражение и называется интегралом энергии. Согласно \eqref{eq:int-energy} и \eqref{eq:ellipse-pol-eq} для скорости тела в произвольной точке орбиты также справедливо выражение
\begin{equation}
	v = \sqrt{\frac{GM}{p}\cdot(1 + 2 e \cos \nu + e^2)},
\end{equation}
где $\nu$~--- истинная аномалия, а $p$~--- фокальный параметр.

Найдем величину момента импульса пробной массы $m$ на эллиптической орбите. В силу постоянства данной величины, можно выбрать любую точку орбиты для её поиска. Проще всего рассмотреть перицентр или апоцентр, рассмотрим первый.
\begin{multline*}
	L
	= m v_q q
	= m \sqrt{\frac{GM}{a} \frac{1+e}{1-e}} \cdot a(1-e) =\\
	= m \sqrt{GMa (1 + e)(1-e)}
	= m \sqrt{GMa(1-e^2)}
	= m \sqrt{GMp}.
\end{multline*}

Для параболической также рассмотрим точку перицентра:
\begin{multline*}
	L
	= m v_q q
	= m v_2(q) q
	= m \sqrt{\frac{2GM}{q}} \cdot q =\\
	= m \sqrt{2GMq}
	= m \sqrt{2GM \cdot \frac{p}{2}}
	= m \sqrt{GMp}.
\end{multline*}


\input{sections/cel-mech.orbit-elem.tex}
\subsection{Точки Лагранжа}

\term{Точки Лагранжа}~--- точки, во вращающейся системе из двух массивных тел,
\begin{wrapfigure}[14]{l}{0.48\tw}
	\centering
	\vspace{-.5pc}
	\includegraphics[width = .48\tw]{lagr-points}
	\captionof{figure}{Точки Лагранжа}
	\label{pic:larg-points}	
\end{wrapfigure}
в которых третье тело с пренебрежимо 
малой массой, не испытывающее воздействие никаких 
других сил, кроме гравитационных, со стороны двух 
первых тел, может оставаться неподвижным относительно 
этих тел. В этих точках гравитационные силы, 
действующие на малое тело, уравновешиваются силами инерции.

Точки $L_1$, $L_2$ и $L_3$ лежат на одной прямой, 
соединяющей два массивных тела. Точки $L_4$ и $L_5$ 
образуют равносторнние треугольники с массивными 
телами.

Для расстояний до точек $L_1$, $L_2$ и $L_3$ от 
центра масс системы справедливы следующие выражения:
\begin{equation}r_1=R\left(1-\sqrt[3]{\frac{\alpha}
{3}}\right), \quad r_2=R\left(1+\sqrt[3]{\frac{\alpha}
{3}}\right), \quad r_3=R\left(1+\frac{5}{12}\alpha\right),
\end{equation}
где $\alpha=M_2 / (M_1 + M_2)$, $R$~--- расстояние между 
телами, $M_1$ --- масса более массивного тела, $M_2$
 --- масса второго тела.

Если $M_2 \ll M_1$, то точки $L_1$ и $L_2$ находятся 
примерно на одинаковом расстоянии от тела $M_2$, равном
\begin{equation}
r\approx R\sqrt[3]{\frac{M_2}{3M_1}}.
\end{equation}

Расстояния от центра масс системы до точек $L_4$ и $L_5$ в координатной системе с центром координат в центре масс системы рассчитываются по  формулам
\begin{equation}
	 r_4 = \left ( \frac{R}{2} \cdot \frac{M_1-M_2}{M_1+M_2} ,   \frac{\sqrt{3}R}{2} \right ), \quad r_5 = \left ( \frac{R}{2} \cdot \frac{M_1-M_2}{M_1+M_2} ,   -\frac{\sqrt{3}R}{2} \right ). 
\end{equation}
\subsection{Приливы и отливы}

\term{Приливы и отливы}~--- периодические вертикальные колебания уровня океана, являющиеся результатом изменения положения Луны и Солнца. Хотя силы тяготения Солнца почти в 200 раз больше, чем силы тяготения Луны, приливные силы, порождаемые Луной, почти вдвое больше порождаемых Солнцем. Это происходит из-за того, что приливные силы зависят не от величины гравитационного поля, а от степени его неоднородности. Высота приливов зависит от взаимного расположения Луны и Солнца: наибольший~---  силы от Луны и от Солнца действуют вдоль одного направления, а наименьший~--- под прямым углом друг к другу.

\begin{minipage}{.24\tw}
Ускорение в центре Земли ($T$) определяется формулой \eqref{eq:g}:
\begin{equation*}
	a_T=\frac{G M}{r^2},
\end{equation*}
$M$~--- масса возмущающего тела,
\end{minipage}
\hfill
\begin{minipage}{0.74\tw}
	\vspace{-.5pc}
	\includegraphics[width = \tw]{Ebb_flow}
	\captionof{figure}{К объяснению приливных сил}\label{Ebb_flow}
\end{minipage}\\[-0.5pc]

$r$~--- расстояние между центрами Земли и данного тела. Аналогично, ускорения в точках $A$ и $B$ равны соответственно
\begin{equation}
	a_A = \frac{G M}{(r - R)^2} \quad \text{и} \quad a_B = \frac{GM}{(r + R)^2},
\end{equation}
где $R$~--- радиус Земли или иного тела, подверженного воздействию приливных сил. Ускорение в точке $A$ относительно точки $T$ равно
\begin{equation}
	a_A - a_T = a_T \cdot \frac{2 r R - R^2}{(r - R)^2} = \frac{GM \left(2 r R - R^2 \right)}{r^2 (r - R)^2} \xrightarrow{R \ll r} \frac{2 G M R}{r^3}.
	\label{eq:ebb-force}
\end{equation}

Под действием лунного притяжения водная оболочка Земли принимает форму 
эллипсоида, который вытянут по направлению к Луне. Близ точек $A$ и $B$ будет 
прилив, а в точках $F$ и $D$ --- отлив (см.~Рис.\,\ref{Ebb_flow}).
\nopagebreak
\subsection{Затмения}
Диаметр тени спутника при полном центральном затмении (когда центры трёх тел лежат на одной прямой) с большой точностью равен 
\begin{equation}
d_\text{тени} = 2 \cdot \frac{R_{\moon}(a_\oplus - R_\oplus) - R_\odot \left( a_{\moonч} - R_\oplus \right)}{a_\oplus - a_{\moon}}.
\end{equation}
Среднее значение  этой величины около 200~км, максимальное~--- около 215~км. При нецентральном затмении максимальный диаметр тени Луны на поверхности Земли может достигать 270~км. Это даёт оценку на продолжительность, равную 7.5 минутам. Большинство полных затмений длятся 2\,--\,4~минуты.

\begin{figure}[h!]
\centering
\vspace{-.5pc}
\includegraphics[width = 10cm]{full_eclipse}
\caption{Полное солнечное затмение со стороны северного полюса эклиптики}
\label{fig:eclipses-full-solar-eslipse}
\end{figure}
При \term{кольцеобразном солнечном затмении} Луна так расположена относительно Земли, что конус её тени не достаёт до поверхности планеты, и вокруг Луны можно наблюдать яркое кольцо незакрытой части солнечного диска.

\begin{figure}[h!]
	\centering
	\includegraphics[width = 10cm]{partly-eclipse}
	\caption{Кольцеобразное солнечное затмение со стороны северного полюса эклиптики}
	\label{fig:eclipses-circle-solar-eslipse}
\end{figure}
При особом расположении Луны и Земли возможны \term{гибридные} затмения, когда в разных пунктах Земли наблюдаются либо \imp{кольцеобразное}, либо \imp{полное} затмение. Причиной такого явления является шарообразность Земли.

\vspace{-1pc}
\begin{figure}[h!]
	\centering
	\includegraphics[width=10cm]{moon-eclipse}
	\caption{Схема лунного затмения со стороны северного полюса эклиптики}
	\label{fig:moon-eclipse-scheme}
\end{figure}
\term{Лунное затмение}, в отличие от солнечного, видно со всего ночного полушария Земли. Диаметр земной тени на расстоянии Луны превышает размер последней примерно в 2.5\,--\,3 раза. Бывают \term{частные}, когда лишь часть Луны попадает в земную тень, \term{полные}~--- Луна полностью погружается в тень Земли, и \term{полутеневые}~--- Луна проходит через полутень Земли, не затрагивая конус тени.

\term{Синодический месяц}~--- промежуток времени между одинаковыми фазами Луны, равен 29.53 суток.

\term{Драконический месяц}~--- промежуток времени между двумя последовательными прохождениями Луны через один и тот же узел орбиты,~--- 27.21 суток.

\term{Сарос}~--- промежуток  времени, по прошествии которого солнечные и лунные затмения повторяются в прежнем порядке. Происходит это из-за того, что каждый сарос Луна, орбита Луны и Солнце возвращаются в прежнее положение относительно далёких звёзд. Сарос длится ровно 242 драконических месяца, или 223 синодических месяца, или 18 лет 11 дней 8 часов.

\begin{wrapfigure}[8]{r}{.42\tw}
	\centering
	\vspace{-1pc}
	\includegraphics[width = 0.2\textwidth]{phases}
	\hfill
	\includegraphics[width = 0.2\textwidth]{phases-2}
	\caption{Частное и полное затмение}
	\label{fig:part-eclipses-scheme}
\end{wrapfigure}
Важной характеристикой любого затмения является его \term{фаза}~--- для \imp{частных} и \imp{кольцеобразных} затмений: отношение закрытой части $x$ диаметра\footnote{Здесь имеется в виду \imp{угловой} диаметр} затмеваемого тела, проходящего через центр затмевающего тела, ко всему диаметру затмеваемого тела $D$; для \imp{полного}: единица плюс отношение расстояния\footnote{Расстояние между окружностями $l_1$ и $l_2$~--- это $\min |L_1L_2|$ по всем $L_1 \in l_1$ и $L_2 \in l_2$.} между краями дисков затмеваемого и затмевающего тел к диаметру затмеваемого тела $D$.
\begin{equation}
\Phi_{\text{част}} = \frac{x}{D} < 1, \quad \quad \quad \Phi_{\text{полн}} =  1 + \frac{\min\{d_1, d_2\}}{D} > 1.
\end{equation}
Иногда вводят такое понятие, как \term{площадная фаза затмения}, т.\,е. отношение площади закрытой части диска затмеваемого тела к полной площади его диска. Чаще всего  площадную фазу используют применительно к двойным звёздам, когда считают падение блеска при затмении одной звезды другой.

\input{sections/cel-mech.planet-config.tex}
\input{sections/cel-mech.phase-angle.tex}
\input{sections/cel-mech.synod-period.tex}
\subsection{Собственное движение звёзд}
\term{Собственным движением} $(\mu)$ называется изменение координат звёзд на небесной сфере, вызванное относительным движением звёзд и Солнца, обычно измеряется в mas/год.
\begin{equation}
	\mu = \frac{V_\tau}{D},
\end{equation}
где $V_\tau$~--- тангенциальная относительная скорость звезды, $D$~--- расстояние до неё.

\change{Разделяют также собственное движение по склонению~--- $\mu_\delta$ и собственное движение по прямому восхождению~--- $\mu_\alpha$, которые определяются следующими выражениями:}
\begin{equation}
  \mu_\delta = \frac{\delta(t_2) - \delta(t_1)}{t_2 - t_1}, \quad \quad \mu_\alpha = \frac{\alpha(t_2) - \alpha(t_1)}{t_2 - t_1}.
\end{equation}
\change{
\begin{wrapfigure}{r}{.4\tw}
\begin{flushright}
	\vspace{-1pc}
	\begin{tikzpicture}
	\footnotesize
	\draw [dashes] (0, 4) arc(90:0:3 and 4);
	\draw [dashes] (0, 4) arc(90:0:2 and 4); 
	%
	\draw [dashes] (3.47, 2) arc(0:-70:3.47 and 1.16);	
	\draw [dashes] (2.64, 3) arc(0:-70:2.64 and 0.88);
	%
	\draw [thick, -latex] (2.3, 2.55) arc(-34:-56:2.64 and 0.88);
	\draw [thick, -latex] (2.3, 2.55) arc(53:29:2 and 4);
	\draw [thick, -latex] (2.3, 2.55) .. controls (2.3, 1.9) and (2.1, 1.4) .. (1.93, 1.03);
	%
	\draw (.9, 2.2) node [anchor=south] {$\delta(t_1)$};
	\draw (1.2, .9) node [anchor=south] {$\delta(t_2)$};
	%
	\draw (2, 0) node [anchor=north] {$\alpha(t_2)$};
	\draw (3, 0) node [anchor=north] {$\alpha(t_1)$};
	%
	\draw [fill=white] (2.3, 2.55) circle (0.03);
	\draw [fill=white] (1.93, 1.03) circle (0.03);
	\draw [fill=white] (0, 4) circle (0.03);
	%
	\draw (0, 4) node [anchor=north] {$P$};
	%
	\draw (1.9, 2.4) node [anchor=south] {$\mu_\alpha$};
	\draw (2.6, 2.05) node [anchor=west] {$\mu_\delta$};
	\draw (2.06, 1.65) node [anchor=south] {$\mu$};
	%
\end{tikzpicture}
\end{flushright}
\end{wrapfigure}
 Как отсюда видно, $\mu_\alpha$ является угловой скоростью по малому кругу, а значит, зависит от $\delta$. Следовательно, полное собственное движение $\mu$ можно найти, как
\begin{equation}
	\mu = \sqrt{\mu_\delta^2 + \mu_\alpha^2 \cos^2 \delta},
\end{equation}
потому что радиус малого круга, состоящего из точек со склонением~$\delta$, равен $R \cos \delta$, где $R$~--- радиус сферы, содержащей этот круг.
}

\begin{figure}[h!]
\begin{subfigure}[b]{0.47\tw}
	\begin{tikzpicture}[scale=1.05]
	\footnotesize
	
%	\foreach \x in {0, .1,...,4} {
%		\draw [line width=.1pt] (\x - 1, 0) -- (\x - 1, 4);
%	};
%	
%	\foreach \x in {0, 1,...,4} {
%		\draw [line width=.4pt] (\x - 1, 0) -- (\x - 1, 4);
%	};
%	
%	\foreach \y in {0, .1,...,4} {
%		\draw [line width=.1pt] (-1, \y) -- (4, \y);
%	};
%	
%	\foreach \y in {0, 1,...,4} {
%		\draw [line width=.4pt] (-1, \y) -- (4, \y);
%	};
	
	\draw [double] (.21, .21) arc (45:104:.3);
	\draw (-.93, 3.71) arc (-76:-35:.3);
	
	\draw (0, 0) -- (-1, 4);
	\draw (0, 0) -- (2, 2);
	\draw (-1, 4) -- (2.6, 1.6);
	
	\draw [thick, -latex] (-1, 4) -- (0, 4.25);
	\draw [thick, -latex] (-1, 4) -- (-.6, 2.4);
	
	\draw [fill=white] (-1, 4) circle (.03);
	\draw [fill=white] (0, 0) circle (.03);
	\draw [fill=white] (2, 2) circle (.03);
	
	\draw (1, 1) node [anchor=north west] {$R$};
	\draw (-.45, 2.1) node [anchor=north east] {$R_0$};
	\draw (.5, 2.95) node [anchor=south west] {$V \Delta t$};
	\draw (0, 0) node [anchor=north] {Солнце};
	\draw (-1, 4) node [anchor=south east] {Звезда};
	
	\draw (.1, .3) node [anchor=south] {$\xi$};
	\draw (-.9, 3.75) node [anchor=north west] {$\gamma$};
	
	\draw (-.5, 4.15) node [anchor=south] {$V_\tau$};
	\draw (-.75, 3.1) node [anchor=east] {$V_r$};
\end{tikzpicture}
\caption{}
\label{pic:phase-angle-1}
\end{subfigure}
\hfill
\begin{subfigure}[b]{0.47\tw}
\begin{tikzpicture}[scale=0.9]
	\footnotesize
	
	\draw (.2, 4.86) arc (-45:-135:0.28);
	\draw [double] (-1.65, 1.51) arc (5:80:0.25);
	
	\draw (0, 5) .. controls (-1.5, 4) and (-2, 2) .. (-2, 0);
	\draw (0, 5) .. controls (1.5, 4) and (2, 2) .. (2, 0);
	\draw (-2, 0) .. controls (-1, -.5) and (1, -.5) .. (2, 0);
	\draw (-1.9, 1.5) .. controls (-1, 1.5) and (1, 2) .. (1.5, 3);
	
	\draw [fill=white] (0, 5) circle (.03);
	\draw [fill=white] (-2, 0) circle (.03);
	\draw [fill=white] (2, 0) circle (.03);
	\draw [fill=white] (-1.9, 1.5) circle (.03);
	\draw [fill=white] (1.5, 3) circle (.03);
	
	\draw (-2, .2) -- (-1.8, .11) -- (-1.8, -.09);
	\draw (2, .2) -- (1.8, .11) -- (1.8, -.09);
	
	\draw (0, 5) node [anchor=south] {$P$};
	\draw (0, 1.9) node [anchor=north] {$\xi$};
	\draw (0, -.4) node [anchor=south] {$\Delta \alpha$};
	\draw (0, 4.8) node [anchor=north] {$\Delta \alpha$};
	\draw (-1, 4) node [anchor=east] {$90^\circ - \delta$};
	\draw (.9, 4.2) node [anchor=west] {$90^\circ - (\delta + \Delta \delta)$};
	\draw (0, -.4) node [anchor=north] {Небесный экватор};
\end{tikzpicture}
\caption{}
%\label{pic:phase-angle-2}
\end{subfigure}
\caption{}
\end{figure}


\change{Получим выражение для координат звезды, имеющей собственное движение $\mu = (\mu_\alpha, \mu_\delta)$, лучевую скорость $V_r$ и параллакс в начальный момент времени $\pi_0$. Найдем сначала тангенциальную скорость:
\begin{equation*}
	V_\tau = R_0 \sqrt{ \mu_\delta^2 + \mu_\alpha^2 \cos^2 \delta} = \frac{\sqrt{ \mu_\delta^2 + \mu_\alpha^2 \cos^2 \delta}}{\pi_0}.
\end{equation*}
Определим теперь угол между лучем зрения и полной скоростью звезды:
\begin{equation*}
	\gamma = \arctan \frac{V_\tau}{V_r}.
\end{equation*}
При этом полная скорость равна
\begin{equation*}
	V_0 = \sqrt{V_\tau^2 + V_r^2}.
\end{equation*}
Из теоремы косинусов можно найти расстояние для звезды через промежуток времени $\Delta t$:
\begin{equation*}
	R = \sqrt{R_0^2 + (V_0 \Delta t)^2 - 2 R_0 V_0 \Delta t \cos \gamma}.
\end{equation*}
Тогда угловое перемещение звезды равно
\begin{equation*}
	\sin \xi = \frac{V_0 \Delta t \sin \alpha}{R}.
\end{equation*}
Через компоненты собственного движения нетрудно получить угол между направлением на полюс и вектором полного собственного движения в начальный момент:
\begin{equation*}
	\tg \psi =  \frac{\mu_a \cos \delta}{\mu_\delta}.
\end{equation*}
Теперь с помощью сферической теоремы косинусов можно определить склонение звезды через время $\Delta t$:
\begin{equation*}
	\sin (\delta - \Delta \delta) = \cos \xi \sin \delta + \sin \xi \cos \delta \cos \psi.
\end{equation*}
Далее из сферической теоремы синусов получаем выражение для изменения прямого восхождения за время $\Delta t$~---
\begin{equation*}
	\sin \Delta \alpha = \frac{\sin \psi \sin \xi}{\cos (\delta - \Delta \delta)}.
\end{equation*}
}






\input{sections/cel-mech.precession.tex}
\section{Конические сечения}
\subsection{Эллипс}
\begin{minipage}{0.5\tw}
{\bfseries \term{Эллипс}} --- плоская замкнутая кривая, сумма расстояний от любой точки которой до двух фиксированных точек, называемых фокусами, постоянна и равна удвоенной большой полуоси эллипса.
\begin{equation}
|F_1 M|+|F_2M|=\const=2a
\end{equation}	
Главные отрезки эллипса: \term{большая полуось} ($a$), \term{малая полуось} ($b$), \term{фокальное расстояние} ($c$). Очевидно,
\end{minipage}
\begin{minipage}{0.5\tw}
	\begin{flushright}
		\vspace{-.5pc}
		\includegraphics[width = .97\tw]{Ellips}
		\captionof{figure}{Эллипс}
	\end{flushright}
\end{minipage}\\
\begin{equation}
	b^2 + c^2 = a^2.
\end{equation}
\term{Эксцентриситет} ($e$)~--- числовая 
характеристика, показывающая степень отклонения от 
окружности. Для эллипса $e$ лежит в интервале $(0, \, 1)$ и
определяется следующей формулой:\begin{equation}
p=\frac{b^2}{a}=a(1-e^2)=b\sqrt{1-e^2}
\end{equation}

\term{Апоцентр}~--- наиболее удаленная точка
от заданного фокуса точка эллипса. Из определения эллипса
вытекает соотношение для расстояния от фокуса до 
апоцентра ($Q$):\begin{equation}
Q = a (1 + e).
\end{equation}

\term{Перицентр}~--- ближайшая точка
точка эллипса к заданному фокусу. Из определения эллипса
вытекает соотношение для расстояния от фокуса до 
перицетра ($q$):\begin{equation}
q = a (1 - e).
\end{equation}

\term{Фокальный параметр}~($p$)~--- длина перпендикуляра,
проведенного из фокуса до точки пересечения с эллипсом.
Из теоремы Пифагора и определения эллипса следует 
нижеприведенная формула для расчета его длины:
\begin{equation}
p = a(1 - e^2).
\end{equation}

\term{Площадь эллипса} ($S$) --- площадь части 
плоскости, ограниченной эллипсом, равна
\begin{equation}
S=\pi ab = \pi a^2 \sqrt{1-e^2}.
\end{equation}

%Радиус кривизны дуги эллипса в зависимости от расстояния 
%$x$ от фокуса:
%\begin{equation}
%R=\frac{(2ax-x^2)^{3/2}}{ab}
%\end{equation}
%\begin{center}
%\includegraphics[width = 0.3\textwidth]{rad-curv}
%\begin{figure}[!h]
%\caption{К вычислению радиуса кривизны эллипса}
%\end{figure}
%\end{center}

{\itshape Уравнение эллипса} в декартовых координатах 
представляет собой уравнение замкнутой кривой второго 
порядка, канонический вид которого:
\begin{equation}
\frac{x^2}{a^2}+\frac{y^2}{b^2}=1.
\end{equation}

Его можно представить параметрическом виде:\begin{equation}
\left\{\begin{aligned}[lcl]
&x=a\cos t,\\
&y=b\sin t;\\
\end{aligned}
\right. \quad\quad t \in [0, \, 2\pi).
\end{equation}
В полярных координатах уравнение принимает следующий вид:
\begin{equation}
r=\frac{p}{1\pm e \cos \varphi},
\label{eq:ellipse-pol-eq}
\end{equation} 
\begin{wrapfigure}[11]{l}{0.5\tw}
	\centering
	\vspace{-.7pc}
	\includegraphics[width = 0.5\tw]{Parabola}
	\captionof{figure}{Парабола \label{pic:the-pic}}
\end{wrapfigure}
где $\varphi$ --- \term{истинная аномалия} --- угол 
{\slshape перицентр -- фокус -- заданная точка}, 
отсчитываемый в сторону движения по эллипсу. При 
знаке плюс перед $e$ второй фокус эллипса будет 
находится в точке $(0, \, 2c)$, а при минус --- в 
точке $(\pi, \, 2c)$.

Кроме этого, эллипс обладает важным {\itshape оптическим 
свойством}, которое можно сформулировать так: свет от источника в одном из фокусов, 
	отражается эллипсом так, что отражённые лучи пересекаются 
	во втором фокусе или, что тоже самое, касательная к эллипсу в заданной точке образует с фокальными радиусами в данной точке равные острые углы.






 

\subsection{Парабола}
{\bfseries \term{Парабола}} --- геометрическое место точек, равноудалённых от данной прямой (называемой \term{директрисой} параболы) и данной точки (называемой \term{фокусом} параболы).

{\itshape Каноническое уравнение параболы}:
\begin{equation}
y^2=2px,
\end{equation}
где $p$~--- \term{фокальный параметр}, равный расстоянию между фокусом параболы и директрисой или удвоенному расстоянию между фокусом параболы и вершиной.

Парабола в полярной системе координат $(r,\varphi)$ с центром в фокусе и нулевым направлением вдоль оси параболы (от фокуса к вершине) может быть представлена в виде уравнения
\begin{equation}
r = \frac{p}{1 + \cos\varphi}.
\end{equation}
Эксцентриситет параболы равен $e=1$. Важно отметить, что парабола не имеет \term{большой} и \term{малой полуоси}.

Как и все конические сечения, парабола обладает \textit{оптическим свойством}, которое формулируется следующим образом: пучок лучей, параллельных оси параболы, отражаясь в параболе, собирается в её фокусе. И наоборот, свет от источника, находящегося в фокусе, отражается параболой в пучок параллельных её оси лучей.
\section{Конические сечения}
\subsection{Эллипс}
\begin{minipage}{0.5\tw}
{\bfseries \term{Эллипс}} --- плоская замкнутая кривая, сумма расстояний от любой точки которой до двух фиксированных точек, называемых фокусами, постоянна и равна удвоенной большой полуоси эллипса.
\begin{equation}
|F_1 M|+|F_2M|=\const=2a
\end{equation}	
Главные отрезки эллипса: \term{большая полуось} ($a$), \term{малая полуось} ($b$), \term{фокальное расстояние} ($c$). Очевидно,
\end{minipage}
\begin{minipage}{0.5\tw}
	\begin{flushright}
		\vspace{-.5pc}
		\includegraphics[width = .97\tw]{Ellips}
		\captionof{figure}{Эллипс}
	\end{flushright}
\end{minipage}\\
\begin{equation}
	b^2 + c^2 = a^2.
\end{equation}
\term{Эксцентриситет} ($e$)~--- числовая 
характеристика, показывающая степень отклонения от 
окружности. Для эллипса $e$ лежит в интервале $(0, \, 1)$ и
определяется следующей формулой:\begin{equation}
p=\frac{b^2}{a}=a(1-e^2)=b\sqrt{1-e^2}
\end{equation}

\term{Апоцентр}~--- наиболее удаленная точка
от заданного фокуса точка эллипса. Из определения эллипса
вытекает соотношение для расстояния от фокуса до 
апоцентра ($Q$):\begin{equation}
Q = a (1 + e).
\end{equation}

\term{Перицентр}~--- ближайшая точка
точка эллипса к заданному фокусу. Из определения эллипса
вытекает соотношение для расстояния от фокуса до 
перицетра ($q$):\begin{equation}
q = a (1 - e).
\end{equation}

\term{Фокальный параметр}~($p$)~--- длина перпендикуляра,
проведенного из фокуса до точки пересечения с эллипсом.
Из теоремы Пифагора и определения эллипса следует 
нижеприведенная формула для расчета его длины:
\begin{equation}
p = a(1 - e^2).
\end{equation}

\term{Площадь эллипса} ($S$) --- площадь части 
плоскости, ограниченной эллипсом, равна
\begin{equation}
S=\pi ab = \pi a^2 \sqrt{1-e^2}.
\end{equation}

%Радиус кривизны дуги эллипса в зависимости от расстояния 
%$x$ от фокуса:
%\begin{equation}
%R=\frac{(2ax-x^2)^{3/2}}{ab}
%\end{equation}
%\begin{center}
%\includegraphics[width = 0.3\textwidth]{rad-curv}
%\begin{figure}[!h]
%\caption{К вычислению радиуса кривизны эллипса}
%\end{figure}
%\end{center}

{\itshape Уравнение эллипса} в декартовых координатах 
представляет собой уравнение замкнутой кривой второго 
порядка, канонический вид которого:
\begin{equation}
\frac{x^2}{a^2}+\frac{y^2}{b^2}=1.
\end{equation}

Его можно представить параметрическом виде:\begin{equation}
\left\{\begin{aligned}[lcl]
&x=a\cos t,\\
&y=b\sin t;\\
\end{aligned}
\right. \quad\quad t \in [0, \, 2\pi).
\end{equation}
В полярных координатах уравнение принимает следующий вид:
\begin{equation}
r=\frac{p}{1\pm e \cos \varphi},
\label{eq:ellipse-pol-eq}
\end{equation} 
\begin{wrapfigure}[11]{l}{0.5\tw}
	\centering
	\vspace{-.7pc}
	\includegraphics[width = 0.5\tw]{Parabola}
	\captionof{figure}{Парабола \label{pic:the-pic}}
\end{wrapfigure}
где $\varphi$ --- \term{истинная аномалия} --- угол 
{\slshape перицентр -- фокус -- заданная точка}, 
отсчитываемый в сторону движения по эллипсу. При 
знаке плюс перед $e$ второй фокус эллипса будет 
находится в точке $(0, \, 2c)$, а при минус --- в 
точке $(\pi, \, 2c)$.

Кроме этого, эллипс обладает важным {\itshape оптическим 
свойством}, которое можно сформулировать так: свет от источника в одном из фокусов, 
	отражается эллипсом так, что отражённые лучи пересекаются 
	во втором фокусе или, что тоже самое, касательная к эллипсу в заданной точке образует с фокальными радиусами в данной точке равные острые углы.






 

\subsection{Парабола}
{\bfseries \term{Парабола}} --- геометрическое место точек, равноудалённых от данной прямой (называемой \term{директрисой} параболы) и данной точки (называемой \term{фокусом} параболы).

{\itshape Каноническое уравнение параболы}:
\begin{equation}
y^2=2px,
\end{equation}
где $p$~--- \term{фокальный параметр}, равный расстоянию между фокусом параболы и директрисой или удвоенному расстоянию между фокусом параболы и вершиной.

Парабола в полярной системе координат $(r,\varphi)$ с центром в фокусе и нулевым направлением вдоль оси параболы (от фокуса к вершине) может быть представлена в виде уравнения
\begin{equation}
r = \frac{p}{1 + \cos\varphi}.
\end{equation}
Эксцентриситет параболы равен $e=1$. Важно отметить, что парабола не имеет \term{большой} и \term{малой полуоси}.

Как и все конические сечения, парабола обладает \textit{оптическим свойством}, которое формулируется следующим образом: пучок лучей, параллельных оси параболы, отражаясь в параболе, собирается в её фокусе. И наоборот, свет от источника, находящегося в фокусе, отражается параболой в пучок параллельных её оси лучей.
\section{Конические сечения}
\input{sections/conic.sec/conic.sec.ellips}
\input{sections/conic.sec/conic.sec.parabola}
\input{sections/conic.sec/conic.sec.hyperbola}
\newpage
\section{Астрофизика}
\subsection{Звёздные величины}
Звёздная величина~--- безразмерная числовая характеристика яркости объекта. Известно, что увеличению светового потока в 100 раз соответствует уменьшение видимой звёздной величины ровно на 5 единиц. Тогда уменьшение звёздной величины на одну единицу означает увеличение светового потока в $\sqrt[5]{100}\approx 2.512$~раз, то есть звёздные величины являются логарифмической шкалой измерения плотности потока. Зависимость, связывающая отношение освещённостей $E_1$ и $E_2$ и разность звёздных величин $m_1$ и $m_2$ двух объектов, называется \term{формулой Погсона} и имеет вид
\begin{equation}
	\frac{E_1}{E_2} = 10^{0.4(m_2 - m_1)} \quad \Longleftrightarrow \quad m_2 = m_1 + 2.5 \lg \frac{E_1}{E_2}.
	\label{eq:Pogson-law}
\end{equation}
Широко используется понятие \term{абсолютной звёздной величины} $M$~--- это видимая звёздная величина $m$ при наблюдении с установленного расстояния: для звёзд~---~10~пк, для тел Солнечной системы~---~1~\au, причем считается, что тело находится в 1~\au~и от наблюдателя и от Солнца, а фаза равна единице, то есть можно считать, что наблюдатель находится в центре Солнца, а~тело~--- в~1~\au~от него. 

Кроме этого, важно понятие \term{болометрической звёздной величины} $m_\text{bol}$~--- это звёздная величина, при расчёте которой учитывается полная мощность излучения источника во всех диапазонах электромагнитных волн. Обычная (видимая) звёздная величина учитывает излучение лишь в видимой части спектра от примерно 380~нм до примерно~780~нм. Разность между болометрической и видимой звёздными величинами называется \term{болометрической поправкой} ($BC$), которая отличается для разных спектральных классов звёзд. Из определения, болометрическая поправка может быть найдена по формуле
\begin{equation}
	BC = m_\text{bol} - m.
\end{equation}
Абсолютную звёздную величину звезды можно получить по формуле Погсона \eqref{eq:Pogson-law} из видимой звёздной величины $m$ и расстояния $r$ до неё в парсеках
\begin{equation}
	M = m + 2.5 \lg \frac{E}{E_\text{абс}} = m + 2.5 \lg \frac{(10~\text{пк})^2}{r^2} = m + 5 - 5\lg r.
	\label{eq:abs-mag}
\end{equation} 
Если принимать к рассмотрению межзвездное поглощение $A$, то формулу  \eqref{eq:abs-mag} необходимо уточнить:
\begin{equation}
	M = m + 5 - 5\lg r - Ar.
\end{equation}
\subsection{Закон Стефана-Больцмана}
\term{Закон Стефана~--- Больцмана} определяет зависимость плотности мощности излучения абсолютно чёрного тела (АЧТ) $u$ от его температуры $T$:
\begin{equation}
u = a T^4,
\end{equation} 
где $a$~--- некая универсальная константа.
Отсюда полная светимость АЧТ с площадью поверхности $S$
	\begin{equation}
	L = S \sigma T^4,
	\label{eq:steff-bol-law}
\end{equation}
константа $\sigma$ называется \term{постоянной Стефана-Больцмана}.
  
Важно отметить, что \imp{закон Стефана-Больцмана}~--- прямое следствие формулы Планка \eqref{Planck's formula}, так как
\begin{equation}
	\sigma T^4 = \int\limits^\infty_0 B(\lambda, T)\,d\lambda \int\limits_0^{\pi/2} \sin \varphi\, d\varphi \int\limits_0^{2\pi} \cos \varphi\, d\theta = \pi \int\limits^\infty_0 B(\lambda, T)\,d\lambda,
\end{equation}
откуда $\sigma = (2\pi^5k^4)/(15c^2h^3) = 5.67 \cdot 10^{-8}~\text{Вт}/(\text{м}^2\cdot \text{К}^4)$.

%Для АЧТ сферической формы с радиусом $R$ формула~\eqref{eq:steff-bol-law} принимает вид
%\begin{equation}
%L=4\pi R^2\sigma T^4.
%\end{equation}
Для звёзд главной последовательности выполняется соотношение $L \sim M^{\alpha}$, где~$\alpha$~--- коэффициент пропорциональности, который зависит от массы звезды следующим образом:
\begin{align*}
\alpha &= 2.5, \quad M < 0.43 M_\odot; & 
\alpha &= 4, \quad 0.43 M_\odot < M < 2 M_\odot;\\ 
\alpha &= 3.2, \quad 2 M_\odot < M < 20 M_\odot; & 
\alpha &= 1, \quad M > 20 M_\odot.
\end{align*}
Также существует примерная зависимость светимости звёзды от её радиуса, имеющая вид  $L\sim R^{5.2}$.
\subsection{Энергия излучения}
\term{Энергия излучения}~--- энергия, переносимая излучением ($Q_e$).\\
\term{Поток излучения}~--- физическая величина, характеризующая мощность, переносимую излучением,
\begin{equation}
 \Phi_e = \frac{d Q_e}{dt}.
\end{equation}
\imp{Теорема Гаусса}: через любую замкнутую поверхность потоки от одинаковых источников равны.

\term{Спектральная плотность потока излучения}~--- поток излучения, приходящийся на малый единичный интервал спектра,
\begin{equation}
\Phi_{e, \lambda}(\lambda) = \frac{d\Phi_e(\lambda)}{d\lambda}, \quad\quad \Phi_{e, \nu}(\nu) = \frac{d\Phi_e(\nu)}{d\nu} =  \frac{\lambda^2}{c}\Phi_{e, \lambda}(\lambda).
\end{equation}

\term{Объемная плотность энергии излучения}~--- количество энергии на единицу объема
\begin{equation}
U_e = \frac{d Q_e}{dV}.
\end{equation}

\term{Светимость}~--- величина, представляющая собой световой поток излучения, испускаемого с малого участка светящейся поверхности единичной площади,
\begin{equation}
M_e = \frac{d \Phi_e}{dS_1},
\end{equation}
здесь $S_1$~--- площадь объекта, испускающего энергию.

\term{Яркость}~--- световой поток, приходящийся на единичный телесный угол, в расчёте на единичную площадку проекции излучающей поверхности на картинную плоскость, 
\begin{equation}
L_e = \frac{d^2 \Phi_e}{d \Omega\,dS_1 \cos \varepsilon},
\end{equation}
где $\varepsilon$~--- угол между направлением потока излучения и нормалью к плоскости излучающей поверхности.

\term{Интегральная яркость}~--- интеграл яркости по видимой поверхности источника. Показывает количество энергии, пришедшее от источника за единицу времени.
\begin{equation}
\Lambda_e = \int \limits_S L_e(\vec{r})\,ds.
\end{equation}
\term{Освещенность}~--- величина, равная отношению светового потока, падающего на малый участок поверхности, к его площади~--- поверхностная плотность потока
\begin{equation}
E_e = \frac{d\Phi_e}{dS_2} \sim \frac{1}{r^2},
\end{equation}
здесь $S_2$~--- площадь поверхности приёмника, $r$~--- расстояние от источника.
\newpage
\section{Астрофизика}
\subsection{Звёздные величины}
Звёздная величина~--- безразмерная числовая характеристика яркости объекта. Известно, что увеличению светового потока в 100 раз соответствует уменьшение видимой звёздной величины ровно на 5 единиц. Тогда уменьшение звёздной величины на одну единицу означает увеличение светового потока в $\sqrt[5]{100}\approx 2.512$~раз, то есть звёздные величины являются логарифмической шкалой измерения плотности потока. Зависимость, связывающая отношение освещённостей $E_1$ и $E_2$ и разность звёздных величин $m_1$ и $m_2$ двух объектов, называется \term{формулой Погсона} и имеет вид
\begin{equation}
	\frac{E_1}{E_2} = 10^{0.4(m_2 - m_1)} \quad \Longleftrightarrow \quad m_2 = m_1 + 2.5 \lg \frac{E_1}{E_2}.
	\label{eq:Pogson-law}
\end{equation}
Широко используется понятие \term{абсолютной звёздной величины} $M$~--- это видимая звёздная величина $m$ при наблюдении с установленного расстояния: для звёзд~---~10~пк, для тел Солнечной системы~---~1~\au, причем считается, что тело находится в 1~\au~и от наблюдателя и от Солнца, а фаза равна единице, то есть можно считать, что наблюдатель находится в центре Солнца, а~тело~--- в~1~\au~от него. 

Кроме этого, важно понятие \term{болометрической звёздной величины} $m_\text{bol}$~--- это звёздная величина, при расчёте которой учитывается полная мощность излучения источника во всех диапазонах электромагнитных волн. Обычная (видимая) звёздная величина учитывает излучение лишь в видимой части спектра от примерно 380~нм до примерно~780~нм. Разность между болометрической и видимой звёздными величинами называется \term{болометрической поправкой} ($BC$), которая отличается для разных спектральных классов звёзд. Из определения, болометрическая поправка может быть найдена по формуле
\begin{equation}
	BC = m_\text{bol} - m.
\end{equation}
Абсолютную звёздную величину звезды можно получить по формуле Погсона \eqref{eq:Pogson-law} из видимой звёздной величины $m$ и расстояния $r$ до неё в парсеках
\begin{equation}
	M = m + 2.5 \lg \frac{E}{E_\text{абс}} = m + 2.5 \lg \frac{(10~\text{пк})^2}{r^2} = m + 5 - 5\lg r.
	\label{eq:abs-mag}
\end{equation} 
Если принимать к рассмотрению межзвездное поглощение $A$, то формулу  \eqref{eq:abs-mag} необходимо уточнить:
\begin{equation}
	M = m + 5 - 5\lg r - Ar.
\end{equation}
\subsection{Закон Стефана-Больцмана}
\term{Закон Стефана~--- Больцмана} определяет зависимость плотности мощности излучения абсолютно чёрного тела (АЧТ) $u$ от его температуры $T$:
\begin{equation}
u = a T^4,
\end{equation} 
где $a$~--- некая универсальная константа.
Отсюда полная светимость АЧТ с площадью поверхности $S$
	\begin{equation}
	L = S \sigma T^4,
	\label{eq:steff-bol-law}
\end{equation}
константа $\sigma$ называется \term{постоянной Стефана-Больцмана}.
  
Важно отметить, что \imp{закон Стефана-Больцмана}~--- прямое следствие формулы Планка \eqref{Planck's formula}, так как
\begin{equation}
	\sigma T^4 = \int\limits^\infty_0 B(\lambda, T)\,d\lambda \int\limits_0^{\pi/2} \sin \varphi\, d\varphi \int\limits_0^{2\pi} \cos \varphi\, d\theta = \pi \int\limits^\infty_0 B(\lambda, T)\,d\lambda,
\end{equation}
откуда $\sigma = (2\pi^5k^4)/(15c^2h^3) = 5.67 \cdot 10^{-8}~\text{Вт}/(\text{м}^2\cdot \text{К}^4)$.

%Для АЧТ сферической формы с радиусом $R$ формула~\eqref{eq:steff-bol-law} принимает вид
%\begin{equation}
%L=4\pi R^2\sigma T^4.
%\end{equation}
Для звёзд главной последовательности выполняется соотношение $L \sim M^{\alpha}$, где~$\alpha$~--- коэффициент пропорциональности, который зависит от массы звезды следующим образом:
\begin{align*}
\alpha &= 2.5, \quad M < 0.43 M_\odot; & 
\alpha &= 4, \quad 0.43 M_\odot < M < 2 M_\odot;\\ 
\alpha &= 3.2, \quad 2 M_\odot < M < 20 M_\odot; & 
\alpha &= 1, \quad M > 20 M_\odot.
\end{align*}
Также существует примерная зависимость светимости звёзды от её радиуса, имеющая вид  $L\sim R^{5.2}$.
\subsection{Энергия излучения}
\term{Энергия излучения}~--- энергия, переносимая излучением ($Q_e$).\\
\term{Поток излучения}~--- физическая величина, характеризующая мощность, переносимую излучением,
\begin{equation}
 \Phi_e = \frac{d Q_e}{dt}.
\end{equation}
\imp{Теорема Гаусса}: через любую замкнутую поверхность потоки от одинаковых источников равны.

\term{Спектральная плотность потока излучения}~--- поток излучения, приходящийся на малый единичный интервал спектра,
\begin{equation}
\Phi_{e, \lambda}(\lambda) = \frac{d\Phi_e(\lambda)}{d\lambda}, \quad\quad \Phi_{e, \nu}(\nu) = \frac{d\Phi_e(\nu)}{d\nu} =  \frac{\lambda^2}{c}\Phi_{e, \lambda}(\lambda).
\end{equation}

\term{Объемная плотность энергии излучения}~--- количество энергии на единицу объема
\begin{equation}
U_e = \frac{d Q_e}{dV}.
\end{equation}

\term{Светимость}~--- величина, представляющая собой световой поток излучения, испускаемого с малого участка светящейся поверхности единичной площади,
\begin{equation}
M_e = \frac{d \Phi_e}{dS_1},
\end{equation}
здесь $S_1$~--- площадь объекта, испускающего энергию.

\term{Яркость}~--- световой поток, приходящийся на единичный телесный угол, в расчёте на единичную площадку проекции излучающей поверхности на картинную плоскость, 
\begin{equation}
L_e = \frac{d^2 \Phi_e}{d \Omega\,dS_1 \cos \varepsilon},
\end{equation}
где $\varepsilon$~--- угол между направлением потока излучения и нормалью к плоскости излучающей поверхности.

\term{Интегральная яркость}~--- интеграл яркости по видимой поверхности источника. Показывает количество энергии, пришедшее от источника за единицу времени.
\begin{equation}
\Lambda_e = \int \limits_S L_e(\vec{r})\,ds.
\end{equation}
\term{Освещенность}~--- величина, равная отношению светового потока, падающего на малый участок поверхности, к его площади~--- поверхностная плотность потока
\begin{equation}
E_e = \frac{d\Phi_e}{dS_2} \sim \frac{1}{r^2},
\end{equation}
здесь $S_2$~--- площадь поверхности приёмника, $r$~--- расстояние от источника.
\input{sections/astrophys.flux-albedo.tex}
\input{sections/astrophys.photon.tex}
\input{sections/astrophys.energy-lines.tex}
\subsection{Формула Планка}
\label{sec:planck-law}
\term{Формула Планка}~--- выражение для спектральной плотности мощности излучения абсолютно чёрного тела на интервале частот $[\nu, \nu + d \nu)$, распространяющейся с телесном угле $d\Omega$, которое было получено Максом Планком в 1900~году. Данное выражение имеет следующий вид:
\begin{equation}
B_\nu(\nu,T)=\frac{2h\nu^3}{c^2}\cdot \frac{1}{\exp\left(\frac{h\nu}{kT}\right)-1} = \left[ \frac{\text{Вт}}{\text{м}^2 \cdot \text{Гц} \cdot \text{ср}}\right],
\label{eq:plancks-law-nu}
\end{equation}
где $\nu$~--- частота излучения, $T$~--- температура АЧТ, $h$~--- постоянная Планка, $k$~--- постоянная Больцмана, $c$~--- скорость света.

Если записать закон излучения Планка \eqref{eq:plancks-law-nu} для длин волн, то
\begin{equation}
B_\lambda(\lambda,T)=\frac{2hc^2}{\lambda^5} \cdot \frac{1}{\exp\left(\frac{hc}{\lambda kT}\right)-1} = \left[ \frac{\text{Вт}}{\text{м}^3 \cdot \text{ср}}\right].
\label{eq:plancks-law-lambda}
\end{equation}
\begin{wrapfigure}[15]{l}{.6\tw}
\centering
\vspace{-.9pc}
 \begin{tikzpicture}
  \begin{axis}[
  				width 	=	.6\tw, 
				height	=	6cm, 
  				ymax	=	1e+14,
  				xmax	=	2000,
  				xmin	=	0,
  				ymin	=	0,
				xlabel	=	{Длина волны $\lambda$,~нм}, 
				ylabel 	= 	{$B_\lambda(\lambda, T)$,~$\text{Вт} \cdot \text{м}^{-3}$}
]
   \addplot+[dashed, thin, black] table[x=l, y=tl] {data/planck.txt};
   \addplot+[black] table[x=l, y=t4] {data/planck.txt} node at (axis cs:870, 1.6e+13) {\tiny{$4500$~K}};
   \addplot+[black] table[x=l, y=t5] {data/planck.txt}node at (axis cs:750, 4.2e+13) {\tiny{$5000$~K}};
   \addplot+[black] table[x=l, y=t58] {data/planck.txt}node at (axis cs:670, 8.5e+13) {\tiny{$5800$~K}};
   \addplot+[black] table[x=l, y=t7] {data/planck.txt}node at (axis cs:1350, 3.5e+13) {\tiny{$7000$~K}};
	%\addplot+[black, smooth] table[x=l, y=t15] {data/planck.txt} node at (axis cs:1670, 5.5e+13) {\tiny{$15000$~K}};
  \end{axis}
 \end{tikzpicture}
\caption{Кривые спектральной плотности мощности изотропного излучения АЧТ с разной температурой}\label{pic:wien-law}
\end{wrapfigure}
Стоит заметить, что при переходе в функции к длинам волн меняется не только частота на длину волны, но и выражение для интервала. 

Формула Планка появилась, когда стало ясно, что формула Рэлея-Джинса удовлетворительно описывает излучение только в области больших длин волн, а~с~убыванием длин волн даёт сильные расхождения с реальными данными. Однако формулу Рэлея-Джинса используют и сейчас для описания кривой Планка на больших длинах волн. 

\change{
Проделаем обратные действия: получим формулу Рэлея-Джинса из формулы Планка. Длинноволновая часть спектра характеризуется соотношением $h\nu \ll kT$, то есть 
\begin{equation*}
	\exp\left( \frac{h\nu}{kT}\right) \approx 1 + \frac{h\nu}{kT}.
\end{equation*}
Подставляя полученное выражение в знаменатель \eqref{eq:plancks-law-nu}, получим
\begin{equation*}
	B_\nu(\nu,T) \approx \frac{2h\nu^3}{c^2}\cdot \frac{1}{1 + \frac{h\nu}{kT} - 1} = \frac{2h\nu^3 }{c^2}\cdot \frac{k T}{ h \nu} = \frac{2 \nu^2 k T}{c^2}.
\end{equation*}
}
\change{
	Проделав то же самое для выражения через длину волны, получим:
}
\begin{equation}
	B(\lambda, T) \simeq \frac{2 c k T}{\lambda^4}, \quad\quad B(\nu, T) \simeq \frac{2 \nu^2 k T}{c^2}.
\label{Ray-Jean}
\end{equation}

\change{
	В коротковолновой области, наоборот, $h \nu \gg kT$, следовательно, в знаменателе формулы Планка единица много меньше стоящей там экспоненты, то есть
	\begin{equation*}
		\frac{1}{\exp\left(\frac{h\nu}{kT}\right)-1} \approx \frac{1}{\exp\left(\frac{h\nu}{kT}\right)} = \exp\left(-\frac{h\nu}{kT}\right).
	\end{equation*} 
	Отсюда получаются выражения, называемые приближением Вина:
}
\begin{equation}
B ( \lambda, T) \simeq \frac{2 h c^2}{\lambda^5} \exp \left( -\frac{h c}{\lambda k T}\right), \quad \quad B( \nu, T ) \simeq \frac{2 h \nu^3}{c^2} \exp \left( -\frac{h \nu}{k T} \right).
\end{equation}
\subsection{Закон смещения Вина}
\term{Закон смещения Вина} --- закон, устанавливающий зависимость длины волны~$\lambda_\text{макс}$, на которой спектральная плотность излучения $B_\lambda(\lambda, T)$ абсолютно чёрного тела достигает своего максимума, от температуры $T$ этого тела:
\begin{equation}
	\lambda_\text{макс} \approx \frac{b}{T} \equiv \frac{0.0029~\text{м} \cdot \text{К}}{T}.
\end{equation}
Закон является следствием исследования функции Планка (см.~\ref{sec:planck-law}) на экстремальность.
\subsection{Эффект Доплера. Красное смещение}
\term{Эффект Доплера}~--- эффект изменения частоты и длины волны электромагнитного излучения, регистрируемого приёмником, вызванный относительным движением источника и приёмника (см.~Рис.\,\ref{doppler-ef}).

При $\Delta \lambda \ll \lambda_0$ с большой точностью выполняется следующее важное соотношение:\begin{equation}
\beta \equiv \dfrac{v}{c} = \dfrac{\lambda - \lambda_0}{\lambda_0} \equiv \dfrac{\Delta \lambda}{\lambda_0},
\label{eq:dopler-ef-simple}
\end{equation}
\begin{wrapfigure}[6]{r}{0.5\tw}
\centering
\vspace{-.5pc}
\includegraphics[width=.5\tw]{doppler-ef}
\caption{Эффект Доплера}
\label{doppler-ef}
\end{wrapfigure}
где $\lambda_0$~--- лабораторная длина волны излучения источника, а $\lambda$~--- наблюдаемая. В действительности же имеет место более общий случай: \imp{релятивистский эффект Доплера}, обусловленный проявлением СТО при $v \simeq c$, для которого формула~\eqref{eq:dopler-ef-simple} усложняется и принимает вид
\begin{equation}
\nu = \nu_0 \cdot \dfrac{\sqrt{1 - \beta^2}}{1 + \beta \cdot \cos\theta},
\label{eq:dopler-ef-rel}
\end{equation}
где $\nu$~--- частота, с которой наблюдатель принимает волны, $\nu_0$~--- частота, с которой источник испускает волны, $v$~--- скорость источника, $\theta$~--- угол между направлением на источник и вектором его скорости в системе отсчёта приёмника. Если источник радиально удаляется от наблюдателя, то $\theta = 0$, если приближается, то $\theta =\pi$. Важно, что~\eqref{eq:dopler-ef-simple} напрямую следует из \eqref{eq:dopler-ef-rel} при $\beta  \ll 1$.

\term{Красное смещение}~--- явление сдвига спектральных линий химических элементов в красную (длинноволновую) сторону, обусловленное относительным движение объектов. Параметр красного смещения определяется из наблюдаемой и лабораторной длин волн как
\begin{equation}
z = \dfrac{\lambda - \lambda_0}{\lambda_0}.
\end{equation}

Доплеровское смещение длины волны в спектре источника, движущегося с лучевой скоростью $v_{r}$ и полной скоростью $v$,
\begin{equation}
z = \dfrac{1 + v_r / c}{\sqrt{1 - \beta^2}}.
\end{equation}

\term{Гравитационное красное смещение}~--- проявление эффекта изменения частоты излучения, испущенного массивным объектом, таким как звезда или чёрная дыра. Наблюдается как сдвиг спектральных линий в спектре источника в красную область спектра. Гравитационное красное смещение определяется из формулы, выведенной Эйнштейном,
\begin{equation}
z_G=\dfrac{GM}{c^2 R}-\dfrac{GM}{c^2 r},
\label{eq:grav-red-shift}
\end{equation}
где $M$~--- масса гравитирующего тела, $R$~--- радиальное расстояние от центра масс тела до точки излучения (радиус источника), $r$~---  радиальное расстояние от центра масс источника до точки наблюдения. В случае, когда наблюдатель находится от источника много дальше его радиуса, т.\,е. выполняется соотношение $r \gg R$, выражение~\eqref{eq:grav-red-shift} можно упростить до
\begin{equation}
z_G \simeq \dfrac{GM}{c^2 R}.
\end{equation}

\input{sections/astrophys.light-pressure.tex}
\input{sections/astrophys.edd.tex}
\input{sections/astrophys.grav-lens.tex}
\subsection{Закон Хаббла}
\term{Закон Хаббла}~--- эмпирический закон, связывающий скорость удаления галактик $V$ и расстояние $R$ до них линейным образом: 
\begin{equation}
	V = H R,
\end{equation}
величина $H=68~\text{км/c} \cdot \text{Мпк})$ называется \imp{постоянной Хаббла}.

При $v \ll c$ можно использовать приближение эффекта Доплера, тогда
\begin{equation}
	V = c z.
\label{eq:hubble-speed}
\end{equation}

Равенство \eqref{eq:hubble-speed} справедливо только при $z \ll 1$, а при б\'{o}льших значениях $z$ космологическое красное смещение нльзя связывать с эффектом Доплера, поэтому можно пользоваться только формулой 
\begin{equation}
	\frac{dz}{dt} = - H(z)(1+z),
\end{equation}
где постоянная Хаббла введена как функция красного смещения.
\subsection{Шкала электромагнитных волн}


\term{Гамма излучение} возникает при радиоактивных распадах ядер, при торможении электронов энергией более $10^5$~эВ и при других взаимодействиях элементарных частиц. Используются в гамма-дефектоскопии, при изучении свойств вещества.

\term{Рентгеновские лучи} излучаются при большом ускорении электронов, например при их торможении в металлах. Получают их при помощи рентгеновской трубки: электроны в вакуумной трубке ускоряются электрическим полем при высоком напряжении, достигая анода, при со­ударении резко тормозятся. При торможении электроны движут­ся с ускорением и излучают электромагнитные волны с малой длиной. 

\begin{figure}[!h]
\centering
\includegraphics[width = 1\textwidth]{scale-wave.pdf}
\caption{Шкала электромагнитных волн}
\end{figure}
\term{Ультрафиолетовые лучи}~--- излучение Солнца, ртутных ламп и т.\,п. Используются в ультрафиолетовой микроскопии, в медицине.

\term{Видимое излучение}~--- часть электромагнитного излучения, воспринимаемая глазом (от фиолетового до от красного).

\term{Инфракрасное излучение}~--- тепловое, излучается любым нагретым телом.

\term{Радиоволны} используются повсеместно в обычной жизни, это и сотовая связь, и радиолокация, и спутниковая связь, и Wi-Fi и многое другое.

\term{Низкочастотные волны}~--- диапазон, традиционно используемый в электротехнике. В промышленной электроэнергетике используется частота 50~Гц, на~которой осуществляется передача электрической энергии по линиям и преобразование напряжений трансформаторными устройствами.
\input{sections/astrophys.spec-theor-rel.tex}
\subsection{Оптическая толщина. Закон Бугера}
\term{Оптическая толщина}~--- безразмерная величина, характеризующая степень непрозрачности среды для проходящего сквозь неё излучения,
\begin{equation}
\tau = \int n(x) \sigma(x)\,dx,
\end{equation}
где $\tau$~--- оптическая толщина среды, $n$~--- концентрация частиц, $\sigma$~--- сечение их взаимодействия.

Поток $I_0$ на входе связан с потоком $I$ на выходе \term{Законом Бугера}:
\begin{equation}
I = I_0 e^{-\tau}.
\end{equation}
\input{sections/astrophys.colour.tex}
\input{sections/astrophys.mkt.tex}
\input{sections/astrophys.earth-atmosphere.tex}

\subsection{Формула Планка}
\label{sec:planck-law}
\term{Формула Планка}~--- выражение для спектральной плотности мощности излучения абсолютно чёрного тела на интервале частот $[\nu, \nu + d \nu)$, распространяющейся с телесном угле $d\Omega$, которое было получено Максом Планком в 1900~году. Данное выражение имеет следующий вид:
\begin{equation}\label{Planck's formula}
B_\nu(\nu,T)=\frac{2h\nu^3}{c^2}\cdot \frac{1}{\exp\left(\frac{h\nu}{kT}\right)-1} = \left[ \frac{\text{Вт}}{\text{м}^2 \cdot \text{Гц} \cdot \text{ср}}\right],
\end{equation}
где $\nu$~--- частота излучения, $T$~--- температура АЧТ, $h$~--- постоянная Планка, $k$~--- постоянная Больцмана, $c$~--- скорость света.

Если записать закон излучения Планка (\ref{Planck's formula}) для длин волн, то
\begin{equation}\label{Planck's formula2}
B_\lambda(\lambda,T)=\frac{2hc^2}{\lambda^5} \cdot \frac{1}{\exp\left(\frac{hc}{\lambda kT}\right)-1} = \left[ \frac{\text{Вт}}{\text{м}^3 \cdot \text{ср}}\right].
\end{equation}
\begin{wrapfigure}[15]{l}{.6\tw}
\centering
\vspace{-.9pc}
 \begin{tikzpicture}
  \begin{axis}[
  				width 	=	.6\tw, 
				height	=	6cm, 
  				ymax	=	1e+14,
  				xmax	=	2000,
  				xmin	=	0,
  				ymin	=	0,
				xlabel	=	{Длина волны $\lambda$,~нм}, 
				ylabel 	= 	{$B_\lambda(\lambda, T)$,~$\text{Вт} \cdot \text{м}^{-3}$}
]
   \addplot+[dashed, thin, black] table[x=l, y=tl] {data/planck.txt};
   \addplot+[black] table[x=l, y=t4] {data/planck.txt} node at (axis cs:870, 1.6e+13) {\tiny{$4500$~K}};
   \addplot+[black] table[x=l, y=t5] {data/planck.txt}node at (axis cs:750, 4.2e+13) {\tiny{$5000$~K}};
   \addplot+[black] table[x=l, y=t58] {data/planck.txt}node at (axis cs:670, 8.5e+13) {\tiny{$5800$~K}};
   \addplot+[black] table[x=l, y=t7] {data/planck.txt}node at (axis cs:1350, 3.5e+13) {\tiny{$7000$~K}};
	%\addplot+[black, smooth] table[x=l, y=t15] {data/planck.txt} node at (axis cs:1670, 5.5e+13) {\tiny{$15000$~K}};
  \end{axis}
 \end{tikzpicture}
\caption{Кривые спектральной плотности мощности изотропного излучения АЧТ с разной температурой}\label{pic:wien-law}
\end{wrapfigure}
Стоит заметить, что при переходе в функции к длинам волн меняется не только частота на длину волны, но и выражение для интервала. 

Формула Планка появилась, когда стало ясно, что формула Рэлея-Джинса удоволетворительно описывает излучение только в области длинных волн, а~с~убыванием длин волн даёт сильные расхождения с реальными данными. Однако формулу Рэлея-Джинса используют и сейчас для описания кривой Планка на больших длинах волн:
\begin{equation}
	B(\lambda, T) \simeq \frac{2 c k T}{\lambda^4}, \quad\quad B(\nu, T) \simeq \frac{2 \nu^2 k T}{c^2}.
\label{Ray-Jean}
\end{equation}
В коротковолновой же области формулу Планка можно приблизить следующими зависимостями:
\begin{equation}
B ( \lambda, T) \simeq \frac{2 h c^2}{\lambda^5} \exp \left( -\frac{h c}{\lambda k T}\right), \quad \quad B( \nu, T ) \simeq \frac{2 h \nu^3}{c^2} \exp \left( -\frac{h \nu}{k T} \right).
\end{equation}
\subsection{Закон смещения Вина}
\term{Закон смещения Вина} --- закон, устанавливающий зависимость длины волны~$\lambda_\text{макс}$, на которой спектральная плотность излучения $B_\lambda(\lambda, T)$ абсолютно чёрного тела достигает своего максимума, от температуры $T$ этого тела:
\begin{equation}
	\lambda_\text{макс} \approx \frac{b}{T} \equiv \frac{0.0029~\text{м} \cdot \text{К}}{T}.
\end{equation}
Закон является следствием исследования функции Планка (см.~\ref{sec:planck-law}) на экстремальность.
\newpage
\section{Астрофизика}
\subsection{Звёздные величины}
Звёздная величина~--- безразмерная числовая характеристика яркости объекта. Известно, что увеличению светового потока в 100 раз соответствует уменьшение видимой звёздной величины ровно на 5 единиц. Тогда уменьшение звёздной величины на одну единицу означает увеличение светового потока в $\sqrt[5]{100}\approx 2.512$~раз, то есть звёздные величины являются логарифмической шкалой измерения плотности потока. Зависимость, связывающая отношение освещённостей $E_1$ и $E_2$ и разность звёздных величин $m_1$ и $m_2$ двух объектов, называется \term{формулой Погсона} и имеет вид
\begin{equation}
	\frac{E_1}{E_2} = 10^{0.4(m_2 - m_1)} \quad \Longleftrightarrow \quad m_2 = m_1 + 2.5 \lg \frac{E_1}{E_2}.
	\label{eq:Pogson-law}
\end{equation}
Широко используется понятие \term{абсолютной звёздной величины} $M$~--- это видимая звёздная величина $m$ при наблюдении с установленного расстояния: для звёзд~---~10~пк, для тел Солнечной системы~---~1~\au, причем считается, что тело находится в 1~\au~и от наблюдателя и от Солнца, а фаза равна единице, то есть можно считать, что наблюдатель находится в центре Солнца, а~тело~--- в~1~\au~от него. 

Кроме этого, важно понятие \term{болометрической звёздной величины} $m_\text{bol}$~--- это звёздная величина, при расчёте которой учитывается полная мощность излучения источника во всех диапазонах электромагнитных волн. Обычная (видимая) звёздная величина учитывает излучение лишь в видимой части спектра от примерно 380~нм до примерно~780~нм. Разность между болометрической и видимой звёздными величинами называется \term{болометрической поправкой} ($BC$), которая отличается для разных спектральных классов звёзд. Из определения, болометрическая поправка может быть найдена по формуле
\begin{equation}
	BC = m_\text{bol} - m.
\end{equation}
Абсолютную звёздную величину звезды можно получить по формуле Погсона \eqref{eq:Pogson-law} из видимой звёздной величины $m$ и расстояния $r$ до неё в парсеках
\begin{equation}
	M = m + 2.5 \lg \frac{E}{E_\text{абс}} = m + 2.5 \lg \frac{(10~\text{пк})^2}{r^2} = m + 5 - 5\lg r.
	\label{eq:abs-mag}
\end{equation} 
Если принимать к рассмотрению межзвездное поглощение $A$, то формулу  \eqref{eq:abs-mag} необходимо уточнить:
\begin{equation}
	M = m + 5 - 5\lg r - Ar.
\end{equation}
\subsection{Закон Стефана-Больцмана}
\term{Закон Стефана~--- Больцмана} определяет зависимость плотности мощности излучения абсолютно чёрного тела (АЧТ) $u$ от его температуры $T$:
\begin{equation}
u = a T^4,
\end{equation} 
где $a$~--- некая универсальная константа.
Отсюда полная светимость АЧТ с площадью поверхности $S$
	\begin{equation}
	L = S \sigma T^4,
	\label{eq:steff-bol-law}
\end{equation}
константа $\sigma$ называется \term{постоянной Стефана-Больцмана}.
  
Важно отметить, что \imp{закон Стефана-Больцмана}~--- прямое следствие формулы Планка \eqref{Planck's formula}, так как
\begin{equation}
	\sigma T^4 = \int\limits^\infty_0 B(\lambda, T)\,d\lambda \int\limits_0^{\pi/2} \sin \varphi\, d\varphi \int\limits_0^{2\pi} \cos \varphi\, d\theta = \pi \int\limits^\infty_0 B(\lambda, T)\,d\lambda,
\end{equation}
откуда $\sigma = (2\pi^5k^4)/(15c^2h^3) = 5.67 \cdot 10^{-8}~\text{Вт}/(\text{м}^2\cdot \text{К}^4)$.

%Для АЧТ сферической формы с радиусом $R$ формула~\eqref{eq:steff-bol-law} принимает вид
%\begin{equation}
%L=4\pi R^2\sigma T^4.
%\end{equation}
Для звёзд главной последовательности выполняется соотношение $L \sim M^{\alpha}$, где~$\alpha$~--- коэффициент пропорциональности, который зависит от массы звезды следующим образом:
\begin{align*}
\alpha &= 2.5, \quad M < 0.43 M_\odot; & 
\alpha &= 4, \quad 0.43 M_\odot < M < 2 M_\odot;\\ 
\alpha &= 3.2, \quad 2 M_\odot < M < 20 M_\odot; & 
\alpha &= 1, \quad M > 20 M_\odot.
\end{align*}
Также существует примерная зависимость светимости звёзды от её радиуса, имеющая вид  $L\sim R^{5.2}$.
\subsection{Энергия излучения}
\term{Энергия излучения}~--- энергия, переносимая излучением ($Q_e$).\\
\term{Поток излучения}~--- физическая величина, характеризующая мощность, переносимую излучением,
\begin{equation}
 \Phi_e = \frac{d Q_e}{dt}.
\end{equation}
\imp{Теорема Гаусса}: через любую замкнутую поверхность потоки от одинаковых источников равны.

\term{Спектральная плотность потока излучения}~--- поток излучения, приходящийся на малый единичный интервал спектра,
\begin{equation}
\Phi_{e, \lambda}(\lambda) = \frac{d\Phi_e(\lambda)}{d\lambda}, \quad\quad \Phi_{e, \nu}(\nu) = \frac{d\Phi_e(\nu)}{d\nu} =  \frac{\lambda^2}{c}\Phi_{e, \lambda}(\lambda).
\end{equation}

\term{Объемная плотность энергии излучения}~--- количество энергии на единицу объема
\begin{equation}
U_e = \frac{d Q_e}{dV}.
\end{equation}

\term{Светимость}~--- величина, представляющая собой световой поток излучения, испускаемого с малого участка светящейся поверхности единичной площади,
\begin{equation}
M_e = \frac{d \Phi_e}{dS_1},
\end{equation}
здесь $S_1$~--- площадь объекта, испускающего энергию.

\term{Яркость}~--- световой поток, приходящийся на единичный телесный угол, в расчёте на единичную площадку проекции излучающей поверхности на картинную плоскость, 
\begin{equation}
L_e = \frac{d^2 \Phi_e}{d \Omega\,dS_1 \cos \varepsilon},
\end{equation}
где $\varepsilon$~--- угол между направлением потока излучения и нормалью к плоскости излучающей поверхности.

\term{Интегральная яркость}~--- интеграл яркости по видимой поверхности источника. Показывает количество энергии, пришедшее от источника за единицу времени.
\begin{equation}
\Lambda_e = \int \limits_S L_e(\vec{r})\,ds.
\end{equation}
\term{Освещенность}~--- величина, равная отношению светового потока, падающего на малый участок поверхности, к его площади~--- поверхностная плотность потока
\begin{equation}
E_e = \frac{d\Phi_e}{dS_2} \sim \frac{1}{r^2},
\end{equation}
здесь $S_2$~--- площадь поверхности приёмника, $r$~--- расстояние от источника.
\input{sections/astrophys.flux-albedo.tex}
\input{sections/astrophys.photon.tex}
\input{sections/astrophys.energy-lines.tex}
\subsection{Формула Планка}
\label{sec:planck-law}
\term{Формула Планка}~--- выражение для спектральной плотности мощности излучения абсолютно чёрного тела на интервале частот $[\nu, \nu + d \nu)$, распространяющейся с телесном угле $d\Omega$, которое было получено Максом Планком в 1900~году. Данное выражение имеет следующий вид:
\begin{equation}
B_\nu(\nu,T)=\frac{2h\nu^3}{c^2}\cdot \frac{1}{\exp\left(\frac{h\nu}{kT}\right)-1} = \left[ \frac{\text{Вт}}{\text{м}^2 \cdot \text{Гц} \cdot \text{ср}}\right],
\label{eq:plancks-law-nu}
\end{equation}
где $\nu$~--- частота излучения, $T$~--- температура АЧТ, $h$~--- постоянная Планка, $k$~--- постоянная Больцмана, $c$~--- скорость света.

Если записать закон излучения Планка \eqref{eq:plancks-law-nu} для длин волн, то
\begin{equation}
B_\lambda(\lambda,T)=\frac{2hc^2}{\lambda^5} \cdot \frac{1}{\exp\left(\frac{hc}{\lambda kT}\right)-1} = \left[ \frac{\text{Вт}}{\text{м}^3 \cdot \text{ср}}\right].
\label{eq:plancks-law-lambda}
\end{equation}
\begin{wrapfigure}[15]{l}{.6\tw}
\centering
\vspace{-.9pc}
 \begin{tikzpicture}
  \begin{axis}[
  				width 	=	.6\tw, 
				height	=	6cm, 
  				ymax	=	1e+14,
  				xmax	=	2000,
  				xmin	=	0,
  				ymin	=	0,
				xlabel	=	{Длина волны $\lambda$,~нм}, 
				ylabel 	= 	{$B_\lambda(\lambda, T)$,~$\text{Вт} \cdot \text{м}^{-3}$}
]
   \addplot+[dashed, thin, black] table[x=l, y=tl] {data/planck.txt};
   \addplot+[black] table[x=l, y=t4] {data/planck.txt} node at (axis cs:870, 1.6e+13) {\tiny{$4500$~K}};
   \addplot+[black] table[x=l, y=t5] {data/planck.txt}node at (axis cs:750, 4.2e+13) {\tiny{$5000$~K}};
   \addplot+[black] table[x=l, y=t58] {data/planck.txt}node at (axis cs:670, 8.5e+13) {\tiny{$5800$~K}};
   \addplot+[black] table[x=l, y=t7] {data/planck.txt}node at (axis cs:1350, 3.5e+13) {\tiny{$7000$~K}};
	%\addplot+[black, smooth] table[x=l, y=t15] {data/planck.txt} node at (axis cs:1670, 5.5e+13) {\tiny{$15000$~K}};
  \end{axis}
 \end{tikzpicture}
\caption{Кривые спектральной плотности мощности изотропного излучения АЧТ с разной температурой}\label{pic:wien-law}
\end{wrapfigure}
Стоит заметить, что при переходе в функции к длинам волн меняется не только частота на длину волны, но и выражение для интервала. 

Формула Планка появилась, когда стало ясно, что формула Рэлея-Джинса удовлетворительно описывает излучение только в области больших длин волн, а~с~убыванием длин волн даёт сильные расхождения с реальными данными. Однако формулу Рэлея-Джинса используют и сейчас для описания кривой Планка на больших длинах волн. 

\change{
Проделаем обратные действия: получим формулу Рэлея-Джинса из формулы Планка. Длинноволновая часть спектра характеризуется соотношением $h\nu \ll kT$, то есть 
\begin{equation*}
	\exp\left( \frac{h\nu}{kT}\right) \approx 1 + \frac{h\nu}{kT}.
\end{equation*}
Подставляя полученное выражение в знаменатель \eqref{eq:plancks-law-nu}, получим
\begin{equation*}
	B_\nu(\nu,T) \approx \frac{2h\nu^3}{c^2}\cdot \frac{1}{1 + \frac{h\nu}{kT} - 1} = \frac{2h\nu^3 }{c^2}\cdot \frac{k T}{ h \nu} = \frac{2 \nu^2 k T}{c^2}.
\end{equation*}
}
\change{
	Проделав то же самое для выражения через длину волны, получим:
}
\begin{equation}
	B(\lambda, T) \simeq \frac{2 c k T}{\lambda^4}, \quad\quad B(\nu, T) \simeq \frac{2 \nu^2 k T}{c^2}.
\label{Ray-Jean}
\end{equation}

\change{
	В коротковолновой области, наоборот, $h \nu \gg kT$, следовательно, в знаменателе формулы Планка единица много меньше стоящей там экспоненты, то есть
	\begin{equation*}
		\frac{1}{\exp\left(\frac{h\nu}{kT}\right)-1} \approx \frac{1}{\exp\left(\frac{h\nu}{kT}\right)} = \exp\left(-\frac{h\nu}{kT}\right).
	\end{equation*} 
	Отсюда получаются выражения, называемые приближением Вина:
}
\begin{equation}
B ( \lambda, T) \simeq \frac{2 h c^2}{\lambda^5} \exp \left( -\frac{h c}{\lambda k T}\right), \quad \quad B( \nu, T ) \simeq \frac{2 h \nu^3}{c^2} \exp \left( -\frac{h \nu}{k T} \right).
\end{equation}
\subsection{Закон смещения Вина}
\term{Закон смещения Вина} --- закон, устанавливающий зависимость длины волны~$\lambda_\text{макс}$, на которой спектральная плотность излучения $B_\lambda(\lambda, T)$ абсолютно чёрного тела достигает своего максимума, от температуры $T$ этого тела:
\begin{equation}
	\lambda_\text{макс} \approx \frac{b}{T} \equiv \frac{0.0029~\text{м} \cdot \text{К}}{T}.
\end{equation}
Закон является следствием исследования функции Планка (см.~\ref{sec:planck-law}) на экстремальность.
\subsection{Эффект Доплера. Красное смещение}
\term{Эффект Доплера}~--- эффект изменения частоты и длины волны электромагнитного излучения, регистрируемого приёмником, вызванный относительным движением источника и приёмника (см.~Рис.\,\ref{doppler-ef}).

При $\Delta \lambda \ll \lambda_0$ с большой точностью выполняется следующее важное соотношение:\begin{equation}
\beta \equiv \dfrac{v}{c} = \dfrac{\lambda - \lambda_0}{\lambda_0} \equiv \dfrac{\Delta \lambda}{\lambda_0},
\label{eq:dopler-ef-simple}
\end{equation}
\begin{wrapfigure}[6]{r}{0.5\tw}
\centering
\vspace{-.5pc}
\includegraphics[width=.5\tw]{doppler-ef}
\caption{Эффект Доплера}
\label{doppler-ef}
\end{wrapfigure}
где $\lambda_0$~--- лабораторная длина волны излучения источника, а $\lambda$~--- наблюдаемая. В действительности же имеет место более общий случай: \imp{релятивистский эффект Доплера}, обусловленный проявлением СТО при $v \simeq c$, для которого формула~\eqref{eq:dopler-ef-simple} усложняется и принимает вид
\begin{equation}
\nu = \nu_0 \cdot \dfrac{\sqrt{1 - \beta^2}}{1 + \beta \cdot \cos\theta},
\label{eq:dopler-ef-rel}
\end{equation}
где $\nu$~--- частота, с которой наблюдатель принимает волны, $\nu_0$~--- частота, с которой источник испускает волны, $v$~--- скорость источника, $\theta$~--- угол между направлением на источник и вектором его скорости в системе отсчёта приёмника. Если источник радиально удаляется от наблюдателя, то $\theta = 0$, если приближается, то $\theta =\pi$. Важно, что~\eqref{eq:dopler-ef-simple} напрямую следует из \eqref{eq:dopler-ef-rel} при $\beta  \ll 1$.

\term{Красное смещение}~--- явление сдвига спектральных линий химических элементов в красную (длинноволновую) сторону, обусловленное относительным движение объектов. Параметр красного смещения определяется из наблюдаемой и лабораторной длин волн как
\begin{equation}
z = \dfrac{\lambda - \lambda_0}{\lambda_0}.
\end{equation}

Доплеровское смещение длины волны в спектре источника, движущегося с лучевой скоростью $v_{r}$ и полной скоростью $v$,
\begin{equation}
z = \dfrac{1 + v_r / c}{\sqrt{1 - \beta^2}}.
\end{equation}

\term{Гравитационное красное смещение}~--- проявление эффекта изменения частоты излучения, испущенного массивным объектом, таким как звезда или чёрная дыра. Наблюдается как сдвиг спектральных линий в спектре источника в красную область спектра. Гравитационное красное смещение определяется из формулы, выведенной Эйнштейном,
\begin{equation}
z_G=\dfrac{GM}{c^2 R}-\dfrac{GM}{c^2 r},
\label{eq:grav-red-shift}
\end{equation}
где $M$~--- масса гравитирующего тела, $R$~--- радиальное расстояние от центра масс тела до точки излучения (радиус источника), $r$~---  радиальное расстояние от центра масс источника до точки наблюдения. В случае, когда наблюдатель находится от источника много дальше его радиуса, т.\,е. выполняется соотношение $r \gg R$, выражение~\eqref{eq:grav-red-shift} можно упростить до
\begin{equation}
z_G \simeq \dfrac{GM}{c^2 R}.
\end{equation}

\input{sections/astrophys.light-pressure.tex}
\input{sections/astrophys.edd.tex}
\input{sections/astrophys.grav-lens.tex}
\subsection{Закон Хаббла}
\term{Закон Хаббла}~--- эмпирический закон, связывающий скорость удаления галактик $V$ и расстояние $R$ до них линейным образом: 
\begin{equation}
	V = H R,
\end{equation}
величина $H=68~\text{км/c} \cdot \text{Мпк})$ называется \imp{постоянной Хаббла}.

При $v \ll c$ можно использовать приближение эффекта Доплера, тогда
\begin{equation}
	V = c z.
\label{eq:hubble-speed}
\end{equation}

Равенство \eqref{eq:hubble-speed} справедливо только при $z \ll 1$, а при б\'{o}льших значениях $z$ космологическое красное смещение нльзя связывать с эффектом Доплера, поэтому можно пользоваться только формулой 
\begin{equation}
	\frac{dz}{dt} = - H(z)(1+z),
\end{equation}
где постоянная Хаббла введена как функция красного смещения.
\subsection{Шкала электромагнитных волн}


\term{Гамма излучение} возникает при радиоактивных распадах ядер, при торможении электронов энергией более $10^5$~эВ и при других взаимодействиях элементарных частиц. Используются в гамма-дефектоскопии, при изучении свойств вещества.

\term{Рентгеновские лучи} излучаются при большом ускорении электронов, например при их торможении в металлах. Получают их при помощи рентгеновской трубки: электроны в вакуумной трубке ускоряются электрическим полем при высоком напряжении, достигая анода, при со­ударении резко тормозятся. При торможении электроны движут­ся с ускорением и излучают электромагнитные волны с малой длиной. 

\begin{figure}[!h]
\centering
\includegraphics[width = 1\textwidth]{scale-wave.pdf}
\caption{Шкала электромагнитных волн}
\end{figure}
\term{Ультрафиолетовые лучи}~--- излучение Солнца, ртутных ламп и т.\,п. Используются в ультрафиолетовой микроскопии, в медицине.

\term{Видимое излучение}~--- часть электромагнитного излучения, воспринимаемая глазом (от фиолетового до от красного).

\term{Инфракрасное излучение}~--- тепловое, излучается любым нагретым телом.

\term{Радиоволны} используются повсеместно в обычной жизни, это и сотовая связь, и радиолокация, и спутниковая связь, и Wi-Fi и многое другое.

\term{Низкочастотные волны}~--- диапазон, традиционно используемый в электротехнике. В промышленной электроэнергетике используется частота 50~Гц, на~которой осуществляется передача электрической энергии по линиям и преобразование напряжений трансформаторными устройствами.
\input{sections/astrophys.spec-theor-rel.tex}
\subsection{Оптическая толщина. Закон Бугера}
\term{Оптическая толщина}~--- безразмерная величина, характеризующая степень непрозрачности среды для проходящего сквозь неё излучения,
\begin{equation}
\tau = \int n(x) \sigma(x)\,dx,
\end{equation}
где $\tau$~--- оптическая толщина среды, $n$~--- концентрация частиц, $\sigma$~--- сечение их взаимодействия.

Поток $I_0$ на входе связан с потоком $I$ на выходе \term{Законом Бугера}:
\begin{equation}
I = I_0 e^{-\tau}.
\end{equation}
\input{sections/astrophys.colour.tex}
\input{sections/astrophys.mkt.tex}
\input{sections/astrophys.earth-atmosphere.tex}

\subsection{Эффект Доплера. Красное смещение}
\term{Эффект Доплера}~--- эффект изменения частоты и длины волны электромагнитного излучения, регистрируемого приёмником, вызванный относительным движением источника и приёмника (см.~Рис.\,\ref{doppler-ef}).

При $\Delta \lambda \ll \lambda_0$ с большой точностью выполняется следующее важное соотношение:\begin{equation}
\beta \equiv \dfrac{v}{c} = \dfrac{\lambda - \lambda_0}{\lambda_0} \equiv \dfrac{\Delta \lambda}{\lambda_0},
\label{eq:dopler-ef-simple}
\end{equation}
\begin{wrapfigure}[6]{r}{0.5\tw}
\centering
\vspace{-.5pc}
\includegraphics[width=.5\tw]{doppler-ef}
\caption{Эффект Доплера}
\label{doppler-ef}
\end{wrapfigure}
где $\lambda_0$~--- лабораторная длина волны излучения источника, а $\lambda$~--- наблюдаемая. В действительности же имеет место более общий случай: \imp{релятивистский эффект Доплера}, обусловленный проявлением СТО при $v \simeq c$, для которого формула~\eqref{eq:dopler-ef-simple} усложняется и принимает вид
\begin{equation}
\nu = \nu_0 \cdot \dfrac{\sqrt{1 - \beta^2}}{1 + \beta \cdot \cos\theta},
\label{eq:dopler-ef-rel}
\end{equation}
где $\nu$~--- частота, с которой наблюдатель принимает волны, $\nu_0$~--- частота, с которой источник испускает волны, $v$~--- скорость источника, $\theta$~--- угол между направлением на источник и вектором его скорости в системе отсчёта приёмника. Если источник радиально удаляется от наблюдателя, то $\theta = 0$, если приближается, то $\theta =\pi$. Важно, что~\eqref{eq:dopler-ef-simple} напрямую следует из \eqref{eq:dopler-ef-rel} при $\beta  \ll 1$.

\term{Красное смещение}~--- явление сдвига спектральных линий химических элементов в красную (длинноволновую) сторону, обусловленное относительным движение объектов. Параметр красного смещения определяется из наблюдаемой и лабораторной длин волн как
\begin{equation}
z = \dfrac{\lambda - \lambda_0}{\lambda_0}.
\end{equation}

Доплеровское смещение длины волны в спектре источника, движущегося с лучевой скоростью $v_{r}$ и полной скоростью $v$,
\begin{equation}
z = \dfrac{1 + v_r / c}{\sqrt{1 - \beta^2}}.
\end{equation}

\term{Гравитационное красное смещение}~--- проявление эффекта изменения частоты излучения, испущенного массивным объектом, таким как звезда или чёрная дыра. Наблюдается как сдвиг спектральных линий в спектре источника в красную область спектра. Гравитационное красное смещение определяется из формулы, выведенной Эйнштейном,
\begin{equation}
z_G=\dfrac{GM}{c^2 R}-\dfrac{GM}{c^2 r},
\label{eq:grav-red-shift}
\end{equation}
где $M$~--- масса гравитирующего тела, $R$~--- радиальное расстояние от центра масс тела до точки излучения (радиус источника), $r$~---  радиальное расстояние от центра масс источника до точки наблюдения. В случае, когда наблюдатель находится от источника много дальше его радиуса, т.\,е. выполняется соотношение $r \gg R$, выражение~\eqref{eq:grav-red-shift} можно упростить до
\begin{equation}
z_G \simeq \dfrac{GM}{c^2 R}.
\end{equation}

\newpage
\section{Астрофизика}
\subsection{Звёздные величины}
Звёздная величина~--- безразмерная числовая характеристика яркости объекта. Известно, что увеличению светового потока в 100 раз соответствует уменьшение видимой звёздной величины ровно на 5 единиц. Тогда уменьшение звёздной величины на одну единицу означает увеличение светового потока в $\sqrt[5]{100}\approx 2.512$~раз, то есть звёздные величины являются логарифмической шкалой измерения плотности потока. Зависимость, связывающая отношение освещённостей $E_1$ и $E_2$ и разность звёздных величин $m_1$ и $m_2$ двух объектов, называется \term{формулой Погсона} и имеет вид
\begin{equation}
	\frac{E_1}{E_2} = 10^{0.4(m_2 - m_1)} \quad \Longleftrightarrow \quad m_2 = m_1 + 2.5 \lg \frac{E_1}{E_2}.
	\label{eq:Pogson-law}
\end{equation}
Широко используется понятие \term{абсолютной звёздной величины} $M$~--- это видимая звёздная величина $m$ при наблюдении с установленного расстояния: для звёзд~---~10~пк, для тел Солнечной системы~---~1~\au, причем считается, что тело находится в 1~\au~и от наблюдателя и от Солнца, а фаза равна единице, то есть можно считать, что наблюдатель находится в центре Солнца, а~тело~--- в~1~\au~от него. 

Кроме этого, важно понятие \term{болометрической звёздной величины} $m_\text{bol}$~--- это звёздная величина, при расчёте которой учитывается полная мощность излучения источника во всех диапазонах электромагнитных волн. Обычная (видимая) звёздная величина учитывает излучение лишь в видимой части спектра от примерно 380~нм до примерно~780~нм. Разность между болометрической и видимой звёздными величинами называется \term{болометрической поправкой} ($BC$), которая отличается для разных спектральных классов звёзд. Из определения, болометрическая поправка может быть найдена по формуле
\begin{equation}
	BC = m_\text{bol} - m.
\end{equation}
Абсолютную звёздную величину звезды можно получить по формуле Погсона \eqref{eq:Pogson-law} из видимой звёздной величины $m$ и расстояния $r$ до неё в парсеках
\begin{equation}
	M = m + 2.5 \lg \frac{E}{E_\text{абс}} = m + 2.5 \lg \frac{(10~\text{пк})^2}{r^2} = m + 5 - 5\lg r.
	\label{eq:abs-mag}
\end{equation} 
Если принимать к рассмотрению межзвездное поглощение $A$, то формулу  \eqref{eq:abs-mag} необходимо уточнить:
\begin{equation}
	M = m + 5 - 5\lg r - Ar.
\end{equation}
\subsection{Закон Стефана-Больцмана}
\term{Закон Стефана~--- Больцмана} определяет зависимость плотности мощности излучения абсолютно чёрного тела (АЧТ) $u$ от его температуры $T$:
\begin{equation}
u = a T^4,
\end{equation} 
где $a$~--- некая универсальная константа.
Отсюда полная светимость АЧТ с площадью поверхности $S$
	\begin{equation}
	L = S \sigma T^4,
	\label{eq:steff-bol-law}
\end{equation}
константа $\sigma$ называется \term{постоянной Стефана-Больцмана}.
  
Важно отметить, что \imp{закон Стефана-Больцмана}~--- прямое следствие формулы Планка \eqref{Planck's formula}, так как
\begin{equation}
	\sigma T^4 = \int\limits^\infty_0 B(\lambda, T)\,d\lambda \int\limits_0^{\pi/2} \sin \varphi\, d\varphi \int\limits_0^{2\pi} \cos \varphi\, d\theta = \pi \int\limits^\infty_0 B(\lambda, T)\,d\lambda,
\end{equation}
откуда $\sigma = (2\pi^5k^4)/(15c^2h^3) = 5.67 \cdot 10^{-8}~\text{Вт}/(\text{м}^2\cdot \text{К}^4)$.

%Для АЧТ сферической формы с радиусом $R$ формула~\eqref{eq:steff-bol-law} принимает вид
%\begin{equation}
%L=4\pi R^2\sigma T^4.
%\end{equation}
Для звёзд главной последовательности выполняется соотношение $L \sim M^{\alpha}$, где~$\alpha$~--- коэффициент пропорциональности, который зависит от массы звезды следующим образом:
\begin{align*}
\alpha &= 2.5, \quad M < 0.43 M_\odot; & 
\alpha &= 4, \quad 0.43 M_\odot < M < 2 M_\odot;\\ 
\alpha &= 3.2, \quad 2 M_\odot < M < 20 M_\odot; & 
\alpha &= 1, \quad M > 20 M_\odot.
\end{align*}
Также существует примерная зависимость светимости звёзды от её радиуса, имеющая вид  $L\sim R^{5.2}$.
\subsection{Энергия излучения}
\term{Энергия излучения}~--- энергия, переносимая излучением ($Q_e$).\\
\term{Поток излучения}~--- физическая величина, характеризующая мощность, переносимую излучением,
\begin{equation}
 \Phi_e = \frac{d Q_e}{dt}.
\end{equation}
\imp{Теорема Гаусса}: через любую замкнутую поверхность потоки от одинаковых источников равны.

\term{Спектральная плотность потока излучения}~--- поток излучения, приходящийся на малый единичный интервал спектра,
\begin{equation}
\Phi_{e, \lambda}(\lambda) = \frac{d\Phi_e(\lambda)}{d\lambda}, \quad\quad \Phi_{e, \nu}(\nu) = \frac{d\Phi_e(\nu)}{d\nu} =  \frac{\lambda^2}{c}\Phi_{e, \lambda}(\lambda).
\end{equation}

\term{Объемная плотность энергии излучения}~--- количество энергии на единицу объема
\begin{equation}
U_e = \frac{d Q_e}{dV}.
\end{equation}

\term{Светимость}~--- величина, представляющая собой световой поток излучения, испускаемого с малого участка светящейся поверхности единичной площади,
\begin{equation}
M_e = \frac{d \Phi_e}{dS_1},
\end{equation}
здесь $S_1$~--- площадь объекта, испускающего энергию.

\term{Яркость}~--- световой поток, приходящийся на единичный телесный угол, в расчёте на единичную площадку проекции излучающей поверхности на картинную плоскость, 
\begin{equation}
L_e = \frac{d^2 \Phi_e}{d \Omega\,dS_1 \cos \varepsilon},
\end{equation}
где $\varepsilon$~--- угол между направлением потока излучения и нормалью к плоскости излучающей поверхности.

\term{Интегральная яркость}~--- интеграл яркости по видимой поверхности источника. Показывает количество энергии, пришедшее от источника за единицу времени.
\begin{equation}
\Lambda_e = \int \limits_S L_e(\vec{r})\,ds.
\end{equation}
\term{Освещенность}~--- величина, равная отношению светового потока, падающего на малый участок поверхности, к его площади~--- поверхностная плотность потока
\begin{equation}
E_e = \frac{d\Phi_e}{dS_2} \sim \frac{1}{r^2},
\end{equation}
здесь $S_2$~--- площадь поверхности приёмника, $r$~--- расстояние от источника.
\input{sections/astrophys.flux-albedo.tex}
\input{sections/astrophys.photon.tex}
\input{sections/astrophys.energy-lines.tex}
\subsection{Формула Планка}
\label{sec:planck-law}
\term{Формула Планка}~--- выражение для спектральной плотности мощности излучения абсолютно чёрного тела на интервале частот $[\nu, \nu + d \nu)$, распространяющейся с телесном угле $d\Omega$, которое было получено Максом Планком в 1900~году. Данное выражение имеет следующий вид:
\begin{equation}
B_\nu(\nu,T)=\frac{2h\nu^3}{c^2}\cdot \frac{1}{\exp\left(\frac{h\nu}{kT}\right)-1} = \left[ \frac{\text{Вт}}{\text{м}^2 \cdot \text{Гц} \cdot \text{ср}}\right],
\label{eq:plancks-law-nu}
\end{equation}
где $\nu$~--- частота излучения, $T$~--- температура АЧТ, $h$~--- постоянная Планка, $k$~--- постоянная Больцмана, $c$~--- скорость света.

Если записать закон излучения Планка \eqref{eq:plancks-law-nu} для длин волн, то
\begin{equation}
B_\lambda(\lambda,T)=\frac{2hc^2}{\lambda^5} \cdot \frac{1}{\exp\left(\frac{hc}{\lambda kT}\right)-1} = \left[ \frac{\text{Вт}}{\text{м}^3 \cdot \text{ср}}\right].
\label{eq:plancks-law-lambda}
\end{equation}
\begin{wrapfigure}[15]{l}{.6\tw}
\centering
\vspace{-.9pc}
 \begin{tikzpicture}
  \begin{axis}[
  				width 	=	.6\tw, 
				height	=	6cm, 
  				ymax	=	1e+14,
  				xmax	=	2000,
  				xmin	=	0,
  				ymin	=	0,
				xlabel	=	{Длина волны $\lambda$,~нм}, 
				ylabel 	= 	{$B_\lambda(\lambda, T)$,~$\text{Вт} \cdot \text{м}^{-3}$}
]
   \addplot+[dashed, thin, black] table[x=l, y=tl] {data/planck.txt};
   \addplot+[black] table[x=l, y=t4] {data/planck.txt} node at (axis cs:870, 1.6e+13) {\tiny{$4500$~K}};
   \addplot+[black] table[x=l, y=t5] {data/planck.txt}node at (axis cs:750, 4.2e+13) {\tiny{$5000$~K}};
   \addplot+[black] table[x=l, y=t58] {data/planck.txt}node at (axis cs:670, 8.5e+13) {\tiny{$5800$~K}};
   \addplot+[black] table[x=l, y=t7] {data/planck.txt}node at (axis cs:1350, 3.5e+13) {\tiny{$7000$~K}};
	%\addplot+[black, smooth] table[x=l, y=t15] {data/planck.txt} node at (axis cs:1670, 5.5e+13) {\tiny{$15000$~K}};
  \end{axis}
 \end{tikzpicture}
\caption{Кривые спектральной плотности мощности изотропного излучения АЧТ с разной температурой}\label{pic:wien-law}
\end{wrapfigure}
Стоит заметить, что при переходе в функции к длинам волн меняется не только частота на длину волны, но и выражение для интервала. 

Формула Планка появилась, когда стало ясно, что формула Рэлея-Джинса удовлетворительно описывает излучение только в области больших длин волн, а~с~убыванием длин волн даёт сильные расхождения с реальными данными. Однако формулу Рэлея-Джинса используют и сейчас для описания кривой Планка на больших длинах волн. 

\change{
Проделаем обратные действия: получим формулу Рэлея-Джинса из формулы Планка. Длинноволновая часть спектра характеризуется соотношением $h\nu \ll kT$, то есть 
\begin{equation*}
	\exp\left( \frac{h\nu}{kT}\right) \approx 1 + \frac{h\nu}{kT}.
\end{equation*}
Подставляя полученное выражение в знаменатель \eqref{eq:plancks-law-nu}, получим
\begin{equation*}
	B_\nu(\nu,T) \approx \frac{2h\nu^3}{c^2}\cdot \frac{1}{1 + \frac{h\nu}{kT} - 1} = \frac{2h\nu^3 }{c^2}\cdot \frac{k T}{ h \nu} = \frac{2 \nu^2 k T}{c^2}.
\end{equation*}
}
\change{
	Проделав то же самое для выражения через длину волны, получим:
}
\begin{equation}
	B(\lambda, T) \simeq \frac{2 c k T}{\lambda^4}, \quad\quad B(\nu, T) \simeq \frac{2 \nu^2 k T}{c^2}.
\label{Ray-Jean}
\end{equation}

\change{
	В коротковолновой области, наоборот, $h \nu \gg kT$, следовательно, в знаменателе формулы Планка единица много меньше стоящей там экспоненты, то есть
	\begin{equation*}
		\frac{1}{\exp\left(\frac{h\nu}{kT}\right)-1} \approx \frac{1}{\exp\left(\frac{h\nu}{kT}\right)} = \exp\left(-\frac{h\nu}{kT}\right).
	\end{equation*} 
	Отсюда получаются выражения, называемые приближением Вина:
}
\begin{equation}
B ( \lambda, T) \simeq \frac{2 h c^2}{\lambda^5} \exp \left( -\frac{h c}{\lambda k T}\right), \quad \quad B( \nu, T ) \simeq \frac{2 h \nu^3}{c^2} \exp \left( -\frac{h \nu}{k T} \right).
\end{equation}
\subsection{Закон смещения Вина}
\term{Закон смещения Вина} --- закон, устанавливающий зависимость длины волны~$\lambda_\text{макс}$, на которой спектральная плотность излучения $B_\lambda(\lambda, T)$ абсолютно чёрного тела достигает своего максимума, от температуры $T$ этого тела:
\begin{equation}
	\lambda_\text{макс} \approx \frac{b}{T} \equiv \frac{0.0029~\text{м} \cdot \text{К}}{T}.
\end{equation}
Закон является следствием исследования функции Планка (см.~\ref{sec:planck-law}) на экстремальность.
\subsection{Эффект Доплера. Красное смещение}
\term{Эффект Доплера}~--- эффект изменения частоты и длины волны электромагнитного излучения, регистрируемого приёмником, вызванный относительным движением источника и приёмника (см.~Рис.\,\ref{doppler-ef}).

При $\Delta \lambda \ll \lambda_0$ с большой точностью выполняется следующее важное соотношение:\begin{equation}
\beta \equiv \dfrac{v}{c} = \dfrac{\lambda - \lambda_0}{\lambda_0} \equiv \dfrac{\Delta \lambda}{\lambda_0},
\label{eq:dopler-ef-simple}
\end{equation}
\begin{wrapfigure}[6]{r}{0.5\tw}
\centering
\vspace{-.5pc}
\includegraphics[width=.5\tw]{doppler-ef}
\caption{Эффект Доплера}
\label{doppler-ef}
\end{wrapfigure}
где $\lambda_0$~--- лабораторная длина волны излучения источника, а $\lambda$~--- наблюдаемая. В действительности же имеет место более общий случай: \imp{релятивистский эффект Доплера}, обусловленный проявлением СТО при $v \simeq c$, для которого формула~\eqref{eq:dopler-ef-simple} усложняется и принимает вид
\begin{equation}
\nu = \nu_0 \cdot \dfrac{\sqrt{1 - \beta^2}}{1 + \beta \cdot \cos\theta},
\label{eq:dopler-ef-rel}
\end{equation}
где $\nu$~--- частота, с которой наблюдатель принимает волны, $\nu_0$~--- частота, с которой источник испускает волны, $v$~--- скорость источника, $\theta$~--- угол между направлением на источник и вектором его скорости в системе отсчёта приёмника. Если источник радиально удаляется от наблюдателя, то $\theta = 0$, если приближается, то $\theta =\pi$. Важно, что~\eqref{eq:dopler-ef-simple} напрямую следует из \eqref{eq:dopler-ef-rel} при $\beta  \ll 1$.

\term{Красное смещение}~--- явление сдвига спектральных линий химических элементов в красную (длинноволновую) сторону, обусловленное относительным движение объектов. Параметр красного смещения определяется из наблюдаемой и лабораторной длин волн как
\begin{equation}
z = \dfrac{\lambda - \lambda_0}{\lambda_0}.
\end{equation}

Доплеровское смещение длины волны в спектре источника, движущегося с лучевой скоростью $v_{r}$ и полной скоростью $v$,
\begin{equation}
z = \dfrac{1 + v_r / c}{\sqrt{1 - \beta^2}}.
\end{equation}

\term{Гравитационное красное смещение}~--- проявление эффекта изменения частоты излучения, испущенного массивным объектом, таким как звезда или чёрная дыра. Наблюдается как сдвиг спектральных линий в спектре источника в красную область спектра. Гравитационное красное смещение определяется из формулы, выведенной Эйнштейном,
\begin{equation}
z_G=\dfrac{GM}{c^2 R}-\dfrac{GM}{c^2 r},
\label{eq:grav-red-shift}
\end{equation}
где $M$~--- масса гравитирующего тела, $R$~--- радиальное расстояние от центра масс тела до точки излучения (радиус источника), $r$~---  радиальное расстояние от центра масс источника до точки наблюдения. В случае, когда наблюдатель находится от источника много дальше его радиуса, т.\,е. выполняется соотношение $r \gg R$, выражение~\eqref{eq:grav-red-shift} можно упростить до
\begin{equation}
z_G \simeq \dfrac{GM}{c^2 R}.
\end{equation}

\input{sections/astrophys.light-pressure.tex}
\input{sections/astrophys.edd.tex}
\input{sections/astrophys.grav-lens.tex}
\subsection{Закон Хаббла}
\term{Закон Хаббла}~--- эмпирический закон, связывающий скорость удаления галактик $V$ и расстояние $R$ до них линейным образом: 
\begin{equation}
	V = H R,
\end{equation}
величина $H=68~\text{км/c} \cdot \text{Мпк})$ называется \imp{постоянной Хаббла}.

При $v \ll c$ можно использовать приближение эффекта Доплера, тогда
\begin{equation}
	V = c z.
\label{eq:hubble-speed}
\end{equation}

Равенство \eqref{eq:hubble-speed} справедливо только при $z \ll 1$, а при б\'{o}льших значениях $z$ космологическое красное смещение нльзя связывать с эффектом Доплера, поэтому можно пользоваться только формулой 
\begin{equation}
	\frac{dz}{dt} = - H(z)(1+z),
\end{equation}
где постоянная Хаббла введена как функция красного смещения.
\subsection{Шкала электромагнитных волн}


\term{Гамма излучение} возникает при радиоактивных распадах ядер, при торможении электронов энергией более $10^5$~эВ и при других взаимодействиях элементарных частиц. Используются в гамма-дефектоскопии, при изучении свойств вещества.

\term{Рентгеновские лучи} излучаются при большом ускорении электронов, например при их торможении в металлах. Получают их при помощи рентгеновской трубки: электроны в вакуумной трубке ускоряются электрическим полем при высоком напряжении, достигая анода, при со­ударении резко тормозятся. При торможении электроны движут­ся с ускорением и излучают электромагнитные волны с малой длиной. 

\begin{figure}[!h]
\centering
\includegraphics[width = 1\textwidth]{scale-wave.pdf}
\caption{Шкала электромагнитных волн}
\end{figure}
\term{Ультрафиолетовые лучи}~--- излучение Солнца, ртутных ламп и т.\,п. Используются в ультрафиолетовой микроскопии, в медицине.

\term{Видимое излучение}~--- часть электромагнитного излучения, воспринимаемая глазом (от фиолетового до от красного).

\term{Инфракрасное излучение}~--- тепловое, излучается любым нагретым телом.

\term{Радиоволны} используются повсеместно в обычной жизни, это и сотовая связь, и радиолокация, и спутниковая связь, и Wi-Fi и многое другое.

\term{Низкочастотные волны}~--- диапазон, традиционно используемый в электротехнике. В промышленной электроэнергетике используется частота 50~Гц, на~которой осуществляется передача электрической энергии по линиям и преобразование напряжений трансформаторными устройствами.
\input{sections/astrophys.spec-theor-rel.tex}
\subsection{Оптическая толщина. Закон Бугера}
\term{Оптическая толщина}~--- безразмерная величина, характеризующая степень непрозрачности среды для проходящего сквозь неё излучения,
\begin{equation}
\tau = \int n(x) \sigma(x)\,dx,
\end{equation}
где $\tau$~--- оптическая толщина среды, $n$~--- концентрация частиц, $\sigma$~--- сечение их взаимодействия.

Поток $I_0$ на входе связан с потоком $I$ на выходе \term{Законом Бугера}:
\begin{equation}
I = I_0 e^{-\tau}.
\end{equation}
\input{sections/astrophys.colour.tex}
\input{sections/astrophys.mkt.tex}
\input{sections/astrophys.earth-atmosphere.tex}

\newpage
\section{Астрофизика}
\subsection{Звёздные величины}
Звёздная величина~--- безразмерная числовая характеристика яркости объекта. Известно, что увеличению светового потока в 100 раз соответствует уменьшение видимой звёздной величины ровно на 5 единиц. Тогда уменьшение звёздной величины на одну единицу означает увеличение светового потока в $\sqrt[5]{100}\approx 2.512$~раз, то есть звёздные величины являются логарифмической шкалой измерения плотности потока. Зависимость, связывающая отношение освещённостей $E_1$ и $E_2$ и разность звёздных величин $m_1$ и $m_2$ двух объектов, называется \term{формулой Погсона} и имеет вид
\begin{equation}
	\frac{E_1}{E_2} = 10^{0.4(m_2 - m_1)} \quad \Longleftrightarrow \quad m_2 = m_1 + 2.5 \lg \frac{E_1}{E_2}.
	\label{eq:Pogson-law}
\end{equation}
Широко используется понятие \term{абсолютной звёздной величины} $M$~--- это видимая звёздная величина $m$ при наблюдении с установленного расстояния: для звёзд~---~10~пк, для тел Солнечной системы~---~1~\au, причем считается, что тело находится в 1~\au~и от наблюдателя и от Солнца, а фаза равна единице, то есть можно считать, что наблюдатель находится в центре Солнца, а~тело~--- в~1~\au~от него. 

Кроме этого, важно понятие \term{болометрической звёздной величины} $m_\text{bol}$~--- это звёздная величина, при расчёте которой учитывается полная мощность излучения источника во всех диапазонах электромагнитных волн. Обычная (видимая) звёздная величина учитывает излучение лишь в видимой части спектра от примерно 380~нм до примерно~780~нм. Разность между болометрической и видимой звёздными величинами называется \term{болометрической поправкой} ($BC$), которая отличается для разных спектральных классов звёзд. Из определения, болометрическая поправка может быть найдена по формуле
\begin{equation}
	BC = m_\text{bol} - m.
\end{equation}
Абсолютную звёздную величину звезды можно получить по формуле Погсона \eqref{eq:Pogson-law} из видимой звёздной величины $m$ и расстояния $r$ до неё в парсеках
\begin{equation}
	M = m + 2.5 \lg \frac{E}{E_\text{абс}} = m + 2.5 \lg \frac{(10~\text{пк})^2}{r^2} = m + 5 - 5\lg r.
	\label{eq:abs-mag}
\end{equation} 
Если принимать к рассмотрению межзвездное поглощение $A$, то формулу  \eqref{eq:abs-mag} необходимо уточнить:
\begin{equation}
	M = m + 5 - 5\lg r - Ar.
\end{equation}
\subsection{Закон Стефана-Больцмана}
\term{Закон Стефана~--- Больцмана} определяет зависимость плотности мощности излучения абсолютно чёрного тела (АЧТ) $u$ от его температуры $T$:
\begin{equation}
u = a T^4,
\end{equation} 
где $a$~--- некая универсальная константа.
Отсюда полная светимость АЧТ с площадью поверхности $S$
	\begin{equation}
	L = S \sigma T^4,
	\label{eq:steff-bol-law}
\end{equation}
константа $\sigma$ называется \term{постоянной Стефана-Больцмана}.
  
Важно отметить, что \imp{закон Стефана-Больцмана}~--- прямое следствие формулы Планка \eqref{Planck's formula}, так как
\begin{equation}
	\sigma T^4 = \int\limits^\infty_0 B(\lambda, T)\,d\lambda \int\limits_0^{\pi/2} \sin \varphi\, d\varphi \int\limits_0^{2\pi} \cos \varphi\, d\theta = \pi \int\limits^\infty_0 B(\lambda, T)\,d\lambda,
\end{equation}
откуда $\sigma = (2\pi^5k^4)/(15c^2h^3) = 5.67 \cdot 10^{-8}~\text{Вт}/(\text{м}^2\cdot \text{К}^4)$.

%Для АЧТ сферической формы с радиусом $R$ формула~\eqref{eq:steff-bol-law} принимает вид
%\begin{equation}
%L=4\pi R^2\sigma T^4.
%\end{equation}
Для звёзд главной последовательности выполняется соотношение $L \sim M^{\alpha}$, где~$\alpha$~--- коэффициент пропорциональности, который зависит от массы звезды следующим образом:
\begin{align*}
\alpha &= 2.5, \quad M < 0.43 M_\odot; & 
\alpha &= 4, \quad 0.43 M_\odot < M < 2 M_\odot;\\ 
\alpha &= 3.2, \quad 2 M_\odot < M < 20 M_\odot; & 
\alpha &= 1, \quad M > 20 M_\odot.
\end{align*}
Также существует примерная зависимость светимости звёзды от её радиуса, имеющая вид  $L\sim R^{5.2}$.
\subsection{Энергия излучения}
\term{Энергия излучения}~--- энергия, переносимая излучением ($Q_e$).\\
\term{Поток излучения}~--- физическая величина, характеризующая мощность, переносимую излучением,
\begin{equation}
 \Phi_e = \frac{d Q_e}{dt}.
\end{equation}
\imp{Теорема Гаусса}: через любую замкнутую поверхность потоки от одинаковых источников равны.

\term{Спектральная плотность потока излучения}~--- поток излучения, приходящийся на малый единичный интервал спектра,
\begin{equation}
\Phi_{e, \lambda}(\lambda) = \frac{d\Phi_e(\lambda)}{d\lambda}, \quad\quad \Phi_{e, \nu}(\nu) = \frac{d\Phi_e(\nu)}{d\nu} =  \frac{\lambda^2}{c}\Phi_{e, \lambda}(\lambda).
\end{equation}

\term{Объемная плотность энергии излучения}~--- количество энергии на единицу объема
\begin{equation}
U_e = \frac{d Q_e}{dV}.
\end{equation}

\term{Светимость}~--- величина, представляющая собой световой поток излучения, испускаемого с малого участка светящейся поверхности единичной площади,
\begin{equation}
M_e = \frac{d \Phi_e}{dS_1},
\end{equation}
здесь $S_1$~--- площадь объекта, испускающего энергию.

\term{Яркость}~--- световой поток, приходящийся на единичный телесный угол, в расчёте на единичную площадку проекции излучающей поверхности на картинную плоскость, 
\begin{equation}
L_e = \frac{d^2 \Phi_e}{d \Omega\,dS_1 \cos \varepsilon},
\end{equation}
где $\varepsilon$~--- угол между направлением потока излучения и нормалью к плоскости излучающей поверхности.

\term{Интегральная яркость}~--- интеграл яркости по видимой поверхности источника. Показывает количество энергии, пришедшее от источника за единицу времени.
\begin{equation}
\Lambda_e = \int \limits_S L_e(\vec{r})\,ds.
\end{equation}
\term{Освещенность}~--- величина, равная отношению светового потока, падающего на малый участок поверхности, к его площади~--- поверхностная плотность потока
\begin{equation}
E_e = \frac{d\Phi_e}{dS_2} \sim \frac{1}{r^2},
\end{equation}
здесь $S_2$~--- площадь поверхности приёмника, $r$~--- расстояние от источника.
\input{sections/astrophys.flux-albedo.tex}
\input{sections/astrophys.photon.tex}
\input{sections/astrophys.energy-lines.tex}
\subsection{Формула Планка}
\label{sec:planck-law}
\term{Формула Планка}~--- выражение для спектральной плотности мощности излучения абсолютно чёрного тела на интервале частот $[\nu, \nu + d \nu)$, распространяющейся с телесном угле $d\Omega$, которое было получено Максом Планком в 1900~году. Данное выражение имеет следующий вид:
\begin{equation}
B_\nu(\nu,T)=\frac{2h\nu^3}{c^2}\cdot \frac{1}{\exp\left(\frac{h\nu}{kT}\right)-1} = \left[ \frac{\text{Вт}}{\text{м}^2 \cdot \text{Гц} \cdot \text{ср}}\right],
\label{eq:plancks-law-nu}
\end{equation}
где $\nu$~--- частота излучения, $T$~--- температура АЧТ, $h$~--- постоянная Планка, $k$~--- постоянная Больцмана, $c$~--- скорость света.

Если записать закон излучения Планка \eqref{eq:plancks-law-nu} для длин волн, то
\begin{equation}
B_\lambda(\lambda,T)=\frac{2hc^2}{\lambda^5} \cdot \frac{1}{\exp\left(\frac{hc}{\lambda kT}\right)-1} = \left[ \frac{\text{Вт}}{\text{м}^3 \cdot \text{ср}}\right].
\label{eq:plancks-law-lambda}
\end{equation}
\begin{wrapfigure}[15]{l}{.6\tw}
\centering
\vspace{-.9pc}
 \begin{tikzpicture}
  \begin{axis}[
  				width 	=	.6\tw, 
				height	=	6cm, 
  				ymax	=	1e+14,
  				xmax	=	2000,
  				xmin	=	0,
  				ymin	=	0,
				xlabel	=	{Длина волны $\lambda$,~нм}, 
				ylabel 	= 	{$B_\lambda(\lambda, T)$,~$\text{Вт} \cdot \text{м}^{-3}$}
]
   \addplot+[dashed, thin, black] table[x=l, y=tl] {data/planck.txt};
   \addplot+[black] table[x=l, y=t4] {data/planck.txt} node at (axis cs:870, 1.6e+13) {\tiny{$4500$~K}};
   \addplot+[black] table[x=l, y=t5] {data/planck.txt}node at (axis cs:750, 4.2e+13) {\tiny{$5000$~K}};
   \addplot+[black] table[x=l, y=t58] {data/planck.txt}node at (axis cs:670, 8.5e+13) {\tiny{$5800$~K}};
   \addplot+[black] table[x=l, y=t7] {data/planck.txt}node at (axis cs:1350, 3.5e+13) {\tiny{$7000$~K}};
	%\addplot+[black, smooth] table[x=l, y=t15] {data/planck.txt} node at (axis cs:1670, 5.5e+13) {\tiny{$15000$~K}};
  \end{axis}
 \end{tikzpicture}
\caption{Кривые спектральной плотности мощности изотропного излучения АЧТ с разной температурой}\label{pic:wien-law}
\end{wrapfigure}
Стоит заметить, что при переходе в функции к длинам волн меняется не только частота на длину волны, но и выражение для интервала. 

Формула Планка появилась, когда стало ясно, что формула Рэлея-Джинса удовлетворительно описывает излучение только в области больших длин волн, а~с~убыванием длин волн даёт сильные расхождения с реальными данными. Однако формулу Рэлея-Джинса используют и сейчас для описания кривой Планка на больших длинах волн. 

\change{
Проделаем обратные действия: получим формулу Рэлея-Джинса из формулы Планка. Длинноволновая часть спектра характеризуется соотношением $h\nu \ll kT$, то есть 
\begin{equation*}
	\exp\left( \frac{h\nu}{kT}\right) \approx 1 + \frac{h\nu}{kT}.
\end{equation*}
Подставляя полученное выражение в знаменатель \eqref{eq:plancks-law-nu}, получим
\begin{equation*}
	B_\nu(\nu,T) \approx \frac{2h\nu^3}{c^2}\cdot \frac{1}{1 + \frac{h\nu}{kT} - 1} = \frac{2h\nu^3 }{c^2}\cdot \frac{k T}{ h \nu} = \frac{2 \nu^2 k T}{c^2}.
\end{equation*}
}
\change{
	Проделав то же самое для выражения через длину волны, получим:
}
\begin{equation}
	B(\lambda, T) \simeq \frac{2 c k T}{\lambda^4}, \quad\quad B(\nu, T) \simeq \frac{2 \nu^2 k T}{c^2}.
\label{Ray-Jean}
\end{equation}

\change{
	В коротковолновой области, наоборот, $h \nu \gg kT$, следовательно, в знаменателе формулы Планка единица много меньше стоящей там экспоненты, то есть
	\begin{equation*}
		\frac{1}{\exp\left(\frac{h\nu}{kT}\right)-1} \approx \frac{1}{\exp\left(\frac{h\nu}{kT}\right)} = \exp\left(-\frac{h\nu}{kT}\right).
	\end{equation*} 
	Отсюда получаются выражения, называемые приближением Вина:
}
\begin{equation}
B ( \lambda, T) \simeq \frac{2 h c^2}{\lambda^5} \exp \left( -\frac{h c}{\lambda k T}\right), \quad \quad B( \nu, T ) \simeq \frac{2 h \nu^3}{c^2} \exp \left( -\frac{h \nu}{k T} \right).
\end{equation}
\subsection{Закон смещения Вина}
\term{Закон смещения Вина} --- закон, устанавливающий зависимость длины волны~$\lambda_\text{макс}$, на которой спектральная плотность излучения $B_\lambda(\lambda, T)$ абсолютно чёрного тела достигает своего максимума, от температуры $T$ этого тела:
\begin{equation}
	\lambda_\text{макс} \approx \frac{b}{T} \equiv \frac{0.0029~\text{м} \cdot \text{К}}{T}.
\end{equation}
Закон является следствием исследования функции Планка (см.~\ref{sec:planck-law}) на экстремальность.
\subsection{Эффект Доплера. Красное смещение}
\term{Эффект Доплера}~--- эффект изменения частоты и длины волны электромагнитного излучения, регистрируемого приёмником, вызванный относительным движением источника и приёмника (см.~Рис.\,\ref{doppler-ef}).

При $\Delta \lambda \ll \lambda_0$ с большой точностью выполняется следующее важное соотношение:\begin{equation}
\beta \equiv \dfrac{v}{c} = \dfrac{\lambda - \lambda_0}{\lambda_0} \equiv \dfrac{\Delta \lambda}{\lambda_0},
\label{eq:dopler-ef-simple}
\end{equation}
\begin{wrapfigure}[6]{r}{0.5\tw}
\centering
\vspace{-.5pc}
\includegraphics[width=.5\tw]{doppler-ef}
\caption{Эффект Доплера}
\label{doppler-ef}
\end{wrapfigure}
где $\lambda_0$~--- лабораторная длина волны излучения источника, а $\lambda$~--- наблюдаемая. В действительности же имеет место более общий случай: \imp{релятивистский эффект Доплера}, обусловленный проявлением СТО при $v \simeq c$, для которого формула~\eqref{eq:dopler-ef-simple} усложняется и принимает вид
\begin{equation}
\nu = \nu_0 \cdot \dfrac{\sqrt{1 - \beta^2}}{1 + \beta \cdot \cos\theta},
\label{eq:dopler-ef-rel}
\end{equation}
где $\nu$~--- частота, с которой наблюдатель принимает волны, $\nu_0$~--- частота, с которой источник испускает волны, $v$~--- скорость источника, $\theta$~--- угол между направлением на источник и вектором его скорости в системе отсчёта приёмника. Если источник радиально удаляется от наблюдателя, то $\theta = 0$, если приближается, то $\theta =\pi$. Важно, что~\eqref{eq:dopler-ef-simple} напрямую следует из \eqref{eq:dopler-ef-rel} при $\beta  \ll 1$.

\term{Красное смещение}~--- явление сдвига спектральных линий химических элементов в красную (длинноволновую) сторону, обусловленное относительным движение объектов. Параметр красного смещения определяется из наблюдаемой и лабораторной длин волн как
\begin{equation}
z = \dfrac{\lambda - \lambda_0}{\lambda_0}.
\end{equation}

Доплеровское смещение длины волны в спектре источника, движущегося с лучевой скоростью $v_{r}$ и полной скоростью $v$,
\begin{equation}
z = \dfrac{1 + v_r / c}{\sqrt{1 - \beta^2}}.
\end{equation}

\term{Гравитационное красное смещение}~--- проявление эффекта изменения частоты излучения, испущенного массивным объектом, таким как звезда или чёрная дыра. Наблюдается как сдвиг спектральных линий в спектре источника в красную область спектра. Гравитационное красное смещение определяется из формулы, выведенной Эйнштейном,
\begin{equation}
z_G=\dfrac{GM}{c^2 R}-\dfrac{GM}{c^2 r},
\label{eq:grav-red-shift}
\end{equation}
где $M$~--- масса гравитирующего тела, $R$~--- радиальное расстояние от центра масс тела до точки излучения (радиус источника), $r$~---  радиальное расстояние от центра масс источника до точки наблюдения. В случае, когда наблюдатель находится от источника много дальше его радиуса, т.\,е. выполняется соотношение $r \gg R$, выражение~\eqref{eq:grav-red-shift} можно упростить до
\begin{equation}
z_G \simeq \dfrac{GM}{c^2 R}.
\end{equation}

\input{sections/astrophys.light-pressure.tex}
\input{sections/astrophys.edd.tex}
\input{sections/astrophys.grav-lens.tex}
\subsection{Закон Хаббла}
\term{Закон Хаббла}~--- эмпирический закон, связывающий скорость удаления галактик $V$ и расстояние $R$ до них линейным образом: 
\begin{equation}
	V = H R,
\end{equation}
величина $H=68~\text{км/c} \cdot \text{Мпк})$ называется \imp{постоянной Хаббла}.

При $v \ll c$ можно использовать приближение эффекта Доплера, тогда
\begin{equation}
	V = c z.
\label{eq:hubble-speed}
\end{equation}

Равенство \eqref{eq:hubble-speed} справедливо только при $z \ll 1$, а при б\'{o}льших значениях $z$ космологическое красное смещение нльзя связывать с эффектом Доплера, поэтому можно пользоваться только формулой 
\begin{equation}
	\frac{dz}{dt} = - H(z)(1+z),
\end{equation}
где постоянная Хаббла введена как функция красного смещения.
\subsection{Шкала электромагнитных волн}


\term{Гамма излучение} возникает при радиоактивных распадах ядер, при торможении электронов энергией более $10^5$~эВ и при других взаимодействиях элементарных частиц. Используются в гамма-дефектоскопии, при изучении свойств вещества.

\term{Рентгеновские лучи} излучаются при большом ускорении электронов, например при их торможении в металлах. Получают их при помощи рентгеновской трубки: электроны в вакуумной трубке ускоряются электрическим полем при высоком напряжении, достигая анода, при со­ударении резко тормозятся. При торможении электроны движут­ся с ускорением и излучают электромагнитные волны с малой длиной. 

\begin{figure}[!h]
\centering
\includegraphics[width = 1\textwidth]{scale-wave.pdf}
\caption{Шкала электромагнитных волн}
\end{figure}
\term{Ультрафиолетовые лучи}~--- излучение Солнца, ртутных ламп и т.\,п. Используются в ультрафиолетовой микроскопии, в медицине.

\term{Видимое излучение}~--- часть электромагнитного излучения, воспринимаемая глазом (от фиолетового до от красного).

\term{Инфракрасное излучение}~--- тепловое, излучается любым нагретым телом.

\term{Радиоволны} используются повсеместно в обычной жизни, это и сотовая связь, и радиолокация, и спутниковая связь, и Wi-Fi и многое другое.

\term{Низкочастотные волны}~--- диапазон, традиционно используемый в электротехнике. В промышленной электроэнергетике используется частота 50~Гц, на~которой осуществляется передача электрической энергии по линиям и преобразование напряжений трансформаторными устройствами.
\input{sections/astrophys.spec-theor-rel.tex}
\subsection{Оптическая толщина. Закон Бугера}
\term{Оптическая толщина}~--- безразмерная величина, характеризующая степень непрозрачности среды для проходящего сквозь неё излучения,
\begin{equation}
\tau = \int n(x) \sigma(x)\,dx,
\end{equation}
где $\tau$~--- оптическая толщина среды, $n$~--- концентрация частиц, $\sigma$~--- сечение их взаимодействия.

Поток $I_0$ на входе связан с потоком $I$ на выходе \term{Законом Бугера}:
\begin{equation}
I = I_0 e^{-\tau}.
\end{equation}
\input{sections/astrophys.colour.tex}
\input{sections/astrophys.mkt.tex}
\input{sections/astrophys.earth-atmosphere.tex}

\newpage
\section{Астрофизика}
\subsection{Звёздные величины}
Звёздная величина~--- безразмерная числовая характеристика яркости объекта. Известно, что увеличению светового потока в 100 раз соответствует уменьшение видимой звёздной величины ровно на 5 единиц. Тогда уменьшение звёздной величины на одну единицу означает увеличение светового потока в $\sqrt[5]{100}\approx 2.512$~раз, то есть звёздные величины являются логарифмической шкалой измерения плотности потока. Зависимость, связывающая отношение освещённостей $E_1$ и $E_2$ и разность звёздных величин $m_1$ и $m_2$ двух объектов, называется \term{формулой Погсона} и имеет вид
\begin{equation}
	\frac{E_1}{E_2} = 10^{0.4(m_2 - m_1)} \quad \Longleftrightarrow \quad m_2 = m_1 + 2.5 \lg \frac{E_1}{E_2}.
	\label{eq:Pogson-law}
\end{equation}
Широко используется понятие \term{абсолютной звёздной величины} $M$~--- это видимая звёздная величина $m$ при наблюдении с установленного расстояния: для звёзд~---~10~пк, для тел Солнечной системы~---~1~\au, причем считается, что тело находится в 1~\au~и от наблюдателя и от Солнца, а фаза равна единице, то есть можно считать, что наблюдатель находится в центре Солнца, а~тело~--- в~1~\au~от него. 

Кроме этого, важно понятие \term{болометрической звёздной величины} $m_\text{bol}$~--- это звёздная величина, при расчёте которой учитывается полная мощность излучения источника во всех диапазонах электромагнитных волн. Обычная (видимая) звёздная величина учитывает излучение лишь в видимой части спектра от примерно 380~нм до примерно~780~нм. Разность между болометрической и видимой звёздными величинами называется \term{болометрической поправкой} ($BC$), которая отличается для разных спектральных классов звёзд. Из определения, болометрическая поправка может быть найдена по формуле
\begin{equation}
	BC = m_\text{bol} - m.
\end{equation}
Абсолютную звёздную величину звезды можно получить по формуле Погсона \eqref{eq:Pogson-law} из видимой звёздной величины $m$ и расстояния $r$ до неё в парсеках
\begin{equation}
	M = m + 2.5 \lg \frac{E}{E_\text{абс}} = m + 2.5 \lg \frac{(10~\text{пк})^2}{r^2} = m + 5 - 5\lg r.
	\label{eq:abs-mag}
\end{equation} 
Если принимать к рассмотрению межзвездное поглощение $A$, то формулу  \eqref{eq:abs-mag} необходимо уточнить:
\begin{equation}
	M = m + 5 - 5\lg r - Ar.
\end{equation}
\subsection{Закон Стефана-Больцмана}
\term{Закон Стефана~--- Больцмана} определяет зависимость плотности мощности излучения абсолютно чёрного тела (АЧТ) $u$ от его температуры $T$:
\begin{equation}
u = a T^4,
\end{equation} 
где $a$~--- некая универсальная константа.
Отсюда полная светимость АЧТ с площадью поверхности $S$
	\begin{equation}
	L = S \sigma T^4,
	\label{eq:steff-bol-law}
\end{equation}
константа $\sigma$ называется \term{постоянной Стефана-Больцмана}.
  
Важно отметить, что \imp{закон Стефана-Больцмана}~--- прямое следствие формулы Планка \eqref{Planck's formula}, так как
\begin{equation}
	\sigma T^4 = \int\limits^\infty_0 B(\lambda, T)\,d\lambda \int\limits_0^{\pi/2} \sin \varphi\, d\varphi \int\limits_0^{2\pi} \cos \varphi\, d\theta = \pi \int\limits^\infty_0 B(\lambda, T)\,d\lambda,
\end{equation}
откуда $\sigma = (2\pi^5k^4)/(15c^2h^3) = 5.67 \cdot 10^{-8}~\text{Вт}/(\text{м}^2\cdot \text{К}^4)$.

%Для АЧТ сферической формы с радиусом $R$ формула~\eqref{eq:steff-bol-law} принимает вид
%\begin{equation}
%L=4\pi R^2\sigma T^4.
%\end{equation}
Для звёзд главной последовательности выполняется соотношение $L \sim M^{\alpha}$, где~$\alpha$~--- коэффициент пропорциональности, который зависит от массы звезды следующим образом:
\begin{align*}
\alpha &= 2.5, \quad M < 0.43 M_\odot; & 
\alpha &= 4, \quad 0.43 M_\odot < M < 2 M_\odot;\\ 
\alpha &= 3.2, \quad 2 M_\odot < M < 20 M_\odot; & 
\alpha &= 1, \quad M > 20 M_\odot.
\end{align*}
Также существует примерная зависимость светимости звёзды от её радиуса, имеющая вид  $L\sim R^{5.2}$.
\subsection{Энергия излучения}
\term{Энергия излучения}~--- энергия, переносимая излучением ($Q_e$).\\
\term{Поток излучения}~--- физическая величина, характеризующая мощность, переносимую излучением,
\begin{equation}
 \Phi_e = \frac{d Q_e}{dt}.
\end{equation}
\imp{Теорема Гаусса}: через любую замкнутую поверхность потоки от одинаковых источников равны.

\term{Спектральная плотность потока излучения}~--- поток излучения, приходящийся на малый единичный интервал спектра,
\begin{equation}
\Phi_{e, \lambda}(\lambda) = \frac{d\Phi_e(\lambda)}{d\lambda}, \quad\quad \Phi_{e, \nu}(\nu) = \frac{d\Phi_e(\nu)}{d\nu} =  \frac{\lambda^2}{c}\Phi_{e, \lambda}(\lambda).
\end{equation}

\term{Объемная плотность энергии излучения}~--- количество энергии на единицу объема
\begin{equation}
U_e = \frac{d Q_e}{dV}.
\end{equation}

\term{Светимость}~--- величина, представляющая собой световой поток излучения, испускаемого с малого участка светящейся поверхности единичной площади,
\begin{equation}
M_e = \frac{d \Phi_e}{dS_1},
\end{equation}
здесь $S_1$~--- площадь объекта, испускающего энергию.

\term{Яркость}~--- световой поток, приходящийся на единичный телесный угол, в расчёте на единичную площадку проекции излучающей поверхности на картинную плоскость, 
\begin{equation}
L_e = \frac{d^2 \Phi_e}{d \Omega\,dS_1 \cos \varepsilon},
\end{equation}
где $\varepsilon$~--- угол между направлением потока излучения и нормалью к плоскости излучающей поверхности.

\term{Интегральная яркость}~--- интеграл яркости по видимой поверхности источника. Показывает количество энергии, пришедшее от источника за единицу времени.
\begin{equation}
\Lambda_e = \int \limits_S L_e(\vec{r})\,ds.
\end{equation}
\term{Освещенность}~--- величина, равная отношению светового потока, падающего на малый участок поверхности, к его площади~--- поверхностная плотность потока
\begin{equation}
E_e = \frac{d\Phi_e}{dS_2} \sim \frac{1}{r^2},
\end{equation}
здесь $S_2$~--- площадь поверхности приёмника, $r$~--- расстояние от источника.
\input{sections/astrophys.flux-albedo.tex}
\input{sections/astrophys.photon.tex}
\input{sections/astrophys.energy-lines.tex}
\subsection{Формула Планка}
\label{sec:planck-law}
\term{Формула Планка}~--- выражение для спектральной плотности мощности излучения абсолютно чёрного тела на интервале частот $[\nu, \nu + d \nu)$, распространяющейся с телесном угле $d\Omega$, которое было получено Максом Планком в 1900~году. Данное выражение имеет следующий вид:
\begin{equation}
B_\nu(\nu,T)=\frac{2h\nu^3}{c^2}\cdot \frac{1}{\exp\left(\frac{h\nu}{kT}\right)-1} = \left[ \frac{\text{Вт}}{\text{м}^2 \cdot \text{Гц} \cdot \text{ср}}\right],
\label{eq:plancks-law-nu}
\end{equation}
где $\nu$~--- частота излучения, $T$~--- температура АЧТ, $h$~--- постоянная Планка, $k$~--- постоянная Больцмана, $c$~--- скорость света.

Если записать закон излучения Планка \eqref{eq:plancks-law-nu} для длин волн, то
\begin{equation}
B_\lambda(\lambda,T)=\frac{2hc^2}{\lambda^5} \cdot \frac{1}{\exp\left(\frac{hc}{\lambda kT}\right)-1} = \left[ \frac{\text{Вт}}{\text{м}^3 \cdot \text{ср}}\right].
\label{eq:plancks-law-lambda}
\end{equation}
\begin{wrapfigure}[15]{l}{.6\tw}
\centering
\vspace{-.9pc}
 \begin{tikzpicture}
  \begin{axis}[
  				width 	=	.6\tw, 
				height	=	6cm, 
  				ymax	=	1e+14,
  				xmax	=	2000,
  				xmin	=	0,
  				ymin	=	0,
				xlabel	=	{Длина волны $\lambda$,~нм}, 
				ylabel 	= 	{$B_\lambda(\lambda, T)$,~$\text{Вт} \cdot \text{м}^{-3}$}
]
   \addplot+[dashed, thin, black] table[x=l, y=tl] {data/planck.txt};
   \addplot+[black] table[x=l, y=t4] {data/planck.txt} node at (axis cs:870, 1.6e+13) {\tiny{$4500$~K}};
   \addplot+[black] table[x=l, y=t5] {data/planck.txt}node at (axis cs:750, 4.2e+13) {\tiny{$5000$~K}};
   \addplot+[black] table[x=l, y=t58] {data/planck.txt}node at (axis cs:670, 8.5e+13) {\tiny{$5800$~K}};
   \addplot+[black] table[x=l, y=t7] {data/planck.txt}node at (axis cs:1350, 3.5e+13) {\tiny{$7000$~K}};
	%\addplot+[black, smooth] table[x=l, y=t15] {data/planck.txt} node at (axis cs:1670, 5.5e+13) {\tiny{$15000$~K}};
  \end{axis}
 \end{tikzpicture}
\caption{Кривые спектральной плотности мощности изотропного излучения АЧТ с разной температурой}\label{pic:wien-law}
\end{wrapfigure}
Стоит заметить, что при переходе в функции к длинам волн меняется не только частота на длину волны, но и выражение для интервала. 

Формула Планка появилась, когда стало ясно, что формула Рэлея-Джинса удовлетворительно описывает излучение только в области больших длин волн, а~с~убыванием длин волн даёт сильные расхождения с реальными данными. Однако формулу Рэлея-Джинса используют и сейчас для описания кривой Планка на больших длинах волн. 

\change{
Проделаем обратные действия: получим формулу Рэлея-Джинса из формулы Планка. Длинноволновая часть спектра характеризуется соотношением $h\nu \ll kT$, то есть 
\begin{equation*}
	\exp\left( \frac{h\nu}{kT}\right) \approx 1 + \frac{h\nu}{kT}.
\end{equation*}
Подставляя полученное выражение в знаменатель \eqref{eq:plancks-law-nu}, получим
\begin{equation*}
	B_\nu(\nu,T) \approx \frac{2h\nu^3}{c^2}\cdot \frac{1}{1 + \frac{h\nu}{kT} - 1} = \frac{2h\nu^3 }{c^2}\cdot \frac{k T}{ h \nu} = \frac{2 \nu^2 k T}{c^2}.
\end{equation*}
}
\change{
	Проделав то же самое для выражения через длину волны, получим:
}
\begin{equation}
	B(\lambda, T) \simeq \frac{2 c k T}{\lambda^4}, \quad\quad B(\nu, T) \simeq \frac{2 \nu^2 k T}{c^2}.
\label{Ray-Jean}
\end{equation}

\change{
	В коротковолновой области, наоборот, $h \nu \gg kT$, следовательно, в знаменателе формулы Планка единица много меньше стоящей там экспоненты, то есть
	\begin{equation*}
		\frac{1}{\exp\left(\frac{h\nu}{kT}\right)-1} \approx \frac{1}{\exp\left(\frac{h\nu}{kT}\right)} = \exp\left(-\frac{h\nu}{kT}\right).
	\end{equation*} 
	Отсюда получаются выражения, называемые приближением Вина:
}
\begin{equation}
B ( \lambda, T) \simeq \frac{2 h c^2}{\lambda^5} \exp \left( -\frac{h c}{\lambda k T}\right), \quad \quad B( \nu, T ) \simeq \frac{2 h \nu^3}{c^2} \exp \left( -\frac{h \nu}{k T} \right).
\end{equation}
\subsection{Закон смещения Вина}
\term{Закон смещения Вина} --- закон, устанавливающий зависимость длины волны~$\lambda_\text{макс}$, на которой спектральная плотность излучения $B_\lambda(\lambda, T)$ абсолютно чёрного тела достигает своего максимума, от температуры $T$ этого тела:
\begin{equation}
	\lambda_\text{макс} \approx \frac{b}{T} \equiv \frac{0.0029~\text{м} \cdot \text{К}}{T}.
\end{equation}
Закон является следствием исследования функции Планка (см.~\ref{sec:planck-law}) на экстремальность.
\subsection{Эффект Доплера. Красное смещение}
\term{Эффект Доплера}~--- эффект изменения частоты и длины волны электромагнитного излучения, регистрируемого приёмником, вызванный относительным движением источника и приёмника (см.~Рис.\,\ref{doppler-ef}).

При $\Delta \lambda \ll \lambda_0$ с большой точностью выполняется следующее важное соотношение:\begin{equation}
\beta \equiv \dfrac{v}{c} = \dfrac{\lambda - \lambda_0}{\lambda_0} \equiv \dfrac{\Delta \lambda}{\lambda_0},
\label{eq:dopler-ef-simple}
\end{equation}
\begin{wrapfigure}[6]{r}{0.5\tw}
\centering
\vspace{-.5pc}
\includegraphics[width=.5\tw]{doppler-ef}
\caption{Эффект Доплера}
\label{doppler-ef}
\end{wrapfigure}
где $\lambda_0$~--- лабораторная длина волны излучения источника, а $\lambda$~--- наблюдаемая. В действительности же имеет место более общий случай: \imp{релятивистский эффект Доплера}, обусловленный проявлением СТО при $v \simeq c$, для которого формула~\eqref{eq:dopler-ef-simple} усложняется и принимает вид
\begin{equation}
\nu = \nu_0 \cdot \dfrac{\sqrt{1 - \beta^2}}{1 + \beta \cdot \cos\theta},
\label{eq:dopler-ef-rel}
\end{equation}
где $\nu$~--- частота, с которой наблюдатель принимает волны, $\nu_0$~--- частота, с которой источник испускает волны, $v$~--- скорость источника, $\theta$~--- угол между направлением на источник и вектором его скорости в системе отсчёта приёмника. Если источник радиально удаляется от наблюдателя, то $\theta = 0$, если приближается, то $\theta =\pi$. Важно, что~\eqref{eq:dopler-ef-simple} напрямую следует из \eqref{eq:dopler-ef-rel} при $\beta  \ll 1$.

\term{Красное смещение}~--- явление сдвига спектральных линий химических элементов в красную (длинноволновую) сторону, обусловленное относительным движение объектов. Параметр красного смещения определяется из наблюдаемой и лабораторной длин волн как
\begin{equation}
z = \dfrac{\lambda - \lambda_0}{\lambda_0}.
\end{equation}

Доплеровское смещение длины волны в спектре источника, движущегося с лучевой скоростью $v_{r}$ и полной скоростью $v$,
\begin{equation}
z = \dfrac{1 + v_r / c}{\sqrt{1 - \beta^2}}.
\end{equation}

\term{Гравитационное красное смещение}~--- проявление эффекта изменения частоты излучения, испущенного массивным объектом, таким как звезда или чёрная дыра. Наблюдается как сдвиг спектральных линий в спектре источника в красную область спектра. Гравитационное красное смещение определяется из формулы, выведенной Эйнштейном,
\begin{equation}
z_G=\dfrac{GM}{c^2 R}-\dfrac{GM}{c^2 r},
\label{eq:grav-red-shift}
\end{equation}
где $M$~--- масса гравитирующего тела, $R$~--- радиальное расстояние от центра масс тела до точки излучения (радиус источника), $r$~---  радиальное расстояние от центра масс источника до точки наблюдения. В случае, когда наблюдатель находится от источника много дальше его радиуса, т.\,е. выполняется соотношение $r \gg R$, выражение~\eqref{eq:grav-red-shift} можно упростить до
\begin{equation}
z_G \simeq \dfrac{GM}{c^2 R}.
\end{equation}

\input{sections/astrophys.light-pressure.tex}
\input{sections/astrophys.edd.tex}
\input{sections/astrophys.grav-lens.tex}
\subsection{Закон Хаббла}
\term{Закон Хаббла}~--- эмпирический закон, связывающий скорость удаления галактик $V$ и расстояние $R$ до них линейным образом: 
\begin{equation}
	V = H R,
\end{equation}
величина $H=68~\text{км/c} \cdot \text{Мпк})$ называется \imp{постоянной Хаббла}.

При $v \ll c$ можно использовать приближение эффекта Доплера, тогда
\begin{equation}
	V = c z.
\label{eq:hubble-speed}
\end{equation}

Равенство \eqref{eq:hubble-speed} справедливо только при $z \ll 1$, а при б\'{o}льших значениях $z$ космологическое красное смещение нльзя связывать с эффектом Доплера, поэтому можно пользоваться только формулой 
\begin{equation}
	\frac{dz}{dt} = - H(z)(1+z),
\end{equation}
где постоянная Хаббла введена как функция красного смещения.
\subsection{Шкала электромагнитных волн}


\term{Гамма излучение} возникает при радиоактивных распадах ядер, при торможении электронов энергией более $10^5$~эВ и при других взаимодействиях элементарных частиц. Используются в гамма-дефектоскопии, при изучении свойств вещества.

\term{Рентгеновские лучи} излучаются при большом ускорении электронов, например при их торможении в металлах. Получают их при помощи рентгеновской трубки: электроны в вакуумной трубке ускоряются электрическим полем при высоком напряжении, достигая анода, при со­ударении резко тормозятся. При торможении электроны движут­ся с ускорением и излучают электромагнитные волны с малой длиной. 

\begin{figure}[!h]
\centering
\includegraphics[width = 1\textwidth]{scale-wave.pdf}
\caption{Шкала электромагнитных волн}
\end{figure}
\term{Ультрафиолетовые лучи}~--- излучение Солнца, ртутных ламп и т.\,п. Используются в ультрафиолетовой микроскопии, в медицине.

\term{Видимое излучение}~--- часть электромагнитного излучения, воспринимаемая глазом (от фиолетового до от красного).

\term{Инфракрасное излучение}~--- тепловое, излучается любым нагретым телом.

\term{Радиоволны} используются повсеместно в обычной жизни, это и сотовая связь, и радиолокация, и спутниковая связь, и Wi-Fi и многое другое.

\term{Низкочастотные волны}~--- диапазон, традиционно используемый в электротехнике. В промышленной электроэнергетике используется частота 50~Гц, на~которой осуществляется передача электрической энергии по линиям и преобразование напряжений трансформаторными устройствами.
\input{sections/astrophys.spec-theor-rel.tex}
\subsection{Оптическая толщина. Закон Бугера}
\term{Оптическая толщина}~--- безразмерная величина, характеризующая степень непрозрачности среды для проходящего сквозь неё излучения,
\begin{equation}
\tau = \int n(x) \sigma(x)\,dx,
\end{equation}
где $\tau$~--- оптическая толщина среды, $n$~--- концентрация частиц, $\sigma$~--- сечение их взаимодействия.

Поток $I_0$ на входе связан с потоком $I$ на выходе \term{Законом Бугера}:
\begin{equation}
I = I_0 e^{-\tau}.
\end{equation}
\input{sections/astrophys.colour.tex}
\input{sections/astrophys.mkt.tex}
\input{sections/astrophys.earth-atmosphere.tex}

\subsection{Закон Хаббла}
\term{Закон Хаббла}~--- эмпирический закон, связывающий скорость удаления галактик $V$ и расстояние $R$ до них линейным образом: 
\begin{equation}
	V = H R,
\end{equation}
величина $H=68~\text{км/c} \cdot \text{Мпк})$ называется \imp{постоянной Хаббла}.

При $v \ll c$ можно использовать приближение эффекта Доплера, тогда
\begin{equation}
	V = c z.
\label{eq:hubble-speed}
\end{equation}

Равенство \eqref{eq:hubble-speed} справедливо только при $z \ll 1$, а при б\'{o}льших значениях $z$ космологическое красное смещение нльзя связывать с эффектом Доплера, поэтому можно пользоваться только формулой 
\begin{equation}
	\frac{dz}{dt} = - H(z)(1+z),
\end{equation}
где постоянная Хаббла введена как функция красного смещения.
\subsection{Шкала электромагнитных волн}


\term{Гамма излучение} возникает при радиоактивных распадах ядер, при торможении электронов энергией более $10^5$~эВ и при других взаимодействиях элементарных частиц. Используются в гамма-дефектоскопии, при изучении свойств вещества.

\term{Рентгеновские лучи} излучаются при большом ускорении электронов, например при их торможении в металлах. Получают их при помощи рентгеновской трубки: электроны в вакуумной трубке ускоряются электрическим полем при высоком напряжении, достигая анода, при со­ударении резко тормозятся. При торможении электроны движут­ся с ускорением и излучают электромагнитные волны с малой длиной. 

\begin{figure}[!h]
\centering
\includegraphics[width = 1\textwidth]{scale-wave.pdf}
\caption{Шкала электромагнитных волн}
\end{figure}
\term{Ультрафиолетовые лучи}~--- излучение Солнца, ртутных ламп и т.\,п. Используются в ультрафиолетовой микроскопии, в медицине.

\term{Видимое излучение}~--- часть электромагнитного излучения, воспринимаемая глазом (от фиолетового до от красного).

\term{Инфракрасное излучение}~--- тепловое, излучается любым нагретым телом.

\term{Радиоволны} используются повсеместно в обычной жизни, это и сотовая связь, и радиолокация, и спутниковая связь, и Wi-Fi и многое другое.

\term{Низкочастотные волны}~--- диапазон, традиционно используемый в электротехнике. В промышленной электроэнергетике используется частота 50~Гц, на~которой осуществляется передача электрической энергии по линиям и преобразование напряжений трансформаторными устройствами.
\newpage
\section{Астрофизика}
\subsection{Звёздные величины}
Звёздная величина~--- безразмерная числовая характеристика яркости объекта. Известно, что увеличению светового потока в 100 раз соответствует уменьшение видимой звёздной величины ровно на 5 единиц. Тогда уменьшение звёздной величины на одну единицу означает увеличение светового потока в $\sqrt[5]{100}\approx 2.512$~раз, то есть звёздные величины являются логарифмической шкалой измерения плотности потока. Зависимость, связывающая отношение освещённостей $E_1$ и $E_2$ и разность звёздных величин $m_1$ и $m_2$ двух объектов, называется \term{формулой Погсона} и имеет вид
\begin{equation}
	\frac{E_1}{E_2} = 10^{0.4(m_2 - m_1)} \quad \Longleftrightarrow \quad m_2 = m_1 + 2.5 \lg \frac{E_1}{E_2}.
	\label{eq:Pogson-law}
\end{equation}
Широко используется понятие \term{абсолютной звёздной величины} $M$~--- это видимая звёздная величина $m$ при наблюдении с установленного расстояния: для звёзд~---~10~пк, для тел Солнечной системы~---~1~\au, причем считается, что тело находится в 1~\au~и от наблюдателя и от Солнца, а фаза равна единице, то есть можно считать, что наблюдатель находится в центре Солнца, а~тело~--- в~1~\au~от него. 

Кроме этого, важно понятие \term{болометрической звёздной величины} $m_\text{bol}$~--- это звёздная величина, при расчёте которой учитывается полная мощность излучения источника во всех диапазонах электромагнитных волн. Обычная (видимая) звёздная величина учитывает излучение лишь в видимой части спектра от примерно 380~нм до примерно~780~нм. Разность между болометрической и видимой звёздными величинами называется \term{болометрической поправкой} ($BC$), которая отличается для разных спектральных классов звёзд. Из определения, болометрическая поправка может быть найдена по формуле
\begin{equation}
	BC = m_\text{bol} - m.
\end{equation}
Абсолютную звёздную величину звезды можно получить по формуле Погсона \eqref{eq:Pogson-law} из видимой звёздной величины $m$ и расстояния $r$ до неё в парсеках
\begin{equation}
	M = m + 2.5 \lg \frac{E}{E_\text{абс}} = m + 2.5 \lg \frac{(10~\text{пк})^2}{r^2} = m + 5 - 5\lg r.
	\label{eq:abs-mag}
\end{equation} 
Если принимать к рассмотрению межзвездное поглощение $A$, то формулу  \eqref{eq:abs-mag} необходимо уточнить:
\begin{equation}
	M = m + 5 - 5\lg r - Ar.
\end{equation}
\subsection{Закон Стефана-Больцмана}
\term{Закон Стефана~--- Больцмана} определяет зависимость плотности мощности излучения абсолютно чёрного тела (АЧТ) $u$ от его температуры $T$:
\begin{equation}
u = a T^4,
\end{equation} 
где $a$~--- некая универсальная константа.
Отсюда полная светимость АЧТ с площадью поверхности $S$
	\begin{equation}
	L = S \sigma T^4,
	\label{eq:steff-bol-law}
\end{equation}
константа $\sigma$ называется \term{постоянной Стефана-Больцмана}.
  
Важно отметить, что \imp{закон Стефана-Больцмана}~--- прямое следствие формулы Планка \eqref{Planck's formula}, так как
\begin{equation}
	\sigma T^4 = \int\limits^\infty_0 B(\lambda, T)\,d\lambda \int\limits_0^{\pi/2} \sin \varphi\, d\varphi \int\limits_0^{2\pi} \cos \varphi\, d\theta = \pi \int\limits^\infty_0 B(\lambda, T)\,d\lambda,
\end{equation}
откуда $\sigma = (2\pi^5k^4)/(15c^2h^3) = 5.67 \cdot 10^{-8}~\text{Вт}/(\text{м}^2\cdot \text{К}^4)$.

%Для АЧТ сферической формы с радиусом $R$ формула~\eqref{eq:steff-bol-law} принимает вид
%\begin{equation}
%L=4\pi R^2\sigma T^4.
%\end{equation}
Для звёзд главной последовательности выполняется соотношение $L \sim M^{\alpha}$, где~$\alpha$~--- коэффициент пропорциональности, который зависит от массы звезды следующим образом:
\begin{align*}
\alpha &= 2.5, \quad M < 0.43 M_\odot; & 
\alpha &= 4, \quad 0.43 M_\odot < M < 2 M_\odot;\\ 
\alpha &= 3.2, \quad 2 M_\odot < M < 20 M_\odot; & 
\alpha &= 1, \quad M > 20 M_\odot.
\end{align*}
Также существует примерная зависимость светимости звёзды от её радиуса, имеющая вид  $L\sim R^{5.2}$.
\subsection{Энергия излучения}
\term{Энергия излучения}~--- энергия, переносимая излучением ($Q_e$).\\
\term{Поток излучения}~--- физическая величина, характеризующая мощность, переносимую излучением,
\begin{equation}
 \Phi_e = \frac{d Q_e}{dt}.
\end{equation}
\imp{Теорема Гаусса}: через любую замкнутую поверхность потоки от одинаковых источников равны.

\term{Спектральная плотность потока излучения}~--- поток излучения, приходящийся на малый единичный интервал спектра,
\begin{equation}
\Phi_{e, \lambda}(\lambda) = \frac{d\Phi_e(\lambda)}{d\lambda}, \quad\quad \Phi_{e, \nu}(\nu) = \frac{d\Phi_e(\nu)}{d\nu} =  \frac{\lambda^2}{c}\Phi_{e, \lambda}(\lambda).
\end{equation}

\term{Объемная плотность энергии излучения}~--- количество энергии на единицу объема
\begin{equation}
U_e = \frac{d Q_e}{dV}.
\end{equation}

\term{Светимость}~--- величина, представляющая собой световой поток излучения, испускаемого с малого участка светящейся поверхности единичной площади,
\begin{equation}
M_e = \frac{d \Phi_e}{dS_1},
\end{equation}
здесь $S_1$~--- площадь объекта, испускающего энергию.

\term{Яркость}~--- световой поток, приходящийся на единичный телесный угол, в расчёте на единичную площадку проекции излучающей поверхности на картинную плоскость, 
\begin{equation}
L_e = \frac{d^2 \Phi_e}{d \Omega\,dS_1 \cos \varepsilon},
\end{equation}
где $\varepsilon$~--- угол между направлением потока излучения и нормалью к плоскости излучающей поверхности.

\term{Интегральная яркость}~--- интеграл яркости по видимой поверхности источника. Показывает количество энергии, пришедшее от источника за единицу времени.
\begin{equation}
\Lambda_e = \int \limits_S L_e(\vec{r})\,ds.
\end{equation}
\term{Освещенность}~--- величина, равная отношению светового потока, падающего на малый участок поверхности, к его площади~--- поверхностная плотность потока
\begin{equation}
E_e = \frac{d\Phi_e}{dS_2} \sim \frac{1}{r^2},
\end{equation}
здесь $S_2$~--- площадь поверхности приёмника, $r$~--- расстояние от источника.
\input{sections/astrophys.flux-albedo.tex}
\input{sections/astrophys.photon.tex}
\input{sections/astrophys.energy-lines.tex}
\subsection{Формула Планка}
\label{sec:planck-law}
\term{Формула Планка}~--- выражение для спектральной плотности мощности излучения абсолютно чёрного тела на интервале частот $[\nu, \nu + d \nu)$, распространяющейся с телесном угле $d\Omega$, которое было получено Максом Планком в 1900~году. Данное выражение имеет следующий вид:
\begin{equation}
B_\nu(\nu,T)=\frac{2h\nu^3}{c^2}\cdot \frac{1}{\exp\left(\frac{h\nu}{kT}\right)-1} = \left[ \frac{\text{Вт}}{\text{м}^2 \cdot \text{Гц} \cdot \text{ср}}\right],
\label{eq:plancks-law-nu}
\end{equation}
где $\nu$~--- частота излучения, $T$~--- температура АЧТ, $h$~--- постоянная Планка, $k$~--- постоянная Больцмана, $c$~--- скорость света.

Если записать закон излучения Планка \eqref{eq:plancks-law-nu} для длин волн, то
\begin{equation}
B_\lambda(\lambda,T)=\frac{2hc^2}{\lambda^5} \cdot \frac{1}{\exp\left(\frac{hc}{\lambda kT}\right)-1} = \left[ \frac{\text{Вт}}{\text{м}^3 \cdot \text{ср}}\right].
\label{eq:plancks-law-lambda}
\end{equation}
\begin{wrapfigure}[15]{l}{.6\tw}
\centering
\vspace{-.9pc}
 \begin{tikzpicture}
  \begin{axis}[
  				width 	=	.6\tw, 
				height	=	6cm, 
  				ymax	=	1e+14,
  				xmax	=	2000,
  				xmin	=	0,
  				ymin	=	0,
				xlabel	=	{Длина волны $\lambda$,~нм}, 
				ylabel 	= 	{$B_\lambda(\lambda, T)$,~$\text{Вт} \cdot \text{м}^{-3}$}
]
   \addplot+[dashed, thin, black] table[x=l, y=tl] {data/planck.txt};
   \addplot+[black] table[x=l, y=t4] {data/planck.txt} node at (axis cs:870, 1.6e+13) {\tiny{$4500$~K}};
   \addplot+[black] table[x=l, y=t5] {data/planck.txt}node at (axis cs:750, 4.2e+13) {\tiny{$5000$~K}};
   \addplot+[black] table[x=l, y=t58] {data/planck.txt}node at (axis cs:670, 8.5e+13) {\tiny{$5800$~K}};
   \addplot+[black] table[x=l, y=t7] {data/planck.txt}node at (axis cs:1350, 3.5e+13) {\tiny{$7000$~K}};
	%\addplot+[black, smooth] table[x=l, y=t15] {data/planck.txt} node at (axis cs:1670, 5.5e+13) {\tiny{$15000$~K}};
  \end{axis}
 \end{tikzpicture}
\caption{Кривые спектральной плотности мощности изотропного излучения АЧТ с разной температурой}\label{pic:wien-law}
\end{wrapfigure}
Стоит заметить, что при переходе в функции к длинам волн меняется не только частота на длину волны, но и выражение для интервала. 

Формула Планка появилась, когда стало ясно, что формула Рэлея-Джинса удовлетворительно описывает излучение только в области больших длин волн, а~с~убыванием длин волн даёт сильные расхождения с реальными данными. Однако формулу Рэлея-Джинса используют и сейчас для описания кривой Планка на больших длинах волн. 

\change{
Проделаем обратные действия: получим формулу Рэлея-Джинса из формулы Планка. Длинноволновая часть спектра характеризуется соотношением $h\nu \ll kT$, то есть 
\begin{equation*}
	\exp\left( \frac{h\nu}{kT}\right) \approx 1 + \frac{h\nu}{kT}.
\end{equation*}
Подставляя полученное выражение в знаменатель \eqref{eq:plancks-law-nu}, получим
\begin{equation*}
	B_\nu(\nu,T) \approx \frac{2h\nu^3}{c^2}\cdot \frac{1}{1 + \frac{h\nu}{kT} - 1} = \frac{2h\nu^3 }{c^2}\cdot \frac{k T}{ h \nu} = \frac{2 \nu^2 k T}{c^2}.
\end{equation*}
}
\change{
	Проделав то же самое для выражения через длину волны, получим:
}
\begin{equation}
	B(\lambda, T) \simeq \frac{2 c k T}{\lambda^4}, \quad\quad B(\nu, T) \simeq \frac{2 \nu^2 k T}{c^2}.
\label{Ray-Jean}
\end{equation}

\change{
	В коротковолновой области, наоборот, $h \nu \gg kT$, следовательно, в знаменателе формулы Планка единица много меньше стоящей там экспоненты, то есть
	\begin{equation*}
		\frac{1}{\exp\left(\frac{h\nu}{kT}\right)-1} \approx \frac{1}{\exp\left(\frac{h\nu}{kT}\right)} = \exp\left(-\frac{h\nu}{kT}\right).
	\end{equation*} 
	Отсюда получаются выражения, называемые приближением Вина:
}
\begin{equation}
B ( \lambda, T) \simeq \frac{2 h c^2}{\lambda^5} \exp \left( -\frac{h c}{\lambda k T}\right), \quad \quad B( \nu, T ) \simeq \frac{2 h \nu^3}{c^2} \exp \left( -\frac{h \nu}{k T} \right).
\end{equation}
\subsection{Закон смещения Вина}
\term{Закон смещения Вина} --- закон, устанавливающий зависимость длины волны~$\lambda_\text{макс}$, на которой спектральная плотность излучения $B_\lambda(\lambda, T)$ абсолютно чёрного тела достигает своего максимума, от температуры $T$ этого тела:
\begin{equation}
	\lambda_\text{макс} \approx \frac{b}{T} \equiv \frac{0.0029~\text{м} \cdot \text{К}}{T}.
\end{equation}
Закон является следствием исследования функции Планка (см.~\ref{sec:planck-law}) на экстремальность.
\subsection{Эффект Доплера. Красное смещение}
\term{Эффект Доплера}~--- эффект изменения частоты и длины волны электромагнитного излучения, регистрируемого приёмником, вызванный относительным движением источника и приёмника (см.~Рис.\,\ref{doppler-ef}).

При $\Delta \lambda \ll \lambda_0$ с большой точностью выполняется следующее важное соотношение:\begin{equation}
\beta \equiv \dfrac{v}{c} = \dfrac{\lambda - \lambda_0}{\lambda_0} \equiv \dfrac{\Delta \lambda}{\lambda_0},
\label{eq:dopler-ef-simple}
\end{equation}
\begin{wrapfigure}[6]{r}{0.5\tw}
\centering
\vspace{-.5pc}
\includegraphics[width=.5\tw]{doppler-ef}
\caption{Эффект Доплера}
\label{doppler-ef}
\end{wrapfigure}
где $\lambda_0$~--- лабораторная длина волны излучения источника, а $\lambda$~--- наблюдаемая. В действительности же имеет место более общий случай: \imp{релятивистский эффект Доплера}, обусловленный проявлением СТО при $v \simeq c$, для которого формула~\eqref{eq:dopler-ef-simple} усложняется и принимает вид
\begin{equation}
\nu = \nu_0 \cdot \dfrac{\sqrt{1 - \beta^2}}{1 + \beta \cdot \cos\theta},
\label{eq:dopler-ef-rel}
\end{equation}
где $\nu$~--- частота, с которой наблюдатель принимает волны, $\nu_0$~--- частота, с которой источник испускает волны, $v$~--- скорость источника, $\theta$~--- угол между направлением на источник и вектором его скорости в системе отсчёта приёмника. Если источник радиально удаляется от наблюдателя, то $\theta = 0$, если приближается, то $\theta =\pi$. Важно, что~\eqref{eq:dopler-ef-simple} напрямую следует из \eqref{eq:dopler-ef-rel} при $\beta  \ll 1$.

\term{Красное смещение}~--- явление сдвига спектральных линий химических элементов в красную (длинноволновую) сторону, обусловленное относительным движение объектов. Параметр красного смещения определяется из наблюдаемой и лабораторной длин волн как
\begin{equation}
z = \dfrac{\lambda - \lambda_0}{\lambda_0}.
\end{equation}

Доплеровское смещение длины волны в спектре источника, движущегося с лучевой скоростью $v_{r}$ и полной скоростью $v$,
\begin{equation}
z = \dfrac{1 + v_r / c}{\sqrt{1 - \beta^2}}.
\end{equation}

\term{Гравитационное красное смещение}~--- проявление эффекта изменения частоты излучения, испущенного массивным объектом, таким как звезда или чёрная дыра. Наблюдается как сдвиг спектральных линий в спектре источника в красную область спектра. Гравитационное красное смещение определяется из формулы, выведенной Эйнштейном,
\begin{equation}
z_G=\dfrac{GM}{c^2 R}-\dfrac{GM}{c^2 r},
\label{eq:grav-red-shift}
\end{equation}
где $M$~--- масса гравитирующего тела, $R$~--- радиальное расстояние от центра масс тела до точки излучения (радиус источника), $r$~---  радиальное расстояние от центра масс источника до точки наблюдения. В случае, когда наблюдатель находится от источника много дальше его радиуса, т.\,е. выполняется соотношение $r \gg R$, выражение~\eqref{eq:grav-red-shift} можно упростить до
\begin{equation}
z_G \simeq \dfrac{GM}{c^2 R}.
\end{equation}

\input{sections/astrophys.light-pressure.tex}
\input{sections/astrophys.edd.tex}
\input{sections/astrophys.grav-lens.tex}
\subsection{Закон Хаббла}
\term{Закон Хаббла}~--- эмпирический закон, связывающий скорость удаления галактик $V$ и расстояние $R$ до них линейным образом: 
\begin{equation}
	V = H R,
\end{equation}
величина $H=68~\text{км/c} \cdot \text{Мпк})$ называется \imp{постоянной Хаббла}.

При $v \ll c$ можно использовать приближение эффекта Доплера, тогда
\begin{equation}
	V = c z.
\label{eq:hubble-speed}
\end{equation}

Равенство \eqref{eq:hubble-speed} справедливо только при $z \ll 1$, а при б\'{o}льших значениях $z$ космологическое красное смещение нльзя связывать с эффектом Доплера, поэтому можно пользоваться только формулой 
\begin{equation}
	\frac{dz}{dt} = - H(z)(1+z),
\end{equation}
где постоянная Хаббла введена как функция красного смещения.
\subsection{Шкала электромагнитных волн}


\term{Гамма излучение} возникает при радиоактивных распадах ядер, при торможении электронов энергией более $10^5$~эВ и при других взаимодействиях элементарных частиц. Используются в гамма-дефектоскопии, при изучении свойств вещества.

\term{Рентгеновские лучи} излучаются при большом ускорении электронов, например при их торможении в металлах. Получают их при помощи рентгеновской трубки: электроны в вакуумной трубке ускоряются электрическим полем при высоком напряжении, достигая анода, при со­ударении резко тормозятся. При торможении электроны движут­ся с ускорением и излучают электромагнитные волны с малой длиной. 

\begin{figure}[!h]
\centering
\includegraphics[width = 1\textwidth]{scale-wave.pdf}
\caption{Шкала электромагнитных волн}
\end{figure}
\term{Ультрафиолетовые лучи}~--- излучение Солнца, ртутных ламп и т.\,п. Используются в ультрафиолетовой микроскопии, в медицине.

\term{Видимое излучение}~--- часть электромагнитного излучения, воспринимаемая глазом (от фиолетового до от красного).

\term{Инфракрасное излучение}~--- тепловое, излучается любым нагретым телом.

\term{Радиоволны} используются повсеместно в обычной жизни, это и сотовая связь, и радиолокация, и спутниковая связь, и Wi-Fi и многое другое.

\term{Низкочастотные волны}~--- диапазон, традиционно используемый в электротехнике. В промышленной электроэнергетике используется частота 50~Гц, на~которой осуществляется передача электрической энергии по линиям и преобразование напряжений трансформаторными устройствами.
\input{sections/astrophys.spec-theor-rel.tex}
\subsection{Оптическая толщина. Закон Бугера}
\term{Оптическая толщина}~--- безразмерная величина, характеризующая степень непрозрачности среды для проходящего сквозь неё излучения,
\begin{equation}
\tau = \int n(x) \sigma(x)\,dx,
\end{equation}
где $\tau$~--- оптическая толщина среды, $n$~--- концентрация частиц, $\sigma$~--- сечение их взаимодействия.

Поток $I_0$ на входе связан с потоком $I$ на выходе \term{Законом Бугера}:
\begin{equation}
I = I_0 e^{-\tau}.
\end{equation}
\input{sections/astrophys.colour.tex}
\input{sections/astrophys.mkt.tex}
\input{sections/astrophys.earth-atmosphere.tex}

\subsection{Оптическая толщина. Закон Бугера}
\term{Оптическая толщина}~--- безразмерная величина, характеризующая степень непрозрачности среды для проходящего сквозь неё излучения,
\begin{equation}
\tau = \int n(x) \sigma(x)\,dx,
\end{equation}
где $\tau$~--- оптическая толщина среды, $n$~--- концентрация частиц, $\sigma$~--- сечение их взаимодействия.

Поток $I_0$ на входе связан с потоком $I$ на выходе \term{Законом Бугера}:
\begin{equation}
I = I_0 e^{-\tau}.
\end{equation}
\newpage
\section{Астрофизика}
\subsection{Звёздные величины}
Звёздная величина~--- безразмерная числовая характеристика яркости объекта. Известно, что увеличению светового потока в 100 раз соответствует уменьшение видимой звёздной величины ровно на 5 единиц. Тогда уменьшение звёздной величины на одну единицу означает увеличение светового потока в $\sqrt[5]{100}\approx 2.512$~раз, то есть звёздные величины являются логарифмической шкалой измерения плотности потока. Зависимость, связывающая отношение освещённостей $E_1$ и $E_2$ и разность звёздных величин $m_1$ и $m_2$ двух объектов, называется \term{формулой Погсона} и имеет вид
\begin{equation}
	\frac{E_1}{E_2} = 10^{0.4(m_2 - m_1)} \quad \Longleftrightarrow \quad m_2 = m_1 + 2.5 \lg \frac{E_1}{E_2}.
	\label{eq:Pogson-law}
\end{equation}
Широко используется понятие \term{абсолютной звёздной величины} $M$~--- это видимая звёздная величина $m$ при наблюдении с установленного расстояния: для звёзд~---~10~пк, для тел Солнечной системы~---~1~\au, причем считается, что тело находится в 1~\au~и от наблюдателя и от Солнца, а фаза равна единице, то есть можно считать, что наблюдатель находится в центре Солнца, а~тело~--- в~1~\au~от него. 

Кроме этого, важно понятие \term{болометрической звёздной величины} $m_\text{bol}$~--- это звёздная величина, при расчёте которой учитывается полная мощность излучения источника во всех диапазонах электромагнитных волн. Обычная (видимая) звёздная величина учитывает излучение лишь в видимой части спектра от примерно 380~нм до примерно~780~нм. Разность между болометрической и видимой звёздными величинами называется \term{болометрической поправкой} ($BC$), которая отличается для разных спектральных классов звёзд. Из определения, болометрическая поправка может быть найдена по формуле
\begin{equation}
	BC = m_\text{bol} - m.
\end{equation}
Абсолютную звёздную величину звезды можно получить по формуле Погсона \eqref{eq:Pogson-law} из видимой звёздной величины $m$ и расстояния $r$ до неё в парсеках
\begin{equation}
	M = m + 2.5 \lg \frac{E}{E_\text{абс}} = m + 2.5 \lg \frac{(10~\text{пк})^2}{r^2} = m + 5 - 5\lg r.
	\label{eq:abs-mag}
\end{equation} 
Если принимать к рассмотрению межзвездное поглощение $A$, то формулу  \eqref{eq:abs-mag} необходимо уточнить:
\begin{equation}
	M = m + 5 - 5\lg r - Ar.
\end{equation}
\subsection{Закон Стефана-Больцмана}
\term{Закон Стефана~--- Больцмана} определяет зависимость плотности мощности излучения абсолютно чёрного тела (АЧТ) $u$ от его температуры $T$:
\begin{equation}
u = a T^4,
\end{equation} 
где $a$~--- некая универсальная константа.
Отсюда полная светимость АЧТ с площадью поверхности $S$
	\begin{equation}
	L = S \sigma T^4,
	\label{eq:steff-bol-law}
\end{equation}
константа $\sigma$ называется \term{постоянной Стефана-Больцмана}.
  
Важно отметить, что \imp{закон Стефана-Больцмана}~--- прямое следствие формулы Планка \eqref{Planck's formula}, так как
\begin{equation}
	\sigma T^4 = \int\limits^\infty_0 B(\lambda, T)\,d\lambda \int\limits_0^{\pi/2} \sin \varphi\, d\varphi \int\limits_0^{2\pi} \cos \varphi\, d\theta = \pi \int\limits^\infty_0 B(\lambda, T)\,d\lambda,
\end{equation}
откуда $\sigma = (2\pi^5k^4)/(15c^2h^3) = 5.67 \cdot 10^{-8}~\text{Вт}/(\text{м}^2\cdot \text{К}^4)$.

%Для АЧТ сферической формы с радиусом $R$ формула~\eqref{eq:steff-bol-law} принимает вид
%\begin{equation}
%L=4\pi R^2\sigma T^4.
%\end{equation}
Для звёзд главной последовательности выполняется соотношение $L \sim M^{\alpha}$, где~$\alpha$~--- коэффициент пропорциональности, который зависит от массы звезды следующим образом:
\begin{align*}
\alpha &= 2.5, \quad M < 0.43 M_\odot; & 
\alpha &= 4, \quad 0.43 M_\odot < M < 2 M_\odot;\\ 
\alpha &= 3.2, \quad 2 M_\odot < M < 20 M_\odot; & 
\alpha &= 1, \quad M > 20 M_\odot.
\end{align*}
Также существует примерная зависимость светимости звёзды от её радиуса, имеющая вид  $L\sim R^{5.2}$.
\subsection{Энергия излучения}
\term{Энергия излучения}~--- энергия, переносимая излучением ($Q_e$).\\
\term{Поток излучения}~--- физическая величина, характеризующая мощность, переносимую излучением,
\begin{equation}
 \Phi_e = \frac{d Q_e}{dt}.
\end{equation}
\imp{Теорема Гаусса}: через любую замкнутую поверхность потоки от одинаковых источников равны.

\term{Спектральная плотность потока излучения}~--- поток излучения, приходящийся на малый единичный интервал спектра,
\begin{equation}
\Phi_{e, \lambda}(\lambda) = \frac{d\Phi_e(\lambda)}{d\lambda}, \quad\quad \Phi_{e, \nu}(\nu) = \frac{d\Phi_e(\nu)}{d\nu} =  \frac{\lambda^2}{c}\Phi_{e, \lambda}(\lambda).
\end{equation}

\term{Объемная плотность энергии излучения}~--- количество энергии на единицу объема
\begin{equation}
U_e = \frac{d Q_e}{dV}.
\end{equation}

\term{Светимость}~--- величина, представляющая собой световой поток излучения, испускаемого с малого участка светящейся поверхности единичной площади,
\begin{equation}
M_e = \frac{d \Phi_e}{dS_1},
\end{equation}
здесь $S_1$~--- площадь объекта, испускающего энергию.

\term{Яркость}~--- световой поток, приходящийся на единичный телесный угол, в расчёте на единичную площадку проекции излучающей поверхности на картинную плоскость, 
\begin{equation}
L_e = \frac{d^2 \Phi_e}{d \Omega\,dS_1 \cos \varepsilon},
\end{equation}
где $\varepsilon$~--- угол между направлением потока излучения и нормалью к плоскости излучающей поверхности.

\term{Интегральная яркость}~--- интеграл яркости по видимой поверхности источника. Показывает количество энергии, пришедшее от источника за единицу времени.
\begin{equation}
\Lambda_e = \int \limits_S L_e(\vec{r})\,ds.
\end{equation}
\term{Освещенность}~--- величина, равная отношению светового потока, падающего на малый участок поверхности, к его площади~--- поверхностная плотность потока
\begin{equation}
E_e = \frac{d\Phi_e}{dS_2} \sim \frac{1}{r^2},
\end{equation}
здесь $S_2$~--- площадь поверхности приёмника, $r$~--- расстояние от источника.
\input{sections/astrophys.flux-albedo.tex}
\input{sections/astrophys.photon.tex}
\input{sections/astrophys.energy-lines.tex}
\subsection{Формула Планка}
\label{sec:planck-law}
\term{Формула Планка}~--- выражение для спектральной плотности мощности излучения абсолютно чёрного тела на интервале частот $[\nu, \nu + d \nu)$, распространяющейся с телесном угле $d\Omega$, которое было получено Максом Планком в 1900~году. Данное выражение имеет следующий вид:
\begin{equation}
B_\nu(\nu,T)=\frac{2h\nu^3}{c^2}\cdot \frac{1}{\exp\left(\frac{h\nu}{kT}\right)-1} = \left[ \frac{\text{Вт}}{\text{м}^2 \cdot \text{Гц} \cdot \text{ср}}\right],
\label{eq:plancks-law-nu}
\end{equation}
где $\nu$~--- частота излучения, $T$~--- температура АЧТ, $h$~--- постоянная Планка, $k$~--- постоянная Больцмана, $c$~--- скорость света.

Если записать закон излучения Планка \eqref{eq:plancks-law-nu} для длин волн, то
\begin{equation}
B_\lambda(\lambda,T)=\frac{2hc^2}{\lambda^5} \cdot \frac{1}{\exp\left(\frac{hc}{\lambda kT}\right)-1} = \left[ \frac{\text{Вт}}{\text{м}^3 \cdot \text{ср}}\right].
\label{eq:plancks-law-lambda}
\end{equation}
\begin{wrapfigure}[15]{l}{.6\tw}
\centering
\vspace{-.9pc}
 \begin{tikzpicture}
  \begin{axis}[
  				width 	=	.6\tw, 
				height	=	6cm, 
  				ymax	=	1e+14,
  				xmax	=	2000,
  				xmin	=	0,
  				ymin	=	0,
				xlabel	=	{Длина волны $\lambda$,~нм}, 
				ylabel 	= 	{$B_\lambda(\lambda, T)$,~$\text{Вт} \cdot \text{м}^{-3}$}
]
   \addplot+[dashed, thin, black] table[x=l, y=tl] {data/planck.txt};
   \addplot+[black] table[x=l, y=t4] {data/planck.txt} node at (axis cs:870, 1.6e+13) {\tiny{$4500$~K}};
   \addplot+[black] table[x=l, y=t5] {data/planck.txt}node at (axis cs:750, 4.2e+13) {\tiny{$5000$~K}};
   \addplot+[black] table[x=l, y=t58] {data/planck.txt}node at (axis cs:670, 8.5e+13) {\tiny{$5800$~K}};
   \addplot+[black] table[x=l, y=t7] {data/planck.txt}node at (axis cs:1350, 3.5e+13) {\tiny{$7000$~K}};
	%\addplot+[black, smooth] table[x=l, y=t15] {data/planck.txt} node at (axis cs:1670, 5.5e+13) {\tiny{$15000$~K}};
  \end{axis}
 \end{tikzpicture}
\caption{Кривые спектральной плотности мощности изотропного излучения АЧТ с разной температурой}\label{pic:wien-law}
\end{wrapfigure}
Стоит заметить, что при переходе в функции к длинам волн меняется не только частота на длину волны, но и выражение для интервала. 

Формула Планка появилась, когда стало ясно, что формула Рэлея-Джинса удовлетворительно описывает излучение только в области больших длин волн, а~с~убыванием длин волн даёт сильные расхождения с реальными данными. Однако формулу Рэлея-Джинса используют и сейчас для описания кривой Планка на больших длинах волн. 

\change{
Проделаем обратные действия: получим формулу Рэлея-Джинса из формулы Планка. Длинноволновая часть спектра характеризуется соотношением $h\nu \ll kT$, то есть 
\begin{equation*}
	\exp\left( \frac{h\nu}{kT}\right) \approx 1 + \frac{h\nu}{kT}.
\end{equation*}
Подставляя полученное выражение в знаменатель \eqref{eq:plancks-law-nu}, получим
\begin{equation*}
	B_\nu(\nu,T) \approx \frac{2h\nu^3}{c^2}\cdot \frac{1}{1 + \frac{h\nu}{kT} - 1} = \frac{2h\nu^3 }{c^2}\cdot \frac{k T}{ h \nu} = \frac{2 \nu^2 k T}{c^2}.
\end{equation*}
}
\change{
	Проделав то же самое для выражения через длину волны, получим:
}
\begin{equation}
	B(\lambda, T) \simeq \frac{2 c k T}{\lambda^4}, \quad\quad B(\nu, T) \simeq \frac{2 \nu^2 k T}{c^2}.
\label{Ray-Jean}
\end{equation}

\change{
	В коротковолновой области, наоборот, $h \nu \gg kT$, следовательно, в знаменателе формулы Планка единица много меньше стоящей там экспоненты, то есть
	\begin{equation*}
		\frac{1}{\exp\left(\frac{h\nu}{kT}\right)-1} \approx \frac{1}{\exp\left(\frac{h\nu}{kT}\right)} = \exp\left(-\frac{h\nu}{kT}\right).
	\end{equation*} 
	Отсюда получаются выражения, называемые приближением Вина:
}
\begin{equation}
B ( \lambda, T) \simeq \frac{2 h c^2}{\lambda^5} \exp \left( -\frac{h c}{\lambda k T}\right), \quad \quad B( \nu, T ) \simeq \frac{2 h \nu^3}{c^2} \exp \left( -\frac{h \nu}{k T} \right).
\end{equation}
\subsection{Закон смещения Вина}
\term{Закон смещения Вина} --- закон, устанавливающий зависимость длины волны~$\lambda_\text{макс}$, на которой спектральная плотность излучения $B_\lambda(\lambda, T)$ абсолютно чёрного тела достигает своего максимума, от температуры $T$ этого тела:
\begin{equation}
	\lambda_\text{макс} \approx \frac{b}{T} \equiv \frac{0.0029~\text{м} \cdot \text{К}}{T}.
\end{equation}
Закон является следствием исследования функции Планка (см.~\ref{sec:planck-law}) на экстремальность.
\subsection{Эффект Доплера. Красное смещение}
\term{Эффект Доплера}~--- эффект изменения частоты и длины волны электромагнитного излучения, регистрируемого приёмником, вызванный относительным движением источника и приёмника (см.~Рис.\,\ref{doppler-ef}).

При $\Delta \lambda \ll \lambda_0$ с большой точностью выполняется следующее важное соотношение:\begin{equation}
\beta \equiv \dfrac{v}{c} = \dfrac{\lambda - \lambda_0}{\lambda_0} \equiv \dfrac{\Delta \lambda}{\lambda_0},
\label{eq:dopler-ef-simple}
\end{equation}
\begin{wrapfigure}[6]{r}{0.5\tw}
\centering
\vspace{-.5pc}
\includegraphics[width=.5\tw]{doppler-ef}
\caption{Эффект Доплера}
\label{doppler-ef}
\end{wrapfigure}
где $\lambda_0$~--- лабораторная длина волны излучения источника, а $\lambda$~--- наблюдаемая. В действительности же имеет место более общий случай: \imp{релятивистский эффект Доплера}, обусловленный проявлением СТО при $v \simeq c$, для которого формула~\eqref{eq:dopler-ef-simple} усложняется и принимает вид
\begin{equation}
\nu = \nu_0 \cdot \dfrac{\sqrt{1 - \beta^2}}{1 + \beta \cdot \cos\theta},
\label{eq:dopler-ef-rel}
\end{equation}
где $\nu$~--- частота, с которой наблюдатель принимает волны, $\nu_0$~--- частота, с которой источник испускает волны, $v$~--- скорость источника, $\theta$~--- угол между направлением на источник и вектором его скорости в системе отсчёта приёмника. Если источник радиально удаляется от наблюдателя, то $\theta = 0$, если приближается, то $\theta =\pi$. Важно, что~\eqref{eq:dopler-ef-simple} напрямую следует из \eqref{eq:dopler-ef-rel} при $\beta  \ll 1$.

\term{Красное смещение}~--- явление сдвига спектральных линий химических элементов в красную (длинноволновую) сторону, обусловленное относительным движение объектов. Параметр красного смещения определяется из наблюдаемой и лабораторной длин волн как
\begin{equation}
z = \dfrac{\lambda - \lambda_0}{\lambda_0}.
\end{equation}

Доплеровское смещение длины волны в спектре источника, движущегося с лучевой скоростью $v_{r}$ и полной скоростью $v$,
\begin{equation}
z = \dfrac{1 + v_r / c}{\sqrt{1 - \beta^2}}.
\end{equation}

\term{Гравитационное красное смещение}~--- проявление эффекта изменения частоты излучения, испущенного массивным объектом, таким как звезда или чёрная дыра. Наблюдается как сдвиг спектральных линий в спектре источника в красную область спектра. Гравитационное красное смещение определяется из формулы, выведенной Эйнштейном,
\begin{equation}
z_G=\dfrac{GM}{c^2 R}-\dfrac{GM}{c^2 r},
\label{eq:grav-red-shift}
\end{equation}
где $M$~--- масса гравитирующего тела, $R$~--- радиальное расстояние от центра масс тела до точки излучения (радиус источника), $r$~---  радиальное расстояние от центра масс источника до точки наблюдения. В случае, когда наблюдатель находится от источника много дальше его радиуса, т.\,е. выполняется соотношение $r \gg R$, выражение~\eqref{eq:grav-red-shift} можно упростить до
\begin{equation}
z_G \simeq \dfrac{GM}{c^2 R}.
\end{equation}

\input{sections/astrophys.light-pressure.tex}
\input{sections/astrophys.edd.tex}
\input{sections/astrophys.grav-lens.tex}
\subsection{Закон Хаббла}
\term{Закон Хаббла}~--- эмпирический закон, связывающий скорость удаления галактик $V$ и расстояние $R$ до них линейным образом: 
\begin{equation}
	V = H R,
\end{equation}
величина $H=68~\text{км/c} \cdot \text{Мпк})$ называется \imp{постоянной Хаббла}.

При $v \ll c$ можно использовать приближение эффекта Доплера, тогда
\begin{equation}
	V = c z.
\label{eq:hubble-speed}
\end{equation}

Равенство \eqref{eq:hubble-speed} справедливо только при $z \ll 1$, а при б\'{o}льших значениях $z$ космологическое красное смещение нльзя связывать с эффектом Доплера, поэтому можно пользоваться только формулой 
\begin{equation}
	\frac{dz}{dt} = - H(z)(1+z),
\end{equation}
где постоянная Хаббла введена как функция красного смещения.
\subsection{Шкала электромагнитных волн}


\term{Гамма излучение} возникает при радиоактивных распадах ядер, при торможении электронов энергией более $10^5$~эВ и при других взаимодействиях элементарных частиц. Используются в гамма-дефектоскопии, при изучении свойств вещества.

\term{Рентгеновские лучи} излучаются при большом ускорении электронов, например при их торможении в металлах. Получают их при помощи рентгеновской трубки: электроны в вакуумной трубке ускоряются электрическим полем при высоком напряжении, достигая анода, при со­ударении резко тормозятся. При торможении электроны движут­ся с ускорением и излучают электромагнитные волны с малой длиной. 

\begin{figure}[!h]
\centering
\includegraphics[width = 1\textwidth]{scale-wave.pdf}
\caption{Шкала электромагнитных волн}
\end{figure}
\term{Ультрафиолетовые лучи}~--- излучение Солнца, ртутных ламп и т.\,п. Используются в ультрафиолетовой микроскопии, в медицине.

\term{Видимое излучение}~--- часть электромагнитного излучения, воспринимаемая глазом (от фиолетового до от красного).

\term{Инфракрасное излучение}~--- тепловое, излучается любым нагретым телом.

\term{Радиоволны} используются повсеместно в обычной жизни, это и сотовая связь, и радиолокация, и спутниковая связь, и Wi-Fi и многое другое.

\term{Низкочастотные волны}~--- диапазон, традиционно используемый в электротехнике. В промышленной электроэнергетике используется частота 50~Гц, на~которой осуществляется передача электрической энергии по линиям и преобразование напряжений трансформаторными устройствами.
\input{sections/astrophys.spec-theor-rel.tex}
\subsection{Оптическая толщина. Закон Бугера}
\term{Оптическая толщина}~--- безразмерная величина, характеризующая степень непрозрачности среды для проходящего сквозь неё излучения,
\begin{equation}
\tau = \int n(x) \sigma(x)\,dx,
\end{equation}
где $\tau$~--- оптическая толщина среды, $n$~--- концентрация частиц, $\sigma$~--- сечение их взаимодействия.

Поток $I_0$ на входе связан с потоком $I$ на выходе \term{Законом Бугера}:
\begin{equation}
I = I_0 e^{-\tau}.
\end{equation}
\input{sections/astrophys.colour.tex}
\input{sections/astrophys.mkt.tex}
\input{sections/astrophys.earth-atmosphere.tex}

\newpage
\section{Астрофизика}
\subsection{Звёздные величины}
Звёздная величина~--- безразмерная числовая характеристика яркости объекта. Известно, что увеличению светового потока в 100 раз соответствует уменьшение видимой звёздной величины ровно на 5 единиц. Тогда уменьшение звёздной величины на одну единицу означает увеличение светового потока в $\sqrt[5]{100}\approx 2.512$~раз, то есть звёздные величины являются логарифмической шкалой измерения плотности потока. Зависимость, связывающая отношение освещённостей $E_1$ и $E_2$ и разность звёздных величин $m_1$ и $m_2$ двух объектов, называется \term{формулой Погсона} и имеет вид
\begin{equation}
	\frac{E_1}{E_2} = 10^{0.4(m_2 - m_1)} \quad \Longleftrightarrow \quad m_2 = m_1 + 2.5 \lg \frac{E_1}{E_2}.
	\label{eq:Pogson-law}
\end{equation}
Широко используется понятие \term{абсолютной звёздной величины} $M$~--- это видимая звёздная величина $m$ при наблюдении с установленного расстояния: для звёзд~---~10~пк, для тел Солнечной системы~---~1~\au, причем считается, что тело находится в 1~\au~и от наблюдателя и от Солнца, а фаза равна единице, то есть можно считать, что наблюдатель находится в центре Солнца, а~тело~--- в~1~\au~от него. 

Кроме этого, важно понятие \term{болометрической звёздной величины} $m_\text{bol}$~--- это звёздная величина, при расчёте которой учитывается полная мощность излучения источника во всех диапазонах электромагнитных волн. Обычная (видимая) звёздная величина учитывает излучение лишь в видимой части спектра от примерно 380~нм до примерно~780~нм. Разность между болометрической и видимой звёздными величинами называется \term{болометрической поправкой} ($BC$), которая отличается для разных спектральных классов звёзд. Из определения, болометрическая поправка может быть найдена по формуле
\begin{equation}
	BC = m_\text{bol} - m.
\end{equation}
Абсолютную звёздную величину звезды можно получить по формуле Погсона \eqref{eq:Pogson-law} из видимой звёздной величины $m$ и расстояния $r$ до неё в парсеках
\begin{equation}
	M = m + 2.5 \lg \frac{E}{E_\text{абс}} = m + 2.5 \lg \frac{(10~\text{пк})^2}{r^2} = m + 5 - 5\lg r.
	\label{eq:abs-mag}
\end{equation} 
Если принимать к рассмотрению межзвездное поглощение $A$, то формулу  \eqref{eq:abs-mag} необходимо уточнить:
\begin{equation}
	M = m + 5 - 5\lg r - Ar.
\end{equation}
\subsection{Закон Стефана-Больцмана}
\term{Закон Стефана~--- Больцмана} определяет зависимость плотности мощности излучения абсолютно чёрного тела (АЧТ) $u$ от его температуры $T$:
\begin{equation}
u = a T^4,
\end{equation} 
где $a$~--- некая универсальная константа.
Отсюда полная светимость АЧТ с площадью поверхности $S$
	\begin{equation}
	L = S \sigma T^4,
	\label{eq:steff-bol-law}
\end{equation}
константа $\sigma$ называется \term{постоянной Стефана-Больцмана}.
  
Важно отметить, что \imp{закон Стефана-Больцмана}~--- прямое следствие формулы Планка \eqref{Planck's formula}, так как
\begin{equation}
	\sigma T^4 = \int\limits^\infty_0 B(\lambda, T)\,d\lambda \int\limits_0^{\pi/2} \sin \varphi\, d\varphi \int\limits_0^{2\pi} \cos \varphi\, d\theta = \pi \int\limits^\infty_0 B(\lambda, T)\,d\lambda,
\end{equation}
откуда $\sigma = (2\pi^5k^4)/(15c^2h^3) = 5.67 \cdot 10^{-8}~\text{Вт}/(\text{м}^2\cdot \text{К}^4)$.

%Для АЧТ сферической формы с радиусом $R$ формула~\eqref{eq:steff-bol-law} принимает вид
%\begin{equation}
%L=4\pi R^2\sigma T^4.
%\end{equation}
Для звёзд главной последовательности выполняется соотношение $L \sim M^{\alpha}$, где~$\alpha$~--- коэффициент пропорциональности, который зависит от массы звезды следующим образом:
\begin{align*}
\alpha &= 2.5, \quad M < 0.43 M_\odot; & 
\alpha &= 4, \quad 0.43 M_\odot < M < 2 M_\odot;\\ 
\alpha &= 3.2, \quad 2 M_\odot < M < 20 M_\odot; & 
\alpha &= 1, \quad M > 20 M_\odot.
\end{align*}
Также существует примерная зависимость светимости звёзды от её радиуса, имеющая вид  $L\sim R^{5.2}$.
\subsection{Энергия излучения}
\term{Энергия излучения}~--- энергия, переносимая излучением ($Q_e$).\\
\term{Поток излучения}~--- физическая величина, характеризующая мощность, переносимую излучением,
\begin{equation}
 \Phi_e = \frac{d Q_e}{dt}.
\end{equation}
\imp{Теорема Гаусса}: через любую замкнутую поверхность потоки от одинаковых источников равны.

\term{Спектральная плотность потока излучения}~--- поток излучения, приходящийся на малый единичный интервал спектра,
\begin{equation}
\Phi_{e, \lambda}(\lambda) = \frac{d\Phi_e(\lambda)}{d\lambda}, \quad\quad \Phi_{e, \nu}(\nu) = \frac{d\Phi_e(\nu)}{d\nu} =  \frac{\lambda^2}{c}\Phi_{e, \lambda}(\lambda).
\end{equation}

\term{Объемная плотность энергии излучения}~--- количество энергии на единицу объема
\begin{equation}
U_e = \frac{d Q_e}{dV}.
\end{equation}

\term{Светимость}~--- величина, представляющая собой световой поток излучения, испускаемого с малого участка светящейся поверхности единичной площади,
\begin{equation}
M_e = \frac{d \Phi_e}{dS_1},
\end{equation}
здесь $S_1$~--- площадь объекта, испускающего энергию.

\term{Яркость}~--- световой поток, приходящийся на единичный телесный угол, в расчёте на единичную площадку проекции излучающей поверхности на картинную плоскость, 
\begin{equation}
L_e = \frac{d^2 \Phi_e}{d \Omega\,dS_1 \cos \varepsilon},
\end{equation}
где $\varepsilon$~--- угол между направлением потока излучения и нормалью к плоскости излучающей поверхности.

\term{Интегральная яркость}~--- интеграл яркости по видимой поверхности источника. Показывает количество энергии, пришедшее от источника за единицу времени.
\begin{equation}
\Lambda_e = \int \limits_S L_e(\vec{r})\,ds.
\end{equation}
\term{Освещенность}~--- величина, равная отношению светового потока, падающего на малый участок поверхности, к его площади~--- поверхностная плотность потока
\begin{equation}
E_e = \frac{d\Phi_e}{dS_2} \sim \frac{1}{r^2},
\end{equation}
здесь $S_2$~--- площадь поверхности приёмника, $r$~--- расстояние от источника.
\input{sections/astrophys.flux-albedo.tex}
\input{sections/astrophys.photon.tex}
\input{sections/astrophys.energy-lines.tex}
\subsection{Формула Планка}
\label{sec:planck-law}
\term{Формула Планка}~--- выражение для спектральной плотности мощности излучения абсолютно чёрного тела на интервале частот $[\nu, \nu + d \nu)$, распространяющейся с телесном угле $d\Omega$, которое было получено Максом Планком в 1900~году. Данное выражение имеет следующий вид:
\begin{equation}
B_\nu(\nu,T)=\frac{2h\nu^3}{c^2}\cdot \frac{1}{\exp\left(\frac{h\nu}{kT}\right)-1} = \left[ \frac{\text{Вт}}{\text{м}^2 \cdot \text{Гц} \cdot \text{ср}}\right],
\label{eq:plancks-law-nu}
\end{equation}
где $\nu$~--- частота излучения, $T$~--- температура АЧТ, $h$~--- постоянная Планка, $k$~--- постоянная Больцмана, $c$~--- скорость света.

Если записать закон излучения Планка \eqref{eq:plancks-law-nu} для длин волн, то
\begin{equation}
B_\lambda(\lambda,T)=\frac{2hc^2}{\lambda^5} \cdot \frac{1}{\exp\left(\frac{hc}{\lambda kT}\right)-1} = \left[ \frac{\text{Вт}}{\text{м}^3 \cdot \text{ср}}\right].
\label{eq:plancks-law-lambda}
\end{equation}
\begin{wrapfigure}[15]{l}{.6\tw}
\centering
\vspace{-.9pc}
 \begin{tikzpicture}
  \begin{axis}[
  				width 	=	.6\tw, 
				height	=	6cm, 
  				ymax	=	1e+14,
  				xmax	=	2000,
  				xmin	=	0,
  				ymin	=	0,
				xlabel	=	{Длина волны $\lambda$,~нм}, 
				ylabel 	= 	{$B_\lambda(\lambda, T)$,~$\text{Вт} \cdot \text{м}^{-3}$}
]
   \addplot+[dashed, thin, black] table[x=l, y=tl] {data/planck.txt};
   \addplot+[black] table[x=l, y=t4] {data/planck.txt} node at (axis cs:870, 1.6e+13) {\tiny{$4500$~K}};
   \addplot+[black] table[x=l, y=t5] {data/planck.txt}node at (axis cs:750, 4.2e+13) {\tiny{$5000$~K}};
   \addplot+[black] table[x=l, y=t58] {data/planck.txt}node at (axis cs:670, 8.5e+13) {\tiny{$5800$~K}};
   \addplot+[black] table[x=l, y=t7] {data/planck.txt}node at (axis cs:1350, 3.5e+13) {\tiny{$7000$~K}};
	%\addplot+[black, smooth] table[x=l, y=t15] {data/planck.txt} node at (axis cs:1670, 5.5e+13) {\tiny{$15000$~K}};
  \end{axis}
 \end{tikzpicture}
\caption{Кривые спектральной плотности мощности изотропного излучения АЧТ с разной температурой}\label{pic:wien-law}
\end{wrapfigure}
Стоит заметить, что при переходе в функции к длинам волн меняется не только частота на длину волны, но и выражение для интервала. 

Формула Планка появилась, когда стало ясно, что формула Рэлея-Джинса удовлетворительно описывает излучение только в области больших длин волн, а~с~убыванием длин волн даёт сильные расхождения с реальными данными. Однако формулу Рэлея-Джинса используют и сейчас для описания кривой Планка на больших длинах волн. 

\change{
Проделаем обратные действия: получим формулу Рэлея-Джинса из формулы Планка. Длинноволновая часть спектра характеризуется соотношением $h\nu \ll kT$, то есть 
\begin{equation*}
	\exp\left( \frac{h\nu}{kT}\right) \approx 1 + \frac{h\nu}{kT}.
\end{equation*}
Подставляя полученное выражение в знаменатель \eqref{eq:plancks-law-nu}, получим
\begin{equation*}
	B_\nu(\nu,T) \approx \frac{2h\nu^3}{c^2}\cdot \frac{1}{1 + \frac{h\nu}{kT} - 1} = \frac{2h\nu^3 }{c^2}\cdot \frac{k T}{ h \nu} = \frac{2 \nu^2 k T}{c^2}.
\end{equation*}
}
\change{
	Проделав то же самое для выражения через длину волны, получим:
}
\begin{equation}
	B(\lambda, T) \simeq \frac{2 c k T}{\lambda^4}, \quad\quad B(\nu, T) \simeq \frac{2 \nu^2 k T}{c^2}.
\label{Ray-Jean}
\end{equation}

\change{
	В коротковолновой области, наоборот, $h \nu \gg kT$, следовательно, в знаменателе формулы Планка единица много меньше стоящей там экспоненты, то есть
	\begin{equation*}
		\frac{1}{\exp\left(\frac{h\nu}{kT}\right)-1} \approx \frac{1}{\exp\left(\frac{h\nu}{kT}\right)} = \exp\left(-\frac{h\nu}{kT}\right).
	\end{equation*} 
	Отсюда получаются выражения, называемые приближением Вина:
}
\begin{equation}
B ( \lambda, T) \simeq \frac{2 h c^2}{\lambda^5} \exp \left( -\frac{h c}{\lambda k T}\right), \quad \quad B( \nu, T ) \simeq \frac{2 h \nu^3}{c^2} \exp \left( -\frac{h \nu}{k T} \right).
\end{equation}
\subsection{Закон смещения Вина}
\term{Закон смещения Вина} --- закон, устанавливающий зависимость длины волны~$\lambda_\text{макс}$, на которой спектральная плотность излучения $B_\lambda(\lambda, T)$ абсолютно чёрного тела достигает своего максимума, от температуры $T$ этого тела:
\begin{equation}
	\lambda_\text{макс} \approx \frac{b}{T} \equiv \frac{0.0029~\text{м} \cdot \text{К}}{T}.
\end{equation}
Закон является следствием исследования функции Планка (см.~\ref{sec:planck-law}) на экстремальность.
\subsection{Эффект Доплера. Красное смещение}
\term{Эффект Доплера}~--- эффект изменения частоты и длины волны электромагнитного излучения, регистрируемого приёмником, вызванный относительным движением источника и приёмника (см.~Рис.\,\ref{doppler-ef}).

При $\Delta \lambda \ll \lambda_0$ с большой точностью выполняется следующее важное соотношение:\begin{equation}
\beta \equiv \dfrac{v}{c} = \dfrac{\lambda - \lambda_0}{\lambda_0} \equiv \dfrac{\Delta \lambda}{\lambda_0},
\label{eq:dopler-ef-simple}
\end{equation}
\begin{wrapfigure}[6]{r}{0.5\tw}
\centering
\vspace{-.5pc}
\includegraphics[width=.5\tw]{doppler-ef}
\caption{Эффект Доплера}
\label{doppler-ef}
\end{wrapfigure}
где $\lambda_0$~--- лабораторная длина волны излучения источника, а $\lambda$~--- наблюдаемая. В действительности же имеет место более общий случай: \imp{релятивистский эффект Доплера}, обусловленный проявлением СТО при $v \simeq c$, для которого формула~\eqref{eq:dopler-ef-simple} усложняется и принимает вид
\begin{equation}
\nu = \nu_0 \cdot \dfrac{\sqrt{1 - \beta^2}}{1 + \beta \cdot \cos\theta},
\label{eq:dopler-ef-rel}
\end{equation}
где $\nu$~--- частота, с которой наблюдатель принимает волны, $\nu_0$~--- частота, с которой источник испускает волны, $v$~--- скорость источника, $\theta$~--- угол между направлением на источник и вектором его скорости в системе отсчёта приёмника. Если источник радиально удаляется от наблюдателя, то $\theta = 0$, если приближается, то $\theta =\pi$. Важно, что~\eqref{eq:dopler-ef-simple} напрямую следует из \eqref{eq:dopler-ef-rel} при $\beta  \ll 1$.

\term{Красное смещение}~--- явление сдвига спектральных линий химических элементов в красную (длинноволновую) сторону, обусловленное относительным движение объектов. Параметр красного смещения определяется из наблюдаемой и лабораторной длин волн как
\begin{equation}
z = \dfrac{\lambda - \lambda_0}{\lambda_0}.
\end{equation}

Доплеровское смещение длины волны в спектре источника, движущегося с лучевой скоростью $v_{r}$ и полной скоростью $v$,
\begin{equation}
z = \dfrac{1 + v_r / c}{\sqrt{1 - \beta^2}}.
\end{equation}

\term{Гравитационное красное смещение}~--- проявление эффекта изменения частоты излучения, испущенного массивным объектом, таким как звезда или чёрная дыра. Наблюдается как сдвиг спектральных линий в спектре источника в красную область спектра. Гравитационное красное смещение определяется из формулы, выведенной Эйнштейном,
\begin{equation}
z_G=\dfrac{GM}{c^2 R}-\dfrac{GM}{c^2 r},
\label{eq:grav-red-shift}
\end{equation}
где $M$~--- масса гравитирующего тела, $R$~--- радиальное расстояние от центра масс тела до точки излучения (радиус источника), $r$~---  радиальное расстояние от центра масс источника до точки наблюдения. В случае, когда наблюдатель находится от источника много дальше его радиуса, т.\,е. выполняется соотношение $r \gg R$, выражение~\eqref{eq:grav-red-shift} можно упростить до
\begin{equation}
z_G \simeq \dfrac{GM}{c^2 R}.
\end{equation}

\input{sections/astrophys.light-pressure.tex}
\input{sections/astrophys.edd.tex}
\input{sections/astrophys.grav-lens.tex}
\subsection{Закон Хаббла}
\term{Закон Хаббла}~--- эмпирический закон, связывающий скорость удаления галактик $V$ и расстояние $R$ до них линейным образом: 
\begin{equation}
	V = H R,
\end{equation}
величина $H=68~\text{км/c} \cdot \text{Мпк})$ называется \imp{постоянной Хаббла}.

При $v \ll c$ можно использовать приближение эффекта Доплера, тогда
\begin{equation}
	V = c z.
\label{eq:hubble-speed}
\end{equation}

Равенство \eqref{eq:hubble-speed} справедливо только при $z \ll 1$, а при б\'{o}льших значениях $z$ космологическое красное смещение нльзя связывать с эффектом Доплера, поэтому можно пользоваться только формулой 
\begin{equation}
	\frac{dz}{dt} = - H(z)(1+z),
\end{equation}
где постоянная Хаббла введена как функция красного смещения.
\subsection{Шкала электромагнитных волн}


\term{Гамма излучение} возникает при радиоактивных распадах ядер, при торможении электронов энергией более $10^5$~эВ и при других взаимодействиях элементарных частиц. Используются в гамма-дефектоскопии, при изучении свойств вещества.

\term{Рентгеновские лучи} излучаются при большом ускорении электронов, например при их торможении в металлах. Получают их при помощи рентгеновской трубки: электроны в вакуумной трубке ускоряются электрическим полем при высоком напряжении, достигая анода, при со­ударении резко тормозятся. При торможении электроны движут­ся с ускорением и излучают электромагнитные волны с малой длиной. 

\begin{figure}[!h]
\centering
\includegraphics[width = 1\textwidth]{scale-wave.pdf}
\caption{Шкала электромагнитных волн}
\end{figure}
\term{Ультрафиолетовые лучи}~--- излучение Солнца, ртутных ламп и т.\,п. Используются в ультрафиолетовой микроскопии, в медицине.

\term{Видимое излучение}~--- часть электромагнитного излучения, воспринимаемая глазом (от фиолетового до от красного).

\term{Инфракрасное излучение}~--- тепловое, излучается любым нагретым телом.

\term{Радиоволны} используются повсеместно в обычной жизни, это и сотовая связь, и радиолокация, и спутниковая связь, и Wi-Fi и многое другое.

\term{Низкочастотные волны}~--- диапазон, традиционно используемый в электротехнике. В промышленной электроэнергетике используется частота 50~Гц, на~которой осуществляется передача электрической энергии по линиям и преобразование напряжений трансформаторными устройствами.
\input{sections/astrophys.spec-theor-rel.tex}
\subsection{Оптическая толщина. Закон Бугера}
\term{Оптическая толщина}~--- безразмерная величина, характеризующая степень непрозрачности среды для проходящего сквозь неё излучения,
\begin{equation}
\tau = \int n(x) \sigma(x)\,dx,
\end{equation}
где $\tau$~--- оптическая толщина среды, $n$~--- концентрация частиц, $\sigma$~--- сечение их взаимодействия.

Поток $I_0$ на входе связан с потоком $I$ на выходе \term{Законом Бугера}:
\begin{equation}
I = I_0 e^{-\tau}.
\end{equation}
\input{sections/astrophys.colour.tex}
\input{sections/astrophys.mkt.tex}
\input{sections/astrophys.earth-atmosphere.tex}

\newpage
\section{Оптика}
\subsection{Телескоп}
\term{Телескоп}~--- устройство для наблюдения удаленных объектов. На данный момент существуют телескопы  для наблюдения во всех  диапазонах электро-магнитного излучения. По наблюдаемому диапазону телескопы делят на \imp{оптические} телескопы, \imp{радиотелескопы}, \imp{рентгеновские} телескопы и \imp{гамма-те\-ле\-скопы}. Каждый из классов в свою очередь содержит множество подклассов. Поговорим подробнее про оптические телескопы.

Оптические телескопы по своей схеме делятся на три типа: \imp{рефлекторы} (диоптрические), \imp{рефракторы} (катаптрические) и \imp{катадиоптрические}.

\vspace{-.3pc}
\begin{figure}[h!]
	\centering
	\begin{subfigure}{0.49\tw}
		\includegraphics[width = \tw]{Galiley}
		\caption{Рефрактор системы Галилея}
	\end{subfigure}
	\hfill
	\begin{subfigure}{0.49\tw}
		\includegraphics[width = \tw]{Kepler}
		\caption{Рефрактор системы Кеплера}
		\label{Kepler}
	\end{subfigure}
	\caption{Оптические схемы телескопов рефракторов}
\end{figure}
\term{Рефрактор} (линзовый телескоп)~---  оптический телескоп, в котором для собирания света используется система линз.

\vspace{-.3pc}
\begin{figure}[h!]
	\begin{subfigure}{0.49\tw}
		\includegraphics[width = \tw]{Cassigren.pdf}
		\caption{Рефлектор системы Кассегрена}
	\end{subfigure}
	\hfill
	\begin{subfigure}{0.49\tw}
		\includegraphics[width = \tw]{Gregory.pdf}
		\caption{Рефлектор системы Грегори}
		\label{Gregory}
	\end{subfigure}
	\vskip4pt
	\begin{subfigure}{0.49\tw}
		\includegraphics[width = \tw]{Newton}
		\caption{Рефлектор системы Ньютона}
	\end{subfigure}
	\hfill
	\begin{subfigure}{0.49\tw}
		\includegraphics[width = \tw]{Lomonosov.pdf}
		\caption{Рефлектор системы Ломоносова}
	\end{subfigure}
	\caption{Оптические схемы телескопов рефлекторов}
\end{figure}
\term{Рефлектор} (зеркальный телескоп)~---  оптический телескоп,  в котором светособирающими элементами являются зеркала.

\term{Катадиоптрический} (зеркально-линзовый) \term{телескоп}~--- оптический телескоп, в котором используется как система линз, так и зеркал.

\newpage
\section{Оптика}
\subsection{Телескоп}
\term{Телескоп}~--- устройство для наблюдения удаленных объектов. На данный момент существуют телескопы  для наблюдения во всех  диапазонах электро-магнитного излучения. По наблюдаемому диапазону телескопы делят на \imp{оптические} телескопы, \imp{радиотелескопы}, \imp{рентгеновские} телескопы и \imp{гамма-те\-ле\-скопы}. Каждый из классов в свою очередь содержит множество подклассов. Поговорим подробнее про оптические телескопы.

Оптические телескопы по своей схеме делятся на три типа: \imp{рефлекторы} (диоптрические), \imp{рефракторы} (катаптрические) и \imp{катадиоптрические}.

\vspace{-.3pc}
\begin{figure}[h!]
	\centering
	\begin{subfigure}{0.49\tw}
		\includegraphics[width = \tw]{Galiley}
		\caption{Рефрактор системы Галилея}
	\end{subfigure}
	\hfill
	\begin{subfigure}{0.49\tw}
		\includegraphics[width = \tw]{Kepler}
		\caption{Рефрактор системы Кеплера}
		\label{Kepler}
	\end{subfigure}
	\caption{Оптические схемы телескопов рефракторов}
\end{figure}
\term{Рефрактор} (линзовый телескоп)~---  оптический телескоп, в котором для собирания света используется система линз.

\vspace{-.3pc}
\begin{figure}[h!]
	\begin{subfigure}{0.49\tw}
		\includegraphics[width = \tw]{Cassigren.pdf}
		\caption{Рефлектор системы Кассегрена}
	\end{subfigure}
	\hfill
	\begin{subfigure}{0.49\tw}
		\includegraphics[width = \tw]{Gregory.pdf}
		\caption{Рефлектор системы Грегори}
		\label{Gregory}
	\end{subfigure}
	\vskip4pt
	\begin{subfigure}{0.49\tw}
		\includegraphics[width = \tw]{Newton}
		\caption{Рефлектор системы Ньютона}
	\end{subfigure}
	\hfill
	\begin{subfigure}{0.49\tw}
		\includegraphics[width = \tw]{Lomonosov.pdf}
		\caption{Рефлектор системы Ломоносова}
	\end{subfigure}
	\caption{Оптические схемы телескопов рефлекторов}
\end{figure}
\term{Рефлектор} (зеркальный телескоп)~---  оптический телескоп,  в котором светособирающими элементами являются зеркала.

\term{Катадиоптрический} (зеркально-линзовый) \term{телескоп}~--- оптический телескоп, в котором используется как система линз, так и зеркал.

\input{sections/optic.mounts.tex}
\input{sections/optic.zoom.tex}
\input{sections/optic.thin-lens.tex}
\input{sections/optic.snell-law.tex}
\subsection{Аберрации в оптике}
\paragraph{Хроматическая аберрация}
Пожалуй главным недостатком оптических схем, содержащий преломляющие оптические элементы (линзы; призмы, за исключением использования их для спектроскопии) являются \imp{хроматические аберрации}. Дело в том, что показатель преломления материала линзы зависит от длины волны падающего излучения. Это приводит к тому, что положения фокуса оптической системы зависит от длины волны излучения. При наблюдениях это проявляется как радужный ореол вокруг объектов, ухудшающий качество изображения.

\begin{wrapfigure}{r}{0.5\tw}
	\centering
	\vspace{-1pc}
	\begin{tikzpicture}
	\begin{axis}[
		height	=	4.5cm,
		width	=	6cm,
		xlabel	=	{$\lambda$, мкм},
		ylabel	=	{$n(\lambda)$},
		ylabel shift	= -1 cm,
		xmin = 0.3,
		xmax = 2.5,
		ymin = 1.48,
		ymax = 1.55,
		]
		
		\addplot[smooth, domain=0.3:2.5] table[x=l, y=n] {data/crown-dispersion.txt};
	\end{axis}
\end{tikzpicture}
\caption{}
\label{pic:crown-dispersion}	
\end{wrapfigure}
Найдем зависимость величины хроматической аберрации от величины дисперсии материала линзы. В качестве линзы рассмотрим плосковыпуклую линзу из оптического стекла~--- \imp{крона} (BK7). Зависимость $n(\lambda)$ её показателя преломления $n$ от длины волны $\lambda$ проходящего излучения представлена на графике (см.~Рис\,\ref{pic:crown-dispersion}). Важно отметить, здесь уже нельзя считать линзу тонкой, так как само по себе понятие тонкой линзы подразумевает отсутствие разного рода аберраций и условие фокусировки лучей в одной точке (фокусе), чего не происходит на практике.

\begin{wrapfigure}[9]{r}{0.63\tw}
	\centering
	\vspace{-.8pc}
	\begin{tikzpicture}
		\footnotesize
		
		\draw [decoration={snake, segment length=.7mm, amplitude=0.2mm}, decorate] (1.35, 1) arc(-31:15:0.26);
		\draw [double, line cap=butt] (.2, 1.135) arc(180:195:0.94);
		\draw  (2.7, 0) arc(180:149:0.3);
		
		\fill [lightgray] (.5, -1.5) -- (.5, 1.5) -- (1, 1.5) arc(20:-20:4.386) -- (.5, -1.5);
		
		\draw [thick] (.5, -1.5) -- (.5, 1.5);
		\draw [thick] (1, -1.5) arc(-20:20:4.386);
		\draw [semithick, dash pattern={on 5pt off 2pt on .5pt off 2pt}] (-3.8, 0) -- (3.2, 0);
%		\draw (-3.12, 0) node{$\times$};
		
		\draw [dashes] (-3.12, 0) -- (1.71, 1.29);
		
		\draw [semithick] (-3.8, 1.135) -- (1.12, 1.135) -- (3, 0);
		\draw [-latex] (-3.5, 1.135)-- (-1, 1.135);
		\draw [-latex] (1.12, 1.135) -- (2.25, 0.45);
		
		\draw [latex-latex] (-3.5, 0) -- (-3.5, 1.135);
		\draw [latex-latex] (1.27, -.5) -- (3, -.5);
		\draw [latex-latex] (1.12, -1.4) -- (3, -1.4);
		\draw [-latex] (0.5, -.5) -- (1.12, -.5);
		
		\draw (1.12, 1.135) -- (1.12, -1.5);
		\draw (1.27, 0) -- (1.27, -.6);
		\draw (3, 0) -- (3, -1.5);
		
		\draw (-3.5, 0.56) node[anchor=east] {$d$};
		\draw (-3.12, 0) node[anchor=north] {$C$};
		\draw (-.7, 0.6) node[anchor=north] {$R$};
		\draw (2.14, -.5) node[anchor=south] {$x$};
		\draw (2.14, -1.4) node[anchor=south] {$\frac{d}{\tg \gamma}$};
		
		\draw (.8, -1.5) node[anchor=south] {$n$};
		
		\draw (.2, 1) node[anchor=east] {$\alpha$};
		\draw (1.4, 1.07) node[anchor=west] {$\beta$};
		\draw (2.65, 0.15) node[anchor=east] {$\gamma$};
		
		
		\draw [fill=white] (3, 0) circle (.03); 
		\draw [fill=white] (-3.12, 0) circle (.03);
	\end{tikzpicture}
	\caption{}
	\label{pic:sphere-aberrations-lens}
\end{wrapfigure}
Итак, пусть радиус кривизны выпуклой поверхности рассматриваемой плос\-ко-вы\-пук\-лой линзы равен $R$. Рассмотрим также луч, параллельный оптической оси данной линзы, на расстоянии $d$ от этой оси (см.~Рис.\,\ref{pic:sphere-aberrations-lens}). Так как передняя поверхность линзы плоская, луч, попадая в линзу, не преломляется. Преломление происходит на выходе из линзы. Нетрудно показать, что для угла падения луча на заднюю поверхность линзы $\alpha$ справедливо, что $\sin \alpha = d/R$. По закону Снеллиуса угол преломления рассматримоего луча $\beta$ определяется соотношением $\sin \beta = n \sin \alpha$. Так как угол между нормалью к выпуклой поверхности линзы и ее оптической осью равен $\alpha$, то угол $\gamma$ между преломленным лучем и оптической осью линзы равен $\beta - \alpha$.

Расстояние до точки пересечения преломлённого луча с оптической осью линзы будем отсчитывать от вершины выпуклой поверхности линзы. Расстояние $h$ между проекцией точки преломления на оптическую ось и вершиной можно найти из теоремы Пифагора:
\begin{equation*}
	h = R - \sqrt{R^2 - d^2}.
\end{equation*}
Тогда координата фокуса для лучей на расстоянии $d$ от оптической оси равно
\begin{equation}
	x = \frac{d}{\tg \gamma} - h = \frac{d}{\tg \left( \arcsin \dfrac{n d}{R} - \arcsin \dfrac{d}{R} \right)} - \left( R - \sqrt{R^2 - d^2} \right).
	\label{eq:optics-aberr-x(d)}
\end{equation}
Введём обозначение $\delta \equiv d/R$ и разделим обе части полученного равенства на $R$, чтобы перейти к относительным единицам:
\begin{equation}
	\frac{x}{R}
	= \frac{\delta}{\tg \left( \arcsin n\delta - \arcsin \delta \right)} -  1 + \sqrt{1 - \delta^2}
	~\xrightarrow{\delta \ll 1}~  \frac{1}{n - 1} - \frac{\delta^2 n^2}{2(n-1)}.\footnote{\scriptsize Вывод приближения:
	\begin{multline*}
		\frac{\delta}{\tg \left( \arcsin n\delta - \arcsin \delta \right)} -  1 + \sqrt{1 - \delta^2} =\\
		= \frac{\delta}{\tg \left[ n \delta + \dfrac{n^3 \delta^3}{6} - \delta - \dfrac{ \delta^3}{6} + o(\delta^3) \right]} -  1 + \left(1 - \frac{\delta^2}{2} + o(\delta^2) \right) =\\
		= \frac{\delta}{\tg \left[ \delta(n-1) + \dfrac{(n^3-1) \delta^3}{6} + o(\delta^3) \right]} - \frac{\delta^2}{2} + o(\delta^2) =\\
		= \frac{\delta}{\delta(n-1) + \dfrac{(n^3-1) \delta^3}{6} + \dfrac{\delta^3}{3}(n-1)^3 + o(\delta^3)} - \frac{\delta^2}{2} + o(\delta^2) =\\
		= \frac{1/(n-1)}{1 + \dfrac{\delta^2}{6}(n^2 + n + 1) + \dfrac{\delta^2}{3}(n-1)^2 + o(\delta^2)} - \frac{\delta^2}{2} + o(\delta^2) =\\
		=\frac{1}{n-1} \left[1 - \frac{\delta^2}{6} (3n^2 -3n + 3) \right] - \frac{\delta^2}{2} + o(\delta^2)
		%	= \frac{1}{n-1} - \frac{\delta^2}{2(n-1)}(n^2 - n + 1 + n - 1) + o(\delta^2)
		\simeq \frac{1}{n-1} - \frac{\delta^2 n^2}{2(n-1)}.
	\end{multline*}
	}
\end{equation}

Найдём область определения функции $x(\delta)$. Прежде всего $\delta \geqslant 0$, потому что $d$~--- это расстояние от оптической оси, которое не может быть отрицательным. С другой стороны радиус линзы не может быть больше радиуса кривизны ее поверхности, следовательно, $\delta < 1$. Однако есть ещё одно условие, которое ограничивает $\delta$ сверху. Это эффект полного внутреннего отражения. Действительно, $\sin \beta$ не может быть больше единицы, следовательно, $\sin \alpha < \sfrac{1}{n}$, а значит, $\delta < \sfrac{1}{n}$. Для стекла, коэффициент преломления которого $n \approx 1.5$, получаем, что $\delta \in [0, \sfrac{2}{3})$.

\begin{wrapfigure}{r}{0.5\tw}
	\centering
	\vspace{-1pc}
	\begin{tikzpicture}
	\begin{axis}[
		height	=	4.5cm,
		width	=	6cm,
		xlabel	=	{$\lambda$, мкм},
		ylabel	=	{$x(\lambda)$},
		ylabel shift	= -1 cm,
		xmin = 0.3,
		xmax = 2.5,
		ymin = 1.8,
		ymax = 2.1,
		]
		
		\addplot[smooth, domain=0.3:2.5] table[x=l, y=x] {data/crown-dispersion.txt};
	\end{axis}
\end{tikzpicture}
\caption{}
\label{pic:crown-dispersion-x}	
\end{wrapfigure}
Как будет показано далее, во избежании проявления \imp{сферической аберрации}, используют линзы с маленьким относительным отверстием ($\forall \ll 1$). Следовательно, и $\delta \ll 1$, возьмем для примера значение $\delta = 0.1$. Для него, очевидно, можно использовать приближение для $x$, поэтому при заданном значении $\delta$ выражение для $x$ принимает вид:
\begin{equation*}
	x = \frac{1}{n-1} - \frac{0.01 n^2}{2(n-1)}.
\end{equation*}
Как видно из графика данной зависимости (см.~Рис.\,\ref{pic:crown-dispersion-x}), для оптического диапазона $x(\lambda)$ принимает значения от примерно 1.85 для коротковолновой (фиолетовой) части до примерно 1.95 для красного цвета. 

Чтобы компенсировать такой разбег совместно с собирающей линзой использую рассеивающую из другого материала. Объективы, где исправлена хроматическая аберрация для двух цветов и частично исправлена сферическая аберрация называют \term{ахроматами}; где хроматическая аберрация исправлена для трёх цветов, а также полностью исправлена сферическая аберрация~--- \term{апохроматами}; с более полной геометрической коррекцией~--- \term{апланатами}.

\paragraph{Сферическая аберрация}
В оптических системах, содержащих \change{сферические поверхности (линзы, зеркала)} может наблюдаться \imp{сферическая аберрация}. Суть такой аберрации состоит в том, что лучи, параллельные оптической оси, идущие на разном расстоянии от неё собираются в разных её местах. Это приводит к тому, что изображения точечных источников размываются.

\begin{wrapfigure}[12]{r}{0.55\tw}
	\centering
	\vspace{-.5pc}
	\begin{tikzpicture}
		\begin{axis}[
			height	=	5cm,
			width	=	6.5cm,
			xlabel	=	{$\delta$},
			ylabel	=	{$x(\delta)/R$},
			ylabel shift	= -1.1 cm,
			extra x ticks ={0.667},
			extra x tick labels={$\frac{1}{n}$},
			xmin=-.05,
			xmax=0.72,
			ymin=-.25,
			ymax=2.25,
			legend cell align=left,
			legend style={
			draw=none,
			fill=none,
			font=\scriptsize,
			at={(axis cs:0, 0.1)}, anchor=south west,
			row sep=.5pc,
			},
			]
			\addplot[smooth, gray] table[x=d, y=simple] {data/shere-aberrations-lens.txt};
			\addplot[smooth] table[x=d, y=x] {data/shere-aberrations-lens.txt};
			\addplot[dashes] coordinates { (0.667, -10) (0.667, 10)};
			\legend{
			$\left. \dfrac{x(\delta)}{R} \right|_{\delta \ll 1}$,
			$\dfrac{x(\delta)}{R}$,
			$\delta = \left.\dfrac{1}{n}\right|_{n=3/2}$,
			}
		\end{axis}
	\end{tikzpicture}
	\caption{}
	\label{pic:sphere-aberrations-lens-plot}
\end{wrapfigure}
Покажем наличие сферической аберрации для плосковыпуклой линзы. Рассмотрим полученное выше выражение \eqref{eq:optics-aberr-x(d)} и его приближение при $\delta \ll 1$. Зафиксируем в них $n=3/2$~--- характерное значение показателя преломления для стекла. Графики получаемых при этом зависимостей представлены на Рис.\,\ref{pic:sphere-aberrations-lens-plot}. Как видно из данных графиков, сферические аберрации проявляются уже на малых расстояниях от оптической оси. 

Чтобы показать важность сферических аберраций рассмотрим небольшой телескоп рефрактор с диаметром плосковы\-пук\-ло\-во\-го\linebreak стеклянного ($n \approx 3/2$) объектива $D = 50$~мм. Характерная точность фокусировки $\Delta x$ для таких маленьких телескопов составляет около 1~мм. Установим, при каком фокусном расстоянии такого телескопа точность фокусировки нивелирует сферическую аберрацию. 

При отсутствии сферической аберрации фокусное расстояние плосковыпуклового объектива $F = R/(n-1) = 2R$. Предельное значение $\delta$, которое нужно рассмотреть, соответствует лучам, проходящим через край объектива, следовательно, $\delta = D/(2R)$. Используя приближение, теперь можно записать выражение для требуемого $\Delta x$, чтобы найти необходимое для этого относительное отверситие $\forall$:
\begin{gather*}
	\Delta x = F - x(\delta) = F - x\left( \frac{D}{2R} \right),\\
	\Delta x = F - R\left( 2 - \frac{9 D^2}{4 \cdot 4R^2} \right),\\
	\Delta x = \frac{9D^2}{16R} = \frac{9D^2}{16F} = \frac{9}{16} D \forall;\\
	\therefore \forall = \frac{\Delta x \cdot 16}{9D} = 0.036.
\end{gather*}

\begin{wrapfigure}[9]{r}{0.5\tw}
	\centering
	\vspace{-.8pc}
	\begin{tikzpicture}
		\footnotesize
		
		\draw [decoration={snake, segment length=.7mm, amplitude=0.2mm}, decorate] (1.35, 1) arc(-31:15:0.26);
		\draw (.2, 1.135) arc(180:209:0.94);
		\draw (-2.18, 0) arc(0:15:0.94);
		
		\fill [lightgray] (1.5, -1.5) -- (1.5, 1.5) -- (1, 1.5) arc(20:-20:4.386) -- (1.5, -1.5);
		
		\draw [thick] (1, -1.5) arc(-20:20:4.386);
		\draw [semithick, dash pattern={on 5pt off 2pt on .5pt off 2pt}] (-3.8, 0) -- (1.7, 0);

		
		\draw [dashes] (-3.12, 0) -- (1.71, 1.29);
		
		\draw [semithick] (-3.8, 1.135) -- (1.12, 1.135) -- (-.85, 0);
		\draw [-latex] (-3.5, 1.135)-- (-1, 1.135);
		\draw [-latex] (1.12, 1.135) -- (-.06, 0.45);
		
		\draw [latex-latex] (-3.5, 0) -- (-3.5, 1.135);
		\draw [latex-latex] (-.85, -1.3) -- (1.27, -1.3);

		\draw (1.27, 0) -- (1.27, -1.5);
		\draw (-.85, 0) -- (-.85, -1.5);
		
		\draw (-3.5, 0.56) node[anchor=east] {$d$};
		\draw (-0.85, 0) node[anchor=north east] {$F$};
		\draw (-3.12, 0) node[anchor=south] {$C$};
		\draw (-1.2, 0.5) node[anchor=south] {$R$};
		\draw (0.27, -1.3) node[anchor=south] {$x_F(d)$};

		
		\draw (.2, 1) node[anchor=east] {$\alpha$};
		\draw (-2.2, .15) node[anchor=west] {$\alpha$};
		\draw (.3, .7) node[anchor=east] {$\alpha$};
	
		
		\draw [fill=white] (-0.85, 0) circle (.03); 
		\draw [fill=white] (-3.12, 0) circle (.03);
	\end{tikzpicture}
	\caption{}
	\label{pic:sphere-aberrations-mirrow}
\end{wrapfigure}
Найдем теперь величину аберрации сферического зеркала.\linebreak Пусть $R$~--- радиус кривизны зеркал. Рассмотрим луч, идущий параллельно оптической оси зеркала на расстоянии $d$ от неё. Он падает на зеркало под углом $\alpha$, причем $\sin \alpha = d/R$. В силу закона отражения: угол падения равен углу отражения, то есть угол отражения также равен $\alpha$. Кроме того, угол между нормалью к зеркалу в точке отражения и оптической осью зеркала также равен $\alpha$ как вертикальный. Следовательно треугольник {\slshape центр кривизны зеркала ($C$) -- точка отражения ($A$) -- точка пересечения отраженного луча с оптической осью (F)} является равнобедренным. Значит расстояние $x_F(d)$ от центра зеркала до <<фокуса>> $F$ можно найти как 
\begin{gather*}
	x_F(d) = R - \frac{R}{2} \cdot \frac{1}{\cos\alpha} = R - \frac{R}{2\sqrt{1 - \sin^2 \alpha}}  = R  - \frac{R}{2\sqrt{1 - \dfrac{d^2}{R^2}}};\\
	\left. \frac{x_F(d)}{R} \right|_{d \ll R} \simeq  1  - \frac{1}{2\left(1 - \dfrac{d^2}{2R^2} \right)} \simeq  1 - \frac{1}{2}\left(1 + \dfrac{d^2}{2R^2} \right)  = \frac{1}{2} -  \dfrac{d^2}{4R^2}.
\end{gather*}
\begin{wrapfigure}[12]{r}{0.55\tw}
	\centering
	\vspace{-.5pc}
	\begin{tikzpicture}
		\begin{axis}[
			height	=	5cm,
			width	=	6.5cm,
			xlabel	=	{$d/R$},
			ylabel	=	{$x(d)/R$},
			ylabel shift	= -1.1 cm,
			extra x ticks ={sqrt(2)/2},
			extra x tick labels={$\frac{\sqrt{2}}{2}$},
			xmin=-.05,
			xmax=0.85,
			ymin=.15,
			ymax=0.55,
			legend cell align=left,
			legend style={
			draw=none,
			fill=none,
			font=\scriptsize,
			at={(axis cs:0, .2)}, anchor=south west,
			row sep=.5pc,
			},
			]
			\addplot[smooth, gray] table[x=d, y=simple] {data/sphere-aberrations-mirrow.txt};
			\addplot[smooth] table[x=d, y=x] {data/sphere-aberrations-mirrow.txt};
			\addplot[dashes] coordinates { (sqrt(2)/2, -10) (sqrt(2)/2, 10)};
			\legend{
			$\left. \dfrac{x_F(d)}{R} \right|_{d \ll R}$,
			$\dfrac{x_F(d)}{R}$,
			$ \dfrac{d}{R} = \dfrac{\sqrt{2}}{2}$,
			}
		\end{axis}
	\end{tikzpicture}
	\caption{}
\end{wrapfigure}
Отсюда получается, что фокус сферического зеркала находится ровно между центром кривизны зеркала и центром этого зеркала. Однако в силу сферической аберрации возникает ошибка фокусировки порядка $d/R$, которая размывает изображение. Причём при $d > R\sqrt{2}/2$ лучи не <<разворачиваются>>, следовательно, не вносят вклада в изображение, так как приходят на приемник с другой стороны.

Для компенсации сферической аберрации используют различные линзы-корректоры, однако они помогают лишь частично избавиться от неё. \change{Поэтому в современных рефлекторах используются параболические зеркала, не подверженные сферическим аберрациям.}

%Напоследок нужно отметить, что сферические аберрации

\paragraph{Астигматизм} Ещё один вид аберраций оптических систем состоящий в разности радиусов кривизны оптических элементов в двух перпендикулярных направлениях. Такое возможно, например, в случае большой массы линзы или зеркала. Когда данный оптический элемент долгое время находится в вертикальном положении (оптическая ось горизонтальна), он деформируется: вдоль горизонтали радиус кривизны сохраняется, а по вертикали из-за сжатия уменьшается.

\imp{Астигматизм} проявляется в том, что пучок лучей, исходящих из какой-либо точки, после прохождения через оптическую систему собирается не в одной точке, а на двух взаимно перпендикулярных отрезках, расположенных на некотором расстоянии друг от друга. Изображения промежуточных сечений имеют форму эллипсов.

\paragraph{Кома} 

\begin{wrapfigure}[13]{l}{0.4\tw}
	\centering
	\vspace{-1pc}
	\includegraphics[width=0.4\tw]{img/optics-aberrations-coma.jpg}	
	\caption{Изображение <<хвоста>> Большой Медведицы, полученное с помощью широкоугольного объектива, страдающего ярко выраженной комой.}
	\label{pic:optics-aberrations-coma}
\end{wrapfigure}
Один из видов аберраций оптических систем~--- аберрация широкого пучка световых лучей, проходящий наклонно к оптической оси системы, как и \imp{сферическая аберрация}, обусловлена неодинаковым преломлением световых лучей различными участками линзовых компонент системы. Кома приводит к нарушению центрированности светового пучка. В результате такой аберрации изображение точки имеет вид несимметричного пятна (см.~Рис.\,\ref{pic:optics-aberrations-coma}), по форме напоминающего запятую (англ. {\itshape comma}).


































\subsection{Диффракция}

\begin{figure}[p]
	\centering
	\begin{subfigure}{\tw}
		\begin{tikzpicture}
			\begin{axis} [
				width			=	10cm,
				colormap 		= 	{GS}{rgb(0cm)=(.1, .1, .1)  rgb(1cm)	=	(1, 1, 1)},
				xlabel 			=	{$x$, $\frac{\lambda}{D}$},
				ylabel 			=	{$y$, $\frac{\lambda}{D}$},
				zlabel 			=	{$I/I_0$},
				ylabel shift 	= -.4 cm,
				xlabel shift 	= -.3 cm,
				ytick			= {-2,0,2},
				colorbar,
				colorbar style 	= {
				ytick 	= 	{0, .2, .4, .6, .8, 1.},
				}
				]
				
				\addplot3[
				samples				=	100,
				samples y			=	100,
				mesh,
				patch type			=	line,
				x filter/.code		=	\def\pgfmathresult{-5},
				smooth
				]
				table[x=x, y=y, z=I] {data/eiry-disk-x.txt};
				%
				\addplot3[
				samples			=	100,
				samples y		=	100,
				mesh,
				patch type		=	line,
				y filter/.code	=	\def\pgfmathresult{4.5},
				smooth
				]
				table[x=x, y=y, z=I] {data/eiry-disk-y.txt};
				
				\addplot3[surf] table[x=x, y=y, z=I] {data/eiry-disk.txt};
			\end{axis}
		\end{tikzpicture}
		\caption{}
		\label{}
	\end{subfigure}\\[2pc]
	\begin{subfigure}{\tw}
		\begin{tikzpicture}
			\begin{axis} [
				width			=	10cm,
				height			=	7.5cm,
				colormap 		= 	{GS}{rgb(0cm)=(.1, .1, .1)  rgb(1cm)	=	(1, 1, 1)},
				view			=	{0}{90},
				ytick 	= 	{-3, -2, ..., 3},
				colorbar,
				colorbar style 	= 	{
				ytick 	= 	{0, .2, .4, .6, .8, 1.},
				},
				xlabel 			=	{$x$, $\frac{\lambda}{D}$},
				ylabel 			=	{$y$, $\frac{\lambda}{D}$},
				]
				
				\addplot3[surf, shader=interp] table[x=x, y=y, z=I] {data/eiry-disk.txt};
			\end{axis}
		\end{tikzpicture}
		\caption{}
		\label{}
	\end{subfigure}
	\caption{}
\end{figure}

\begin{figure}[p]
	\centering
	\begin{subfigure}{\tw}
		\begin{tikzpicture}
			\begin{axis} [
				width			=	10cm,
				colormap 		= 	{GSW}{rgb(0cm)=(.1, .1, .1) rgb(.05cm)=(.99, .99, .99) rgb(1cm)	=	(1, 1, 1)},
				xlabel 			=	{$x$, $\frac{\lambda}{D}$},
				ylabel 			=	{$y$, $\frac{\lambda}{D}$},
				zlabel 			=	{$I/I_0$},
				ylabel shift 	= -.4 cm,
				xlabel shift 	= -.3 cm,
				ytick			= {-2,0,2},
				colorbar,
				colorbar style 	= {
				ytick 	= 	{0, .2, .4, .6, .8, 1.},
				}
				]
				
				\addplot3[
				samples				=	100,
				samples y			=	100,
				mesh,
				patch type			=	line,
				x filter/.code		=	\def\pgfmathresult{-5},
				smooth
				]
				table[x=x, y=y, z=I] {data/eiry-disk-x.txt};
				%
				\addplot3[
				samples			=	100,
				samples y		=	100,
				mesh,
				patch type		=	line,
				y filter/.code	=	\def\pgfmathresult{4.5},
				smooth
				]
				table[x=x, y=y, z=I] {data/eiry-disk-y.txt};
				
				\addplot3[surf] table[x=x, y=y, z=I] {data/eiry-disk.txt};
			\end{axis}
		\end{tikzpicture}
		\caption{}
		\label{}
	\end{subfigure}\\[2pc]
	\begin{subfigure}{\tw}
		\begin{tikzpicture}
			\begin{axis} [
				width			=	10cm,
				height			=	7.5cm,
				colormap 		= 	{GSW}{rgb(0cm)=(.1, .1, .1) rgb(.05cm)=(.99, .99, .99) rgb(1cm)	=	(1, 1, 1)},
				view			=	{0}{90},
				ytick 	= 	{-3, -2, ..., 3},
				colorbar,
				colorbar style 	= 	{
				ytick 	= 	{0, .2, .4, .6, .8, 1.},
				},
				xlabel 			=	{$x$, $\frac{\lambda}{D}$},
				ylabel 			=	{$y$, $\frac{\lambda}{D}$},
				]
				
				\addplot3[surf, shader=interp] table[x=x, y=y, z=I] {data/eiry-disk.txt};
			\end{axis}
		\end{tikzpicture}
		\caption{}
		\label{}
	\end{subfigure}
	\caption{}
\end{figure}

\begin{figure}[p]
	\begin{subfigure}[t]{\tw}
		\includegraphics[width=4.7cm]{eiry-disk-0}\hfill
		\begin{tikzpicture}
			\begin{axis}[
				height	=	4.125cm,
				width	=	5.5cm,
				xlabel	=	{$x$, $\frac{\lambda}{D}$},
				ylabel	=	{$I/I_0$},
				ylabel shift	= -1 cm,
				]
				
				\addplot[smooth] table[x=x, y=e0]{data/eiry-disk-profile.txt};
			\end{axis}
		\end{tikzpicture}
		\caption{Диффракционное изображение от одного источника}
	\end{subfigure}\\
	\begin{subfigure}[t]{\tw}
		\includegraphics[width=4.7cm]{eiry-disk-1}\hfill
		\begin{tikzpicture}
			\begin{axis}[
				height	=	4.125cm,
				width	=	5.5cm,
				xlabel	=	{$x$, $\frac{\lambda}{D}$},
				ylabel	=	{$I/I_0$},
				ylabel shift	= -1 cm,
				]
				
				\addplot[smooth] table[x=x, y=e1]{data/eiry-disk-profile.txt};
			\end{axis}
		\end{tikzpicture}
		\caption{Диффракционное изображение от двух источников с разделением~$1.22\lambda/D$}
	\end{subfigure}\\
	\begin{subfigure}{\tw}
		\includegraphics[width=4.7cm]{eiry-disk-2}\hfill
		\begin{tikzpicture}
			\begin{axis}[
				height	=	4.125cm,
				width	=	5.5cm,
				xlabel	=	{$x$, $\frac{\lambda}{D}$},
				ylabel	=	{$I/I_0$},
				ylabel shift	= -1 cm,
				]
				
				\addplot[smooth] table[x=x, y=e2]{data/eiry-disk-profile.txt};
			\end{axis}
		\end{tikzpicture}
		\caption{Диффракционное изображение от двух источников с разделением~$2 \cdot 1.22\lambda/D$}
	\end{subfigure}\\
	\begin{subfigure}{\tw}
		\includegraphics[width=4.7cm]{eiry-disk-3}\hfill
		\begin{tikzpicture}
			\begin{axis}[
				height	=	4.125cm,
				width	=	5.5cm,
				xlabel	=	{$x$, $\frac{\lambda}{D}$},
				ylabel	=	{$I/I_0$},
				ylabel shift	= -1 cm,
				]
				
				\addplot[smooth] table[x=x, y=e3]{data/eiry-disk-profile.txt};
			\end{axis}
		\end{tikzpicture}
		\caption{Диффракционное изображение от двух источников с разделением~$3 \cdot 1.22\lambda/D$}
	\end{subfigure}
	\caption{}
\end{figure}




\newpage
\section{Оптика}
\subsection{Телескоп}
\term{Телескоп}~--- устройство для наблюдения удаленных объектов. На данный момент существуют телескопы  для наблюдения во всех  диапазонах электро-магнитного излучения. По наблюдаемому диапазону телескопы делят на \imp{оптические} телескопы, \imp{радиотелескопы}, \imp{рентгеновские} телескопы и \imp{гамма-те\-ле\-скопы}. Каждый из классов в свою очередь содержит множество подклассов. Поговорим подробнее про оптические телескопы.

Оптические телескопы по своей схеме делятся на три типа: \imp{рефлекторы} (диоптрические), \imp{рефракторы} (катаптрические) и \imp{катадиоптрические}.

\vspace{-.3pc}
\begin{figure}[h!]
	\centering
	\begin{subfigure}{0.49\tw}
		\includegraphics[width = \tw]{Galiley}
		\caption{Рефрактор системы Галилея}
	\end{subfigure}
	\hfill
	\begin{subfigure}{0.49\tw}
		\includegraphics[width = \tw]{Kepler}
		\caption{Рефрактор системы Кеплера}
		\label{Kepler}
	\end{subfigure}
	\caption{Оптические схемы телескопов рефракторов}
\end{figure}
\term{Рефрактор} (линзовый телескоп)~---  оптический телескоп, в котором для собирания света используется система линз.

\vspace{-.3pc}
\begin{figure}[h!]
	\begin{subfigure}{0.49\tw}
		\includegraphics[width = \tw]{Cassigren.pdf}
		\caption{Рефлектор системы Кассегрена}
	\end{subfigure}
	\hfill
	\begin{subfigure}{0.49\tw}
		\includegraphics[width = \tw]{Gregory.pdf}
		\caption{Рефлектор системы Грегори}
		\label{Gregory}
	\end{subfigure}
	\vskip4pt
	\begin{subfigure}{0.49\tw}
		\includegraphics[width = \tw]{Newton}
		\caption{Рефлектор системы Ньютона}
	\end{subfigure}
	\hfill
	\begin{subfigure}{0.49\tw}
		\includegraphics[width = \tw]{Lomonosov.pdf}
		\caption{Рефлектор системы Ломоносова}
	\end{subfigure}
	\caption{Оптические схемы телескопов рефлекторов}
\end{figure}
\term{Рефлектор} (зеркальный телескоп)~---  оптический телескоп,  в котором светособирающими элементами являются зеркала.

\term{Катадиоптрический} (зеркально-линзовый) \term{телескоп}~--- оптический телескоп, в котором используется как система линз, так и зеркал.

\input{sections/optic.mounts.tex}
\input{sections/optic.zoom.tex}
\input{sections/optic.thin-lens.tex}
\input{sections/optic.snell-law.tex}
\subsection{Аберрации в оптике}
\paragraph{Хроматическая аберрация}
Пожалуй главным недостатком оптических схем, содержащий преломляющие оптические элементы (линзы; призмы, за исключением использования их для спектроскопии) являются \imp{хроматические аберрации}. Дело в том, что показатель преломления материала линзы зависит от длины волны падающего излучения. Это приводит к тому, что положения фокуса оптической системы зависит от длины волны излучения. При наблюдениях это проявляется как радужный ореол вокруг объектов, ухудшающий качество изображения.

\begin{wrapfigure}{r}{0.5\tw}
	\centering
	\vspace{-1pc}
	\begin{tikzpicture}
	\begin{axis}[
		height	=	4.5cm,
		width	=	6cm,
		xlabel	=	{$\lambda$, мкм},
		ylabel	=	{$n(\lambda)$},
		ylabel shift	= -1 cm,
		xmin = 0.3,
		xmax = 2.5,
		ymin = 1.48,
		ymax = 1.55,
		]
		
		\addplot[smooth, domain=0.3:2.5] table[x=l, y=n] {data/crown-dispersion.txt};
	\end{axis}
\end{tikzpicture}
\caption{}
\label{pic:crown-dispersion}	
\end{wrapfigure}
Найдем зависимость величины хроматической аберрации от величины дисперсии материала линзы. В качестве линзы рассмотрим плосковыпуклую линзу из оптического стекла~--- \imp{крона} (BK7). Зависимость $n(\lambda)$ её показателя преломления $n$ от длины волны $\lambda$ проходящего излучения представлена на графике (см.~Рис\,\ref{pic:crown-dispersion}). Важно отметить, здесь уже нельзя считать линзу тонкой, так как само по себе понятие тонкой линзы подразумевает отсутствие разного рода аберраций и условие фокусировки лучей в одной точке (фокусе), чего не происходит на практике.

\begin{wrapfigure}[9]{r}{0.63\tw}
	\centering
	\vspace{-.8pc}
	\begin{tikzpicture}
		\footnotesize
		
		\draw [decoration={snake, segment length=.7mm, amplitude=0.2mm}, decorate] (1.35, 1) arc(-31:15:0.26);
		\draw [double, line cap=butt] (.2, 1.135) arc(180:195:0.94);
		\draw  (2.7, 0) arc(180:149:0.3);
		
		\fill [lightgray] (.5, -1.5) -- (.5, 1.5) -- (1, 1.5) arc(20:-20:4.386) -- (.5, -1.5);
		
		\draw [thick] (.5, -1.5) -- (.5, 1.5);
		\draw [thick] (1, -1.5) arc(-20:20:4.386);
		\draw [semithick, dash pattern={on 5pt off 2pt on .5pt off 2pt}] (-3.8, 0) -- (3.2, 0);
%		\draw (-3.12, 0) node{$\times$};
		
		\draw [dashes] (-3.12, 0) -- (1.71, 1.29);
		
		\draw [semithick] (-3.8, 1.135) -- (1.12, 1.135) -- (3, 0);
		\draw [-latex] (-3.5, 1.135)-- (-1, 1.135);
		\draw [-latex] (1.12, 1.135) -- (2.25, 0.45);
		
		\draw [latex-latex] (-3.5, 0) -- (-3.5, 1.135);
		\draw [latex-latex] (1.27, -.5) -- (3, -.5);
		\draw [latex-latex] (1.12, -1.4) -- (3, -1.4);
		\draw [-latex] (0.5, -.5) -- (1.12, -.5);
		
		\draw (1.12, 1.135) -- (1.12, -1.5);
		\draw (1.27, 0) -- (1.27, -.6);
		\draw (3, 0) -- (3, -1.5);
		
		\draw (-3.5, 0.56) node[anchor=east] {$d$};
		\draw (-3.12, 0) node[anchor=north] {$C$};
		\draw (-.7, 0.6) node[anchor=north] {$R$};
		\draw (2.14, -.5) node[anchor=south] {$x$};
		\draw (2.14, -1.4) node[anchor=south] {$\frac{d}{\tg \gamma}$};
		
		\draw (.8, -1.5) node[anchor=south] {$n$};
		
		\draw (.2, 1) node[anchor=east] {$\alpha$};
		\draw (1.4, 1.07) node[anchor=west] {$\beta$};
		\draw (2.65, 0.15) node[anchor=east] {$\gamma$};
		
		
		\draw [fill=white] (3, 0) circle (.03); 
		\draw [fill=white] (-3.12, 0) circle (.03);
	\end{tikzpicture}
	\caption{}
	\label{pic:sphere-aberrations-lens}
\end{wrapfigure}
Итак, пусть радиус кривизны выпуклой поверхности рассматриваемой плос\-ко-вы\-пук\-лой линзы равен $R$. Рассмотрим также луч, параллельный оптической оси данной линзы, на расстоянии $d$ от этой оси (см.~Рис.\,\ref{pic:sphere-aberrations-lens}). Так как передняя поверхность линзы плоская, луч, попадая в линзу, не преломляется. Преломление происходит на выходе из линзы. Нетрудно показать, что для угла падения луча на заднюю поверхность линзы $\alpha$ справедливо, что $\sin \alpha = d/R$. По закону Снеллиуса угол преломления рассматримоего луча $\beta$ определяется соотношением $\sin \beta = n \sin \alpha$. Так как угол между нормалью к выпуклой поверхности линзы и ее оптической осью равен $\alpha$, то угол $\gamma$ между преломленным лучем и оптической осью линзы равен $\beta - \alpha$.

Расстояние до точки пересечения преломлённого луча с оптической осью линзы будем отсчитывать от вершины выпуклой поверхности линзы. Расстояние $h$ между проекцией точки преломления на оптическую ось и вершиной можно найти из теоремы Пифагора:
\begin{equation*}
	h = R - \sqrt{R^2 - d^2}.
\end{equation*}
Тогда координата фокуса для лучей на расстоянии $d$ от оптической оси равно
\begin{equation}
	x = \frac{d}{\tg \gamma} - h = \frac{d}{\tg \left( \arcsin \dfrac{n d}{R} - \arcsin \dfrac{d}{R} \right)} - \left( R - \sqrt{R^2 - d^2} \right).
	\label{eq:optics-aberr-x(d)}
\end{equation}
Введём обозначение $\delta \equiv d/R$ и разделим обе части полученного равенства на $R$, чтобы перейти к относительным единицам:
\begin{equation}
	\frac{x}{R}
	= \frac{\delta}{\tg \left( \arcsin n\delta - \arcsin \delta \right)} -  1 + \sqrt{1 - \delta^2}
	~\xrightarrow{\delta \ll 1}~  \frac{1}{n - 1} - \frac{\delta^2 n^2}{2(n-1)}.\footnote{\scriptsize Вывод приближения:
	\begin{multline*}
		\frac{\delta}{\tg \left( \arcsin n\delta - \arcsin \delta \right)} -  1 + \sqrt{1 - \delta^2} =\\
		= \frac{\delta}{\tg \left[ n \delta + \dfrac{n^3 \delta^3}{6} - \delta - \dfrac{ \delta^3}{6} + o(\delta^3) \right]} -  1 + \left(1 - \frac{\delta^2}{2} + o(\delta^2) \right) =\\
		= \frac{\delta}{\tg \left[ \delta(n-1) + \dfrac{(n^3-1) \delta^3}{6} + o(\delta^3) \right]} - \frac{\delta^2}{2} + o(\delta^2) =\\
		= \frac{\delta}{\delta(n-1) + \dfrac{(n^3-1) \delta^3}{6} + \dfrac{\delta^3}{3}(n-1)^3 + o(\delta^3)} - \frac{\delta^2}{2} + o(\delta^2) =\\
		= \frac{1/(n-1)}{1 + \dfrac{\delta^2}{6}(n^2 + n + 1) + \dfrac{\delta^2}{3}(n-1)^2 + o(\delta^2)} - \frac{\delta^2}{2} + o(\delta^2) =\\
		=\frac{1}{n-1} \left[1 - \frac{\delta^2}{6} (3n^2 -3n + 3) \right] - \frac{\delta^2}{2} + o(\delta^2)
		%	= \frac{1}{n-1} - \frac{\delta^2}{2(n-1)}(n^2 - n + 1 + n - 1) + o(\delta^2)
		\simeq \frac{1}{n-1} - \frac{\delta^2 n^2}{2(n-1)}.
	\end{multline*}
	}
\end{equation}

Найдём область определения функции $x(\delta)$. Прежде всего $\delta \geqslant 0$, потому что $d$~--- это расстояние от оптической оси, которое не может быть отрицательным. С другой стороны радиус линзы не может быть больше радиуса кривизны ее поверхности, следовательно, $\delta < 1$. Однако есть ещё одно условие, которое ограничивает $\delta$ сверху. Это эффект полного внутреннего отражения. Действительно, $\sin \beta$ не может быть больше единицы, следовательно, $\sin \alpha < \sfrac{1}{n}$, а значит, $\delta < \sfrac{1}{n}$. Для стекла, коэффициент преломления которого $n \approx 1.5$, получаем, что $\delta \in [0, \sfrac{2}{3})$.

\begin{wrapfigure}{r}{0.5\tw}
	\centering
	\vspace{-1pc}
	\begin{tikzpicture}
	\begin{axis}[
		height	=	4.5cm,
		width	=	6cm,
		xlabel	=	{$\lambda$, мкм},
		ylabel	=	{$x(\lambda)$},
		ylabel shift	= -1 cm,
		xmin = 0.3,
		xmax = 2.5,
		ymin = 1.8,
		ymax = 2.1,
		]
		
		\addplot[smooth, domain=0.3:2.5] table[x=l, y=x] {data/crown-dispersion.txt};
	\end{axis}
\end{tikzpicture}
\caption{}
\label{pic:crown-dispersion-x}	
\end{wrapfigure}
Как будет показано далее, во избежании проявления \imp{сферической аберрации}, используют линзы с маленьким относительным отверстием ($\forall \ll 1$). Следовательно, и $\delta \ll 1$, возьмем для примера значение $\delta = 0.1$. Для него, очевидно, можно использовать приближение для $x$, поэтому при заданном значении $\delta$ выражение для $x$ принимает вид:
\begin{equation*}
	x = \frac{1}{n-1} - \frac{0.01 n^2}{2(n-1)}.
\end{equation*}
Как видно из графика данной зависимости (см.~Рис.\,\ref{pic:crown-dispersion-x}), для оптического диапазона $x(\lambda)$ принимает значения от примерно 1.85 для коротковолновой (фиолетовой) части до примерно 1.95 для красного цвета. 

Чтобы компенсировать такой разбег совместно с собирающей линзой использую рассеивающую из другого материала. Объективы, где исправлена хроматическая аберрация для двух цветов и частично исправлена сферическая аберрация называют \term{ахроматами}; где хроматическая аберрация исправлена для трёх цветов, а также полностью исправлена сферическая аберрация~--- \term{апохроматами}; с более полной геометрической коррекцией~--- \term{апланатами}.

\paragraph{Сферическая аберрация}
В оптических системах, содержащих \change{сферические поверхности (линзы, зеркала)} может наблюдаться \imp{сферическая аберрация}. Суть такой аберрации состоит в том, что лучи, параллельные оптической оси, идущие на разном расстоянии от неё собираются в разных её местах. Это приводит к тому, что изображения точечных источников размываются.

\begin{wrapfigure}[12]{r}{0.55\tw}
	\centering
	\vspace{-.5pc}
	\begin{tikzpicture}
		\begin{axis}[
			height	=	5cm,
			width	=	6.5cm,
			xlabel	=	{$\delta$},
			ylabel	=	{$x(\delta)/R$},
			ylabel shift	= -1.1 cm,
			extra x ticks ={0.667},
			extra x tick labels={$\frac{1}{n}$},
			xmin=-.05,
			xmax=0.72,
			ymin=-.25,
			ymax=2.25,
			legend cell align=left,
			legend style={
			draw=none,
			fill=none,
			font=\scriptsize,
			at={(axis cs:0, 0.1)}, anchor=south west,
			row sep=.5pc,
			},
			]
			\addplot[smooth, gray] table[x=d, y=simple] {data/shere-aberrations-lens.txt};
			\addplot[smooth] table[x=d, y=x] {data/shere-aberrations-lens.txt};
			\addplot[dashes] coordinates { (0.667, -10) (0.667, 10)};
			\legend{
			$\left. \dfrac{x(\delta)}{R} \right|_{\delta \ll 1}$,
			$\dfrac{x(\delta)}{R}$,
			$\delta = \left.\dfrac{1}{n}\right|_{n=3/2}$,
			}
		\end{axis}
	\end{tikzpicture}
	\caption{}
	\label{pic:sphere-aberrations-lens-plot}
\end{wrapfigure}
Покажем наличие сферической аберрации для плосковыпуклой линзы. Рассмотрим полученное выше выражение \eqref{eq:optics-aberr-x(d)} и его приближение при $\delta \ll 1$. Зафиксируем в них $n=3/2$~--- характерное значение показателя преломления для стекла. Графики получаемых при этом зависимостей представлены на Рис.\,\ref{pic:sphere-aberrations-lens-plot}. Как видно из данных графиков, сферические аберрации проявляются уже на малых расстояниях от оптической оси. 

Чтобы показать важность сферических аберраций рассмотрим небольшой телескоп рефрактор с диаметром плосковы\-пук\-ло\-во\-го\linebreak стеклянного ($n \approx 3/2$) объектива $D = 50$~мм. Характерная точность фокусировки $\Delta x$ для таких маленьких телескопов составляет около 1~мм. Установим, при каком фокусном расстоянии такого телескопа точность фокусировки нивелирует сферическую аберрацию. 

При отсутствии сферической аберрации фокусное расстояние плосковыпуклового объектива $F = R/(n-1) = 2R$. Предельное значение $\delta$, которое нужно рассмотреть, соответствует лучам, проходящим через край объектива, следовательно, $\delta = D/(2R)$. Используя приближение, теперь можно записать выражение для требуемого $\Delta x$, чтобы найти необходимое для этого относительное отверситие $\forall$:
\begin{gather*}
	\Delta x = F - x(\delta) = F - x\left( \frac{D}{2R} \right),\\
	\Delta x = F - R\left( 2 - \frac{9 D^2}{4 \cdot 4R^2} \right),\\
	\Delta x = \frac{9D^2}{16R} = \frac{9D^2}{16F} = \frac{9}{16} D \forall;\\
	\therefore \forall = \frac{\Delta x \cdot 16}{9D} = 0.036.
\end{gather*}

\begin{wrapfigure}[9]{r}{0.5\tw}
	\centering
	\vspace{-.8pc}
	\begin{tikzpicture}
		\footnotesize
		
		\draw [decoration={snake, segment length=.7mm, amplitude=0.2mm}, decorate] (1.35, 1) arc(-31:15:0.26);
		\draw (.2, 1.135) arc(180:209:0.94);
		\draw (-2.18, 0) arc(0:15:0.94);
		
		\fill [lightgray] (1.5, -1.5) -- (1.5, 1.5) -- (1, 1.5) arc(20:-20:4.386) -- (1.5, -1.5);
		
		\draw [thick] (1, -1.5) arc(-20:20:4.386);
		\draw [semithick, dash pattern={on 5pt off 2pt on .5pt off 2pt}] (-3.8, 0) -- (1.7, 0);

		
		\draw [dashes] (-3.12, 0) -- (1.71, 1.29);
		
		\draw [semithick] (-3.8, 1.135) -- (1.12, 1.135) -- (-.85, 0);
		\draw [-latex] (-3.5, 1.135)-- (-1, 1.135);
		\draw [-latex] (1.12, 1.135) -- (-.06, 0.45);
		
		\draw [latex-latex] (-3.5, 0) -- (-3.5, 1.135);
		\draw [latex-latex] (-.85, -1.3) -- (1.27, -1.3);

		\draw (1.27, 0) -- (1.27, -1.5);
		\draw (-.85, 0) -- (-.85, -1.5);
		
		\draw (-3.5, 0.56) node[anchor=east] {$d$};
		\draw (-0.85, 0) node[anchor=north east] {$F$};
		\draw (-3.12, 0) node[anchor=south] {$C$};
		\draw (-1.2, 0.5) node[anchor=south] {$R$};
		\draw (0.27, -1.3) node[anchor=south] {$x_F(d)$};

		
		\draw (.2, 1) node[anchor=east] {$\alpha$};
		\draw (-2.2, .15) node[anchor=west] {$\alpha$};
		\draw (.3, .7) node[anchor=east] {$\alpha$};
	
		
		\draw [fill=white] (-0.85, 0) circle (.03); 
		\draw [fill=white] (-3.12, 0) circle (.03);
	\end{tikzpicture}
	\caption{}
	\label{pic:sphere-aberrations-mirrow}
\end{wrapfigure}
Найдем теперь величину аберрации сферического зеркала.\linebreak Пусть $R$~--- радиус кривизны зеркал. Рассмотрим луч, идущий параллельно оптической оси зеркала на расстоянии $d$ от неё. Он падает на зеркало под углом $\alpha$, причем $\sin \alpha = d/R$. В силу закона отражения: угол падения равен углу отражения, то есть угол отражения также равен $\alpha$. Кроме того, угол между нормалью к зеркалу в точке отражения и оптической осью зеркала также равен $\alpha$ как вертикальный. Следовательно треугольник {\slshape центр кривизны зеркала ($C$) -- точка отражения ($A$) -- точка пересечения отраженного луча с оптической осью (F)} является равнобедренным. Значит расстояние $x_F(d)$ от центра зеркала до <<фокуса>> $F$ можно найти как 
\begin{gather*}
	x_F(d) = R - \frac{R}{2} \cdot \frac{1}{\cos\alpha} = R - \frac{R}{2\sqrt{1 - \sin^2 \alpha}}  = R  - \frac{R}{2\sqrt{1 - \dfrac{d^2}{R^2}}};\\
	\left. \frac{x_F(d)}{R} \right|_{d \ll R} \simeq  1  - \frac{1}{2\left(1 - \dfrac{d^2}{2R^2} \right)} \simeq  1 - \frac{1}{2}\left(1 + \dfrac{d^2}{2R^2} \right)  = \frac{1}{2} -  \dfrac{d^2}{4R^2}.
\end{gather*}
\begin{wrapfigure}[12]{r}{0.55\tw}
	\centering
	\vspace{-.5pc}
	\begin{tikzpicture}
		\begin{axis}[
			height	=	5cm,
			width	=	6.5cm,
			xlabel	=	{$d/R$},
			ylabel	=	{$x(d)/R$},
			ylabel shift	= -1.1 cm,
			extra x ticks ={sqrt(2)/2},
			extra x tick labels={$\frac{\sqrt{2}}{2}$},
			xmin=-.05,
			xmax=0.85,
			ymin=.15,
			ymax=0.55,
			legend cell align=left,
			legend style={
			draw=none,
			fill=none,
			font=\scriptsize,
			at={(axis cs:0, .2)}, anchor=south west,
			row sep=.5pc,
			},
			]
			\addplot[smooth, gray] table[x=d, y=simple] {data/sphere-aberrations-mirrow.txt};
			\addplot[smooth] table[x=d, y=x] {data/sphere-aberrations-mirrow.txt};
			\addplot[dashes] coordinates { (sqrt(2)/2, -10) (sqrt(2)/2, 10)};
			\legend{
			$\left. \dfrac{x_F(d)}{R} \right|_{d \ll R}$,
			$\dfrac{x_F(d)}{R}$,
			$ \dfrac{d}{R} = \dfrac{\sqrt{2}}{2}$,
			}
		\end{axis}
	\end{tikzpicture}
	\caption{}
\end{wrapfigure}
Отсюда получается, что фокус сферического зеркала находится ровно между центром кривизны зеркала и центром этого зеркала. Однако в силу сферической аберрации возникает ошибка фокусировки порядка $d/R$, которая размывает изображение. Причём при $d > R\sqrt{2}/2$ лучи не <<разворачиваются>>, следовательно, не вносят вклада в изображение, так как приходят на приемник с другой стороны.

Для компенсации сферической аберрации используют различные линзы-корректоры, однако они помогают лишь частично избавиться от неё. \change{Поэтому в современных рефлекторах используются параболические зеркала, не подверженные сферическим аберрациям.}

%Напоследок нужно отметить, что сферические аберрации

\paragraph{Астигматизм} Ещё один вид аберраций оптических систем состоящий в разности радиусов кривизны оптических элементов в двух перпендикулярных направлениях. Такое возможно, например, в случае большой массы линзы или зеркала. Когда данный оптический элемент долгое время находится в вертикальном положении (оптическая ось горизонтальна), он деформируется: вдоль горизонтали радиус кривизны сохраняется, а по вертикали из-за сжатия уменьшается.

\imp{Астигматизм} проявляется в том, что пучок лучей, исходящих из какой-либо точки, после прохождения через оптическую систему собирается не в одной точке, а на двух взаимно перпендикулярных отрезках, расположенных на некотором расстоянии друг от друга. Изображения промежуточных сечений имеют форму эллипсов.

\paragraph{Кома} 

\begin{wrapfigure}[13]{l}{0.4\tw}
	\centering
	\vspace{-1pc}
	\includegraphics[width=0.4\tw]{img/optics-aberrations-coma.jpg}	
	\caption{Изображение <<хвоста>> Большой Медведицы, полученное с помощью широкоугольного объектива, страдающего ярко выраженной комой.}
	\label{pic:optics-aberrations-coma}
\end{wrapfigure}
Один из видов аберраций оптических систем~--- аберрация широкого пучка световых лучей, проходящий наклонно к оптической оси системы, как и \imp{сферическая аберрация}, обусловлена неодинаковым преломлением световых лучей различными участками линзовых компонент системы. Кома приводит к нарушению центрированности светового пучка. В результате такой аберрации изображение точки имеет вид несимметричного пятна (см.~Рис.\,\ref{pic:optics-aberrations-coma}), по форме напоминающего запятую (англ. {\itshape comma}).


































\subsection{Диффракция}

\begin{figure}[p]
	\centering
	\begin{subfigure}{\tw}
		\begin{tikzpicture}
			\begin{axis} [
				width			=	10cm,
				colormap 		= 	{GS}{rgb(0cm)=(.1, .1, .1)  rgb(1cm)	=	(1, 1, 1)},
				xlabel 			=	{$x$, $\frac{\lambda}{D}$},
				ylabel 			=	{$y$, $\frac{\lambda}{D}$},
				zlabel 			=	{$I/I_0$},
				ylabel shift 	= -.4 cm,
				xlabel shift 	= -.3 cm,
				ytick			= {-2,0,2},
				colorbar,
				colorbar style 	= {
				ytick 	= 	{0, .2, .4, .6, .8, 1.},
				}
				]
				
				\addplot3[
				samples				=	100,
				samples y			=	100,
				mesh,
				patch type			=	line,
				x filter/.code		=	\def\pgfmathresult{-5},
				smooth
				]
				table[x=x, y=y, z=I] {data/eiry-disk-x.txt};
				%
				\addplot3[
				samples			=	100,
				samples y		=	100,
				mesh,
				patch type		=	line,
				y filter/.code	=	\def\pgfmathresult{4.5},
				smooth
				]
				table[x=x, y=y, z=I] {data/eiry-disk-y.txt};
				
				\addplot3[surf] table[x=x, y=y, z=I] {data/eiry-disk.txt};
			\end{axis}
		\end{tikzpicture}
		\caption{}
		\label{}
	\end{subfigure}\\[2pc]
	\begin{subfigure}{\tw}
		\begin{tikzpicture}
			\begin{axis} [
				width			=	10cm,
				height			=	7.5cm,
				colormap 		= 	{GS}{rgb(0cm)=(.1, .1, .1)  rgb(1cm)	=	(1, 1, 1)},
				view			=	{0}{90},
				ytick 	= 	{-3, -2, ..., 3},
				colorbar,
				colorbar style 	= 	{
				ytick 	= 	{0, .2, .4, .6, .8, 1.},
				},
				xlabel 			=	{$x$, $\frac{\lambda}{D}$},
				ylabel 			=	{$y$, $\frac{\lambda}{D}$},
				]
				
				\addplot3[surf, shader=interp] table[x=x, y=y, z=I] {data/eiry-disk.txt};
			\end{axis}
		\end{tikzpicture}
		\caption{}
		\label{}
	\end{subfigure}
	\caption{}
\end{figure}

\begin{figure}[p]
	\centering
	\begin{subfigure}{\tw}
		\begin{tikzpicture}
			\begin{axis} [
				width			=	10cm,
				colormap 		= 	{GSW}{rgb(0cm)=(.1, .1, .1) rgb(.05cm)=(.99, .99, .99) rgb(1cm)	=	(1, 1, 1)},
				xlabel 			=	{$x$, $\frac{\lambda}{D}$},
				ylabel 			=	{$y$, $\frac{\lambda}{D}$},
				zlabel 			=	{$I/I_0$},
				ylabel shift 	= -.4 cm,
				xlabel shift 	= -.3 cm,
				ytick			= {-2,0,2},
				colorbar,
				colorbar style 	= {
				ytick 	= 	{0, .2, .4, .6, .8, 1.},
				}
				]
				
				\addplot3[
				samples				=	100,
				samples y			=	100,
				mesh,
				patch type			=	line,
				x filter/.code		=	\def\pgfmathresult{-5},
				smooth
				]
				table[x=x, y=y, z=I] {data/eiry-disk-x.txt};
				%
				\addplot3[
				samples			=	100,
				samples y		=	100,
				mesh,
				patch type		=	line,
				y filter/.code	=	\def\pgfmathresult{4.5},
				smooth
				]
				table[x=x, y=y, z=I] {data/eiry-disk-y.txt};
				
				\addplot3[surf] table[x=x, y=y, z=I] {data/eiry-disk.txt};
			\end{axis}
		\end{tikzpicture}
		\caption{}
		\label{}
	\end{subfigure}\\[2pc]
	\begin{subfigure}{\tw}
		\begin{tikzpicture}
			\begin{axis} [
				width			=	10cm,
				height			=	7.5cm,
				colormap 		= 	{GSW}{rgb(0cm)=(.1, .1, .1) rgb(.05cm)=(.99, .99, .99) rgb(1cm)	=	(1, 1, 1)},
				view			=	{0}{90},
				ytick 	= 	{-3, -2, ..., 3},
				colorbar,
				colorbar style 	= 	{
				ytick 	= 	{0, .2, .4, .6, .8, 1.},
				},
				xlabel 			=	{$x$, $\frac{\lambda}{D}$},
				ylabel 			=	{$y$, $\frac{\lambda}{D}$},
				]
				
				\addplot3[surf, shader=interp] table[x=x, y=y, z=I] {data/eiry-disk.txt};
			\end{axis}
		\end{tikzpicture}
		\caption{}
		\label{}
	\end{subfigure}
	\caption{}
\end{figure}

\begin{figure}[p]
	\begin{subfigure}[t]{\tw}
		\includegraphics[width=4.7cm]{eiry-disk-0}\hfill
		\begin{tikzpicture}
			\begin{axis}[
				height	=	4.125cm,
				width	=	5.5cm,
				xlabel	=	{$x$, $\frac{\lambda}{D}$},
				ylabel	=	{$I/I_0$},
				ylabel shift	= -1 cm,
				]
				
				\addplot[smooth] table[x=x, y=e0]{data/eiry-disk-profile.txt};
			\end{axis}
		\end{tikzpicture}
		\caption{Диффракционное изображение от одного источника}
	\end{subfigure}\\
	\begin{subfigure}[t]{\tw}
		\includegraphics[width=4.7cm]{eiry-disk-1}\hfill
		\begin{tikzpicture}
			\begin{axis}[
				height	=	4.125cm,
				width	=	5.5cm,
				xlabel	=	{$x$, $\frac{\lambda}{D}$},
				ylabel	=	{$I/I_0$},
				ylabel shift	= -1 cm,
				]
				
				\addplot[smooth] table[x=x, y=e1]{data/eiry-disk-profile.txt};
			\end{axis}
		\end{tikzpicture}
		\caption{Диффракционное изображение от двух источников с разделением~$1.22\lambda/D$}
	\end{subfigure}\\
	\begin{subfigure}{\tw}
		\includegraphics[width=4.7cm]{eiry-disk-2}\hfill
		\begin{tikzpicture}
			\begin{axis}[
				height	=	4.125cm,
				width	=	5.5cm,
				xlabel	=	{$x$, $\frac{\lambda}{D}$},
				ylabel	=	{$I/I_0$},
				ylabel shift	= -1 cm,
				]
				
				\addplot[smooth] table[x=x, y=e2]{data/eiry-disk-profile.txt};
			\end{axis}
		\end{tikzpicture}
		\caption{Диффракционное изображение от двух источников с разделением~$2 \cdot 1.22\lambda/D$}
	\end{subfigure}\\
	\begin{subfigure}{\tw}
		\includegraphics[width=4.7cm]{eiry-disk-3}\hfill
		\begin{tikzpicture}
			\begin{axis}[
				height	=	4.125cm,
				width	=	5.5cm,
				xlabel	=	{$x$, $\frac{\lambda}{D}$},
				ylabel	=	{$I/I_0$},
				ylabel shift	= -1 cm,
				]
				
				\addplot[smooth] table[x=x, y=e3]{data/eiry-disk-profile.txt};
			\end{axis}
		\end{tikzpicture}
		\caption{Диффракционное изображение от двух источников с разделением~$3 \cdot 1.22\lambda/D$}
	\end{subfigure}
	\caption{}
\end{figure}




\section{Сферическая астрономия}
\section{Сферическая астрономия}
\section{Сферическая астрономия}
\input{sections/spher.astro/spher.astro.coordin-sys}
\input{sections/spher.astro/spher.astro.sky-phenom}
\input{sections/spher.astro/spher.astro.spher-trigonom.tex}
\input{sections/spher.astro/spher.astro.sun-time.tex}
\input{sections/spher.astro/spher.astro.change-coordin.tex}
\input{sections/spher.astro/spher.astro.refrac.tex}
\input{sections/spher.astro/spher.astro.dusk}
\subsection{Суточное вращение небесной сферы}
Вследствие вращения Земли вокруг своей оси для наблюдателя на поверхности небесные объекты совершают суточное движение параллельно небесному экватору, плоскость которого совпадает с плоскостью экватора Земли. Очевидно, в ходе такого движения высота светил постоянно меняется и в некоторые моменты времени достигает своего максимального и минимального значения. 

\term{Верхняя} и \term{нижняя кульминация}~--- моменты пересечения светилом небесного меридиана, причём при верхней кульминации светило имеет наибольшую высоту, а при нижней~--- наименьшую.

Высота светила в верхней и нижней кульминации со склонением $|\delta| < |\varphi|$, соответственно:
\begin{equation}
h_{\text{в}}= 90^\circ - \varphi + \delta, \quad\quad
h_{\text{н}}= - 90^\circ + \varphi  + \delta.
\end{equation}

Если же светило имеет склонение $|\delta| > |\varphi|$, то высота в верхней и нижней кульминации вычисляется так:
\begin{equation}
h_{\text{в}}= 90^\circ + \varphi - \delta, \quad\quad
h_{\text{н}}= - 90^\circ -\varphi - \delta.
\end{equation}

Из формул для высоты в нижней кульминации вытекает условие, определяющее, пересекает ли звезда горизонт:
\begin{equation}
\begin{cases}
	h_\text{в}= +90^\circ - |\varphi + \delta| > 0^\circ,\\
	h_\text{н} = - 90^\circ + |\varphi + \delta| < 0^\circ;	
\end{cases}
\quad \Longleftrightarrow \quad~~ |\delta|< 90^{\circ} - |\varphi|.
\end{equation}

Используя формулы сферической тригонометрии (см.\,\ref{sec:spher-trig}), можно выразить зависимость часового угла светила от его зенитного расстояния:
\begin{equation}
\cos t=\frac{\cos z-\sin\varphi\sin\delta}{\cos\varphi\cos\delta}. 
\end{equation}
Отсюда следует, что для часового угла захода и восхода светила справедливо равенство:
\begin{equation}
	\cos t_{\uparrow\downarrow}=-\tg\varphi\cdot\tg\delta.
\end{equation} 

Аналогично, для вычисления азимута светила верна формула
\begin{equation}
\cos A=\frac{\cos\delta\cos t-\cos\varphi\cos z}{\sin\varphi\sin z}.
\end{equation}
Следовательно, азимуты точек восхода и захода
\begin{equation}
	A_\uparrow = \arccos \left(-\dfrac{\sin\delta}{\cos \varphi} \right)\quad\text{и}\quad A_\downarrow = - A_\uparrow.
\end{equation}

\term{Звёздное время}~$z$~--- часовой угол точки весеннего равноденствия. Из определений прямого восхождения и часового угла следует справедливость равенства\begin{equation}
z = \alpha + t.
\end{equation}
\subsection{Сферическая тригонометрия}
\label{sec:spher-trig}
\begin{wrapfigure}[10]{r}{.3\tw}
	\centering
	\vspace{-1pc}
 	\includegraphics[width=0.3\textwidth]{spher-trigonom}
 	\caption{Сферический треугольник}
\end{wrapfigure}
Для решения некоторых задач астрономии, связанных с видимыми положениями небесных тел, требуются знания о сферической тригонометрии. \imp{Сферический треугольник}~--- фигура на поверхности сферы, состоящая из трёх точек и трёх дуг больших кругов, соединяющих эти точки. Пусть $A$, $B$ и $C$~--- углы сферического треугольника, а $a$, $b$ и $c$~--- его стороны.

Сферические треугольники обладают следующими свойствами:
\begin{enumerate}
\item Два сферических треугольника равны, если они подобны.
\item Каждая сторона меньше суммы двух других сторон и больше их разности.
\item Сумма всех сторон $a+b+c$ всегда меньше $2\pi$.
\item Сумма углов сферического треугольника $\pi < A + B + C < 3\pi$.
\item Разность суммы двух углов и третьего угла меньше $\pi$
\end{enumerate}

Площадь сферического треугольника определяется по формуле:
\begin{equation}
	S = R^2( A + B + C - \pi),
\end{equation}
где $A + B + C - \pi$~--- \imp{сферический избыток}.

Рассмотрим сферический треугольник $ABC$, радиус векторы вершин соответственно $\vec{a}$, $\vec{b}$ и $\vec{c}$. причем из определения сферы $|\vec{a}| = |\vec{b}| = |\vec{c}| = r$. Пусть против вершин $A$, $B$ и $C$ лежат стороны с угловой мерой $a$, $b$ и $c$ соответсвенно. Повернем сферические координаты и нормируем так, чтобы $\vec{a} = (0, 0, 1)$, $\vec{b} = (\sin c, 0, \cos c)$, тогда $ \vec{c} = (\sin b \cos A, \sin b \sin A, \cos b)$.
	
	Теперь запишем выражение для $\scalar{b}{c}$:
	\begin{equation}
		\scalar{b}{c} = \cos a = \sin c \sin b \cos A + \cos c \cos b.
		\label{eq:spher-astro-cos-1}
	\end{equation}
	Аналогично,
	\begin{gather}
		\scalar{a}{c} = \cos b = \sin a \sin c \cos B +  \cos a \cos c,\\
		\scalar{a}{b} = \cos c = \sin a \sin b \cos C + \cos a \cos b.
		\label{eq:spher-astro-cos-1-1}
	\end{gather}
	Выразим отсюда $\cos A$:
	\begin{equation}
		\cos A = \frac{\cos a - \cos c \cos b}{\sin c \sin b}.
		\label{eq:spher-astro-cos-2}
	\end{equation}
	Формулы \eqref{eq:spher-astro-cos-1}\,--\,\eqref{eq:spher-astro-cos-2} называются \term{сферической теоремой косинусов} \imp{для стороны} \eqref{eq:spher-astro-cos-1}\,--\,\eqref{eq:spher-astro-cos-1-1} и, соответственно \imp{для угла} \eqref{eq:spher-astro-cos-2}.
	
	Из основного тригонометрического тождества имеем:
	\begin{multline*}
		\sin^2 A = 1 - \cos^2 A = 1 - \left[ \frac{\cos a - \cos c \cos b}{\sin c \sin b} \right]^2 = \\
		= \frac{\sin^2 c \sin^2 b - \cos^2 a + 2\cos a \cos c \cos b - \cos^2 c \cos^2 b}{\sin^2 c \sin^2 b}=\\
		= \frac{(1 - \cos^2 c)(1 -  \cos^2 b) - \cos^2 a + 2\cos a \cos c \cos b - \cos^2 c \cos^2 b}{\sin^2 c \sin^2 b}=\\
		= \frac{1 - \cos^2 c - \cos^2 b + \cos^2 c \cos^2 b -\cos^2 a}{\sin^2 c \sin^2 b} + \\
		+ \frac{2\cos a \cos c \cos b - \cos^2 c \cos^2 b}{\sin^2 c \sin^2 b} = \\
		= \frac{1 - \cos^2 c - \cos^2 b - \cos^2 a + 2\cos a \cos c \cos b}{\sin^2 c \sin^2 b}.
	\end{multline*}
	Извлекая квадратный корень из левой и правой части и деля их на $\sin a$ имеем
	\begin{equation*}
		\frac{\sin{A}}{\sin a} = \frac{\sqrt{1 - \cos^2 c - \cos^2 b - \cos^2 a + 2\cos a \cos c \cos b}}{\sin a \sin b \sin c}.
	\end{equation*}
	Заметим, что правая часть равенства циклична по переменным $a$, $b$ и $c$, следовательно, \term{сферическая теорема синусов} имеет вид
	\begin{equation}
		\frac{\sin A}{\sin a} = \frac{\sin B}{\sin b} = \frac{\sin C}{\sin c}.
	\end{equation}
	
	Напоследок получим \term{формулу пяти элементов}. Для этого запишем теорему косинусов в выразим в ней один из косинусов, применяя ее же:
	\begin{gather*}
		\cos a = \sin c \sin b \cos A + \cos c \cos b,\\
		\cos a = \sin c \sin b \cos A + \left( \sin a \sin b \cos C + \cos a \cos b \right)\cos b,\\
		\cos a - \cos a \cos^2 b = \sin c \sin b \cos A + \sin a \sin b \cos b \cos C,\\
		\cos a \sin^2 b = \sin c \sin b \cos A + \sin a \sin b \cos b \cos C,
	\end{gather*}
	\begin{equation}
		\cos a \sin b = \sin a \cos b \cos C + \sin c \cos A.
	\end{equation}

\term{Параллактический треугольник}~--- треугольник на небесной  сфере, образованный пересечением небесного меридиана, вертикального круга и часового круга светила. \imp{Вертикальный круг}~--- большой круг небесной сферы, проходящий через надир, зенит и светило. \imp{Часовой круг}~--- большой круг небесной сферы, проходящий через полюса мира и наблюдаемое светило.

Применяя теоремы синусов и косинусов к параллактическому треугольнику, нетрудно получить следующие соотношения:
\begin{gather}
\cos z=\sin\varphi\sin\delta+\cos\varphi\cos\delta\cos t\\
\sin z\sin A=\cos\delta\sin t\\
\sin z\cos A=-\cos\varphi\sin\delta+\sin\varphi\cos\delta\cos t
\end{gather}
\input{sections/spher.astro/spher.astro.sun-time.tex}
\subsection{Годичное движение Солнца}
В течение сидерического года Земля совершает полный оборот вокруг Солнца. Вследствие этого Солнце движется относительно далёких звёзд для наблюдателя на Земле. Это движение совершается по большому кругу небесной сферы, называемому \term{эклиптикой} и совпадающему с плоскостью орбиты Земли. Однако, в силу прецессии земной оси с периодом около 25765~лет, период такого движения равен \imp{тропическому году}, который длиннее сидерического года примерно на 20~мин~25~сек.

\begin{wrapfigure}[12]{r}{0.5\tw}
	\centering
	\vspace{-.9pc}
	\begin{tikzpicture}
		\begin{axis}[
			width	=	.5\tw,
			height	=	4.5cm,
			xlabel	=	{Прямое восхождение $\alpha^h$},
			ylabel	=	{Склонение $\delta^{\circ}$},
			extra y ticks	=	{23.44, -23.44},
			ytick = {-20, -10, 0, 10, 20},
			ymax	=	25,
			ymin	=	-25,
			xmax	=	24,
			xmin	=	0,
			xtick	=	{0, 4, 8, 12, 16, 20, 24},
			x dir = reverse
			]
			\addplot [domain=0:24, samples=100] {atan(sin(x*15)*tan(23.44))};
		\end{axis}
	\end{tikzpicture}
	\caption{График зависимости склонения Солнца от его прямого восхождения}
\end{wrapfigure}
В моменты, когда Солнце находится в \imp{точке весеннего равноденствия}  (20~марта, реже~21) его координаты: $\alpha=0^h$, $\delta=0^{\circ}$. Во время прохождения этой точки обе координаты Солнца растут. Так происходит до момента, пока Солнце не пройдет \imp{точку летнего солнцестояния} (21~июня, реже~20), после этого склонение Солнца начинает уменьшаться. В момент прохождения \imp{точки осеннего равноденствия} (22~или 23~сентября), координаты Солнца составляют $\alpha=12^h$, $\delta=0^{\circ}$. После прохождения \imp{точки зимнего солнцестояния} (22~или 21~декабря) склонение Солнца начинает увеличиваться.

Пренебрегая сферическими искажениями, годичный путь Солнца по небесной сфере можно считать синусоидой, откуда
\begin{equation}
	\delta=\varepsilon\cdot\sin \frac{2 \pi t}{T},
\end{equation}
где $t$~--- время, прошедшее с момента весеннего равноденствия, $T$~--- тропический год.

Более точная формула следует из сферической тригонометрии и имеет вид
\begin{equation}
	\delta=\arcsin\left(\sin\varepsilon\cdot\sin \frac{2 \pi t}{T}\right).
	\label{eq:delta-sun}
\end{equation}

Известно, что движение Солнца по эклиптике происходит неравномерно, поэтому данные формулы не являются абсолютно точными.

Прямое восхождение Солнца связано со склонением формулой
\begin{equation}
	\sin\alpha=\frac{\tg\delta}{\tg\varepsilon}.
	\label{eq:sin-alpha}
\end{equation}

Выражения \eqref{eq:delta-sun} и \eqref{eq:sin-alpha} следуют из формул перехода между экваториальной и эклиптической системами координат, получаемых из сферической тригонометрии.

\subsection{Рефракция}
\term{Рефракция}~--- явление преломления световых лучей, приходящих от небесных светил, в атмосфере планеты. Для наблюдателя на поверхности планеты с атмосферой положение светила будет отличаться от истинного на некоторый угол. Средняя величина рефракции у горизонта для земной атмосферы равна $35'$.

Для зенитного расстояния $z < 70^\circ$ величины рефракции можно определить по формуле
\begin{equation}
	\rho = 60.25'' \cdot \tg z' \cdot \frac{p}{760} \frac{273^{\circ}}{273^{\circ}+ t^{\circ}},
	\label{eq:refrac}
\end{equation}
где $t^{\circ}$~--- температура воздуха в$~^\circ$C, $p$~--- атмосферное давление в мм~рт.\,ст., $z'$~--- видимое зенитное расстояние. При н.~у.: $p = 760$ мм~рт.\,ст. и $t = 0^{\circ}$C, формула \eqref{eq:refrac} принимает вид
\begin{equation}
	\rho = 60.25'' \cdot \tg z'.
\end{equation}

\subsection{Сумерки}
\term{Сумерки}~--- часть суток, когда Солнце находится неглубоко под горизонтом. 
В зависимости от высоты Солнца под горизонтом различают \imp{гражданские}, \imp{навигационные} и \imp{астрономические} сумерки:\\
\begin{minipage}{0.54\tw}
	\begin{enumerate}
		\item Гражданские~--- от $0^{\circ}$ до $-6^{\circ}$
		\item Навигационные~--- от $-6^{\circ}$ до $-12^{\circ}$
		\item Астрономические~--- от $-12^{\circ}$ до $-18^{\circ}$
	\end{enumerate}
	Когда Солнце опускается ниже $-18^{\circ}$, наступает ночь.
\end{minipage}
\hfill
\begin{minipage}{0.44\tw}
	\centering
 	\includegraphics[width=\tw]{spher-astro-dusk.pdf}
	\captionof{figure}{Сумерки}
\end{minipage}
\newpage

\subsection{Суточное вращение небесной сферы}
Вследствие вращения Земли вокруг своей оси для наблюдателя на поверхности небесные объекты совершают суточное движение параллельно небесному экватору, плоскость которого совпадает с плоскостью экватора Земли. Очевидно, в ходе такого движения высота светил постоянно меняется и в некоторые моменты времени достигает своего максимального и минимального значения. 

\term{Верхняя} и \term{нижняя кульминация}~--- моменты пересечения светилом небесного меридиана, причём при верхней кульминации светило имеет наибольшую высоту, а при нижней~--- наименьшую.

Высота светила в верхней и нижней кульминации со склонением $|\delta| < |\varphi|$, соответственно:
\begin{equation}
h_{\text{в}}= 90^\circ - \varphi + \delta, \quad\quad
h_{\text{н}}= - 90^\circ + \varphi  + \delta.
\end{equation}

Если же светило имеет склонение $|\delta| > |\varphi|$, то высота в верхней и нижней кульминации вычисляется так:
\begin{equation}
h_{\text{в}}= 90^\circ + \varphi - \delta, \quad\quad
h_{\text{н}}= - 90^\circ -\varphi - \delta.
\end{equation}

Из формул для высоты в нижней кульминации вытекает условие, определяющее, пересекает ли звезда горизонт:
\begin{equation}
\begin{cases}
	h_\text{в}= +90^\circ - |\varphi + \delta| > 0^\circ,\\
	h_\text{н} = - 90^\circ + |\varphi + \delta| < 0^\circ;	
\end{cases}
\quad \Longleftrightarrow \quad~~ |\delta|< 90^{\circ} - |\varphi|.
\end{equation}

Используя формулы сферической тригонометрии (см.\,\ref{sec:spher-trig}), можно выразить зависимость часового угла светила от его зенитного расстояния:
\begin{equation}
\cos t=\frac{\cos z-\sin\varphi\sin\delta}{\cos\varphi\cos\delta}. 
\end{equation}
Отсюда следует, что для часового угла захода и восхода светила справедливо равенство:
\begin{equation}
	\cos t_{\uparrow\downarrow}=-\tg\varphi\cdot\tg\delta.
\end{equation} 

Аналогично, для вычисления азимута светила верна формула
\begin{equation}
\cos A=\frac{\cos\delta\cos t-\cos\varphi\cos z}{\sin\varphi\sin z}.
\end{equation}
Следовательно, азимуты точек восхода и захода
\begin{equation}
	A_\uparrow = \arccos \left(-\dfrac{\sin\delta}{\cos \varphi} \right)\quad\text{и}\quad A_\downarrow = - A_\uparrow.
\end{equation}

\term{Звёздное время}~$z$~--- часовой угол точки весеннего равноденствия. Из определений прямого восхождения и часового угла следует справедливость равенства\begin{equation}
z = \alpha + t.
\end{equation}
\subsection{Сферическая тригонометрия}
\label{sec:spher-trig}
\begin{wrapfigure}[10]{r}{.3\tw}
	\centering
	\vspace{-1pc}
 	\includegraphics[width=0.3\textwidth]{spher-trigonom}
 	\caption{Сферический треугольник}
\end{wrapfigure}
Для решения некоторых задач астрономии, связанных с видимыми положениями небесных тел, требуются знания о сферической тригонометрии. \imp{Сферический треугольник}~--- фигура на поверхности сферы, состоящая из трёх точек и трёх дуг больших кругов, соединяющих эти точки. Пусть $A$, $B$ и $C$~--- углы сферического треугольника, а $a$, $b$ и $c$~--- его стороны.

Сферические треугольники обладают следующими свойствами:
\begin{enumerate}
\item Два сферических треугольника равны, если они подобны.
\item Каждая сторона меньше суммы двух других сторон и больше их разности.
\item Сумма всех сторон $a+b+c$ всегда меньше $2\pi$.
\item Сумма углов сферического треугольника $\pi < A + B + C < 3\pi$.
\item Разность суммы двух углов и третьего угла меньше $\pi$
\end{enumerate}

Площадь сферического треугольника определяется по формуле:
\begin{equation}
	S = R^2( A + B + C - \pi),
\end{equation}
где $A + B + C - \pi$~--- \imp{сферический избыток}.

Рассмотрим сферический треугольник $ABC$, радиус векторы вершин соответственно $\vec{a}$, $\vec{b}$ и $\vec{c}$. причем из определения сферы $|\vec{a}| = |\vec{b}| = |\vec{c}| = r$. Пусть против вершин $A$, $B$ и $C$ лежат стороны с угловой мерой $a$, $b$ и $c$ соответсвенно. Повернем сферические координаты и нормируем так, чтобы $\vec{a} = (0, 0, 1)$, $\vec{b} = (\sin c, 0, \cos c)$, тогда $ \vec{c} = (\sin b \cos A, \sin b \sin A, \cos b)$.
	
	Теперь запишем выражение для $\scalar{b}{c}$:
	\begin{equation}
		\scalar{b}{c} = \cos a = \sin c \sin b \cos A + \cos c \cos b.
		\label{eq:spher-astro-cos-1}
	\end{equation}
	Аналогично,
	\begin{gather}
		\scalar{a}{c} = \cos b = \sin a \sin c \cos B +  \cos a \cos c,\\
		\scalar{a}{b} = \cos c = \sin a \sin b \cos C + \cos a \cos b.
		\label{eq:spher-astro-cos-1-1}
	\end{gather}
	Выразим отсюда $\cos A$:
	\begin{equation}
		\cos A = \frac{\cos a - \cos c \cos b}{\sin c \sin b}.
		\label{eq:spher-astro-cos-2}
	\end{equation}
	Формулы \eqref{eq:spher-astro-cos-1}\,--\,\eqref{eq:spher-astro-cos-2} называются \term{сферической теоремой косинусов} \imp{для стороны} \eqref{eq:spher-astro-cos-1}\,--\,\eqref{eq:spher-astro-cos-1-1} и, соответственно \imp{для угла} \eqref{eq:spher-astro-cos-2}.
	
	Из основного тригонометрического тождества имеем:
	\begin{multline*}
		\sin^2 A = 1 - \cos^2 A = 1 - \left[ \frac{\cos a - \cos c \cos b}{\sin c \sin b} \right]^2 = \\
		= \frac{\sin^2 c \sin^2 b - \cos^2 a + 2\cos a \cos c \cos b - \cos^2 c \cos^2 b}{\sin^2 c \sin^2 b}=\\
		= \frac{(1 - \cos^2 c)(1 -  \cos^2 b) - \cos^2 a + 2\cos a \cos c \cos b - \cos^2 c \cos^2 b}{\sin^2 c \sin^2 b}=\\
		= \frac{1 - \cos^2 c - \cos^2 b + \cos^2 c \cos^2 b -\cos^2 a}{\sin^2 c \sin^2 b} + \\
		+ \frac{2\cos a \cos c \cos b - \cos^2 c \cos^2 b}{\sin^2 c \sin^2 b} = \\
		= \frac{1 - \cos^2 c - \cos^2 b - \cos^2 a + 2\cos a \cos c \cos b}{\sin^2 c \sin^2 b}.
	\end{multline*}
	Извлекая квадратный корень из левой и правой части и деля их на $\sin a$ имеем
	\begin{equation*}
		\frac{\sin{A}}{\sin a} = \frac{\sqrt{1 - \cos^2 c - \cos^2 b - \cos^2 a + 2\cos a \cos c \cos b}}{\sin a \sin b \sin c}.
	\end{equation*}
	Заметим, что правая часть равенства циклична по переменным $a$, $b$ и $c$, следовательно, \term{сферическая теорема синусов} имеет вид
	\begin{equation}
		\frac{\sin A}{\sin a} = \frac{\sin B}{\sin b} = \frac{\sin C}{\sin c}.
	\end{equation}
	
	Напоследок получим \term{формулу пяти элементов}. Для этого запишем теорему косинусов в выразим в ней один из косинусов, применяя ее же:
	\begin{gather*}
		\cos a = \sin c \sin b \cos A + \cos c \cos b,\\
		\cos a = \sin c \sin b \cos A + \left( \sin a \sin b \cos C + \cos a \cos b \right)\cos b,\\
		\cos a - \cos a \cos^2 b = \sin c \sin b \cos A + \sin a \sin b \cos b \cos C,\\
		\cos a \sin^2 b = \sin c \sin b \cos A + \sin a \sin b \cos b \cos C,
	\end{gather*}
	\begin{equation}
		\cos a \sin b = \sin a \cos b \cos C + \sin c \cos A.
	\end{equation}

\term{Параллактический треугольник}~--- треугольник на небесной  сфере, образованный пересечением небесного меридиана, вертикального круга и часового круга светила. \imp{Вертикальный круг}~--- большой круг небесной сферы, проходящий через надир, зенит и светило. \imp{Часовой круг}~--- большой круг небесной сферы, проходящий через полюса мира и наблюдаемое светило.

Применяя теоремы синусов и косинусов к параллактическому треугольнику, нетрудно получить следующие соотношения:
\begin{gather}
\cos z=\sin\varphi\sin\delta+\cos\varphi\cos\delta\cos t\\
\sin z\sin A=\cos\delta\sin t\\
\sin z\cos A=-\cos\varphi\sin\delta+\sin\varphi\cos\delta\cos t
\end{gather}
\input{sections/spher.astro/spher.astro.sun-time.tex}
\subsection{Годичное движение Солнца}
В течение сидерического года Земля совершает полный оборот вокруг Солнца. Вследствие этого Солнце движется относительно далёких звёзд для наблюдателя на Земле. Это движение совершается по большому кругу небесной сферы, называемому \term{эклиптикой} и совпадающему с плоскостью орбиты Земли. Однако, в силу прецессии земной оси с периодом около 25765~лет, период такого движения равен \imp{тропическому году}, который длиннее сидерического года примерно на 20~мин~25~сек.

\begin{wrapfigure}[12]{r}{0.5\tw}
	\centering
	\vspace{-.9pc}
	\begin{tikzpicture}
		\begin{axis}[
			width	=	.5\tw,
			height	=	4.5cm,
			xlabel	=	{Прямое восхождение $\alpha^h$},
			ylabel	=	{Склонение $\delta^{\circ}$},
			extra y ticks	=	{23.44, -23.44},
			ytick = {-20, -10, 0, 10, 20},
			ymax	=	25,
			ymin	=	-25,
			xmax	=	24,
			xmin	=	0,
			xtick	=	{0, 4, 8, 12, 16, 20, 24},
			x dir = reverse
			]
			\addplot [domain=0:24, samples=100] {atan(sin(x*15)*tan(23.44))};
		\end{axis}
	\end{tikzpicture}
	\caption{График зависимости склонения Солнца от его прямого восхождения}
\end{wrapfigure}
В моменты, когда Солнце находится в \imp{точке весеннего равноденствия}  (20~марта, реже~21) его координаты: $\alpha=0^h$, $\delta=0^{\circ}$. Во время прохождения этой точки обе координаты Солнца растут. Так происходит до момента, пока Солнце не пройдет \imp{точку летнего солнцестояния} (21~июня, реже~20), после этого склонение Солнца начинает уменьшаться. В момент прохождения \imp{точки осеннего равноденствия} (22~или 23~сентября), координаты Солнца составляют $\alpha=12^h$, $\delta=0^{\circ}$. После прохождения \imp{точки зимнего солнцестояния} (22~или 21~декабря) склонение Солнца начинает увеличиваться.

Пренебрегая сферическими искажениями, годичный путь Солнца по небесной сфере можно считать синусоидой, откуда
\begin{equation}
	\delta=\varepsilon\cdot\sin \frac{2 \pi t}{T},
\end{equation}
где $t$~--- время, прошедшее с момента весеннего равноденствия, $T$~--- тропический год.

Более точная формула следует из сферической тригонометрии и имеет вид
\begin{equation}
	\delta=\arcsin\left(\sin\varepsilon\cdot\sin \frac{2 \pi t}{T}\right).
	\label{eq:delta-sun}
\end{equation}

Известно, что движение Солнца по эклиптике происходит неравномерно, поэтому данные формулы не являются абсолютно точными.

Прямое восхождение Солнца связано со склонением формулой
\begin{equation}
	\sin\alpha=\frac{\tg\delta}{\tg\varepsilon}.
	\label{eq:sin-alpha}
\end{equation}

Выражения \eqref{eq:delta-sun} и \eqref{eq:sin-alpha} следуют из формул перехода между экваториальной и эклиптической системами координат, получаемых из сферической тригонометрии.

\subsection{Рефракция}
\term{Рефракция}~--- явление преломления световых лучей, приходящих от небесных светил, в атмосфере планеты. Для наблюдателя на поверхности планеты с атмосферой положение светила будет отличаться от истинного на некоторый угол. Средняя величина рефракции у горизонта для земной атмосферы равна $35'$.

Для зенитного расстояния $z < 70^\circ$ величины рефракции можно определить по формуле
\begin{equation}
	\rho = 60.25'' \cdot \tg z' \cdot \frac{p}{760} \frac{273^{\circ}}{273^{\circ}+ t^{\circ}},
	\label{eq:refrac}
\end{equation}
где $t^{\circ}$~--- температура воздуха в$~^\circ$C, $p$~--- атмосферное давление в мм~рт.\,ст., $z'$~--- видимое зенитное расстояние. При н.~у.: $p = 760$ мм~рт.\,ст. и $t = 0^{\circ}$C, формула \eqref{eq:refrac} принимает вид
\begin{equation}
	\rho = 60.25'' \cdot \tg z'.
\end{equation}

\subsection{Сумерки}
\term{Сумерки}~--- часть суток, когда Солнце находится неглубоко под горизонтом. 
В зависимости от высоты Солнца под горизонтом различают \imp{гражданские}, \imp{навигационные} и \imp{астрономические} сумерки:\\
\begin{minipage}{0.54\tw}
	\begin{enumerate}
		\item Гражданские~--- от $0^{\circ}$ до $-6^{\circ}$
		\item Навигационные~--- от $-6^{\circ}$ до $-12^{\circ}$
		\item Астрономические~--- от $-12^{\circ}$ до $-18^{\circ}$
	\end{enumerate}
	Когда Солнце опускается ниже $-18^{\circ}$, наступает ночь.
\end{minipage}
\hfill
\begin{minipage}{0.44\tw}
	\centering
 	\includegraphics[width=\tw]{spher-astro-dusk.pdf}
	\captionof{figure}{Сумерки}
\end{minipage}
\newpage

\section{Объекты космоса}
\newpage
\section{Астрофизика}
\subsection{Звёздные величины}
Звёздная величина~--- безразмерная числовая характеристика яркости объекта. Известно, что увеличению светового потока в 100 раз соответствует уменьшение видимой звёздной величины ровно на 5 единиц. Тогда уменьшение звёздной величины на одну единицу означает увеличение светового потока в $\sqrt[5]{100}\approx 2.512$~раз, то есть звёздные величины являются логарифмической шкалой измерения плотности потока. Зависимость, связывающая отношение освещённостей $E_1$ и $E_2$ и разность звёздных величин $m_1$ и $m_2$ двух объектов, называется \term{формулой Погсона} и имеет вид
\begin{equation}
	\frac{E_1}{E_2} = 10^{0.4(m_2 - m_1)} \quad \Longleftrightarrow \quad m_2 = m_1 + 2.5 \lg \frac{E_1}{E_2}.
	\label{eq:Pogson-law}
\end{equation}
Широко используется понятие \term{абсолютной звёздной величины} $M$~--- это видимая звёздная величина $m$ при наблюдении с установленного расстояния: для звёзд~---~10~пк, для тел Солнечной системы~---~1~\au, причем считается, что тело находится в 1~\au~и от наблюдателя и от Солнца, а фаза равна единице, то есть можно считать, что наблюдатель находится в центре Солнца, а~тело~--- в~1~\au~от него. 

Кроме этого, важно понятие \term{болометрической звёздной величины} $m_\text{bol}$~--- это звёздная величина, при расчёте которой учитывается полная мощность излучения источника во всех диапазонах электромагнитных волн. Обычная (видимая) звёздная величина учитывает излучение лишь в видимой части спектра от примерно 380~нм до примерно~780~нм. Разность между болометрической и видимой звёздными величинами называется \term{болометрической поправкой} ($BC$), которая отличается для разных спектральных классов звёзд. Из определения, болометрическая поправка может быть найдена по формуле
\begin{equation}
	BC = m_\text{bol} - m.
\end{equation}
Абсолютную звёздную величину звезды можно получить по формуле Погсона \eqref{eq:Pogson-law} из видимой звёздной величины $m$ и расстояния $r$ до неё в парсеках
\begin{equation}
	M = m + 2.5 \lg \frac{E}{E_\text{абс}} = m + 2.5 \lg \frac{(10~\text{пк})^2}{r^2} = m + 5 - 5\lg r.
	\label{eq:abs-mag}
\end{equation} 
Если принимать к рассмотрению межзвездное поглощение $A$, то формулу  \eqref{eq:abs-mag} необходимо уточнить:
\begin{equation}
	M = m + 5 - 5\lg r - Ar.
\end{equation}
\subsection{Закон Стефана-Больцмана}
\term{Закон Стефана~--- Больцмана} определяет зависимость плотности мощности излучения абсолютно чёрного тела (АЧТ) $u$ от его температуры $T$:
\begin{equation}
u = a T^4,
\end{equation} 
где $a$~--- некая универсальная константа.
Отсюда полная светимость АЧТ с площадью поверхности $S$
	\begin{equation}
	L = S \sigma T^4,
	\label{eq:steff-bol-law}
\end{equation}
константа $\sigma$ называется \term{постоянной Стефана-Больцмана}.
  
Важно отметить, что \imp{закон Стефана-Больцмана}~--- прямое следствие формулы Планка \eqref{Planck's formula}, так как
\begin{equation}
	\sigma T^4 = \int\limits^\infty_0 B(\lambda, T)\,d\lambda \int\limits_0^{\pi/2} \sin \varphi\, d\varphi \int\limits_0^{2\pi} \cos \varphi\, d\theta = \pi \int\limits^\infty_0 B(\lambda, T)\,d\lambda,
\end{equation}
откуда $\sigma = (2\pi^5k^4)/(15c^2h^3) = 5.67 \cdot 10^{-8}~\text{Вт}/(\text{м}^2\cdot \text{К}^4)$.

%Для АЧТ сферической формы с радиусом $R$ формула~\eqref{eq:steff-bol-law} принимает вид
%\begin{equation}
%L=4\pi R^2\sigma T^4.
%\end{equation}
Для звёзд главной последовательности выполняется соотношение $L \sim M^{\alpha}$, где~$\alpha$~--- коэффициент пропорциональности, который зависит от массы звезды следующим образом:
\begin{align*}
\alpha &= 2.5, \quad M < 0.43 M_\odot; & 
\alpha &= 4, \quad 0.43 M_\odot < M < 2 M_\odot;\\ 
\alpha &= 3.2, \quad 2 M_\odot < M < 20 M_\odot; & 
\alpha &= 1, \quad M > 20 M_\odot.
\end{align*}
Также существует примерная зависимость светимости звёзды от её радиуса, имеющая вид  $L\sim R^{5.2}$.
\subsection{Энергия излучения}
\term{Энергия излучения}~--- энергия, переносимая излучением ($Q_e$).\\
\term{Поток излучения}~--- физическая величина, характеризующая мощность, переносимую излучением,
\begin{equation}
 \Phi_e = \frac{d Q_e}{dt}.
\end{equation}
\imp{Теорема Гаусса}: через любую замкнутую поверхность потоки от одинаковых источников равны.

\term{Спектральная плотность потока излучения}~--- поток излучения, приходящийся на малый единичный интервал спектра,
\begin{equation}
\Phi_{e, \lambda}(\lambda) = \frac{d\Phi_e(\lambda)}{d\lambda}, \quad\quad \Phi_{e, \nu}(\nu) = \frac{d\Phi_e(\nu)}{d\nu} =  \frac{\lambda^2}{c}\Phi_{e, \lambda}(\lambda).
\end{equation}

\term{Объемная плотность энергии излучения}~--- количество энергии на единицу объема
\begin{equation}
U_e = \frac{d Q_e}{dV}.
\end{equation}

\term{Светимость}~--- величина, представляющая собой световой поток излучения, испускаемого с малого участка светящейся поверхности единичной площади,
\begin{equation}
M_e = \frac{d \Phi_e}{dS_1},
\end{equation}
здесь $S_1$~--- площадь объекта, испускающего энергию.

\term{Яркость}~--- световой поток, приходящийся на единичный телесный угол, в расчёте на единичную площадку проекции излучающей поверхности на картинную плоскость, 
\begin{equation}
L_e = \frac{d^2 \Phi_e}{d \Omega\,dS_1 \cos \varepsilon},
\end{equation}
где $\varepsilon$~--- угол между направлением потока излучения и нормалью к плоскости излучающей поверхности.

\term{Интегральная яркость}~--- интеграл яркости по видимой поверхности источника. Показывает количество энергии, пришедшее от источника за единицу времени.
\begin{equation}
\Lambda_e = \int \limits_S L_e(\vec{r})\,ds.
\end{equation}
\term{Освещенность}~--- величина, равная отношению светового потока, падающего на малый участок поверхности, к его площади~--- поверхностная плотность потока
\begin{equation}
E_e = \frac{d\Phi_e}{dS_2} \sim \frac{1}{r^2},
\end{equation}
здесь $S_2$~--- площадь поверхности приёмника, $r$~--- расстояние от источника.
\input{sections/astrophys.flux-albedo.tex}
\input{sections/astrophys.photon.tex}
\input{sections/astrophys.energy-lines.tex}
\subsection{Формула Планка}
\label{sec:planck-law}
\term{Формула Планка}~--- выражение для спектральной плотности мощности излучения абсолютно чёрного тела на интервале частот $[\nu, \nu + d \nu)$, распространяющейся с телесном угле $d\Omega$, которое было получено Максом Планком в 1900~году. Данное выражение имеет следующий вид:
\begin{equation}
B_\nu(\nu,T)=\frac{2h\nu^3}{c^2}\cdot \frac{1}{\exp\left(\frac{h\nu}{kT}\right)-1} = \left[ \frac{\text{Вт}}{\text{м}^2 \cdot \text{Гц} \cdot \text{ср}}\right],
\label{eq:plancks-law-nu}
\end{equation}
где $\nu$~--- частота излучения, $T$~--- температура АЧТ, $h$~--- постоянная Планка, $k$~--- постоянная Больцмана, $c$~--- скорость света.

Если записать закон излучения Планка \eqref{eq:plancks-law-nu} для длин волн, то
\begin{equation}
B_\lambda(\lambda,T)=\frac{2hc^2}{\lambda^5} \cdot \frac{1}{\exp\left(\frac{hc}{\lambda kT}\right)-1} = \left[ \frac{\text{Вт}}{\text{м}^3 \cdot \text{ср}}\right].
\label{eq:plancks-law-lambda}
\end{equation}
\begin{wrapfigure}[15]{l}{.6\tw}
\centering
\vspace{-.9pc}
 \begin{tikzpicture}
  \begin{axis}[
  				width 	=	.6\tw, 
				height	=	6cm, 
  				ymax	=	1e+14,
  				xmax	=	2000,
  				xmin	=	0,
  				ymin	=	0,
				xlabel	=	{Длина волны $\lambda$,~нм}, 
				ylabel 	= 	{$B_\lambda(\lambda, T)$,~$\text{Вт} \cdot \text{м}^{-3}$}
]
   \addplot+[dashed, thin, black] table[x=l, y=tl] {data/planck.txt};
   \addplot+[black] table[x=l, y=t4] {data/planck.txt} node at (axis cs:870, 1.6e+13) {\tiny{$4500$~K}};
   \addplot+[black] table[x=l, y=t5] {data/planck.txt}node at (axis cs:750, 4.2e+13) {\tiny{$5000$~K}};
   \addplot+[black] table[x=l, y=t58] {data/planck.txt}node at (axis cs:670, 8.5e+13) {\tiny{$5800$~K}};
   \addplot+[black] table[x=l, y=t7] {data/planck.txt}node at (axis cs:1350, 3.5e+13) {\tiny{$7000$~K}};
	%\addplot+[black, smooth] table[x=l, y=t15] {data/planck.txt} node at (axis cs:1670, 5.5e+13) {\tiny{$15000$~K}};
  \end{axis}
 \end{tikzpicture}
\caption{Кривые спектральной плотности мощности изотропного излучения АЧТ с разной температурой}\label{pic:wien-law}
\end{wrapfigure}
Стоит заметить, что при переходе в функции к длинам волн меняется не только частота на длину волны, но и выражение для интервала. 

Формула Планка появилась, когда стало ясно, что формула Рэлея-Джинса удовлетворительно описывает излучение только в области больших длин волн, а~с~убыванием длин волн даёт сильные расхождения с реальными данными. Однако формулу Рэлея-Джинса используют и сейчас для описания кривой Планка на больших длинах волн. 

\change{
Проделаем обратные действия: получим формулу Рэлея-Джинса из формулы Планка. Длинноволновая часть спектра характеризуется соотношением $h\nu \ll kT$, то есть 
\begin{equation*}
	\exp\left( \frac{h\nu}{kT}\right) \approx 1 + \frac{h\nu}{kT}.
\end{equation*}
Подставляя полученное выражение в знаменатель \eqref{eq:plancks-law-nu}, получим
\begin{equation*}
	B_\nu(\nu,T) \approx \frac{2h\nu^3}{c^2}\cdot \frac{1}{1 + \frac{h\nu}{kT} - 1} = \frac{2h\nu^3 }{c^2}\cdot \frac{k T}{ h \nu} = \frac{2 \nu^2 k T}{c^2}.
\end{equation*}
}
\change{
	Проделав то же самое для выражения через длину волны, получим:
}
\begin{equation}
	B(\lambda, T) \simeq \frac{2 c k T}{\lambda^4}, \quad\quad B(\nu, T) \simeq \frac{2 \nu^2 k T}{c^2}.
\label{Ray-Jean}
\end{equation}

\change{
	В коротковолновой области, наоборот, $h \nu \gg kT$, следовательно, в знаменателе формулы Планка единица много меньше стоящей там экспоненты, то есть
	\begin{equation*}
		\frac{1}{\exp\left(\frac{h\nu}{kT}\right)-1} \approx \frac{1}{\exp\left(\frac{h\nu}{kT}\right)} = \exp\left(-\frac{h\nu}{kT}\right).
	\end{equation*} 
	Отсюда получаются выражения, называемые приближением Вина:
}
\begin{equation}
B ( \lambda, T) \simeq \frac{2 h c^2}{\lambda^5} \exp \left( -\frac{h c}{\lambda k T}\right), \quad \quad B( \nu, T ) \simeq \frac{2 h \nu^3}{c^2} \exp \left( -\frac{h \nu}{k T} \right).
\end{equation}
\subsection{Закон смещения Вина}
\term{Закон смещения Вина} --- закон, устанавливающий зависимость длины волны~$\lambda_\text{макс}$, на которой спектральная плотность излучения $B_\lambda(\lambda, T)$ абсолютно чёрного тела достигает своего максимума, от температуры $T$ этого тела:
\begin{equation}
	\lambda_\text{макс} \approx \frac{b}{T} \equiv \frac{0.0029~\text{м} \cdot \text{К}}{T}.
\end{equation}
Закон является следствием исследования функции Планка (см.~\ref{sec:planck-law}) на экстремальность.
\subsection{Эффект Доплера. Красное смещение}
\term{Эффект Доплера}~--- эффект изменения частоты и длины волны электромагнитного излучения, регистрируемого приёмником, вызванный относительным движением источника и приёмника (см.~Рис.\,\ref{doppler-ef}).

При $\Delta \lambda \ll \lambda_0$ с большой точностью выполняется следующее важное соотношение:\begin{equation}
\beta \equiv \dfrac{v}{c} = \dfrac{\lambda - \lambda_0}{\lambda_0} \equiv \dfrac{\Delta \lambda}{\lambda_0},
\label{eq:dopler-ef-simple}
\end{equation}
\begin{wrapfigure}[6]{r}{0.5\tw}
\centering
\vspace{-.5pc}
\includegraphics[width=.5\tw]{doppler-ef}
\caption{Эффект Доплера}
\label{doppler-ef}
\end{wrapfigure}
где $\lambda_0$~--- лабораторная длина волны излучения источника, а $\lambda$~--- наблюдаемая. В действительности же имеет место более общий случай: \imp{релятивистский эффект Доплера}, обусловленный проявлением СТО при $v \simeq c$, для которого формула~\eqref{eq:dopler-ef-simple} усложняется и принимает вид
\begin{equation}
\nu = \nu_0 \cdot \dfrac{\sqrt{1 - \beta^2}}{1 + \beta \cdot \cos\theta},
\label{eq:dopler-ef-rel}
\end{equation}
где $\nu$~--- частота, с которой наблюдатель принимает волны, $\nu_0$~--- частота, с которой источник испускает волны, $v$~--- скорость источника, $\theta$~--- угол между направлением на источник и вектором его скорости в системе отсчёта приёмника. Если источник радиально удаляется от наблюдателя, то $\theta = 0$, если приближается, то $\theta =\pi$. Важно, что~\eqref{eq:dopler-ef-simple} напрямую следует из \eqref{eq:dopler-ef-rel} при $\beta  \ll 1$.

\term{Красное смещение}~--- явление сдвига спектральных линий химических элементов в красную (длинноволновую) сторону, обусловленное относительным движение объектов. Параметр красного смещения определяется из наблюдаемой и лабораторной длин волн как
\begin{equation}
z = \dfrac{\lambda - \lambda_0}{\lambda_0}.
\end{equation}

Доплеровское смещение длины волны в спектре источника, движущегося с лучевой скоростью $v_{r}$ и полной скоростью $v$,
\begin{equation}
z = \dfrac{1 + v_r / c}{\sqrt{1 - \beta^2}}.
\end{equation}

\term{Гравитационное красное смещение}~--- проявление эффекта изменения частоты излучения, испущенного массивным объектом, таким как звезда или чёрная дыра. Наблюдается как сдвиг спектральных линий в спектре источника в красную область спектра. Гравитационное красное смещение определяется из формулы, выведенной Эйнштейном,
\begin{equation}
z_G=\dfrac{GM}{c^2 R}-\dfrac{GM}{c^2 r},
\label{eq:grav-red-shift}
\end{equation}
где $M$~--- масса гравитирующего тела, $R$~--- радиальное расстояние от центра масс тела до точки излучения (радиус источника), $r$~---  радиальное расстояние от центра масс источника до точки наблюдения. В случае, когда наблюдатель находится от источника много дальше его радиуса, т.\,е. выполняется соотношение $r \gg R$, выражение~\eqref{eq:grav-red-shift} можно упростить до
\begin{equation}
z_G \simeq \dfrac{GM}{c^2 R}.
\end{equation}

\input{sections/astrophys.light-pressure.tex}
\input{sections/astrophys.edd.tex}
\input{sections/astrophys.grav-lens.tex}
\subsection{Закон Хаббла}
\term{Закон Хаббла}~--- эмпирический закон, связывающий скорость удаления галактик $V$ и расстояние $R$ до них линейным образом: 
\begin{equation}
	V = H R,
\end{equation}
величина $H=68~\text{км/c} \cdot \text{Мпк})$ называется \imp{постоянной Хаббла}.

При $v \ll c$ можно использовать приближение эффекта Доплера, тогда
\begin{equation}
	V = c z.
\label{eq:hubble-speed}
\end{equation}

Равенство \eqref{eq:hubble-speed} справедливо только при $z \ll 1$, а при б\'{o}льших значениях $z$ космологическое красное смещение нльзя связывать с эффектом Доплера, поэтому можно пользоваться только формулой 
\begin{equation}
	\frac{dz}{dt} = - H(z)(1+z),
\end{equation}
где постоянная Хаббла введена как функция красного смещения.
\subsection{Шкала электромагнитных волн}


\term{Гамма излучение} возникает при радиоактивных распадах ядер, при торможении электронов энергией более $10^5$~эВ и при других взаимодействиях элементарных частиц. Используются в гамма-дефектоскопии, при изучении свойств вещества.

\term{Рентгеновские лучи} излучаются при большом ускорении электронов, например при их торможении в металлах. Получают их при помощи рентгеновской трубки: электроны в вакуумной трубке ускоряются электрическим полем при высоком напряжении, достигая анода, при со­ударении резко тормозятся. При торможении электроны движут­ся с ускорением и излучают электромагнитные волны с малой длиной. 

\begin{figure}[!h]
\centering
\includegraphics[width = 1\textwidth]{scale-wave.pdf}
\caption{Шкала электромагнитных волн}
\end{figure}
\term{Ультрафиолетовые лучи}~--- излучение Солнца, ртутных ламп и т.\,п. Используются в ультрафиолетовой микроскопии, в медицине.

\term{Видимое излучение}~--- часть электромагнитного излучения, воспринимаемая глазом (от фиолетового до от красного).

\term{Инфракрасное излучение}~--- тепловое, излучается любым нагретым телом.

\term{Радиоволны} используются повсеместно в обычной жизни, это и сотовая связь, и радиолокация, и спутниковая связь, и Wi-Fi и многое другое.

\term{Низкочастотные волны}~--- диапазон, традиционно используемый в электротехнике. В промышленной электроэнергетике используется частота 50~Гц, на~которой осуществляется передача электрической энергии по линиям и преобразование напряжений трансформаторными устройствами.
\input{sections/astrophys.spec-theor-rel.tex}
\subsection{Оптическая толщина. Закон Бугера}
\term{Оптическая толщина}~--- безразмерная величина, характеризующая степень непрозрачности среды для проходящего сквозь неё излучения,
\begin{equation}
\tau = \int n(x) \sigma(x)\,dx,
\end{equation}
где $\tau$~--- оптическая толщина среды, $n$~--- концентрация частиц, $\sigma$~--- сечение их взаимодействия.

Поток $I_0$ на входе связан с потоком $I$ на выходе \term{Законом Бугера}:
\begin{equation}
I = I_0 e^{-\tau}.
\end{equation}
\input{sections/astrophys.colour.tex}
\input{sections/astrophys.mkt.tex}
\input{sections/astrophys.earth-atmosphere.tex}

\newpage
\section{Астрофизика}
\subsection{Звёздные величины}
Звёздная величина~--- безразмерная числовая характеристика яркости объекта. Известно, что увеличению светового потока в 100 раз соответствует уменьшение видимой звёздной величины ровно на 5 единиц. Тогда уменьшение звёздной величины на одну единицу означает увеличение светового потока в $\sqrt[5]{100}\approx 2.512$~раз, то есть звёздные величины являются логарифмической шкалой измерения плотности потока. Зависимость, связывающая отношение освещённостей $E_1$ и $E_2$ и разность звёздных величин $m_1$ и $m_2$ двух объектов, называется \term{формулой Погсона} и имеет вид
\begin{equation}
	\frac{E_1}{E_2} = 10^{0.4(m_2 - m_1)} \quad \Longleftrightarrow \quad m_2 = m_1 + 2.5 \lg \frac{E_1}{E_2}.
	\label{eq:Pogson-law}
\end{equation}
Широко используется понятие \term{абсолютной звёздной величины} $M$~--- это видимая звёздная величина $m$ при наблюдении с установленного расстояния: для звёзд~---~10~пк, для тел Солнечной системы~---~1~\au, причем считается, что тело находится в 1~\au~и от наблюдателя и от Солнца, а фаза равна единице, то есть можно считать, что наблюдатель находится в центре Солнца, а~тело~--- в~1~\au~от него. 

Кроме этого, важно понятие \term{болометрической звёздной величины} $m_\text{bol}$~--- это звёздная величина, при расчёте которой учитывается полная мощность излучения источника во всех диапазонах электромагнитных волн. Обычная (видимая) звёздная величина учитывает излучение лишь в видимой части спектра от примерно 380~нм до примерно~780~нм. Разность между болометрической и видимой звёздными величинами называется \term{болометрической поправкой} ($BC$), которая отличается для разных спектральных классов звёзд. Из определения, болометрическая поправка может быть найдена по формуле
\begin{equation}
	BC = m_\text{bol} - m.
\end{equation}
Абсолютную звёздную величину звезды можно получить по формуле Погсона \eqref{eq:Pogson-law} из видимой звёздной величины $m$ и расстояния $r$ до неё в парсеках
\begin{equation}
	M = m + 2.5 \lg \frac{E}{E_\text{абс}} = m + 2.5 \lg \frac{(10~\text{пк})^2}{r^2} = m + 5 - 5\lg r.
	\label{eq:abs-mag}
\end{equation} 
Если принимать к рассмотрению межзвездное поглощение $A$, то формулу  \eqref{eq:abs-mag} необходимо уточнить:
\begin{equation}
	M = m + 5 - 5\lg r - Ar.
\end{equation}
\subsection{Закон Стефана-Больцмана}
\term{Закон Стефана~--- Больцмана} определяет зависимость плотности мощности излучения абсолютно чёрного тела (АЧТ) $u$ от его температуры $T$:
\begin{equation}
u = a T^4,
\end{equation} 
где $a$~--- некая универсальная константа.
Отсюда полная светимость АЧТ с площадью поверхности $S$
	\begin{equation}
	L = S \sigma T^4,
	\label{eq:steff-bol-law}
\end{equation}
константа $\sigma$ называется \term{постоянной Стефана-Больцмана}.
  
Важно отметить, что \imp{закон Стефана-Больцмана}~--- прямое следствие формулы Планка \eqref{Planck's formula}, так как
\begin{equation}
	\sigma T^4 = \int\limits^\infty_0 B(\lambda, T)\,d\lambda \int\limits_0^{\pi/2} \sin \varphi\, d\varphi \int\limits_0^{2\pi} \cos \varphi\, d\theta = \pi \int\limits^\infty_0 B(\lambda, T)\,d\lambda,
\end{equation}
откуда $\sigma = (2\pi^5k^4)/(15c^2h^3) = 5.67 \cdot 10^{-8}~\text{Вт}/(\text{м}^2\cdot \text{К}^4)$.

%Для АЧТ сферической формы с радиусом $R$ формула~\eqref{eq:steff-bol-law} принимает вид
%\begin{equation}
%L=4\pi R^2\sigma T^4.
%\end{equation}
Для звёзд главной последовательности выполняется соотношение $L \sim M^{\alpha}$, где~$\alpha$~--- коэффициент пропорциональности, который зависит от массы звезды следующим образом:
\begin{align*}
\alpha &= 2.5, \quad M < 0.43 M_\odot; & 
\alpha &= 4, \quad 0.43 M_\odot < M < 2 M_\odot;\\ 
\alpha &= 3.2, \quad 2 M_\odot < M < 20 M_\odot; & 
\alpha &= 1, \quad M > 20 M_\odot.
\end{align*}
Также существует примерная зависимость светимости звёзды от её радиуса, имеющая вид  $L\sim R^{5.2}$.
\subsection{Энергия излучения}
\term{Энергия излучения}~--- энергия, переносимая излучением ($Q_e$).\\
\term{Поток излучения}~--- физическая величина, характеризующая мощность, переносимую излучением,
\begin{equation}
 \Phi_e = \frac{d Q_e}{dt}.
\end{equation}
\imp{Теорема Гаусса}: через любую замкнутую поверхность потоки от одинаковых источников равны.

\term{Спектральная плотность потока излучения}~--- поток излучения, приходящийся на малый единичный интервал спектра,
\begin{equation}
\Phi_{e, \lambda}(\lambda) = \frac{d\Phi_e(\lambda)}{d\lambda}, \quad\quad \Phi_{e, \nu}(\nu) = \frac{d\Phi_e(\nu)}{d\nu} =  \frac{\lambda^2}{c}\Phi_{e, \lambda}(\lambda).
\end{equation}

\term{Объемная плотность энергии излучения}~--- количество энергии на единицу объема
\begin{equation}
U_e = \frac{d Q_e}{dV}.
\end{equation}

\term{Светимость}~--- величина, представляющая собой световой поток излучения, испускаемого с малого участка светящейся поверхности единичной площади,
\begin{equation}
M_e = \frac{d \Phi_e}{dS_1},
\end{equation}
здесь $S_1$~--- площадь объекта, испускающего энергию.

\term{Яркость}~--- световой поток, приходящийся на единичный телесный угол, в расчёте на единичную площадку проекции излучающей поверхности на картинную плоскость, 
\begin{equation}
L_e = \frac{d^2 \Phi_e}{d \Omega\,dS_1 \cos \varepsilon},
\end{equation}
где $\varepsilon$~--- угол между направлением потока излучения и нормалью к плоскости излучающей поверхности.

\term{Интегральная яркость}~--- интеграл яркости по видимой поверхности источника. Показывает количество энергии, пришедшее от источника за единицу времени.
\begin{equation}
\Lambda_e = \int \limits_S L_e(\vec{r})\,ds.
\end{equation}
\term{Освещенность}~--- величина, равная отношению светового потока, падающего на малый участок поверхности, к его площади~--- поверхностная плотность потока
\begin{equation}
E_e = \frac{d\Phi_e}{dS_2} \sim \frac{1}{r^2},
\end{equation}
здесь $S_2$~--- площадь поверхности приёмника, $r$~--- расстояние от источника.
\input{sections/astrophys.flux-albedo.tex}
\input{sections/astrophys.photon.tex}
\input{sections/astrophys.energy-lines.tex}
\subsection{Формула Планка}
\label{sec:planck-law}
\term{Формула Планка}~--- выражение для спектральной плотности мощности излучения абсолютно чёрного тела на интервале частот $[\nu, \nu + d \nu)$, распространяющейся с телесном угле $d\Omega$, которое было получено Максом Планком в 1900~году. Данное выражение имеет следующий вид:
\begin{equation}
B_\nu(\nu,T)=\frac{2h\nu^3}{c^2}\cdot \frac{1}{\exp\left(\frac{h\nu}{kT}\right)-1} = \left[ \frac{\text{Вт}}{\text{м}^2 \cdot \text{Гц} \cdot \text{ср}}\right],
\label{eq:plancks-law-nu}
\end{equation}
где $\nu$~--- частота излучения, $T$~--- температура АЧТ, $h$~--- постоянная Планка, $k$~--- постоянная Больцмана, $c$~--- скорость света.

Если записать закон излучения Планка \eqref{eq:plancks-law-nu} для длин волн, то
\begin{equation}
B_\lambda(\lambda,T)=\frac{2hc^2}{\lambda^5} \cdot \frac{1}{\exp\left(\frac{hc}{\lambda kT}\right)-1} = \left[ \frac{\text{Вт}}{\text{м}^3 \cdot \text{ср}}\right].
\label{eq:plancks-law-lambda}
\end{equation}
\begin{wrapfigure}[15]{l}{.6\tw}
\centering
\vspace{-.9pc}
 \begin{tikzpicture}
  \begin{axis}[
  				width 	=	.6\tw, 
				height	=	6cm, 
  				ymax	=	1e+14,
  				xmax	=	2000,
  				xmin	=	0,
  				ymin	=	0,
				xlabel	=	{Длина волны $\lambda$,~нм}, 
				ylabel 	= 	{$B_\lambda(\lambda, T)$,~$\text{Вт} \cdot \text{м}^{-3}$}
]
   \addplot+[dashed, thin, black] table[x=l, y=tl] {data/planck.txt};
   \addplot+[black] table[x=l, y=t4] {data/planck.txt} node at (axis cs:870, 1.6e+13) {\tiny{$4500$~K}};
   \addplot+[black] table[x=l, y=t5] {data/planck.txt}node at (axis cs:750, 4.2e+13) {\tiny{$5000$~K}};
   \addplot+[black] table[x=l, y=t58] {data/planck.txt}node at (axis cs:670, 8.5e+13) {\tiny{$5800$~K}};
   \addplot+[black] table[x=l, y=t7] {data/planck.txt}node at (axis cs:1350, 3.5e+13) {\tiny{$7000$~K}};
	%\addplot+[black, smooth] table[x=l, y=t15] {data/planck.txt} node at (axis cs:1670, 5.5e+13) {\tiny{$15000$~K}};
  \end{axis}
 \end{tikzpicture}
\caption{Кривые спектральной плотности мощности изотропного излучения АЧТ с разной температурой}\label{pic:wien-law}
\end{wrapfigure}
Стоит заметить, что при переходе в функции к длинам волн меняется не только частота на длину волны, но и выражение для интервала. 

Формула Планка появилась, когда стало ясно, что формула Рэлея-Джинса удовлетворительно описывает излучение только в области больших длин волн, а~с~убыванием длин волн даёт сильные расхождения с реальными данными. Однако формулу Рэлея-Джинса используют и сейчас для описания кривой Планка на больших длинах волн. 

\change{
Проделаем обратные действия: получим формулу Рэлея-Джинса из формулы Планка. Длинноволновая часть спектра характеризуется соотношением $h\nu \ll kT$, то есть 
\begin{equation*}
	\exp\left( \frac{h\nu}{kT}\right) \approx 1 + \frac{h\nu}{kT}.
\end{equation*}
Подставляя полученное выражение в знаменатель \eqref{eq:plancks-law-nu}, получим
\begin{equation*}
	B_\nu(\nu,T) \approx \frac{2h\nu^3}{c^2}\cdot \frac{1}{1 + \frac{h\nu}{kT} - 1} = \frac{2h\nu^3 }{c^2}\cdot \frac{k T}{ h \nu} = \frac{2 \nu^2 k T}{c^2}.
\end{equation*}
}
\change{
	Проделав то же самое для выражения через длину волны, получим:
}
\begin{equation}
	B(\lambda, T) \simeq \frac{2 c k T}{\lambda^4}, \quad\quad B(\nu, T) \simeq \frac{2 \nu^2 k T}{c^2}.
\label{Ray-Jean}
\end{equation}

\change{
	В коротковолновой области, наоборот, $h \nu \gg kT$, следовательно, в знаменателе формулы Планка единица много меньше стоящей там экспоненты, то есть
	\begin{equation*}
		\frac{1}{\exp\left(\frac{h\nu}{kT}\right)-1} \approx \frac{1}{\exp\left(\frac{h\nu}{kT}\right)} = \exp\left(-\frac{h\nu}{kT}\right).
	\end{equation*} 
	Отсюда получаются выражения, называемые приближением Вина:
}
\begin{equation}
B ( \lambda, T) \simeq \frac{2 h c^2}{\lambda^5} \exp \left( -\frac{h c}{\lambda k T}\right), \quad \quad B( \nu, T ) \simeq \frac{2 h \nu^3}{c^2} \exp \left( -\frac{h \nu}{k T} \right).
\end{equation}
\subsection{Закон смещения Вина}
\term{Закон смещения Вина} --- закон, устанавливающий зависимость длины волны~$\lambda_\text{макс}$, на которой спектральная плотность излучения $B_\lambda(\lambda, T)$ абсолютно чёрного тела достигает своего максимума, от температуры $T$ этого тела:
\begin{equation}
	\lambda_\text{макс} \approx \frac{b}{T} \equiv \frac{0.0029~\text{м} \cdot \text{К}}{T}.
\end{equation}
Закон является следствием исследования функции Планка (см.~\ref{sec:planck-law}) на экстремальность.
\subsection{Эффект Доплера. Красное смещение}
\term{Эффект Доплера}~--- эффект изменения частоты и длины волны электромагнитного излучения, регистрируемого приёмником, вызванный относительным движением источника и приёмника (см.~Рис.\,\ref{doppler-ef}).

При $\Delta \lambda \ll \lambda_0$ с большой точностью выполняется следующее важное соотношение:\begin{equation}
\beta \equiv \dfrac{v}{c} = \dfrac{\lambda - \lambda_0}{\lambda_0} \equiv \dfrac{\Delta \lambda}{\lambda_0},
\label{eq:dopler-ef-simple}
\end{equation}
\begin{wrapfigure}[6]{r}{0.5\tw}
\centering
\vspace{-.5pc}
\includegraphics[width=.5\tw]{doppler-ef}
\caption{Эффект Доплера}
\label{doppler-ef}
\end{wrapfigure}
где $\lambda_0$~--- лабораторная длина волны излучения источника, а $\lambda$~--- наблюдаемая. В действительности же имеет место более общий случай: \imp{релятивистский эффект Доплера}, обусловленный проявлением СТО при $v \simeq c$, для которого формула~\eqref{eq:dopler-ef-simple} усложняется и принимает вид
\begin{equation}
\nu = \nu_0 \cdot \dfrac{\sqrt{1 - \beta^2}}{1 + \beta \cdot \cos\theta},
\label{eq:dopler-ef-rel}
\end{equation}
где $\nu$~--- частота, с которой наблюдатель принимает волны, $\nu_0$~--- частота, с которой источник испускает волны, $v$~--- скорость источника, $\theta$~--- угол между направлением на источник и вектором его скорости в системе отсчёта приёмника. Если источник радиально удаляется от наблюдателя, то $\theta = 0$, если приближается, то $\theta =\pi$. Важно, что~\eqref{eq:dopler-ef-simple} напрямую следует из \eqref{eq:dopler-ef-rel} при $\beta  \ll 1$.

\term{Красное смещение}~--- явление сдвига спектральных линий химических элементов в красную (длинноволновую) сторону, обусловленное относительным движение объектов. Параметр красного смещения определяется из наблюдаемой и лабораторной длин волн как
\begin{equation}
z = \dfrac{\lambda - \lambda_0}{\lambda_0}.
\end{equation}

Доплеровское смещение длины волны в спектре источника, движущегося с лучевой скоростью $v_{r}$ и полной скоростью $v$,
\begin{equation}
z = \dfrac{1 + v_r / c}{\sqrt{1 - \beta^2}}.
\end{equation}

\term{Гравитационное красное смещение}~--- проявление эффекта изменения частоты излучения, испущенного массивным объектом, таким как звезда или чёрная дыра. Наблюдается как сдвиг спектральных линий в спектре источника в красную область спектра. Гравитационное красное смещение определяется из формулы, выведенной Эйнштейном,
\begin{equation}
z_G=\dfrac{GM}{c^2 R}-\dfrac{GM}{c^2 r},
\label{eq:grav-red-shift}
\end{equation}
где $M$~--- масса гравитирующего тела, $R$~--- радиальное расстояние от центра масс тела до точки излучения (радиус источника), $r$~---  радиальное расстояние от центра масс источника до точки наблюдения. В случае, когда наблюдатель находится от источника много дальше его радиуса, т.\,е. выполняется соотношение $r \gg R$, выражение~\eqref{eq:grav-red-shift} можно упростить до
\begin{equation}
z_G \simeq \dfrac{GM}{c^2 R}.
\end{equation}

\input{sections/astrophys.light-pressure.tex}
\input{sections/astrophys.edd.tex}
\input{sections/astrophys.grav-lens.tex}
\subsection{Закон Хаббла}
\term{Закон Хаббла}~--- эмпирический закон, связывающий скорость удаления галактик $V$ и расстояние $R$ до них линейным образом: 
\begin{equation}
	V = H R,
\end{equation}
величина $H=68~\text{км/c} \cdot \text{Мпк})$ называется \imp{постоянной Хаббла}.

При $v \ll c$ можно использовать приближение эффекта Доплера, тогда
\begin{equation}
	V = c z.
\label{eq:hubble-speed}
\end{equation}

Равенство \eqref{eq:hubble-speed} справедливо только при $z \ll 1$, а при б\'{o}льших значениях $z$ космологическое красное смещение нльзя связывать с эффектом Доплера, поэтому можно пользоваться только формулой 
\begin{equation}
	\frac{dz}{dt} = - H(z)(1+z),
\end{equation}
где постоянная Хаббла введена как функция красного смещения.
\subsection{Шкала электромагнитных волн}


\term{Гамма излучение} возникает при радиоактивных распадах ядер, при торможении электронов энергией более $10^5$~эВ и при других взаимодействиях элементарных частиц. Используются в гамма-дефектоскопии, при изучении свойств вещества.

\term{Рентгеновские лучи} излучаются при большом ускорении электронов, например при их торможении в металлах. Получают их при помощи рентгеновской трубки: электроны в вакуумной трубке ускоряются электрическим полем при высоком напряжении, достигая анода, при со­ударении резко тормозятся. При торможении электроны движут­ся с ускорением и излучают электромагнитные волны с малой длиной. 

\begin{figure}[!h]
\centering
\includegraphics[width = 1\textwidth]{scale-wave.pdf}
\caption{Шкала электромагнитных волн}
\end{figure}
\term{Ультрафиолетовые лучи}~--- излучение Солнца, ртутных ламп и т.\,п. Используются в ультрафиолетовой микроскопии, в медицине.

\term{Видимое излучение}~--- часть электромагнитного излучения, воспринимаемая глазом (от фиолетового до от красного).

\term{Инфракрасное излучение}~--- тепловое, излучается любым нагретым телом.

\term{Радиоволны} используются повсеместно в обычной жизни, это и сотовая связь, и радиолокация, и спутниковая связь, и Wi-Fi и многое другое.

\term{Низкочастотные волны}~--- диапазон, традиционно используемый в электротехнике. В промышленной электроэнергетике используется частота 50~Гц, на~которой осуществляется передача электрической энергии по линиям и преобразование напряжений трансформаторными устройствами.
\input{sections/astrophys.spec-theor-rel.tex}
\subsection{Оптическая толщина. Закон Бугера}
\term{Оптическая толщина}~--- безразмерная величина, характеризующая степень непрозрачности среды для проходящего сквозь неё излучения,
\begin{equation}
\tau = \int n(x) \sigma(x)\,dx,
\end{equation}
где $\tau$~--- оптическая толщина среды, $n$~--- концентрация частиц, $\sigma$~--- сечение их взаимодействия.

Поток $I_0$ на входе связан с потоком $I$ на выходе \term{Законом Бугера}:
\begin{equation}
I = I_0 e^{-\tau}.
\end{equation}
\input{sections/astrophys.colour.tex}
\input{sections/astrophys.mkt.tex}
\input{sections/astrophys.earth-atmosphere.tex}

\newpage
\section{Астрофизика}
\subsection{Звёздные величины}
Звёздная величина~--- безразмерная числовая характеристика яркости объекта. Известно, что увеличению светового потока в 100 раз соответствует уменьшение видимой звёздной величины ровно на 5 единиц. Тогда уменьшение звёздной величины на одну единицу означает увеличение светового потока в $\sqrt[5]{100}\approx 2.512$~раз, то есть звёздные величины являются логарифмической шкалой измерения плотности потока. Зависимость, связывающая отношение освещённостей $E_1$ и $E_2$ и разность звёздных величин $m_1$ и $m_2$ двух объектов, называется \term{формулой Погсона} и имеет вид
\begin{equation}
	\frac{E_1}{E_2} = 10^{0.4(m_2 - m_1)} \quad \Longleftrightarrow \quad m_2 = m_1 + 2.5 \lg \frac{E_1}{E_2}.
	\label{eq:Pogson-law}
\end{equation}
Широко используется понятие \term{абсолютной звёздной величины} $M$~--- это видимая звёздная величина $m$ при наблюдении с установленного расстояния: для звёзд~---~10~пк, для тел Солнечной системы~---~1~\au, причем считается, что тело находится в 1~\au~и от наблюдателя и от Солнца, а фаза равна единице, то есть можно считать, что наблюдатель находится в центре Солнца, а~тело~--- в~1~\au~от него. 

Кроме этого, важно понятие \term{болометрической звёздной величины} $m_\text{bol}$~--- это звёздная величина, при расчёте которой учитывается полная мощность излучения источника во всех диапазонах электромагнитных волн. Обычная (видимая) звёздная величина учитывает излучение лишь в видимой части спектра от примерно 380~нм до примерно~780~нм. Разность между болометрической и видимой звёздными величинами называется \term{болометрической поправкой} ($BC$), которая отличается для разных спектральных классов звёзд. Из определения, болометрическая поправка может быть найдена по формуле
\begin{equation}
	BC = m_\text{bol} - m.
\end{equation}
Абсолютную звёздную величину звезды можно получить по формуле Погсона \eqref{eq:Pogson-law} из видимой звёздной величины $m$ и расстояния $r$ до неё в парсеках
\begin{equation}
	M = m + 2.5 \lg \frac{E}{E_\text{абс}} = m + 2.5 \lg \frac{(10~\text{пк})^2}{r^2} = m + 5 - 5\lg r.
	\label{eq:abs-mag}
\end{equation} 
Если принимать к рассмотрению межзвездное поглощение $A$, то формулу  \eqref{eq:abs-mag} необходимо уточнить:
\begin{equation}
	M = m + 5 - 5\lg r - Ar.
\end{equation}
\subsection{Закон Стефана-Больцмана}
\term{Закон Стефана~--- Больцмана} определяет зависимость плотности мощности излучения абсолютно чёрного тела (АЧТ) $u$ от его температуры $T$:
\begin{equation}
u = a T^4,
\end{equation} 
где $a$~--- некая универсальная константа.
Отсюда полная светимость АЧТ с площадью поверхности $S$
	\begin{equation}
	L = S \sigma T^4,
	\label{eq:steff-bol-law}
\end{equation}
константа $\sigma$ называется \term{постоянной Стефана-Больцмана}.
  
Важно отметить, что \imp{закон Стефана-Больцмана}~--- прямое следствие формулы Планка \eqref{Planck's formula}, так как
\begin{equation}
	\sigma T^4 = \int\limits^\infty_0 B(\lambda, T)\,d\lambda \int\limits_0^{\pi/2} \sin \varphi\, d\varphi \int\limits_0^{2\pi} \cos \varphi\, d\theta = \pi \int\limits^\infty_0 B(\lambda, T)\,d\lambda,
\end{equation}
откуда $\sigma = (2\pi^5k^4)/(15c^2h^3) = 5.67 \cdot 10^{-8}~\text{Вт}/(\text{м}^2\cdot \text{К}^4)$.

%Для АЧТ сферической формы с радиусом $R$ формула~\eqref{eq:steff-bol-law} принимает вид
%\begin{equation}
%L=4\pi R^2\sigma T^4.
%\end{equation}
Для звёзд главной последовательности выполняется соотношение $L \sim M^{\alpha}$, где~$\alpha$~--- коэффициент пропорциональности, который зависит от массы звезды следующим образом:
\begin{align*}
\alpha &= 2.5, \quad M < 0.43 M_\odot; & 
\alpha &= 4, \quad 0.43 M_\odot < M < 2 M_\odot;\\ 
\alpha &= 3.2, \quad 2 M_\odot < M < 20 M_\odot; & 
\alpha &= 1, \quad M > 20 M_\odot.
\end{align*}
Также существует примерная зависимость светимости звёзды от её радиуса, имеющая вид  $L\sim R^{5.2}$.
\subsection{Энергия излучения}
\term{Энергия излучения}~--- энергия, переносимая излучением ($Q_e$).\\
\term{Поток излучения}~--- физическая величина, характеризующая мощность, переносимую излучением,
\begin{equation}
 \Phi_e = \frac{d Q_e}{dt}.
\end{equation}
\imp{Теорема Гаусса}: через любую замкнутую поверхность потоки от одинаковых источников равны.

\term{Спектральная плотность потока излучения}~--- поток излучения, приходящийся на малый единичный интервал спектра,
\begin{equation}
\Phi_{e, \lambda}(\lambda) = \frac{d\Phi_e(\lambda)}{d\lambda}, \quad\quad \Phi_{e, \nu}(\nu) = \frac{d\Phi_e(\nu)}{d\nu} =  \frac{\lambda^2}{c}\Phi_{e, \lambda}(\lambda).
\end{equation}

\term{Объемная плотность энергии излучения}~--- количество энергии на единицу объема
\begin{equation}
U_e = \frac{d Q_e}{dV}.
\end{equation}

\term{Светимость}~--- величина, представляющая собой световой поток излучения, испускаемого с малого участка светящейся поверхности единичной площади,
\begin{equation}
M_e = \frac{d \Phi_e}{dS_1},
\end{equation}
здесь $S_1$~--- площадь объекта, испускающего энергию.

\term{Яркость}~--- световой поток, приходящийся на единичный телесный угол, в расчёте на единичную площадку проекции излучающей поверхности на картинную плоскость, 
\begin{equation}
L_e = \frac{d^2 \Phi_e}{d \Omega\,dS_1 \cos \varepsilon},
\end{equation}
где $\varepsilon$~--- угол между направлением потока излучения и нормалью к плоскости излучающей поверхности.

\term{Интегральная яркость}~--- интеграл яркости по видимой поверхности источника. Показывает количество энергии, пришедшее от источника за единицу времени.
\begin{equation}
\Lambda_e = \int \limits_S L_e(\vec{r})\,ds.
\end{equation}
\term{Освещенность}~--- величина, равная отношению светового потока, падающего на малый участок поверхности, к его площади~--- поверхностная плотность потока
\begin{equation}
E_e = \frac{d\Phi_e}{dS_2} \sim \frac{1}{r^2},
\end{equation}
здесь $S_2$~--- площадь поверхности приёмника, $r$~--- расстояние от источника.
\input{sections/astrophys.flux-albedo.tex}
\input{sections/astrophys.photon.tex}
\input{sections/astrophys.energy-lines.tex}
\subsection{Формула Планка}
\label{sec:planck-law}
\term{Формула Планка}~--- выражение для спектральной плотности мощности излучения абсолютно чёрного тела на интервале частот $[\nu, \nu + d \nu)$, распространяющейся с телесном угле $d\Omega$, которое было получено Максом Планком в 1900~году. Данное выражение имеет следующий вид:
\begin{equation}
B_\nu(\nu,T)=\frac{2h\nu^3}{c^2}\cdot \frac{1}{\exp\left(\frac{h\nu}{kT}\right)-1} = \left[ \frac{\text{Вт}}{\text{м}^2 \cdot \text{Гц} \cdot \text{ср}}\right],
\label{eq:plancks-law-nu}
\end{equation}
где $\nu$~--- частота излучения, $T$~--- температура АЧТ, $h$~--- постоянная Планка, $k$~--- постоянная Больцмана, $c$~--- скорость света.

Если записать закон излучения Планка \eqref{eq:plancks-law-nu} для длин волн, то
\begin{equation}
B_\lambda(\lambda,T)=\frac{2hc^2}{\lambda^5} \cdot \frac{1}{\exp\left(\frac{hc}{\lambda kT}\right)-1} = \left[ \frac{\text{Вт}}{\text{м}^3 \cdot \text{ср}}\right].
\label{eq:plancks-law-lambda}
\end{equation}
\begin{wrapfigure}[15]{l}{.6\tw}
\centering
\vspace{-.9pc}
 \begin{tikzpicture}
  \begin{axis}[
  				width 	=	.6\tw, 
				height	=	6cm, 
  				ymax	=	1e+14,
  				xmax	=	2000,
  				xmin	=	0,
  				ymin	=	0,
				xlabel	=	{Длина волны $\lambda$,~нм}, 
				ylabel 	= 	{$B_\lambda(\lambda, T)$,~$\text{Вт} \cdot \text{м}^{-3}$}
]
   \addplot+[dashed, thin, black] table[x=l, y=tl] {data/planck.txt};
   \addplot+[black] table[x=l, y=t4] {data/planck.txt} node at (axis cs:870, 1.6e+13) {\tiny{$4500$~K}};
   \addplot+[black] table[x=l, y=t5] {data/planck.txt}node at (axis cs:750, 4.2e+13) {\tiny{$5000$~K}};
   \addplot+[black] table[x=l, y=t58] {data/planck.txt}node at (axis cs:670, 8.5e+13) {\tiny{$5800$~K}};
   \addplot+[black] table[x=l, y=t7] {data/planck.txt}node at (axis cs:1350, 3.5e+13) {\tiny{$7000$~K}};
	%\addplot+[black, smooth] table[x=l, y=t15] {data/planck.txt} node at (axis cs:1670, 5.5e+13) {\tiny{$15000$~K}};
  \end{axis}
 \end{tikzpicture}
\caption{Кривые спектральной плотности мощности изотропного излучения АЧТ с разной температурой}\label{pic:wien-law}
\end{wrapfigure}
Стоит заметить, что при переходе в функции к длинам волн меняется не только частота на длину волны, но и выражение для интервала. 

Формула Планка появилась, когда стало ясно, что формула Рэлея-Джинса удовлетворительно описывает излучение только в области больших длин волн, а~с~убыванием длин волн даёт сильные расхождения с реальными данными. Однако формулу Рэлея-Джинса используют и сейчас для описания кривой Планка на больших длинах волн. 

\change{
Проделаем обратные действия: получим формулу Рэлея-Джинса из формулы Планка. Длинноволновая часть спектра характеризуется соотношением $h\nu \ll kT$, то есть 
\begin{equation*}
	\exp\left( \frac{h\nu}{kT}\right) \approx 1 + \frac{h\nu}{kT}.
\end{equation*}
Подставляя полученное выражение в знаменатель \eqref{eq:plancks-law-nu}, получим
\begin{equation*}
	B_\nu(\nu,T) \approx \frac{2h\nu^3}{c^2}\cdot \frac{1}{1 + \frac{h\nu}{kT} - 1} = \frac{2h\nu^3 }{c^2}\cdot \frac{k T}{ h \nu} = \frac{2 \nu^2 k T}{c^2}.
\end{equation*}
}
\change{
	Проделав то же самое для выражения через длину волны, получим:
}
\begin{equation}
	B(\lambda, T) \simeq \frac{2 c k T}{\lambda^4}, \quad\quad B(\nu, T) \simeq \frac{2 \nu^2 k T}{c^2}.
\label{Ray-Jean}
\end{equation}

\change{
	В коротковолновой области, наоборот, $h \nu \gg kT$, следовательно, в знаменателе формулы Планка единица много меньше стоящей там экспоненты, то есть
	\begin{equation*}
		\frac{1}{\exp\left(\frac{h\nu}{kT}\right)-1} \approx \frac{1}{\exp\left(\frac{h\nu}{kT}\right)} = \exp\left(-\frac{h\nu}{kT}\right).
	\end{equation*} 
	Отсюда получаются выражения, называемые приближением Вина:
}
\begin{equation}
B ( \lambda, T) \simeq \frac{2 h c^2}{\lambda^5} \exp \left( -\frac{h c}{\lambda k T}\right), \quad \quad B( \nu, T ) \simeq \frac{2 h \nu^3}{c^2} \exp \left( -\frac{h \nu}{k T} \right).
\end{equation}
\subsection{Закон смещения Вина}
\term{Закон смещения Вина} --- закон, устанавливающий зависимость длины волны~$\lambda_\text{макс}$, на которой спектральная плотность излучения $B_\lambda(\lambda, T)$ абсолютно чёрного тела достигает своего максимума, от температуры $T$ этого тела:
\begin{equation}
	\lambda_\text{макс} \approx \frac{b}{T} \equiv \frac{0.0029~\text{м} \cdot \text{К}}{T}.
\end{equation}
Закон является следствием исследования функции Планка (см.~\ref{sec:planck-law}) на экстремальность.
\subsection{Эффект Доплера. Красное смещение}
\term{Эффект Доплера}~--- эффект изменения частоты и длины волны электромагнитного излучения, регистрируемого приёмником, вызванный относительным движением источника и приёмника (см.~Рис.\,\ref{doppler-ef}).

При $\Delta \lambda \ll \lambda_0$ с большой точностью выполняется следующее важное соотношение:\begin{equation}
\beta \equiv \dfrac{v}{c} = \dfrac{\lambda - \lambda_0}{\lambda_0} \equiv \dfrac{\Delta \lambda}{\lambda_0},
\label{eq:dopler-ef-simple}
\end{equation}
\begin{wrapfigure}[6]{r}{0.5\tw}
\centering
\vspace{-.5pc}
\includegraphics[width=.5\tw]{doppler-ef}
\caption{Эффект Доплера}
\label{doppler-ef}
\end{wrapfigure}
где $\lambda_0$~--- лабораторная длина волны излучения источника, а $\lambda$~--- наблюдаемая. В действительности же имеет место более общий случай: \imp{релятивистский эффект Доплера}, обусловленный проявлением СТО при $v \simeq c$, для которого формула~\eqref{eq:dopler-ef-simple} усложняется и принимает вид
\begin{equation}
\nu = \nu_0 \cdot \dfrac{\sqrt{1 - \beta^2}}{1 + \beta \cdot \cos\theta},
\label{eq:dopler-ef-rel}
\end{equation}
где $\nu$~--- частота, с которой наблюдатель принимает волны, $\nu_0$~--- частота, с которой источник испускает волны, $v$~--- скорость источника, $\theta$~--- угол между направлением на источник и вектором его скорости в системе отсчёта приёмника. Если источник радиально удаляется от наблюдателя, то $\theta = 0$, если приближается, то $\theta =\pi$. Важно, что~\eqref{eq:dopler-ef-simple} напрямую следует из \eqref{eq:dopler-ef-rel} при $\beta  \ll 1$.

\term{Красное смещение}~--- явление сдвига спектральных линий химических элементов в красную (длинноволновую) сторону, обусловленное относительным движение объектов. Параметр красного смещения определяется из наблюдаемой и лабораторной длин волн как
\begin{equation}
z = \dfrac{\lambda - \lambda_0}{\lambda_0}.
\end{equation}

Доплеровское смещение длины волны в спектре источника, движущегося с лучевой скоростью $v_{r}$ и полной скоростью $v$,
\begin{equation}
z = \dfrac{1 + v_r / c}{\sqrt{1 - \beta^2}}.
\end{equation}

\term{Гравитационное красное смещение}~--- проявление эффекта изменения частоты излучения, испущенного массивным объектом, таким как звезда или чёрная дыра. Наблюдается как сдвиг спектральных линий в спектре источника в красную область спектра. Гравитационное красное смещение определяется из формулы, выведенной Эйнштейном,
\begin{equation}
z_G=\dfrac{GM}{c^2 R}-\dfrac{GM}{c^2 r},
\label{eq:grav-red-shift}
\end{equation}
где $M$~--- масса гравитирующего тела, $R$~--- радиальное расстояние от центра масс тела до точки излучения (радиус источника), $r$~---  радиальное расстояние от центра масс источника до точки наблюдения. В случае, когда наблюдатель находится от источника много дальше его радиуса, т.\,е. выполняется соотношение $r \gg R$, выражение~\eqref{eq:grav-red-shift} можно упростить до
\begin{equation}
z_G \simeq \dfrac{GM}{c^2 R}.
\end{equation}

\input{sections/astrophys.light-pressure.tex}
\input{sections/astrophys.edd.tex}
\input{sections/astrophys.grav-lens.tex}
\subsection{Закон Хаббла}
\term{Закон Хаббла}~--- эмпирический закон, связывающий скорость удаления галактик $V$ и расстояние $R$ до них линейным образом: 
\begin{equation}
	V = H R,
\end{equation}
величина $H=68~\text{км/c} \cdot \text{Мпк})$ называется \imp{постоянной Хаббла}.

При $v \ll c$ можно использовать приближение эффекта Доплера, тогда
\begin{equation}
	V = c z.
\label{eq:hubble-speed}
\end{equation}

Равенство \eqref{eq:hubble-speed} справедливо только при $z \ll 1$, а при б\'{o}льших значениях $z$ космологическое красное смещение нльзя связывать с эффектом Доплера, поэтому можно пользоваться только формулой 
\begin{equation}
	\frac{dz}{dt} = - H(z)(1+z),
\end{equation}
где постоянная Хаббла введена как функция красного смещения.
\subsection{Шкала электромагнитных волн}


\term{Гамма излучение} возникает при радиоактивных распадах ядер, при торможении электронов энергией более $10^5$~эВ и при других взаимодействиях элементарных частиц. Используются в гамма-дефектоскопии, при изучении свойств вещества.

\term{Рентгеновские лучи} излучаются при большом ускорении электронов, например при их торможении в металлах. Получают их при помощи рентгеновской трубки: электроны в вакуумной трубке ускоряются электрическим полем при высоком напряжении, достигая анода, при со­ударении резко тормозятся. При торможении электроны движут­ся с ускорением и излучают электромагнитные волны с малой длиной. 

\begin{figure}[!h]
\centering
\includegraphics[width = 1\textwidth]{scale-wave.pdf}
\caption{Шкала электромагнитных волн}
\end{figure}
\term{Ультрафиолетовые лучи}~--- излучение Солнца, ртутных ламп и т.\,п. Используются в ультрафиолетовой микроскопии, в медицине.

\term{Видимое излучение}~--- часть электромагнитного излучения, воспринимаемая глазом (от фиолетового до от красного).

\term{Инфракрасное излучение}~--- тепловое, излучается любым нагретым телом.

\term{Радиоволны} используются повсеместно в обычной жизни, это и сотовая связь, и радиолокация, и спутниковая связь, и Wi-Fi и многое другое.

\term{Низкочастотные волны}~--- диапазон, традиционно используемый в электротехнике. В промышленной электроэнергетике используется частота 50~Гц, на~которой осуществляется передача электрической энергии по линиям и преобразование напряжений трансформаторными устройствами.
\input{sections/astrophys.spec-theor-rel.tex}
\subsection{Оптическая толщина. Закон Бугера}
\term{Оптическая толщина}~--- безразмерная величина, характеризующая степень непрозрачности среды для проходящего сквозь неё излучения,
\begin{equation}
\tau = \int n(x) \sigma(x)\,dx,
\end{equation}
где $\tau$~--- оптическая толщина среды, $n$~--- концентрация частиц, $\sigma$~--- сечение их взаимодействия.

Поток $I_0$ на входе связан с потоком $I$ на выходе \term{Законом Бугера}:
\begin{equation}
I = I_0 e^{-\tau}.
\end{equation}
\input{sections/astrophys.colour.tex}
\input{sections/astrophys.mkt.tex}
\input{sections/astrophys.earth-atmosphere.tex}

\newpage
\section{Астрофизика}
\subsection{Звёздные величины}
Звёздная величина~--- безразмерная числовая характеристика яркости объекта. Известно, что увеличению светового потока в 100 раз соответствует уменьшение видимой звёздной величины ровно на 5 единиц. Тогда уменьшение звёздной величины на одну единицу означает увеличение светового потока в $\sqrt[5]{100}\approx 2.512$~раз, то есть звёздные величины являются логарифмической шкалой измерения плотности потока. Зависимость, связывающая отношение освещённостей $E_1$ и $E_2$ и разность звёздных величин $m_1$ и $m_2$ двух объектов, называется \term{формулой Погсона} и имеет вид
\begin{equation}
	\frac{E_1}{E_2} = 10^{0.4(m_2 - m_1)} \quad \Longleftrightarrow \quad m_2 = m_1 + 2.5 \lg \frac{E_1}{E_2}.
	\label{eq:Pogson-law}
\end{equation}
Широко используется понятие \term{абсолютной звёздной величины} $M$~--- это видимая звёздная величина $m$ при наблюдении с установленного расстояния: для звёзд~---~10~пк, для тел Солнечной системы~---~1~\au, причем считается, что тело находится в 1~\au~и от наблюдателя и от Солнца, а фаза равна единице, то есть можно считать, что наблюдатель находится в центре Солнца, а~тело~--- в~1~\au~от него. 

Кроме этого, важно понятие \term{болометрической звёздной величины} $m_\text{bol}$~--- это звёздная величина, при расчёте которой учитывается полная мощность излучения источника во всех диапазонах электромагнитных волн. Обычная (видимая) звёздная величина учитывает излучение лишь в видимой части спектра от примерно 380~нм до примерно~780~нм. Разность между болометрической и видимой звёздными величинами называется \term{болометрической поправкой} ($BC$), которая отличается для разных спектральных классов звёзд. Из определения, болометрическая поправка может быть найдена по формуле
\begin{equation}
	BC = m_\text{bol} - m.
\end{equation}
Абсолютную звёздную величину звезды можно получить по формуле Погсона \eqref{eq:Pogson-law} из видимой звёздной величины $m$ и расстояния $r$ до неё в парсеках
\begin{equation}
	M = m + 2.5 \lg \frac{E}{E_\text{абс}} = m + 2.5 \lg \frac{(10~\text{пк})^2}{r^2} = m + 5 - 5\lg r.
	\label{eq:abs-mag}
\end{equation} 
Если принимать к рассмотрению межзвездное поглощение $A$, то формулу  \eqref{eq:abs-mag} необходимо уточнить:
\begin{equation}
	M = m + 5 - 5\lg r - Ar.
\end{equation}
\subsection{Закон Стефана-Больцмана}
\term{Закон Стефана~--- Больцмана} определяет зависимость плотности мощности излучения абсолютно чёрного тела (АЧТ) $u$ от его температуры $T$:
\begin{equation}
u = a T^4,
\end{equation} 
где $a$~--- некая универсальная константа.
Отсюда полная светимость АЧТ с площадью поверхности $S$
	\begin{equation}
	L = S \sigma T^4,
	\label{eq:steff-bol-law}
\end{equation}
константа $\sigma$ называется \term{постоянной Стефана-Больцмана}.
  
Важно отметить, что \imp{закон Стефана-Больцмана}~--- прямое следствие формулы Планка \eqref{Planck's formula}, так как
\begin{equation}
	\sigma T^4 = \int\limits^\infty_0 B(\lambda, T)\,d\lambda \int\limits_0^{\pi/2} \sin \varphi\, d\varphi \int\limits_0^{2\pi} \cos \varphi\, d\theta = \pi \int\limits^\infty_0 B(\lambda, T)\,d\lambda,
\end{equation}
откуда $\sigma = (2\pi^5k^4)/(15c^2h^3) = 5.67 \cdot 10^{-8}~\text{Вт}/(\text{м}^2\cdot \text{К}^4)$.

%Для АЧТ сферической формы с радиусом $R$ формула~\eqref{eq:steff-bol-law} принимает вид
%\begin{equation}
%L=4\pi R^2\sigma T^4.
%\end{equation}
Для звёзд главной последовательности выполняется соотношение $L \sim M^{\alpha}$, где~$\alpha$~--- коэффициент пропорциональности, который зависит от массы звезды следующим образом:
\begin{align*}
\alpha &= 2.5, \quad M < 0.43 M_\odot; & 
\alpha &= 4, \quad 0.43 M_\odot < M < 2 M_\odot;\\ 
\alpha &= 3.2, \quad 2 M_\odot < M < 20 M_\odot; & 
\alpha &= 1, \quad M > 20 M_\odot.
\end{align*}
Также существует примерная зависимость светимости звёзды от её радиуса, имеющая вид  $L\sim R^{5.2}$.
\subsection{Энергия излучения}
\term{Энергия излучения}~--- энергия, переносимая излучением ($Q_e$).\\
\term{Поток излучения}~--- физическая величина, характеризующая мощность, переносимую излучением,
\begin{equation}
 \Phi_e = \frac{d Q_e}{dt}.
\end{equation}
\imp{Теорема Гаусса}: через любую замкнутую поверхность потоки от одинаковых источников равны.

\term{Спектральная плотность потока излучения}~--- поток излучения, приходящийся на малый единичный интервал спектра,
\begin{equation}
\Phi_{e, \lambda}(\lambda) = \frac{d\Phi_e(\lambda)}{d\lambda}, \quad\quad \Phi_{e, \nu}(\nu) = \frac{d\Phi_e(\nu)}{d\nu} =  \frac{\lambda^2}{c}\Phi_{e, \lambda}(\lambda).
\end{equation}

\term{Объемная плотность энергии излучения}~--- количество энергии на единицу объема
\begin{equation}
U_e = \frac{d Q_e}{dV}.
\end{equation}

\term{Светимость}~--- величина, представляющая собой световой поток излучения, испускаемого с малого участка светящейся поверхности единичной площади,
\begin{equation}
M_e = \frac{d \Phi_e}{dS_1},
\end{equation}
здесь $S_1$~--- площадь объекта, испускающего энергию.

\term{Яркость}~--- световой поток, приходящийся на единичный телесный угол, в расчёте на единичную площадку проекции излучающей поверхности на картинную плоскость, 
\begin{equation}
L_e = \frac{d^2 \Phi_e}{d \Omega\,dS_1 \cos \varepsilon},
\end{equation}
где $\varepsilon$~--- угол между направлением потока излучения и нормалью к плоскости излучающей поверхности.

\term{Интегральная яркость}~--- интеграл яркости по видимой поверхности источника. Показывает количество энергии, пришедшее от источника за единицу времени.
\begin{equation}
\Lambda_e = \int \limits_S L_e(\vec{r})\,ds.
\end{equation}
\term{Освещенность}~--- величина, равная отношению светового потока, падающего на малый участок поверхности, к его площади~--- поверхностная плотность потока
\begin{equation}
E_e = \frac{d\Phi_e}{dS_2} \sim \frac{1}{r^2},
\end{equation}
здесь $S_2$~--- площадь поверхности приёмника, $r$~--- расстояние от источника.
\input{sections/astrophys.flux-albedo.tex}
\input{sections/astrophys.photon.tex}
\input{sections/astrophys.energy-lines.tex}
\subsection{Формула Планка}
\label{sec:planck-law}
\term{Формула Планка}~--- выражение для спектральной плотности мощности излучения абсолютно чёрного тела на интервале частот $[\nu, \nu + d \nu)$, распространяющейся с телесном угле $d\Omega$, которое было получено Максом Планком в 1900~году. Данное выражение имеет следующий вид:
\begin{equation}
B_\nu(\nu,T)=\frac{2h\nu^3}{c^2}\cdot \frac{1}{\exp\left(\frac{h\nu}{kT}\right)-1} = \left[ \frac{\text{Вт}}{\text{м}^2 \cdot \text{Гц} \cdot \text{ср}}\right],
\label{eq:plancks-law-nu}
\end{equation}
где $\nu$~--- частота излучения, $T$~--- температура АЧТ, $h$~--- постоянная Планка, $k$~--- постоянная Больцмана, $c$~--- скорость света.

Если записать закон излучения Планка \eqref{eq:plancks-law-nu} для длин волн, то
\begin{equation}
B_\lambda(\lambda,T)=\frac{2hc^2}{\lambda^5} \cdot \frac{1}{\exp\left(\frac{hc}{\lambda kT}\right)-1} = \left[ \frac{\text{Вт}}{\text{м}^3 \cdot \text{ср}}\right].
\label{eq:plancks-law-lambda}
\end{equation}
\begin{wrapfigure}[15]{l}{.6\tw}
\centering
\vspace{-.9pc}
 \begin{tikzpicture}
  \begin{axis}[
  				width 	=	.6\tw, 
				height	=	6cm, 
  				ymax	=	1e+14,
  				xmax	=	2000,
  				xmin	=	0,
  				ymin	=	0,
				xlabel	=	{Длина волны $\lambda$,~нм}, 
				ylabel 	= 	{$B_\lambda(\lambda, T)$,~$\text{Вт} \cdot \text{м}^{-3}$}
]
   \addplot+[dashed, thin, black] table[x=l, y=tl] {data/planck.txt};
   \addplot+[black] table[x=l, y=t4] {data/planck.txt} node at (axis cs:870, 1.6e+13) {\tiny{$4500$~K}};
   \addplot+[black] table[x=l, y=t5] {data/planck.txt}node at (axis cs:750, 4.2e+13) {\tiny{$5000$~K}};
   \addplot+[black] table[x=l, y=t58] {data/planck.txt}node at (axis cs:670, 8.5e+13) {\tiny{$5800$~K}};
   \addplot+[black] table[x=l, y=t7] {data/planck.txt}node at (axis cs:1350, 3.5e+13) {\tiny{$7000$~K}};
	%\addplot+[black, smooth] table[x=l, y=t15] {data/planck.txt} node at (axis cs:1670, 5.5e+13) {\tiny{$15000$~K}};
  \end{axis}
 \end{tikzpicture}
\caption{Кривые спектральной плотности мощности изотропного излучения АЧТ с разной температурой}\label{pic:wien-law}
\end{wrapfigure}
Стоит заметить, что при переходе в функции к длинам волн меняется не только частота на длину волны, но и выражение для интервала. 

Формула Планка появилась, когда стало ясно, что формула Рэлея-Джинса удовлетворительно описывает излучение только в области больших длин волн, а~с~убыванием длин волн даёт сильные расхождения с реальными данными. Однако формулу Рэлея-Джинса используют и сейчас для описания кривой Планка на больших длинах волн. 

\change{
Проделаем обратные действия: получим формулу Рэлея-Джинса из формулы Планка. Длинноволновая часть спектра характеризуется соотношением $h\nu \ll kT$, то есть 
\begin{equation*}
	\exp\left( \frac{h\nu}{kT}\right) \approx 1 + \frac{h\nu}{kT}.
\end{equation*}
Подставляя полученное выражение в знаменатель \eqref{eq:plancks-law-nu}, получим
\begin{equation*}
	B_\nu(\nu,T) \approx \frac{2h\nu^3}{c^2}\cdot \frac{1}{1 + \frac{h\nu}{kT} - 1} = \frac{2h\nu^3 }{c^2}\cdot \frac{k T}{ h \nu} = \frac{2 \nu^2 k T}{c^2}.
\end{equation*}
}
\change{
	Проделав то же самое для выражения через длину волны, получим:
}
\begin{equation}
	B(\lambda, T) \simeq \frac{2 c k T}{\lambda^4}, \quad\quad B(\nu, T) \simeq \frac{2 \nu^2 k T}{c^2}.
\label{Ray-Jean}
\end{equation}

\change{
	В коротковолновой области, наоборот, $h \nu \gg kT$, следовательно, в знаменателе формулы Планка единица много меньше стоящей там экспоненты, то есть
	\begin{equation*}
		\frac{1}{\exp\left(\frac{h\nu}{kT}\right)-1} \approx \frac{1}{\exp\left(\frac{h\nu}{kT}\right)} = \exp\left(-\frac{h\nu}{kT}\right).
	\end{equation*} 
	Отсюда получаются выражения, называемые приближением Вина:
}
\begin{equation}
B ( \lambda, T) \simeq \frac{2 h c^2}{\lambda^5} \exp \left( -\frac{h c}{\lambda k T}\right), \quad \quad B( \nu, T ) \simeq \frac{2 h \nu^3}{c^2} \exp \left( -\frac{h \nu}{k T} \right).
\end{equation}
\subsection{Закон смещения Вина}
\term{Закон смещения Вина} --- закон, устанавливающий зависимость длины волны~$\lambda_\text{макс}$, на которой спектральная плотность излучения $B_\lambda(\lambda, T)$ абсолютно чёрного тела достигает своего максимума, от температуры $T$ этого тела:
\begin{equation}
	\lambda_\text{макс} \approx \frac{b}{T} \equiv \frac{0.0029~\text{м} \cdot \text{К}}{T}.
\end{equation}
Закон является следствием исследования функции Планка (см.~\ref{sec:planck-law}) на экстремальность.
\subsection{Эффект Доплера. Красное смещение}
\term{Эффект Доплера}~--- эффект изменения частоты и длины волны электромагнитного излучения, регистрируемого приёмником, вызванный относительным движением источника и приёмника (см.~Рис.\,\ref{doppler-ef}).

При $\Delta \lambda \ll \lambda_0$ с большой точностью выполняется следующее важное соотношение:\begin{equation}
\beta \equiv \dfrac{v}{c} = \dfrac{\lambda - \lambda_0}{\lambda_0} \equiv \dfrac{\Delta \lambda}{\lambda_0},
\label{eq:dopler-ef-simple}
\end{equation}
\begin{wrapfigure}[6]{r}{0.5\tw}
\centering
\vspace{-.5pc}
\includegraphics[width=.5\tw]{doppler-ef}
\caption{Эффект Доплера}
\label{doppler-ef}
\end{wrapfigure}
где $\lambda_0$~--- лабораторная длина волны излучения источника, а $\lambda$~--- наблюдаемая. В действительности же имеет место более общий случай: \imp{релятивистский эффект Доплера}, обусловленный проявлением СТО при $v \simeq c$, для которого формула~\eqref{eq:dopler-ef-simple} усложняется и принимает вид
\begin{equation}
\nu = \nu_0 \cdot \dfrac{\sqrt{1 - \beta^2}}{1 + \beta \cdot \cos\theta},
\label{eq:dopler-ef-rel}
\end{equation}
где $\nu$~--- частота, с которой наблюдатель принимает волны, $\nu_0$~--- частота, с которой источник испускает волны, $v$~--- скорость источника, $\theta$~--- угол между направлением на источник и вектором его скорости в системе отсчёта приёмника. Если источник радиально удаляется от наблюдателя, то $\theta = 0$, если приближается, то $\theta =\pi$. Важно, что~\eqref{eq:dopler-ef-simple} напрямую следует из \eqref{eq:dopler-ef-rel} при $\beta  \ll 1$.

\term{Красное смещение}~--- явление сдвига спектральных линий химических элементов в красную (длинноволновую) сторону, обусловленное относительным движение объектов. Параметр красного смещения определяется из наблюдаемой и лабораторной длин волн как
\begin{equation}
z = \dfrac{\lambda - \lambda_0}{\lambda_0}.
\end{equation}

Доплеровское смещение длины волны в спектре источника, движущегося с лучевой скоростью $v_{r}$ и полной скоростью $v$,
\begin{equation}
z = \dfrac{1 + v_r / c}{\sqrt{1 - \beta^2}}.
\end{equation}

\term{Гравитационное красное смещение}~--- проявление эффекта изменения частоты излучения, испущенного массивным объектом, таким как звезда или чёрная дыра. Наблюдается как сдвиг спектральных линий в спектре источника в красную область спектра. Гравитационное красное смещение определяется из формулы, выведенной Эйнштейном,
\begin{equation}
z_G=\dfrac{GM}{c^2 R}-\dfrac{GM}{c^2 r},
\label{eq:grav-red-shift}
\end{equation}
где $M$~--- масса гравитирующего тела, $R$~--- радиальное расстояние от центра масс тела до точки излучения (радиус источника), $r$~---  радиальное расстояние от центра масс источника до точки наблюдения. В случае, когда наблюдатель находится от источника много дальше его радиуса, т.\,е. выполняется соотношение $r \gg R$, выражение~\eqref{eq:grav-red-shift} можно упростить до
\begin{equation}
z_G \simeq \dfrac{GM}{c^2 R}.
\end{equation}

\input{sections/astrophys.light-pressure.tex}
\input{sections/astrophys.edd.tex}
\input{sections/astrophys.grav-lens.tex}
\subsection{Закон Хаббла}
\term{Закон Хаббла}~--- эмпирический закон, связывающий скорость удаления галактик $V$ и расстояние $R$ до них линейным образом: 
\begin{equation}
	V = H R,
\end{equation}
величина $H=68~\text{км/c} \cdot \text{Мпк})$ называется \imp{постоянной Хаббла}.

При $v \ll c$ можно использовать приближение эффекта Доплера, тогда
\begin{equation}
	V = c z.
\label{eq:hubble-speed}
\end{equation}

Равенство \eqref{eq:hubble-speed} справедливо только при $z \ll 1$, а при б\'{o}льших значениях $z$ космологическое красное смещение нльзя связывать с эффектом Доплера, поэтому можно пользоваться только формулой 
\begin{equation}
	\frac{dz}{dt} = - H(z)(1+z),
\end{equation}
где постоянная Хаббла введена как функция красного смещения.
\subsection{Шкала электромагнитных волн}


\term{Гамма излучение} возникает при радиоактивных распадах ядер, при торможении электронов энергией более $10^5$~эВ и при других взаимодействиях элементарных частиц. Используются в гамма-дефектоскопии, при изучении свойств вещества.

\term{Рентгеновские лучи} излучаются при большом ускорении электронов, например при их торможении в металлах. Получают их при помощи рентгеновской трубки: электроны в вакуумной трубке ускоряются электрическим полем при высоком напряжении, достигая анода, при со­ударении резко тормозятся. При торможении электроны движут­ся с ускорением и излучают электромагнитные волны с малой длиной. 

\begin{figure}[!h]
\centering
\includegraphics[width = 1\textwidth]{scale-wave.pdf}
\caption{Шкала электромагнитных волн}
\end{figure}
\term{Ультрафиолетовые лучи}~--- излучение Солнца, ртутных ламп и т.\,п. Используются в ультрафиолетовой микроскопии, в медицине.

\term{Видимое излучение}~--- часть электромагнитного излучения, воспринимаемая глазом (от фиолетового до от красного).

\term{Инфракрасное излучение}~--- тепловое, излучается любым нагретым телом.

\term{Радиоволны} используются повсеместно в обычной жизни, это и сотовая связь, и радиолокация, и спутниковая связь, и Wi-Fi и многое другое.

\term{Низкочастотные волны}~--- диапазон, традиционно используемый в электротехнике. В промышленной электроэнергетике используется частота 50~Гц, на~которой осуществляется передача электрической энергии по линиям и преобразование напряжений трансформаторными устройствами.
\input{sections/astrophys.spec-theor-rel.tex}
\subsection{Оптическая толщина. Закон Бугера}
\term{Оптическая толщина}~--- безразмерная величина, характеризующая степень непрозрачности среды для проходящего сквозь неё излучения,
\begin{equation}
\tau = \int n(x) \sigma(x)\,dx,
\end{equation}
где $\tau$~--- оптическая толщина среды, $n$~--- концентрация частиц, $\sigma$~--- сечение их взаимодействия.

Поток $I_0$ на входе связан с потоком $I$ на выходе \term{Законом Бугера}:
\begin{equation}
I = I_0 e^{-\tau}.
\end{equation}
\input{sections/astrophys.colour.tex}
\input{sections/astrophys.mkt.tex}
\input{sections/astrophys.earth-atmosphere.tex}

\newpage
\section{Астрофизика}
\subsection{Звёздные величины}
Звёздная величина~--- безразмерная числовая характеристика яркости объекта. Известно, что увеличению светового потока в 100 раз соответствует уменьшение видимой звёздной величины ровно на 5 единиц. Тогда уменьшение звёздной величины на одну единицу означает увеличение светового потока в $\sqrt[5]{100}\approx 2.512$~раз, то есть звёздные величины являются логарифмической шкалой измерения плотности потока. Зависимость, связывающая отношение освещённостей $E_1$ и $E_2$ и разность звёздных величин $m_1$ и $m_2$ двух объектов, называется \term{формулой Погсона} и имеет вид
\begin{equation}
	\frac{E_1}{E_2} = 10^{0.4(m_2 - m_1)} \quad \Longleftrightarrow \quad m_2 = m_1 + 2.5 \lg \frac{E_1}{E_2}.
	\label{eq:Pogson-law}
\end{equation}
Широко используется понятие \term{абсолютной звёздной величины} $M$~--- это видимая звёздная величина $m$ при наблюдении с установленного расстояния: для звёзд~---~10~пк, для тел Солнечной системы~---~1~\au, причем считается, что тело находится в 1~\au~и от наблюдателя и от Солнца, а фаза равна единице, то есть можно считать, что наблюдатель находится в центре Солнца, а~тело~--- в~1~\au~от него. 

Кроме этого, важно понятие \term{болометрической звёздной величины} $m_\text{bol}$~--- это звёздная величина, при расчёте которой учитывается полная мощность излучения источника во всех диапазонах электромагнитных волн. Обычная (видимая) звёздная величина учитывает излучение лишь в видимой части спектра от примерно 380~нм до примерно~780~нм. Разность между болометрической и видимой звёздными величинами называется \term{болометрической поправкой} ($BC$), которая отличается для разных спектральных классов звёзд. Из определения, болометрическая поправка может быть найдена по формуле
\begin{equation}
	BC = m_\text{bol} - m.
\end{equation}
Абсолютную звёздную величину звезды можно получить по формуле Погсона \eqref{eq:Pogson-law} из видимой звёздной величины $m$ и расстояния $r$ до неё в парсеках
\begin{equation}
	M = m + 2.5 \lg \frac{E}{E_\text{абс}} = m + 2.5 \lg \frac{(10~\text{пк})^2}{r^2} = m + 5 - 5\lg r.
	\label{eq:abs-mag}
\end{equation} 
Если принимать к рассмотрению межзвездное поглощение $A$, то формулу  \eqref{eq:abs-mag} необходимо уточнить:
\begin{equation}
	M = m + 5 - 5\lg r - Ar.
\end{equation}
\subsection{Закон Стефана-Больцмана}
\term{Закон Стефана~--- Больцмана} определяет зависимость плотности мощности излучения абсолютно чёрного тела (АЧТ) $u$ от его температуры $T$:
\begin{equation}
u = a T^4,
\end{equation} 
где $a$~--- некая универсальная константа.
Отсюда полная светимость АЧТ с площадью поверхности $S$
	\begin{equation}
	L = S \sigma T^4,
	\label{eq:steff-bol-law}
\end{equation}
константа $\sigma$ называется \term{постоянной Стефана-Больцмана}.
  
Важно отметить, что \imp{закон Стефана-Больцмана}~--- прямое следствие формулы Планка \eqref{Planck's formula}, так как
\begin{equation}
	\sigma T^4 = \int\limits^\infty_0 B(\lambda, T)\,d\lambda \int\limits_0^{\pi/2} \sin \varphi\, d\varphi \int\limits_0^{2\pi} \cos \varphi\, d\theta = \pi \int\limits^\infty_0 B(\lambda, T)\,d\lambda,
\end{equation}
откуда $\sigma = (2\pi^5k^4)/(15c^2h^3) = 5.67 \cdot 10^{-8}~\text{Вт}/(\text{м}^2\cdot \text{К}^4)$.

%Для АЧТ сферической формы с радиусом $R$ формула~\eqref{eq:steff-bol-law} принимает вид
%\begin{equation}
%L=4\pi R^2\sigma T^4.
%\end{equation}
Для звёзд главной последовательности выполняется соотношение $L \sim M^{\alpha}$, где~$\alpha$~--- коэффициент пропорциональности, который зависит от массы звезды следующим образом:
\begin{align*}
\alpha &= 2.5, \quad M < 0.43 M_\odot; & 
\alpha &= 4, \quad 0.43 M_\odot < M < 2 M_\odot;\\ 
\alpha &= 3.2, \quad 2 M_\odot < M < 20 M_\odot; & 
\alpha &= 1, \quad M > 20 M_\odot.
\end{align*}
Также существует примерная зависимость светимости звёзды от её радиуса, имеющая вид  $L\sim R^{5.2}$.
\subsection{Энергия излучения}
\term{Энергия излучения}~--- энергия, переносимая излучением ($Q_e$).\\
\term{Поток излучения}~--- физическая величина, характеризующая мощность, переносимую излучением,
\begin{equation}
 \Phi_e = \frac{d Q_e}{dt}.
\end{equation}
\imp{Теорема Гаусса}: через любую замкнутую поверхность потоки от одинаковых источников равны.

\term{Спектральная плотность потока излучения}~--- поток излучения, приходящийся на малый единичный интервал спектра,
\begin{equation}
\Phi_{e, \lambda}(\lambda) = \frac{d\Phi_e(\lambda)}{d\lambda}, \quad\quad \Phi_{e, \nu}(\nu) = \frac{d\Phi_e(\nu)}{d\nu} =  \frac{\lambda^2}{c}\Phi_{e, \lambda}(\lambda).
\end{equation}

\term{Объемная плотность энергии излучения}~--- количество энергии на единицу объема
\begin{equation}
U_e = \frac{d Q_e}{dV}.
\end{equation}

\term{Светимость}~--- величина, представляющая собой световой поток излучения, испускаемого с малого участка светящейся поверхности единичной площади,
\begin{equation}
M_e = \frac{d \Phi_e}{dS_1},
\end{equation}
здесь $S_1$~--- площадь объекта, испускающего энергию.

\term{Яркость}~--- световой поток, приходящийся на единичный телесный угол, в расчёте на единичную площадку проекции излучающей поверхности на картинную плоскость, 
\begin{equation}
L_e = \frac{d^2 \Phi_e}{d \Omega\,dS_1 \cos \varepsilon},
\end{equation}
где $\varepsilon$~--- угол между направлением потока излучения и нормалью к плоскости излучающей поверхности.

\term{Интегральная яркость}~--- интеграл яркости по видимой поверхности источника. Показывает количество энергии, пришедшее от источника за единицу времени.
\begin{equation}
\Lambda_e = \int \limits_S L_e(\vec{r})\,ds.
\end{equation}
\term{Освещенность}~--- величина, равная отношению светового потока, падающего на малый участок поверхности, к его площади~--- поверхностная плотность потока
\begin{equation}
E_e = \frac{d\Phi_e}{dS_2} \sim \frac{1}{r^2},
\end{equation}
здесь $S_2$~--- площадь поверхности приёмника, $r$~--- расстояние от источника.
\input{sections/astrophys.flux-albedo.tex}
\input{sections/astrophys.photon.tex}
\input{sections/astrophys.energy-lines.tex}
\subsection{Формула Планка}
\label{sec:planck-law}
\term{Формула Планка}~--- выражение для спектральной плотности мощности излучения абсолютно чёрного тела на интервале частот $[\nu, \nu + d \nu)$, распространяющейся с телесном угле $d\Omega$, которое было получено Максом Планком в 1900~году. Данное выражение имеет следующий вид:
\begin{equation}
B_\nu(\nu,T)=\frac{2h\nu^3}{c^2}\cdot \frac{1}{\exp\left(\frac{h\nu}{kT}\right)-1} = \left[ \frac{\text{Вт}}{\text{м}^2 \cdot \text{Гц} \cdot \text{ср}}\right],
\label{eq:plancks-law-nu}
\end{equation}
где $\nu$~--- частота излучения, $T$~--- температура АЧТ, $h$~--- постоянная Планка, $k$~--- постоянная Больцмана, $c$~--- скорость света.

Если записать закон излучения Планка \eqref{eq:plancks-law-nu} для длин волн, то
\begin{equation}
B_\lambda(\lambda,T)=\frac{2hc^2}{\lambda^5} \cdot \frac{1}{\exp\left(\frac{hc}{\lambda kT}\right)-1} = \left[ \frac{\text{Вт}}{\text{м}^3 \cdot \text{ср}}\right].
\label{eq:plancks-law-lambda}
\end{equation}
\begin{wrapfigure}[15]{l}{.6\tw}
\centering
\vspace{-.9pc}
 \begin{tikzpicture}
  \begin{axis}[
  				width 	=	.6\tw, 
				height	=	6cm, 
  				ymax	=	1e+14,
  				xmax	=	2000,
  				xmin	=	0,
  				ymin	=	0,
				xlabel	=	{Длина волны $\lambda$,~нм}, 
				ylabel 	= 	{$B_\lambda(\lambda, T)$,~$\text{Вт} \cdot \text{м}^{-3}$}
]
   \addplot+[dashed, thin, black] table[x=l, y=tl] {data/planck.txt};
   \addplot+[black] table[x=l, y=t4] {data/planck.txt} node at (axis cs:870, 1.6e+13) {\tiny{$4500$~K}};
   \addplot+[black] table[x=l, y=t5] {data/planck.txt}node at (axis cs:750, 4.2e+13) {\tiny{$5000$~K}};
   \addplot+[black] table[x=l, y=t58] {data/planck.txt}node at (axis cs:670, 8.5e+13) {\tiny{$5800$~K}};
   \addplot+[black] table[x=l, y=t7] {data/planck.txt}node at (axis cs:1350, 3.5e+13) {\tiny{$7000$~K}};
	%\addplot+[black, smooth] table[x=l, y=t15] {data/planck.txt} node at (axis cs:1670, 5.5e+13) {\tiny{$15000$~K}};
  \end{axis}
 \end{tikzpicture}
\caption{Кривые спектральной плотности мощности изотропного излучения АЧТ с разной температурой}\label{pic:wien-law}
\end{wrapfigure}
Стоит заметить, что при переходе в функции к длинам волн меняется не только частота на длину волны, но и выражение для интервала. 

Формула Планка появилась, когда стало ясно, что формула Рэлея-Джинса удовлетворительно описывает излучение только в области больших длин волн, а~с~убыванием длин волн даёт сильные расхождения с реальными данными. Однако формулу Рэлея-Джинса используют и сейчас для описания кривой Планка на больших длинах волн. 

\change{
Проделаем обратные действия: получим формулу Рэлея-Джинса из формулы Планка. Длинноволновая часть спектра характеризуется соотношением $h\nu \ll kT$, то есть 
\begin{equation*}
	\exp\left( \frac{h\nu}{kT}\right) \approx 1 + \frac{h\nu}{kT}.
\end{equation*}
Подставляя полученное выражение в знаменатель \eqref{eq:plancks-law-nu}, получим
\begin{equation*}
	B_\nu(\nu,T) \approx \frac{2h\nu^3}{c^2}\cdot \frac{1}{1 + \frac{h\nu}{kT} - 1} = \frac{2h\nu^3 }{c^2}\cdot \frac{k T}{ h \nu} = \frac{2 \nu^2 k T}{c^2}.
\end{equation*}
}
\change{
	Проделав то же самое для выражения через длину волны, получим:
}
\begin{equation}
	B(\lambda, T) \simeq \frac{2 c k T}{\lambda^4}, \quad\quad B(\nu, T) \simeq \frac{2 \nu^2 k T}{c^2}.
\label{Ray-Jean}
\end{equation}

\change{
	В коротковолновой области, наоборот, $h \nu \gg kT$, следовательно, в знаменателе формулы Планка единица много меньше стоящей там экспоненты, то есть
	\begin{equation*}
		\frac{1}{\exp\left(\frac{h\nu}{kT}\right)-1} \approx \frac{1}{\exp\left(\frac{h\nu}{kT}\right)} = \exp\left(-\frac{h\nu}{kT}\right).
	\end{equation*} 
	Отсюда получаются выражения, называемые приближением Вина:
}
\begin{equation}
B ( \lambda, T) \simeq \frac{2 h c^2}{\lambda^5} \exp \left( -\frac{h c}{\lambda k T}\right), \quad \quad B( \nu, T ) \simeq \frac{2 h \nu^3}{c^2} \exp \left( -\frac{h \nu}{k T} \right).
\end{equation}
\subsection{Закон смещения Вина}
\term{Закон смещения Вина} --- закон, устанавливающий зависимость длины волны~$\lambda_\text{макс}$, на которой спектральная плотность излучения $B_\lambda(\lambda, T)$ абсолютно чёрного тела достигает своего максимума, от температуры $T$ этого тела:
\begin{equation}
	\lambda_\text{макс} \approx \frac{b}{T} \equiv \frac{0.0029~\text{м} \cdot \text{К}}{T}.
\end{equation}
Закон является следствием исследования функции Планка (см.~\ref{sec:planck-law}) на экстремальность.
\subsection{Эффект Доплера. Красное смещение}
\term{Эффект Доплера}~--- эффект изменения частоты и длины волны электромагнитного излучения, регистрируемого приёмником, вызванный относительным движением источника и приёмника (см.~Рис.\,\ref{doppler-ef}).

При $\Delta \lambda \ll \lambda_0$ с большой точностью выполняется следующее важное соотношение:\begin{equation}
\beta \equiv \dfrac{v}{c} = \dfrac{\lambda - \lambda_0}{\lambda_0} \equiv \dfrac{\Delta \lambda}{\lambda_0},
\label{eq:dopler-ef-simple}
\end{equation}
\begin{wrapfigure}[6]{r}{0.5\tw}
\centering
\vspace{-.5pc}
\includegraphics[width=.5\tw]{doppler-ef}
\caption{Эффект Доплера}
\label{doppler-ef}
\end{wrapfigure}
где $\lambda_0$~--- лабораторная длина волны излучения источника, а $\lambda$~--- наблюдаемая. В действительности же имеет место более общий случай: \imp{релятивистский эффект Доплера}, обусловленный проявлением СТО при $v \simeq c$, для которого формула~\eqref{eq:dopler-ef-simple} усложняется и принимает вид
\begin{equation}
\nu = \nu_0 \cdot \dfrac{\sqrt{1 - \beta^2}}{1 + \beta \cdot \cos\theta},
\label{eq:dopler-ef-rel}
\end{equation}
где $\nu$~--- частота, с которой наблюдатель принимает волны, $\nu_0$~--- частота, с которой источник испускает волны, $v$~--- скорость источника, $\theta$~--- угол между направлением на источник и вектором его скорости в системе отсчёта приёмника. Если источник радиально удаляется от наблюдателя, то $\theta = 0$, если приближается, то $\theta =\pi$. Важно, что~\eqref{eq:dopler-ef-simple} напрямую следует из \eqref{eq:dopler-ef-rel} при $\beta  \ll 1$.

\term{Красное смещение}~--- явление сдвига спектральных линий химических элементов в красную (длинноволновую) сторону, обусловленное относительным движение объектов. Параметр красного смещения определяется из наблюдаемой и лабораторной длин волн как
\begin{equation}
z = \dfrac{\lambda - \lambda_0}{\lambda_0}.
\end{equation}

Доплеровское смещение длины волны в спектре источника, движущегося с лучевой скоростью $v_{r}$ и полной скоростью $v$,
\begin{equation}
z = \dfrac{1 + v_r / c}{\sqrt{1 - \beta^2}}.
\end{equation}

\term{Гравитационное красное смещение}~--- проявление эффекта изменения частоты излучения, испущенного массивным объектом, таким как звезда или чёрная дыра. Наблюдается как сдвиг спектральных линий в спектре источника в красную область спектра. Гравитационное красное смещение определяется из формулы, выведенной Эйнштейном,
\begin{equation}
z_G=\dfrac{GM}{c^2 R}-\dfrac{GM}{c^2 r},
\label{eq:grav-red-shift}
\end{equation}
где $M$~--- масса гравитирующего тела, $R$~--- радиальное расстояние от центра масс тела до точки излучения (радиус источника), $r$~---  радиальное расстояние от центра масс источника до точки наблюдения. В случае, когда наблюдатель находится от источника много дальше его радиуса, т.\,е. выполняется соотношение $r \gg R$, выражение~\eqref{eq:grav-red-shift} можно упростить до
\begin{equation}
z_G \simeq \dfrac{GM}{c^2 R}.
\end{equation}

\input{sections/astrophys.light-pressure.tex}
\input{sections/astrophys.edd.tex}
\input{sections/astrophys.grav-lens.tex}
\subsection{Закон Хаббла}
\term{Закон Хаббла}~--- эмпирический закон, связывающий скорость удаления галактик $V$ и расстояние $R$ до них линейным образом: 
\begin{equation}
	V = H R,
\end{equation}
величина $H=68~\text{км/c} \cdot \text{Мпк})$ называется \imp{постоянной Хаббла}.

При $v \ll c$ можно использовать приближение эффекта Доплера, тогда
\begin{equation}
	V = c z.
\label{eq:hubble-speed}
\end{equation}

Равенство \eqref{eq:hubble-speed} справедливо только при $z \ll 1$, а при б\'{o}льших значениях $z$ космологическое красное смещение нльзя связывать с эффектом Доплера, поэтому можно пользоваться только формулой 
\begin{equation}
	\frac{dz}{dt} = - H(z)(1+z),
\end{equation}
где постоянная Хаббла введена как функция красного смещения.
\subsection{Шкала электромагнитных волн}


\term{Гамма излучение} возникает при радиоактивных распадах ядер, при торможении электронов энергией более $10^5$~эВ и при других взаимодействиях элементарных частиц. Используются в гамма-дефектоскопии, при изучении свойств вещества.

\term{Рентгеновские лучи} излучаются при большом ускорении электронов, например при их торможении в металлах. Получают их при помощи рентгеновской трубки: электроны в вакуумной трубке ускоряются электрическим полем при высоком напряжении, достигая анода, при со­ударении резко тормозятся. При торможении электроны движут­ся с ускорением и излучают электромагнитные волны с малой длиной. 

\begin{figure}[!h]
\centering
\includegraphics[width = 1\textwidth]{scale-wave.pdf}
\caption{Шкала электромагнитных волн}
\end{figure}
\term{Ультрафиолетовые лучи}~--- излучение Солнца, ртутных ламп и т.\,п. Используются в ультрафиолетовой микроскопии, в медицине.

\term{Видимое излучение}~--- часть электромагнитного излучения, воспринимаемая глазом (от фиолетового до от красного).

\term{Инфракрасное излучение}~--- тепловое, излучается любым нагретым телом.

\term{Радиоволны} используются повсеместно в обычной жизни, это и сотовая связь, и радиолокация, и спутниковая связь, и Wi-Fi и многое другое.

\term{Низкочастотные волны}~--- диапазон, традиционно используемый в электротехнике. В промышленной электроэнергетике используется частота 50~Гц, на~которой осуществляется передача электрической энергии по линиям и преобразование напряжений трансформаторными устройствами.
\input{sections/astrophys.spec-theor-rel.tex}
\subsection{Оптическая толщина. Закон Бугера}
\term{Оптическая толщина}~--- безразмерная величина, характеризующая степень непрозрачности среды для проходящего сквозь неё излучения,
\begin{equation}
\tau = \int n(x) \sigma(x)\,dx,
\end{equation}
где $\tau$~--- оптическая толщина среды, $n$~--- концентрация частиц, $\sigma$~--- сечение их взаимодействия.

Поток $I_0$ на входе связан с потоком $I$ на выходе \term{Законом Бугера}:
\begin{equation}
I = I_0 e^{-\tau}.
\end{equation}
\input{sections/astrophys.colour.tex}
\input{sections/astrophys.mkt.tex}
\input{sections/astrophys.earth-atmosphere.tex}

\subsection{Галактики}
\term{Морфологическая классификация галактик}~--- система разделения галактик на группы по визуальным признакам, используемая в астрономии. Наиболее известной является классификация, разработанная Хабблом и дополненная другими учеными. 
	\begin{figure}[h!]
		\centering
		\vspace{-.9pc}
		\includegraphics[width=0.65\tw]{hubble-fork.pdf}
		\caption{<<Вилка Хаббла>>}
	\end{figure}
	
Согласно данной классфикации галактики делятся на 4 типа:
\begin{enumerate}[itemsep=3pt, label={\arabic*.}, leftmargin=1pc]
	\item{\term{Эллиптические галактики} имеют гладкую эллиптическую форму без отличительных деталей с равномерным уменьшением яркости от центра к периферии. Обозначаются буквой E с индексом. Индекс можно рассчитать по формуле
		\begin{equation}
		i = 10 \times \left(1 - \frac{b}{a}\right),
		\end{equation}
		где $a$ и $b$~--- большая и малая полуоси видимого эллипса.}
	\item{\term{Спиральные галактики} состоят из уплощенного диска из звезд и газа, в центре которого находится сферическое уплотнение, называемое балджем, а также обширного сферического гало. Спиральные галактики обозначаются SB при наличии бара (перемычки между рукавами) или S при отсутствии бара. В зависимости от размеров ядра и балджа галактики делят на 3 группы: a, b и c. Для галактик Sa характерен большой балдж, для галактик Sc~--- маленький. Галактики Sb представляют собой нечто среднее между галактиками Sa и Sb.
	
	Светимость спиральных галактик $L$ связана с их максимальной скоростью вращения $v_\text{макс}$ \term{соотношением Талли-Фишера}:
	\begin{equation}
		L \propto v_\text{макс}^4.
	\end{equation}
	Абсолютная звёздная величина Млечного пути $M_\text{MW} \simeq -21^m$.}
	\item{\term{Неправильные или иррегулярные галактики}~--- галактика, лишенная как вращательной симметрии, так и значительного ядра. Обозначение: Irr.}
	\item{\term{Линзовидные галактики}~--- галактики, являющиеся переходными между спиральными и эллиптическими. Обозначения: S0, SB0.
	
	К линзовидным галактикам с абсолютной звёздной величиной около $-21^m$ применимо соотношение Фабер-Джексона:
	\begin{equation}
		L \propto v^4,
	\end{equation}
	где $v$~--- скорость вращения вещества.}
\end{enumerate}


\subsection{Другие объекты}
\begin{minipage}{0.6\tw}
\term{Шаровые звёздные скопления}~--- это скопление звезд, состоящее из нескольких сотен тысяч светил, тесно связанных гравитацией и вращающееся в качестве спутника вокруг центра галактики. Млечный путь насчитывает около 160 шаровых звездных скоплений. Диаметры шаровых скоплений составляют 20\,--\,60~пк, массы~--- 104\,--\,106~солнечных.

\term{Планетарная туманность}~--- система из звезды, называемой ядром туманности, и симметрично окружающей ее светящейся газовой оболочки. Планетарные туманности образуются при сбросе внешних слоёв (оболочек) красных гигантов и сверхгигантов с массой от $0.8M_\odot$ до $8M_\odot$ на завершающей стадии их эволюции.
\end{minipage}
\hfill
\begin{minipage}{0.35\tw}
	\centering
	\vspace{-0.5pc}
	\includegraphics[width = .52\tw]{m13}
	\captionof{figure}{Шаровое скопление M13 (негатив)}
	\vspace{1pc}
	\includegraphics[width = .52\tw]{m57}
	\captionof{figure}{Пла\-не\-тар\-ная туманность M57 (негатив)}
\end{minipage}

\newpage
\section{Математика}
\subsection{Производная}
\term{Производная в точке}~--- предел отношения приращения функции к приращению её аргумента при стремлении приращения аргумента к нулю, если такой предел существует. \imp{Геометрический смысл производной}: значение производной в точке численно равно тангенсу угла наклона касательной к графику функции в данной точке. Следовательно, точки, где производная обнуляется, соответствуют локальным минимумам и максимумам функции.
\begin{equation}
f^\prime(x_0) = \lim_{\Delta x \to 0}\frac{f(x_0 + \Delta x) - f(x_0)}{\Delta x}
\end{equation}
Общепринятые обозначения для производной функции $y = f(x)$ в точке $x_0$:
\begin{equation}
f^\prime(x_0) = f^\prime_x(x_0) = D f(x_0) = \frac{d f}{d x}(x_0) = \dot{f} (x_0).
\end{equation}
Правила дифференцирования:\\[-0.5pc]
\begin{minipage}{0.5\textwidth}
\begin{align*}
(f+g)^\prime &= f^\prime + g^\prime;\\
(Cf)^\prime &= Cf^\prime;\\
(fg)^\prime &= f^\prime g + f g^\prime;
\end{align*}
\end{minipage}
\begin{minipage}{0.5\textwidth}
\begin{align*}
\left(\dfrac{f}{g}\right)^\prime &= \dfrac{f^\prime g - f g^\prime}{g^2};\\
\dfrac{d}{dx}f\bigl(g(x)\bigr) &= \dfrac{df(g)}{dg}\dfrac{dg(x)}{dx}.
\end{align*}
\end{minipage}\\[0.5pc]
Таблица производных:
\begin{align*} 
(x^a)^\prime &= a x^{a-1};
& (\cos x)^\prime &= - \sin x;
& (\arccos x)^\prime &= - \dfrac{1}{\sqrt{1 - x^2}};\\
(a^x)^\prime &= a^x \ln a;
& (\log_a x)^\prime &= \dfrac{1}{x \ln a}; 
& (\arctg x)^\prime &= \dfrac{1}{1 + x^2};\\
(\sin x)^\prime &= \cos x; 
& (\arcsin x)^\prime &= 
\dfrac{1}{\sqrt{1 - x^2}};
&  (\arcctg x)^\prime &= - \dfrac{1}{1 + x^2}.
\end{align*}
\subsection{Интеграл}
\term{Неопределенным интегралом} функции $f(x)$ называется такая функция $F(x)$, производная которой равна $f(x)$.
\begin{equation}
F(x) = \int f(x)dx,\quad F^\prime(x)=f(x).
\end{equation}
\term{Определенный интеграл} характеризуется верхним и нижним пределом интегрирования. Значение определенного интеграла численно равно площади под графиком функции на данном интервале.
\begin{equation}
\int\limits^b_a f(x)\,dx = F(x) \biggr|^b_a = F(b) - F(a)
\end{equation}
Правила интегрирования:
\begin{align*}	
\int c f(x) \,dx &= c \int f(x) \,dx;  &&&&&\int f(ax + b) \,dx &= \dfrac{1}{a}F(ax + b) + C;\\
\int f \,dg &= fg - \int g \,df; &&&&& \int \bigl[f(x) + g(x)\bigr] \,dx &= \int f(x) \,dx + \int g(x) \,dx;
\end{align*}

Таблица интегралов:
\begin{align*}
\int  x^a \,dx &= \dfrac{x^{a+1}}{a+1} + C,\quad a \neq -1;\quad & 
\int \dfrac{dx}{\sqrt{a^2 - x^2}} &= \arcsin\dfrac{x}{a} + C;\\
\int \frac{dx}{x} &= \ln x + C; &
\int \dfrac{dx}{-\sqrt{a^2 - x^2}} &= \arccos\dfrac{x}{a} + C;\\
\int a^x \,dx &= \dfrac{a^x}{\ln a} + C; & 
\int \dfrac{dx}{x^2 + a^2} &= \dfrac{1}{a} \arctg \dfrac{x}{a} + C; \\
\int \cos x \,dx &= \sin x + C; & 
\int \dfrac{dx}{x^2 - a^2} &= \dfrac{1}{2a} \ln \dfrac{|x - a|}{|x + a|} + C;\\
\int \sin x \,dx &= -\cos x + C; &
\int \dfrac{dx}{\sqrt{x^2 + a}} &= \ln \left| x + \sqrt{x^2 + a} \right| + C.
\end{align*}
\newpage
\section{Математика}
\subsection{Производная}
\term{Производная в точке}~--- предел отношения приращения функции к приращению её аргумента при стремлении приращения аргумента к нулю, если такой предел существует. \imp{Геометрический смысл производной}: значение производной в точке численно равно тангенсу угла наклона касательной к графику функции в данной точке. Следовательно, точки, где производная обнуляется, соответствуют локальным минимумам и максимумам функции.
\begin{equation}
f^\prime(x_0) = \lim_{\Delta x \to 0}\frac{f(x_0 + \Delta x) - f(x_0)}{\Delta x}
\end{equation}
Общепринятые обозначения для производной функции $y = f(x)$ в точке $x_0$:
\begin{equation}
f^\prime(x_0) = f^\prime_x(x_0) = D f(x_0) = \frac{d f}{d x}(x_0) = \dot{f} (x_0).
\end{equation}
Правила дифференцирования:\\[-0.5pc]
\begin{minipage}{0.5\textwidth}
\begin{align*}
(f+g)^\prime &= f^\prime + g^\prime;\\
(Cf)^\prime &= Cf^\prime;\\
(fg)^\prime &= f^\prime g + f g^\prime;
\end{align*}
\end{minipage}
\begin{minipage}{0.5\textwidth}
\begin{align*}
\left(\dfrac{f}{g}\right)^\prime &= \dfrac{f^\prime g - f g^\prime}{g^2};\\
\dfrac{d}{dx}f\bigl(g(x)\bigr) &= \dfrac{df(g)}{dg}\dfrac{dg(x)}{dx}.
\end{align*}
\end{minipage}\\[0.5pc]
Таблица производных:
\begin{align*} 
(x^a)^\prime &= a x^{a-1};
& (\cos x)^\prime &= - \sin x;
& (\arccos x)^\prime &= - \dfrac{1}{\sqrt{1 - x^2}};\\
(a^x)^\prime &= a^x \ln a;
& (\log_a x)^\prime &= \dfrac{1}{x \ln a}; 
& (\arctg x)^\prime &= \dfrac{1}{1 + x^2};\\
(\sin x)^\prime &= \cos x; 
& (\arcsin x)^\prime &= 
\dfrac{1}{\sqrt{1 - x^2}};
&  (\arcctg x)^\prime &= - \dfrac{1}{1 + x^2}.
\end{align*}
\subsection{Интеграл}
\term{Неопределенным интегралом} функции $f(x)$ называется такая функция $F(x)$, производная которой равна $f(x)$.
\begin{equation}
F(x) = \int f(x)dx,\quad F^\prime(x)=f(x).
\end{equation}
\term{Определенный интеграл} характеризуется верхним и нижним пределом интегрирования. Значение определенного интеграла численно равно площади под графиком функции на данном интервале.
\begin{equation}
\int\limits^b_a f(x)\,dx = F(x) \biggr|^b_a = F(b) - F(a)
\end{equation}
Правила интегрирования:
\begin{align*}	
\int c f(x) \,dx &= c \int f(x) \,dx;  &&&&&\int f(ax + b) \,dx &= \dfrac{1}{a}F(ax + b) + C;\\
\int f \,dg &= fg - \int g \,df; &&&&& \int \bigl[f(x) + g(x)\bigr] \,dx &= \int f(x) \,dx + \int g(x) \,dx;
\end{align*}

Таблица интегралов:
\begin{align*}
\int  x^a \,dx &= \dfrac{x^{a+1}}{a+1} + C,\quad a \neq -1;\quad & 
\int \dfrac{dx}{\sqrt{a^2 - x^2}} &= \arcsin\dfrac{x}{a} + C;\\
\int \frac{dx}{x} &= \ln x + C; &
\int \dfrac{dx}{-\sqrt{a^2 - x^2}} &= \arccos\dfrac{x}{a} + C;\\
\int a^x \,dx &= \dfrac{a^x}{\ln a} + C; & 
\int \dfrac{dx}{x^2 + a^2} &= \dfrac{1}{a} \arctg \dfrac{x}{a} + C; \\
\int \cos x \,dx &= \sin x + C; & 
\int \dfrac{dx}{x^2 - a^2} &= \dfrac{1}{2a} \ln \dfrac{|x - a|}{|x + a|} + C;\\
\int \sin x \,dx &= -\cos x + C; &
\int \dfrac{dx}{\sqrt{x^2 + a}} &= \ln \left| x + \sqrt{x^2 + a} \right| + C.
\end{align*}
\input{sections/math.angle.tex}
\subsection{Формулы приближенного вычисления}
Формулы приблженного вычисления при $x \ll 1$:
\begin{align*}
\sin x &\approx x - \frac{x^3}{6} \approx x & \cos x &\approx 1 - \frac{x^2}{2} \\
\tg x &\approx x & \ln(1+x) &\approx x \\
(1 + x)^\alpha &\approx 1 + \alpha x & e^x &\approx 1 + x \\
\sin (\theta + x) &\approx \sin \theta + x \cos \theta & \cos(\theta + x) &\approx \cos \theta - x \sin \theta
\end{align*}
Также существует равенство для нескольких малых переменных:
\begin{equation*}
(1 + a)^\alpha (1 + b)^\beta \ldots \approx 1 + \alpha a + \beta b + \ldots
\end{equation*}
\input{sections/math.ols.tex}
\subsection{Гармонические колебания}
\term{Гармонические колебания}~--- колебания, при которых физическая величина изменяется с течением времени по гармоническому закону. Уравнение гармонических колебаний представляет собой дифференциальное уравнение второго порядка, вида
\begin{equation}
\ddot{x} + \omega^2 x = 0.
\end{equation}
Его общее решение имеет вид
\begin{equation}
	x(t) = C_1 \sin \omega t + C_2 \cos \omega t = C_0 \sin (\omega t + \varphi_0),
\end{equation}
где~$C_0$, $C_1$~и~$C_2$~--- некоторые константы, определяющие амплитуду колебаний,~а~$\varphi_0$~--- начальная фаза колебаний. Отсюда видно, что для периода колебаний справедливо соотношение
\begin{equation}
T = \frac{2 \pi}{\omega}.
\end{equation}
Гармонические колебания совершаются под действием упругих или квазиупругих сил, значение которых пропорционально отклонению со знаком минус. Примерами гармонических колебаний могут служить математический и пружинный маятники.
\input{sections/math.error.tex}
\subsection{Формулы приближенного вычисления}
Формулы приблженного вычисления при $x \ll 1$:
\begin{align*}
\sin x &\approx x - \frac{x^3}{6} \approx x & \cos x &\approx 1 - \frac{x^2}{2} \\
\tg x &\approx x & \ln(1+x) &\approx x \\
(1 + x)^\alpha &\approx 1 + \alpha x & e^x &\approx 1 + x \\
\sin (\theta + x) &\approx \sin \theta + x \cos \theta & \cos(\theta + x) &\approx \cos \theta - x \sin \theta
\end{align*}
Также существует равенство для нескольких малых переменных:
\begin{equation*}
(1 + a)^\alpha (1 + b)^\beta \ldots \approx 1 + \alpha a + \beta b + \ldots
\end{equation*}
\newpage
\section{Математика}
\subsection{Производная}
\term{Производная в точке}~--- предел отношения приращения функции к приращению её аргумента при стремлении приращения аргумента к нулю, если такой предел существует. \imp{Геометрический смысл производной}: значение производной в точке численно равно тангенсу угла наклона касательной к графику функции в данной точке. Следовательно, точки, где производная обнуляется, соответствуют локальным минимумам и максимумам функции.
\begin{equation}
f^\prime(x_0) = \lim_{\Delta x \to 0}\frac{f(x_0 + \Delta x) - f(x_0)}{\Delta x}
\end{equation}
Общепринятые обозначения для производной функции $y = f(x)$ в точке $x_0$:
\begin{equation}
f^\prime(x_0) = f^\prime_x(x_0) = D f(x_0) = \frac{d f}{d x}(x_0) = \dot{f} (x_0).
\end{equation}
Правила дифференцирования:\\[-0.5pc]
\begin{minipage}{0.5\textwidth}
\begin{align*}
(f+g)^\prime &= f^\prime + g^\prime;\\
(Cf)^\prime &= Cf^\prime;\\
(fg)^\prime &= f^\prime g + f g^\prime;
\end{align*}
\end{minipage}
\begin{minipage}{0.5\textwidth}
\begin{align*}
\left(\dfrac{f}{g}\right)^\prime &= \dfrac{f^\prime g - f g^\prime}{g^2};\\
\dfrac{d}{dx}f\bigl(g(x)\bigr) &= \dfrac{df(g)}{dg}\dfrac{dg(x)}{dx}.
\end{align*}
\end{minipage}\\[0.5pc]
Таблица производных:
\begin{align*} 
(x^a)^\prime &= a x^{a-1};
& (\cos x)^\prime &= - \sin x;
& (\arccos x)^\prime &= - \dfrac{1}{\sqrt{1 - x^2}};\\
(a^x)^\prime &= a^x \ln a;
& (\log_a x)^\prime &= \dfrac{1}{x \ln a}; 
& (\arctg x)^\prime &= \dfrac{1}{1 + x^2};\\
(\sin x)^\prime &= \cos x; 
& (\arcsin x)^\prime &= 
\dfrac{1}{\sqrt{1 - x^2}};
&  (\arcctg x)^\prime &= - \dfrac{1}{1 + x^2}.
\end{align*}
\subsection{Интеграл}
\term{Неопределенным интегралом} функции $f(x)$ называется такая функция $F(x)$, производная которой равна $f(x)$.
\begin{equation}
F(x) = \int f(x)dx,\quad F^\prime(x)=f(x).
\end{equation}
\term{Определенный интеграл} характеризуется верхним и нижним пределом интегрирования. Значение определенного интеграла численно равно площади под графиком функции на данном интервале.
\begin{equation}
\int\limits^b_a f(x)\,dx = F(x) \biggr|^b_a = F(b) - F(a)
\end{equation}
Правила интегрирования:
\begin{align*}	
\int c f(x) \,dx &= c \int f(x) \,dx;  &&&&&\int f(ax + b) \,dx &= \dfrac{1}{a}F(ax + b) + C;\\
\int f \,dg &= fg - \int g \,df; &&&&& \int \bigl[f(x) + g(x)\bigr] \,dx &= \int f(x) \,dx + \int g(x) \,dx;
\end{align*}

Таблица интегралов:
\begin{align*}
\int  x^a \,dx &= \dfrac{x^{a+1}}{a+1} + C,\quad a \neq -1;\quad & 
\int \dfrac{dx}{\sqrt{a^2 - x^2}} &= \arcsin\dfrac{x}{a} + C;\\
\int \frac{dx}{x} &= \ln x + C; &
\int \dfrac{dx}{-\sqrt{a^2 - x^2}} &= \arccos\dfrac{x}{a} + C;\\
\int a^x \,dx &= \dfrac{a^x}{\ln a} + C; & 
\int \dfrac{dx}{x^2 + a^2} &= \dfrac{1}{a} \arctg \dfrac{x}{a} + C; \\
\int \cos x \,dx &= \sin x + C; & 
\int \dfrac{dx}{x^2 - a^2} &= \dfrac{1}{2a} \ln \dfrac{|x - a|}{|x + a|} + C;\\
\int \sin x \,dx &= -\cos x + C; &
\int \dfrac{dx}{\sqrt{x^2 + a}} &= \ln \left| x + \sqrt{x^2 + a} \right| + C.
\end{align*}
\input{sections/math.angle.tex}
\subsection{Формулы приближенного вычисления}
Формулы приблженного вычисления при $x \ll 1$:
\begin{align*}
\sin x &\approx x - \frac{x^3}{6} \approx x & \cos x &\approx 1 - \frac{x^2}{2} \\
\tg x &\approx x & \ln(1+x) &\approx x \\
(1 + x)^\alpha &\approx 1 + \alpha x & e^x &\approx 1 + x \\
\sin (\theta + x) &\approx \sin \theta + x \cos \theta & \cos(\theta + x) &\approx \cos \theta - x \sin \theta
\end{align*}
Также существует равенство для нескольких малых переменных:
\begin{equation*}
(1 + a)^\alpha (1 + b)^\beta \ldots \approx 1 + \alpha a + \beta b + \ldots
\end{equation*}
\input{sections/math.ols.tex}
\subsection{Гармонические колебания}
\term{Гармонические колебания}~--- колебания, при которых физическая величина изменяется с течением времени по гармоническому закону. Уравнение гармонических колебаний представляет собой дифференциальное уравнение второго порядка, вида
\begin{equation}
\ddot{x} + \omega^2 x = 0.
\end{equation}
Его общее решение имеет вид
\begin{equation}
	x(t) = C_1 \sin \omega t + C_2 \cos \omega t = C_0 \sin (\omega t + \varphi_0),
\end{equation}
где~$C_0$, $C_1$~и~$C_2$~--- некоторые константы, определяющие амплитуду колебаний,~а~$\varphi_0$~--- начальная фаза колебаний. Отсюда видно, что для периода колебаний справедливо соотношение
\begin{equation}
T = \frac{2 \pi}{\omega}.
\end{equation}
Гармонические колебания совершаются под действием упругих или квазиупругих сил, значение которых пропорционально отклонению со знаком минус. Примерами гармонических колебаний могут служить математический и пружинный маятники.
\input{sections/math.error.tex}
\subsection{Гармонические колебания}
\term{Гармонические колебания}~--- колебания, при которых физическая величина изменяется с течением времени по гармоническому закону. Уравнение гармонических колебаний представляет собой дифференциальное уравнение второго порядка, вида
\begin{equation}
\ddot{x} + \omega^2 x = 0.
\end{equation}
Его общее решение имеет вид
\begin{equation}
	x(t) = C_1 \sin \omega t + C_2 \cos \omega t = C_0 \sin (\omega t + \varphi_0),
\end{equation}
где~$C_0$, $C_1$~и~$C_2$~--- некоторые константы, определяющие амплитуду колебаний,~а~$\varphi_0$~--- начальная фаза колебаний. Отсюда видно, что для периода колебаний справедливо соотношение
\begin{equation}
T = \frac{2 \pi}{\omega}.
\end{equation}
Гармонические колебания совершаются под действием упругих или квазиупругих сил, значение которых пропорционально отклонению со знаком минус. Примерами гармонических колебаний могут служить математический и пружинный маятники.
\newpage
\section{Математика}
\subsection{Производная}
\term{Производная в точке}~--- предел отношения приращения функции к приращению её аргумента при стремлении приращения аргумента к нулю, если такой предел существует. \imp{Геометрический смысл производной}: значение производной в точке численно равно тангенсу угла наклона касательной к графику функции в данной точке. Следовательно, точки, где производная обнуляется, соответствуют локальным минимумам и максимумам функции.
\begin{equation}
f^\prime(x_0) = \lim_{\Delta x \to 0}\frac{f(x_0 + \Delta x) - f(x_0)}{\Delta x}
\end{equation}
Общепринятые обозначения для производной функции $y = f(x)$ в точке $x_0$:
\begin{equation}
f^\prime(x_0) = f^\prime_x(x_0) = D f(x_0) = \frac{d f}{d x}(x_0) = \dot{f} (x_0).
\end{equation}
Правила дифференцирования:\\[-0.5pc]
\begin{minipage}{0.5\textwidth}
\begin{align*}
(f+g)^\prime &= f^\prime + g^\prime;\\
(Cf)^\prime &= Cf^\prime;\\
(fg)^\prime &= f^\prime g + f g^\prime;
\end{align*}
\end{minipage}
\begin{minipage}{0.5\textwidth}
\begin{align*}
\left(\dfrac{f}{g}\right)^\prime &= \dfrac{f^\prime g - f g^\prime}{g^2};\\
\dfrac{d}{dx}f\bigl(g(x)\bigr) &= \dfrac{df(g)}{dg}\dfrac{dg(x)}{dx}.
\end{align*}
\end{minipage}\\[0.5pc]
Таблица производных:
\begin{align*} 
(x^a)^\prime &= a x^{a-1};
& (\cos x)^\prime &= - \sin x;
& (\arccos x)^\prime &= - \dfrac{1}{\sqrt{1 - x^2}};\\
(a^x)^\prime &= a^x \ln a;
& (\log_a x)^\prime &= \dfrac{1}{x \ln a}; 
& (\arctg x)^\prime &= \dfrac{1}{1 + x^2};\\
(\sin x)^\prime &= \cos x; 
& (\arcsin x)^\prime &= 
\dfrac{1}{\sqrt{1 - x^2}};
&  (\arcctg x)^\prime &= - \dfrac{1}{1 + x^2}.
\end{align*}
\subsection{Интеграл}
\term{Неопределенным интегралом} функции $f(x)$ называется такая функция $F(x)$, производная которой равна $f(x)$.
\begin{equation}
F(x) = \int f(x)dx,\quad F^\prime(x)=f(x).
\end{equation}
\term{Определенный интеграл} характеризуется верхним и нижним пределом интегрирования. Значение определенного интеграла численно равно площади под графиком функции на данном интервале.
\begin{equation}
\int\limits^b_a f(x)\,dx = F(x) \biggr|^b_a = F(b) - F(a)
\end{equation}
Правила интегрирования:
\begin{align*}	
\int c f(x) \,dx &= c \int f(x) \,dx;  &&&&&\int f(ax + b) \,dx &= \dfrac{1}{a}F(ax + b) + C;\\
\int f \,dg &= fg - \int g \,df; &&&&& \int \bigl[f(x) + g(x)\bigr] \,dx &= \int f(x) \,dx + \int g(x) \,dx;
\end{align*}

Таблица интегралов:
\begin{align*}
\int  x^a \,dx &= \dfrac{x^{a+1}}{a+1} + C,\quad a \neq -1;\quad & 
\int \dfrac{dx}{\sqrt{a^2 - x^2}} &= \arcsin\dfrac{x}{a} + C;\\
\int \frac{dx}{x} &= \ln x + C; &
\int \dfrac{dx}{-\sqrt{a^2 - x^2}} &= \arccos\dfrac{x}{a} + C;\\
\int a^x \,dx &= \dfrac{a^x}{\ln a} + C; & 
\int \dfrac{dx}{x^2 + a^2} &= \dfrac{1}{a} \arctg \dfrac{x}{a} + C; \\
\int \cos x \,dx &= \sin x + C; & 
\int \dfrac{dx}{x^2 - a^2} &= \dfrac{1}{2a} \ln \dfrac{|x - a|}{|x + a|} + C;\\
\int \sin x \,dx &= -\cos x + C; &
\int \dfrac{dx}{\sqrt{x^2 + a}} &= \ln \left| x + \sqrt{x^2 + a} \right| + C.
\end{align*}
\input{sections/math.angle.tex}
\subsection{Формулы приближенного вычисления}
Формулы приблженного вычисления при $x \ll 1$:
\begin{align*}
\sin x &\approx x - \frac{x^3}{6} \approx x & \cos x &\approx 1 - \frac{x^2}{2} \\
\tg x &\approx x & \ln(1+x) &\approx x \\
(1 + x)^\alpha &\approx 1 + \alpha x & e^x &\approx 1 + x \\
\sin (\theta + x) &\approx \sin \theta + x \cos \theta & \cos(\theta + x) &\approx \cos \theta - x \sin \theta
\end{align*}
Также существует равенство для нескольких малых переменных:
\begin{equation*}
(1 + a)^\alpha (1 + b)^\beta \ldots \approx 1 + \alpha a + \beta b + \ldots
\end{equation*}
\input{sections/math.ols.tex}
\subsection{Гармонические колебания}
\term{Гармонические колебания}~--- колебания, при которых физическая величина изменяется с течением времени по гармоническому закону. Уравнение гармонических колебаний представляет собой дифференциальное уравнение второго порядка, вида
\begin{equation}
\ddot{x} + \omega^2 x = 0.
\end{equation}
Его общее решение имеет вид
\begin{equation}
	x(t) = C_1 \sin \omega t + C_2 \cos \omega t = C_0 \sin (\omega t + \varphi_0),
\end{equation}
где~$C_0$, $C_1$~и~$C_2$~--- некоторые константы, определяющие амплитуду колебаний,~а~$\varphi_0$~--- начальная фаза колебаний. Отсюда видно, что для периода колебаний справедливо соотношение
\begin{equation}
T = \frac{2 \pi}{\omega}.
\end{equation}
Гармонические колебания совершаются под действием упругих или квазиупругих сил, значение которых пропорционально отклонению со знаком минус. Примерами гармонических колебаний могут служить математический и пружинный маятники.
\input{sections/math.error.tex}
\newpage

\section*{Приложение}
\addcontentsline{toc}{section}{Приложение}
\renewcommand{\leftmark}[1]{Приложение}

\begin{table}[h!]
    \scriptsize
    \renewcommand{\arraystretch}{1.5}
    \renewcommand{\tabcolsep}{1pt}
    \centering
    \begin{tabularx}{\tw}{|C{.13}|C{.125}|C{.15}|C{.125}|C{.115}|C{.16}|C{.15}|}
        \hline
        \quad\quad\quad Объект &  Большая полуось  $\mathbf{a}$,~а.\,е. & Сиде\-ри\-чес\-кий пе\-риод $\mathbf{T}$,~год & Эксцен\-триситет $\mathbf{e}$ & Накло\-нение $\mathbf{i}$,~$~^\circ$ & Долгота восходящего узла $\mathbf{\Omega}$,~$~^\circ$ & Аргумент перицентра $\boldsymbol{\omega} $, $~^\circ$\\
        \hline
        Меркурий & 0.387 & 0.241 & 0.206 & 7.00  & 48.3  & 29.1\\

        Венера     & 0.723 & 0.615 & 0.007 & 3.39  & 76.7  & 54.9\\

        Земля    & 1.000 & 1.000 & 0.017 & 0.00    & 348.7 & 114.2\\

        Марс     & 1.524 & 1.881  & 0.093 & 1.85  & 49.6  & 286.5\\

        Церера   & 2.765 & 4.601  & 0.079 & 10.6 & 80.4  & 2.83\\

        Юпитер   & 5.204 & 11.86 & 0.048 & 1.31  & 100.6 & 275.1\\

        Сатурн   & 9.582 & 29.46 & 0.056 & 2.49  & 113.6 & 336.0\\

        Уран     & 19.23 & 84.01 & 0.044 & 0.77  & 74.0  & 96.5\\

        Нептун   & 30.06  & 164.8 & 0.011 & 1.77  & 131.8 & 265.6\\

        Плутон   & 39.48 & 247.9 & 0.249 & 17.1 & 110.2 & 113.8\\
        \hline
    \end{tabularx}
    \caption{Параметры орбит больших тел Солнечной системы}
\end{table}
\begin{table}[h!]
    \footnotesize
    \renewcommand{\arraystretch}{1.5}
    \renewcommand{\tabcolsep}{2pt}
    \centering
    \begin{tabularx}{\tw}{|C{0.14}|C{0.16}|C{0.17}|C{0.17}|C{0.17}|C{0.2}|}
        \hline
        Объект & Символ & Масса $\mathbf M$,~кг & Радиус $\mathbf R$,~м & Период $\mathbf T$,~ч & Наклон оси $\mathbf i$,~$~^\circ$\\
        \hline
        Солнце & $\odot$ & $1.99 \times 10^{30}$ & $6.97 \times 10^8$ & 609.1 &7.25\\

        Меркурий & $\mercury$ & $3.33 \times 10^{23}$ & $2.44 \times 10^6$ & 1408. & 0.035\\

        Венера   &  $\venus$  & $4.87 \times 10^{24}$ & $6.05 \times 10^6$ & 5833. & 177.4\\

        Земля    & $\oplus$   & $5.97 \times 10^{24}$ & $6.37 \times 10^6$ & 23.93 & 23.44 \\
        Луна    &    $\rightmoon$ & $7.35 \times 10^{22}$ & $1.74 \times 10^6$ &  655.7 & 1.54\\

        Марс    & $\mars$ & $6.42 \times 10^{23} $ & $3.39 \times 10^6 $  & 24.62 & 25.19 \\

        Церера  &  & $9.39 \times 10^{20}$ & $4.63 \times 10^{5}$  & 9.077 & 3 \\

        Юпитер   &$\jupiter$ & $1.90 \times 10^{27}$ & $7.00 \times 10^{7}$ & 9.925 & 3.13 \\

        Сатурн   &$\saturn$& $5.68 \times 10^{26}$ & $5.82 \times 10^{7}$ & 10.53 & 26.73  \\

        Уран    & $\uranus$& $8.68 \times 10^{25}$ & $2.54 \times 10^7$ & 17.24 & 97.77  \\

        Нептун   &$\neptune$& $1.02 \times 10^{26}$  & $2.46 \times 10^7$ & 15.97 & 28.32  \\

        Плутон   &$\pluto$& $1.30 \times 10^{22}$ & $1.19 \times 10^6 $ & 153.3 & 119.6 \\
        \hline
    \end{tabularx}
    \caption{Физические характеристики больших тел Солнечной системы}
\end{table}
\noindent Светимость Солнца $L_\odot$ \hfill $3.828 \times 10^{26}$~Вт\\
Видимая звёздная величина Солнца $m_\odot$ \hfill $-26.74^m$\\
Абсолютная звёздная величина Солнца $M_\odot$ \hfill $+4.83^m$\\
Показатель цвета Солнца $(B - V)_\odot$ \hfill $+0.67^m$\\
Эффективная температура Солнца $T_\odot$ \hfill $5778$~К\\
Большая полуось орбиты Луны $a_{\moon}$ \hfill $384399$~км\\
Эксцентриситет орбиты Луны $e_{\moon}$ \hfill $0.055$\\
Наклонение плоскости орбиты Луны к эклиптике $i_{\moon}$ \hfill $5.15^\circ$\\
Сидерический период Луны $T_{\moon}$ \hfill $27.3217$~сут\\
Геометрическое альбедо Луны $A_{\moon}$ \hfill $0.12$\\
Видимая звёздная величина Луны в полнолунии $m_{\odot}$ \hfill $-12.7^m$\\
Геометрическое Альбедо Земли $A_\oplus$ \hfill $0.37$\\[-5pt]
\rule{\tw}{.7pt}\\
Заряд электрона $e$ \hfill $-1.6 \times 10^{-19}$~Кл\\
Постоянная Планка $h$ \hfill $6.626 \times 10^{-34}~\text{Дж}\cdot\text{с}$\\
Постоянная Стефана-Больцмана $\sigma$ \hfill $5.670 \times 10^{-8}~\text{Вт} \cdot \text{м}^{-2} \cdot \text{К}^{-4}$
Гравитационная постоянная $G$ \hfill $6.672 \times 10^{-11}~\text{м}^3 \cdot \text{с}^{-2} \cdot \text{кг}^{-1}$\\
Постоянная Больцмана $k$ \hfill $1.381 \times 10^{-23}~\text{Дж} \cdot \text{К}^{-1}$\\
Постоянная Хаббла $H$ \hfill $67~\text{км} \cdot \text{с}^{-1} \cdot \text{Мпк}^{-1}$


\begin{figure}[h!]
    \centering
    \vspace{-.5pc}
    \includegraphics[width=\tw]{sky-map.pdf}
    \caption{Карта звёздного неба до $-45^\circ$ по склонению}
    \vspace{-1cm}
\end{figure}

\newpage
\thispagestyle{empty}
\begin{center}
	{ \bfseries \itshape Для заметок}
\end{center}
%\newpage
%\thispagestyle{empty}
~
\vfill
\centering
\begin{minipage}{0.8\tw}
	\small
	\centering
	Подписано в печать \signed.\\
	Формат 84$\times$108/32. Бумага офсетная. Печать офсетная.\\
	Тираж 500 экз.\\ 
	Отпечатано в ГУП МО <<Коломенская типография>>.\\ 140400, г.~Коломна, ул.~III Интернационала, д.~2a
\end{minipage}
\label{pg:last-page}

\end{document}