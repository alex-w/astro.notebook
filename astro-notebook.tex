\documentclass[10pt, a5paper, twoside]{article}

%%% Дата
\newcommand{\publish}{3}
\newcommand{\signed}{22.12.2020}
\renewcommand{\year}{2023}
\newcommand{\isbn}{ISBN 978-5-9909877-2-2}

%%% Русский шрифт
\usepackage[utf8]{inputenc}
\usepackage[russian]{babel}
\usepackage[OT1]{fontenc}

\usepackage{pscyr}

%%% Математика

% Шрифты для математики
\usepackage{amsmath}
\usepackage{amsfonts}
\usepackage{amssymb}
\usepackage{cancel}
\usepackage{mathrsfs}
\usepackage{mathtools}
\usepackage{upgreek}
\usepackage{xfrac}

% Математические команды
\renewcommand{\vec}[1]{\mathbf{#1}}
\newcommand{\R}{\mathbb{R}}
\renewcommand{\L}{\mathcal{L}}
\newcommand{\scalar}[2]{\left(\vec{#1} \cdot \vec{#2}\right)}
\newcommand{\scalarNoVec}[2]{\left(#1 \cdot #2 \right)}
\newcommand{\cross}[2]{\left[\vec{#1} \times \vec{#2}\right]}
\newcommand{\triple}[3]{\left( \vec{#1}, \vec{#2}, \vec{#3} \right)}
\DeclareMathOperator{\T}{T}
\DeclareMathOperator{\pr}{pr}
\DeclareMathOperator{\argmax}{argmax}
\newcommand{\LL}{\mathcal{L}}

% Операторы
\DeclareMathOperator{\const}{const}
\DeclareMathOperator{\area}{Area}

% Астрономические символы
\usepackage{wasysym}
\usepackage{starfont} % Значок Цереры

% Настройки для математики
\setcounter{MaxMatrixCols}{20}


%%% Поля и разметка страницы
%\usepackage[inner=1.5cm, outer=0.9cm, top=1.5cm, bottom=1.2 cm, headsep=4mm]{geometry} % Старая
\usepackage[inner=2.0cm, outer=1.5cm, top=2cm, bottom=1.5 cm, headsep=4mm]{geometry} % Новая


%%% Набор текста

% Разрывы страниц в формулах
\allowdisplaybreaks

% Последняя строка абзаца
\widowpenalty10000
\clubpenalty10000

%%% Иллюстрации
\usepackage{graphicx}
\usepackage{wrapfig}
\usepackage[export]{adjustbox}
\graphicspath{{./img/}}

% Графики
\usepackage{pgfplots}
\pgfplotsset{compat = 1.3}
\usepgfplotslibrary{patchplots}
%\usepackage{patchplots}
\pgfplotsset{	width	=	14	cm,
					x label style={
        					font = {\small\sffamily},
        					yshift = 1mm
      				},
      				tick label style={
      					font = {\scriptsize},
      				},
      				y label style={
      					font = {\small\sffamily},
      					yshift = -1mm,
      					at={(ticklabel cs:0.5)},
%      					rotate=90,
      					anchor=near ticklabel
      				},
      				every tick/.style	=	{
						black, 
						line width 	= 	.5	pt
					},
					axis line style 	= 	{
						line width 	= 	.5	pt
					},
					grid style	=	{
						gray,
						dotted
					},
					minor x tick num = 1,
 					minor y tick num = 1,
 					no markers,
 					grid = major,
 					every axis/.append style	=	{
 						line width	=	.7	pt
 					}
				}
				
%Подписи
\usepackage		[margin		= 10	pt,
					font		= footnotesize, 
					labelfont	= bf, 
					labelsep	= endash, 
					labelfont	= bf,
					textfont	= sl,
					margin		= 0 	pt,  
					aboveskip 	= 4		pt, 
					belowskip 	= -6	pt]	{caption}
\usepackage		[margin		= 10	pt,
					font		= footnotesize, 
					labelfont	= bf, 
					labelsep	= endash, 
					labelfont	= bf,
					textfont	= sl,
					margin		= 0 	pt,  
					aboveskip 	= 4		pt, 
					belowskip 	= 6	pt]	{subcaption}

%%% Insert pdf pages
\usepackage[final]{pdfpages}


%%% Color highlight
\usepackage{xcolor}


%%% TiKZ
\usepackage{tikz}
\usepackage{tkz-euclide}

\usetikzlibrary{external}
\tikzexternalize[prefix=tikz/resource/,optimize command away=\includepdf]
\usetikzlibrary{calc, fadings, decorations.markings, shapes, snakes}

\tkzSetUpLine[line width=0.4pt]
\tkzSetUpPoint[fill=white, size=2]
\tkzSetUpLabel[font=\scriptsize]


\def\centerarc[#1](#2)(#3:#4:#5)% [draw options] (center) (initial angle:final angle:radius)
  { \draw [#1] ($(#2)+({#5*cos(#3)},{#5*sin(#3)})$) arc (#3:#4:#5); }
  

\def\point(#1){% (center)
    \draw (#1) [fill=white] circle (.03);%
}


\def\pointStar(#1){% (center)
    \draw (#1) node[star, star points=5, star point ratio=2.25, fill=black, scale=0.2] {};%
}


\def\earth(#1){% (center)
    \tkzDefShiftPoint[#1](0,-0.08cm){B}%
    \tkzDefShiftPoint[#1](0,0.08cm){U}%
    \tkzDefShiftPoint[#1](0.08cm,0){R}%
    \tkzDefShiftPoint[#1](-0.08cm,0){L}%
    \tkzDrawCircle[color=black, fill=white, line width=.5pt](#1, L)%
    \tkzDrawSegment[line width=.5pt](L,R)%
    \tkzDrawSegment[line width=.5pt](U,B)%
}


\def\sun(#1){% (center)
    \tkzDefShiftPoint[#1](0,0.08cm){O}%
    \tkzDefShiftPoint[#1](0,0.015cm){I}%
    \tkzDrawCircle[color=black, fill=white, line width=.5pt](#1, O)%
    \tkzDrawCircle[color=black, fill=black](#1, I)%
}


% todo: Убрать, когда перепишутся рисунки, от него зависимые
\usetikzlibrary{ipe}
\tikzstyle{ipe stylesheet} = [
  ipe import,
  even odd rule,
  line join=round,
  line cap=round,
  ipe pen normal/.style={line width=0.4},
  ipe pen heavier/.style={line width=0.8},
  ipe pen fat/.style={line width=1.2},
  ipe pen ultrafat/.style={line width=2},
  ipe pen normal,
  ipe mark normal/.style={ipe mark scale=3},
  ipe mark large/.style={ipe mark scale=5},
  ipe mark small/.style={ipe mark scale=2},
  ipe mark tiny/.style={ipe mark scale=1.1},
  ipe mark normal,
  /pgf/arrow keys/.cd,
  ipe arrow normal/.style={scale=7},
  ipe arrow large/.style={scale=10},
  ipe arrow small/.style={scale=5},
  ipe arrow tiny/.style={scale=3},
  ipe arrow normal,
  /tikz/.cd,
  ipe arrows, % update arrows
  <->/.tip = ipe normal,
  ipe dash normal/.style={dash pattern=},
  ipe dash dashed/.style={dash pattern=on 4pt off 4pt},
  ipe dash dotted/.style={dash pattern=on 5pt off 2pt on .5pt off 2pt},
  ipe dash dash dotted/.style={dash pattern=on 4bp off 2bp on 1bp off 2bp},
  ipe dash dash dot dotted/.style={dash pattern=on 4bp off 2bp on 1bp off 2bp on 1bp off 2bp},
  ipe dash normal,
  ipe node/.append style={font=\normalsize},
  ipe stretch normal/.style={ipe node stretch=1},
  ipe stretch normal,
  ipe opacity 10/.style={opacity=0.1},
  ipe opacity 30/.style={opacity=0.3},
  ipe opacity 50/.style={opacity=0.5},
  ipe opacity 75/.style={opacity=0.75},
  ipe opacity opaque/.style={opacity=1},
  ipe opacity opaque,
]
%\usetikzlibrary{decorations.text}

\tikzset{every picture/.style={line cap=round}}

\makeatletter

\tikzset{
% dot diameter/.store in=\dot@diameter,
% dot diameter=3pt,
    dot spacing/.store in=\dot@spacing,
    dot spacing = 4pt,
    dashes/.style={
%        line width=\dot@diameter,
        line cap=round,
        dash pattern=on 4pt off \dot@spacing
    }
}
\makeatother

%%% Таблицы
\usepackage{tabularx}
\newcolumntype{L}[1]{>{\hsize=#1\hsize\raggedright\arraybackslash}X}%
\newcolumntype{R}[1]{>{\hsize=#1\hsize\raggedleft\arraybackslash}X}%
\newcolumntype{C}[1]{>{\hsize=#1\hsize\centering\arraybackslash}X}


%%% Нумерованные списки
\usepackage{enumitem}
\setlist[enumerate]{itemsep=1pt, label={\arabic*.}, leftmargin=1pc}


%%% Новые команды
\newcommand{\term}[1]{\,\!{\sffamily\bfseries#1}}
\newcommand{\moon}{\!\!\rightmoon}
\newcommand{\au}{\text{а.\,е.}}
\newcommand{\tw}{\textwidth}
\newcommand{\imp}[1]{{\itshape#1}}

\newcommand{\change}[1]{\textcolor{red}{#1}}

\newcommand*{\SectorRadius}{1ex}
\newcommand*{\SectorHalfAngle}{40}
\newcommand*{\SectorLineWidth}{.5pt}


\newcommand{\picRef}[1]{Рис.\;\ref{#1}}
\newcommand{\lookPicRef}[1]{см.~\picRef{#1}}

\newcommand*{\sector}{%
    \begin{pgfpicture}
        \pgfpathmoveto{\pgforigin}%
        \pgfpathlineto{\pgfpointpolar{90-(\SectorHalfAngle)}{\SectorRadius}}%
        \pgfarc{90-(\SectorHalfAngle)}{90+\SectorHalfAngle}{\SectorRadius}%
        \pgfpathclose
        \pgfsetlinewidth{\SectorLineWidth}%
        \pgfusepath{stroke}%
    \end{pgfpicture}%
}


%%% Cчетчики
\renewcommand{\theequation}{\thesection.\arabic{equation}}
\numberwithin{equation}{section}


%%% Размеры
\setlength {\parskip} { 4pt }
%\setlength {\parindent} { 0pt }
\renewcommand {\baselinestretch}{ 1.02 }


%%% Гиперссылки
\usepackage{hyperref}
\hypersetup	{
    colorlinks=false,
    linktoc=all
}


%%% Колонтитулы
\usepackage{fancyhdr}
\pagestyle{fancy}
\fancyhf{}
\fancyhead[CO]{\slshape \small \leftmark}
\fancyhead[LE, RO]{\slshape \small \thepage}
\fancyhead[CE]{\slshape \hyperref[toc]{Астрадь}}


\usepackage{url}

%%% main
\begin{document}
%\includepdf[pages=-]{sys/empty-page.pdf}
    \newpage
\thispagestyle{empty}
\begin{center}
\includegraphics[width=0.95\tw]{sys/cover.pdf}\\[1pc]
{\scshape Жуковский \\ \year}	
\end{center}

%	\thispagestyle{empty}
\begin{center}
	{\sffamily \large А.\,С.~Шепелев, Д.\,А.~Долгов, С.\,Д.~Молчанов, С.\,Б.~Борисов}\\[-7pt]
	
	\rule{0.88\tw}{0.7pt} \\[12pc]

	
	
	\scalebox{2}{\Huge \bfseries \scshape Астрадь}\\[1pc]
	{\Large \scshape краткий сборник теории\\[.5pc] по астрономии}\\[2pc]
	\vfill
	{\slshape второе издание, исправленное}
	\vfill
	{\scshape Астрономический кружок\\ им.~Е.\,П.~Левитана}
	\vfill
	{\scshape Жуковский \\ 2020}
\end{center}
	\setcounter{page}{1}
\thispagestyle{empty}
\vspace*{1cm}
{\hspace{1pc} {\bfseries А.\,С.~Шепелев, Д.\,А.~Долгов, С.\,Д.~Молчанов, С.\,Б.~Борисов.} Астрадь~--- краткий сборник теории по астрономии. 2018~---~60~с.}
\vskip4cm
\begin{center}
	{\slshape 2-е издание\\
	переработанное}\\[5mm]
	\small  Тираж 20 экз.
\end{center}
\vskip6cm
\begin{center}
Астрономический кружок им.~Е.\,П.~Левитана\\[3mm]
г.\,Жуковский\\[1cm]
	2018
\end{center}
\newpage
	\setcounter{tocdepth}{2}
	{
		\small
		\renewcommand{\baselinestretch}{ 0.95 }
		\tableofcontents
		\label{toc}
		\renewcommand {\baselinestretch}{ 1.02 }
	}
	\newpage
	\section{Небесная механика}
\subsection{Расстояние и размеры}
\term{Астрономическая единица}~--- единица измерения расстояния в астрономии, равная большой полуоси орбиты Земли. \begin{equation}
	1~\au = 149\:597\:870\:700~\text{м} \simeq 1.5 \times 10^{11}~\text{м}.
\end{equation}

\term{Годичный параллакс}\footnote{Важно отметить, здесь x$\pi$~--- лишь обозначение, ничего общего с числом $\pi$ не имеющее.} ($\pi$) объекта~--- это угол, под которым видно 
орбиту Земли из окрестностей данного объекта. Применяется к объектам вне 
Солнечной системы. \begin{equation}
	\tg \pi = \frac{a_\oplus}{r},
	\label{eq:parallax-sin}	
\end{equation}
где $a_\oplus$~--- большая полуось орбиты Земли и $r$~--- расстояние до объекта 
имеют одинаковые единицы измерений. Учитывая малость угла $\pi$, можно считать $\tg\pi \simeq \pi$ в \eqref{eq:parallax-sin}, тогда
\begin{equation}
	\pi = \frac{a_\oplus}{r}.
	\label{eq:parallax}
\end{equation} 
\begin{figure}[h!]
	\centering
	\vspace{-1pc}
	\includegraphics[width = 0.7\tw]{parallax.pdf}
	\caption{Схема годичного параллакса}
\end{figure}

Расстояние $r$, с которого большая полуось орбиты Земли $a_\oplus$ видна под углом $\pi = 1''$ называется \term{1 парсеком}. Так как \begin{equation}
	1~\text{рад} = \frac{180^\circ}{\pi} \simeq  3 438' \simeq 206265'' 
\quad \Longrightarrow \quad \mathsf{1~\text{\sffamily пк} = 
206265~\text{\sffamily а.\,е.}},
\end{equation} 
следовательно, записывая большую полуось орбиты Земли в \au, а расстояние до звезды в парсеках, получаем параллакс в секундах. Таким образом,
\begin{equation}
	r_\text{пк} = \frac{1~\au}{\pi''}.
\end{equation}

\term{Угловой размер объекта}~--- это угол, под которым видно объект. Для сферически симметричных объектов с радиусом $R$, угловой размер (диаметр) при наблюдении с расстояния $r$ определяется как
\begin{equation}
\rho = 2 \arcsin \frac{R}{r}.
\end{equation}
В случае, когда $r\gg R$, можно считать, что $\sin \rho \simeq \rho$, тогда
\begin{equation}
	\rho \simeq \frac{2 R}{r}.
\end{equation}

\vspace{-1.5pc}
\begin{figure}[h!]
	\begin{minipage}[b]{0.5\tw}
		\begin{flushleft}
			\includegraphics[width = 0.93\tw]{angle-size}
			\captionof{figure}{Угловой размер}
		\end{flushleft}
	\end{minipage}
	\begin{minipage}[b]{0.5\tw}
		\centering
		\includegraphics[width = \tw]{parallax-horiz}
		\captionof{figure}{Горизонтальный параллакс}
	\end{minipage}
\end{figure}

\term{Горизонтальный параллакс}~$(p)$~--- это угловой радиус Земли при наблюдении с объекта:
\begin{equation}
\sin p=\frac{R_\oplus}{r}.
\end{equation}

\term{Правило Тициуса-Боде} --- эмпирическая формула, приблизительно описывающая 
радиусы орбит планет в Солнечной системе:
\begin{equation}r=\frac{3\cdot 2^n+4}{10}, \quad n=-\infty, 0, 1, 2...
\end{equation}


\subsection{Закон всемирного тяготения}
Согласно \imp{закону всемирного тяготения}, сила притяжения
между двумя точечными телами с массами $M$ и $m$,
находящимися на расстоянии $r$, равна
\begin{equation}
	F=\frac{GMm}{r^2}, \label{eq:grav-law}
\end{equation}\nopagebreak где $G\simeq 6.67\cdot 10^{-11}~\text{м}^3 /
\left( \text{кг} \cdot \text{с}^2 \right)$~---
\term{гравитационная постоянная}.

\term{Гравитационный потенциал} поля точечной (или сферически
симметричной) массы $M$ на расстоянии $r$ от нее равен
работе, которую необходимо затратить, чтобы принести
единичную массу с бесконечности в данную точку. Так как
гравитационные силы между двумя массами --- это силы
притяжения, то эта работа отрицательна. Данная
величина также является \term{потенциальной энергией} точечной
массы на расстоянии $r$ от массы $M$, а выражение для нее имеет
следующий вид:
\begin{equation}
	U=-\frac{GM}{r}.
\end{equation}

Напряженность гравитационного поля $dU/dr$ часто называют
\term{ускорением свободного падения} $g$, она вычисляется по формуле
\begin{equation}
	g = \frac{GM}{r^2}.
	\label{eq:g}
\end{equation}
Тогда (\ref{eq:grav-law}) можно записать как
\begin{equation}
	F = mg.
\end{equation}

\input{sections/cel-mech.energy-conserv.tex}
\input{sections/cel-mech.kepler-laws.tex}
\subsection{Движение по орбите}
\begin{figure}[t]
	\centering
	\begin{tikzpicture}
		\footnotesize
		
		%	\foreach \x in {0, .1,...,5} {
		%		\draw [line width=.1pt] (\x, -3) -- (\x, 3);
		%	};
		%
		%	\foreach \y in {-3, -2.9,...,3} {
		%		\draw [line width=.1pt] (0, \y) -- (5, \y);
		%	};
		
		\draw [thick] (0, 0) .. controls (2, 4) and (3, -1) .. (5, -1);
		\draw [-latex] (0, -2) -- (1.25, 1.5);
		\draw [-latex] (0, -2) -- (3.3, .1);
		\draw [-latex] (1.25, 1.5) -- (3.3, .1);
		
		\draw (.3, -1.8) arc(31:70:0.36);
		
		\draw (.6, -.3) node [anchor = east] {$\vec{r}(t)$};
		\draw (1.6, -.9) node [anchor = north west] {$\vec{r}(t + dt)$};
		\draw (2.3, 0.9) node [anchor = north east] {$d\vec{r}$};
		\draw (0.2, -1.5) node [anchor = south west] {$\boldsymbol{\omega} \,dt$};
		
		\draw[fill=white] (1.25, 1.5) circle (0.03);
		\draw[fill=white] (3.3, .1) circle (0.03);
		\draw[fill=white] (0, -2) circle (0.03);
		
	\end{tikzpicture}
	\caption{}
\end{figure}

Рассмотрим такую физическую величину, как \term{секториальная скорость}~--- это векторная величина, описывающая ориентированную площадь, заметаемую радиус вектором тела за единицу времени. Пусть в момент времени $t$ тело находилось в точке $\vec{r}(t)$, а через промежуток времени $dt$~--- в точке $\vec{r}(t + dt)$. Обозначим перемещение тела за этот промежуток времени как $d\vec{r}$. Его можно выразить через скорость тела в момент времени $t$, считая ее постоянной на промежутке от $t$ до $t + dt$: $d\vec{r} = \vec{v} \, dt$. Площадь, которую заметает радиус-вектор тело $\vec{r}(t)$ равна половине параллелограмма, построенного на векторах $\vec{r}(t)$ и $d\vec{r}$. Поэтому можно записать
\begin{equation*}
	\vec{s} = \frac{1}{2} [\vec{r} \times \vec{v} dt],
\end{equation*}
следовательно секториальная скорость равна
\begin{equation*}
	\boldsymbol{\sigma} = \frac{d \vec{s}}{dt} = \frac{1}{2} [\vec{r} \times \vec{v}] = \frac{\vec{l}}{2} = \frac{\vec{L}}{2m},
\end{equation*}
где $\vec{l}$~--- удельный момент импульса (на единицу массы). Полученное выражение доказывает \imp{второй закон Кеплера}.

С другой стороны, перемещение $d\vec{r}$ можно выразить через угловую скорость $\boldsymbol{\omega}$, как $d \vec{r} = [\vec{r} \times \boldsymbol{\omega}\,dt]$. Тогда
\begin{equation*}
	\boldsymbol{\sigma}
	= \frac{1}{2} \big[ \vec{r} \times [\vec{r} \times \boldsymbol{\omega} ]\big]
	= \vec{r} \underbrace{(\vec{r}, \boldsymbol{\omega})}_0 - \boldsymbol{\omega} ( \vec{r}, \vec{r} )
	= r^2 \boldsymbol{\omega}.
\end{equation*}
Получим \imp{третий закон Кеплера}, заметив, что модуль секториальной скорости можно записать, как
\begin{gather*}
	\sigma
	= \frac{S_\text{эл}}{T}
	= \frac{\pi a b}{T}
	= \frac{L}{2m},\\
	\frac{\pi a^2 \sqrt{1 - e^2}}{T}
	= \frac{m \sqrt{\dfrac{GM}{a} \cdot \dfrac{1 + e}{1 - e}} \cdot a(1-e)}{2m},\\
	\frac{4\pi^2 a^4 (1 - e^2)}{T^2}
	= a^2(1-e)^2 \cdot \frac{GM}{a} \cdot \frac{1 + e}{1-e}
\end{gather*}
\begin{equation}
	\frac{T^2}{a^3} = \frac{4\pi^2}{GM}.чч
\end{equation}

Получим еще одно важное соотношение~--- \term{интеграл энергии}~--- формулу для скорости тела на орбите с большой полуосью $a$ в точке, удалённой на расстояние~$r$ от центрального тела с массой $M$. Для этого рассмотрим  сначала точку перицентра ($q$, <<п>>) и апоцентра ($Q$, <<a>>) данной орбиты, запишем для них закон сохранения энергии и закон сохранения момента импульса:
\begin{gather*}
	-\frac{GMm}{q} + \frac{m v^2_\text{п}}{2} = -\frac{GMm}{Q} + \frac{m v^2_\text{а}}{2},\\
	mv_\text{п}q = mv_\text{a}Q.
\end{gather*}
Из ЗСМИ и выражений для перицентрического~$q$ и апоцентрического~$Q$ расстояний через большую полуось $a$ и эксцентриситет $e$ имеем:
\begin{equation*}
	\frac{v_\text{а}}{v_\text{п}} = \frac{1 - e}{1 + e}.
\end{equation*}
Использую это соотношения, преобразуем ЗСЭ:
\begin{gather}
	\frac{v_\text{п}^2}{2} \left( 1 - \frac{(1 -e)^2}{(1 + e)^2} \right) = GM \left( \frac{1}{a(1-e)} - \frac{1}{a(1+e)} \right),\\
	\frac{v_\text{п}^2}{2} \cdot \frac{ 1 + 2e + e^2 - 1 + 2e - e^2}{(1+e)^2} = \frac{GM}{a} \cdot \frac{1 + e - 1 +  e}{(1+e)(1-e)},\\
	v_\text{п} = \sqrt{\frac{GM}{a}}\sqrt{\frac{1+e}{1-e}}, \quad \quad v_\text{a} = \sqrt{\frac{GM}{a}}\sqrt{\frac{1-e}{1+e}}.
\end{gather}
Запишем теперь ЗСЭ для перицентра и произвольной точки орбиты на расстоянии $r$:
\begin{gather*}
	-\frac{GMm}{q} + \frac{m v^2_\text{п}}{2} = -\frac{GMm}{r} + \frac{m v^2}{2},\\
	-\frac{GMm}{q} + \frac{GMm}{2a} \cdot \frac{1+e}{1-e} = -\frac{GMm}{r} + \frac{m v^2}{2},\\
	v^2 = GM \left( \frac{2}{r} - \frac{2}{a(1 - e)} + \frac{1+e}{a (1-e) }\right) = GM \left( \frac{2}{r} - \frac{1}{a} \right),
\end{gather*}
\begin{equation}
	v = \sqrt{ GM \left( \frac{2}{r} - \frac{1}{a} \right)}.
	\label{eq:int-energy}
\end{equation}
Полученное выражение и называется интегралом энергии. Согласно \eqref{eq:int-energy} и \eqref{eq:ellipse-pol-eq} для скорости тела в произвольной точке орбиты также справедливо выражение
\begin{equation}
	v = \sqrt{\frac{GM}{p}\cdot(1 + 2 e \cos \nu + e^2)},
\end{equation}
где $\nu$~--- истинная аномалия, а $p$~--- фокальный параметр.

Найдем величину момента импульса пробной массы $m$ на эллиптической орбите. В силу постоянства данной величины, можно выбрать любую точку орбиты для её поиска. Проще всего рассмотреть перицентр или апоцентр, рассмотрим первый.
\begin{multline*}
	L
	= m v_q q
	= m \sqrt{\frac{GM}{a} \frac{1+e}{1-e}} \cdot a(1-e) =\\
	= m \sqrt{GMa (1 + e)(1-e)}
	= m \sqrt{GMa(1-e^2)}
	= m \sqrt{GMp}.
\end{multline*}

Для параболической также рассмотрим точку перицентра:
\begin{multline*}
	L
	= m v_q q
	= m v_2(q) q
	= m \sqrt{\frac{2GM}{q}} \cdot q =\\
	= m \sqrt{2GMq}
	= m \sqrt{2GM \cdot \frac{p}{2}}
	= m \sqrt{GMp}.
\end{multline*}


\input{sections/cel-mech.orbit-elem.tex}
\subsection{Точки Лагранжа}

\term{Точки Лагранжа}~--- точки, во вращающейся системе из двух массивных тел,
\begin{wrapfigure}[14]{l}{0.48\tw}
	\centering
	\vspace{-.5pc}
	\includegraphics[width = .48\tw]{lagr-points}
	\captionof{figure}{Точки Лагранжа}
	\label{pic:larg-points}	
\end{wrapfigure}
в которых третье тело с пренебрежимо 
малой массой, не испытывающее воздействие никаких 
других сил, кроме гравитационных, со стороны двух 
первых тел, может оставаться неподвижным относительно 
этих тел. В этих точках гравитационные силы, 
действующие на малое тело, уравновешиваются силами инерции.

Точки $L_1$, $L_2$ и $L_3$ лежат на одной прямой, 
соединяющей два массивных тела. Точки $L_4$ и $L_5$ 
образуют равносторнние треугольники с массивными 
телами.

Для расстояний до точек $L_1$, $L_2$ и $L_3$ от 
центра масс системы справедливы следующие выражения:
\begin{equation}r_1=R\left(1-\sqrt[3]{\frac{\alpha}
{3}}\right), \quad r_2=R\left(1+\sqrt[3]{\frac{\alpha}
{3}}\right), \quad r_3=R\left(1+\frac{5}{12}\alpha\right),
\end{equation}
где $\alpha=M_2 / (M_1 + M_2)$, $R$~--- расстояние между 
телами, $M_1$ --- масса более массивного тела, $M_2$
 --- масса второго тела.

Если $M_2 \ll M_1$, то точки $L_1$ и $L_2$ находятся 
примерно на одинаковом расстоянии от тела $M_2$, равном
\begin{equation}
r\approx R\sqrt[3]{\frac{M_2}{3M_1}}.
\end{equation}

Расстояния от центра масс системы до точек $L_4$ и $L_5$ в координатной системе с центром координат в центре масс системы рассчитываются по  формулам
\begin{equation}
	 r_4 = \left ( \frac{R}{2} \cdot \frac{M_1-M_2}{M_1+M_2} ,   \frac{\sqrt{3}R}{2} \right ), \quad r_5 = \left ( \frac{R}{2} \cdot \frac{M_1-M_2}{M_1+M_2} ,   -\frac{\sqrt{3}R}{2} \right ). 
\end{equation}
\subsection{Приливы и отливы}

\term{Приливы и отливы}~--- периодические вертикальные колебания уровня океана, являющиеся результатом изменения положения Луны и Солнца. Хотя силы тяготения Солнца почти в 200 раз больше, чем силы тяготения Луны, приливные силы, порождаемые Луной, почти вдвое больше порождаемых Солнцем. Это происходит из-за того, что приливные силы зависят не от величины гравитационного поля, а от степени его неоднородности. Высота приливов зависит от взаимного расположения Луны и Солнца: наибольший~---  силы от Луны и от Солнца действуют вдоль одного направления, а наименьший~--- под прямым углом друг к другу.

\begin{minipage}{.24\tw}
Ускорение в центре Земли ($T$) определяется формулой \eqref{eq:g}:
\begin{equation*}
	a_T=\frac{G M}{r^2},
\end{equation*}
$M$~--- масса возмущающего тела,
\end{minipage}
\hfill
\begin{minipage}{0.74\tw}
	\vspace{-.5pc}
	\includegraphics[width = \tw]{Ebb_flow}
	\captionof{figure}{К объяснению приливных сил}\label{Ebb_flow}
\end{minipage}\\[-0.5pc]

$r$~--- расстояние между центрами Земли и данного тела. Аналогично, ускорения в точках $A$ и $B$ равны соответственно
\begin{equation}
	a_A = \frac{G M}{(r - R)^2} \quad \text{и} \quad a_B = \frac{GM}{(r + R)^2},
\end{equation}
где $R$~--- радиус Земли или иного тела, подверженного воздействию приливных сил. Ускорение в точке $A$ относительно точки $T$ равно
\begin{equation}
	a_A - a_T = a_T \cdot \frac{2 r R - R^2}{(r - R)^2} = \frac{GM \left(2 r R - R^2 \right)}{r^2 (r - R)^2} \xrightarrow{R \ll r} \frac{2 G M R}{r^3}.
	\label{eq:ebb-force}
\end{equation}

Под действием лунного притяжения водная оболочка Земли принимает форму 
эллипсоида, который вытянут по направлению к Луне. Близ точек $A$ и $B$ будет 
прилив, а в точках $F$ и $D$ --- отлив (см.~Рис.\,\ref{Ebb_flow}).
\nopagebreak
\subsection{Затмения}
Диаметр тени спутника при полном центральном затмении (когда центры трёх тел лежат на одной прямой) с большой точностью равен 
\begin{equation}
d_\text{тени} = 2 \cdot \frac{R_{\moon}(a_\oplus - R_\oplus) - R_\odot \left( a_{\moonч} - R_\oplus \right)}{a_\oplus - a_{\moon}}.
\end{equation}
Среднее значение  этой величины около 200~км, максимальное~--- около 215~км. При нецентральном затмении максимальный диаметр тени Луны на поверхности Земли может достигать 270~км. Это даёт оценку на продолжительность, равную 7.5 минутам. Большинство полных затмений длятся 2\,--\,4~минуты.

\begin{figure}[h!]
\centering
\vspace{-.5pc}
\includegraphics[width = 10cm]{full_eclipse}
\caption{Полное солнечное затмение со стороны северного полюса эклиптики}
\label{fig:eclipses-full-solar-eslipse}
\end{figure}
При \term{кольцеобразном солнечном затмении} Луна так расположена относительно Земли, что конус её тени не достаёт до поверхности планеты, и вокруг Луны можно наблюдать яркое кольцо незакрытой части солнечного диска.

\begin{figure}[h!]
	\centering
	\includegraphics[width = 10cm]{partly-eclipse}
	\caption{Кольцеобразное солнечное затмение со стороны северного полюса эклиптики}
	\label{fig:eclipses-circle-solar-eslipse}
\end{figure}
При особом расположении Луны и Земли возможны \term{гибридные} затмения, когда в разных пунктах Земли наблюдаются либо \imp{кольцеобразное}, либо \imp{полное} затмение. Причиной такого явления является шарообразность Земли.

\vspace{-1pc}
\begin{figure}[h!]
	\centering
	\includegraphics[width=10cm]{moon-eclipse}
	\caption{Схема лунного затмения со стороны северного полюса эклиптики}
	\label{fig:moon-eclipse-scheme}
\end{figure}
\term{Лунное затмение}, в отличие от солнечного, видно со всего ночного полушария Земли. Диаметр земной тени на расстоянии Луны превышает размер последней примерно в 2.5\,--\,3 раза. Бывают \term{частные}, когда лишь часть Луны попадает в земную тень, \term{полные}~--- Луна полностью погружается в тень Земли, и \term{полутеневые}~--- Луна проходит через полутень Земли, не затрагивая конус тени.

\term{Синодический месяц}~--- промежуток времени между одинаковыми фазами Луны, равен 29.53 суток.

\term{Драконический месяц}~--- промежуток времени между двумя последовательными прохождениями Луны через один и тот же узел орбиты,~--- 27.21 суток.

\term{Сарос}~--- промежуток  времени, по прошествии которого солнечные и лунные затмения повторяются в прежнем порядке. Происходит это из-за того, что каждый сарос Луна, орбита Луны и Солнце возвращаются в прежнее положение относительно далёких звёзд. Сарос длится ровно 242 драконических месяца, или 223 синодических месяца, или 18 лет 11 дней 8 часов.

\begin{wrapfigure}[8]{r}{.42\tw}
	\centering
	\vspace{-1pc}
	\includegraphics[width = 0.2\textwidth]{phases}
	\hfill
	\includegraphics[width = 0.2\textwidth]{phases-2}
	\caption{Частное и полное затмение}
	\label{fig:part-eclipses-scheme}
\end{wrapfigure}
Важной характеристикой любого затмения является его \term{фаза}~--- для \imp{частных} и \imp{кольцеобразных} затмений: отношение закрытой части $x$ диаметра\footnote{Здесь имеется в виду \imp{угловой} диаметр} затмеваемого тела, проходящего через центр затмевающего тела, ко всему диаметру затмеваемого тела $D$; для \imp{полного}: единица плюс отношение расстояния\footnote{Расстояние между окружностями $l_1$ и $l_2$~--- это $\min |L_1L_2|$ по всем $L_1 \in l_1$ и $L_2 \in l_2$.} между краями дисков затмеваемого и затмевающего тел к диаметру затмеваемого тела $D$.
\begin{equation}
\Phi_{\text{част}} = \frac{x}{D} < 1, \quad \quad \quad \Phi_{\text{полн}} =  1 + \frac{\min\{d_1, d_2\}}{D} > 1.
\end{equation}
Иногда вводят такое понятие, как \term{площадная фаза затмения}, т.\,е. отношение площади закрытой части диска затмеваемого тела к полной площади его диска. Чаще всего  площадную фазу используют применительно к двойным звёздам, когда считают падение блеска при затмении одной звезды другой.

\input{sections/cel-mech.planet-config.tex}
\input{sections/cel-mech.phase-angle.tex}
\input{sections/cel-mech.synod-period.tex}
\subsection{Собственное движение звёзд}
\term{Собственным движением} $(\mu)$ называется изменение координат звёзд на небесной сфере, вызванное относительным движением звёзд и Солнца, обычно измеряется в mas/год.
\begin{equation}
	\mu = \frac{V_\tau}{D},
\end{equation}
где $V_\tau$~--- тангенциальная относительная скорость звезды, $D$~--- расстояние до неё.

\change{Разделяют также собственное движение по склонению~--- $\mu_\delta$ и собственное движение по прямому восхождению~--- $\mu_\alpha$, которые определяются следующими выражениями:}
\begin{equation}
  \mu_\delta = \frac{\delta(t_2) - \delta(t_1)}{t_2 - t_1}, \quad \quad \mu_\alpha = \frac{\alpha(t_2) - \alpha(t_1)}{t_2 - t_1}.
\end{equation}
\change{
\begin{wrapfigure}{r}{.4\tw}
\begin{flushright}
	\vspace{-1pc}
	\begin{tikzpicture}
	\footnotesize
	\draw [dashes] (0, 4) arc(90:0:3 and 4);
	\draw [dashes] (0, 4) arc(90:0:2 and 4); 
	%
	\draw [dashes] (3.47, 2) arc(0:-70:3.47 and 1.16);	
	\draw [dashes] (2.64, 3) arc(0:-70:2.64 and 0.88);
	%
	\draw [thick, -latex] (2.3, 2.55) arc(-34:-56:2.64 and 0.88);
	\draw [thick, -latex] (2.3, 2.55) arc(53:29:2 and 4);
	\draw [thick, -latex] (2.3, 2.55) .. controls (2.3, 1.9) and (2.1, 1.4) .. (1.93, 1.03);
	%
	\draw (.9, 2.2) node [anchor=south] {$\delta(t_1)$};
	\draw (1.2, .9) node [anchor=south] {$\delta(t_2)$};
	%
	\draw (2, 0) node [anchor=north] {$\alpha(t_2)$};
	\draw (3, 0) node [anchor=north] {$\alpha(t_1)$};
	%
	\draw [fill=white] (2.3, 2.55) circle (0.03);
	\draw [fill=white] (1.93, 1.03) circle (0.03);
	\draw [fill=white] (0, 4) circle (0.03);
	%
	\draw (0, 4) node [anchor=north] {$P$};
	%
	\draw (1.9, 2.4) node [anchor=south] {$\mu_\alpha$};
	\draw (2.6, 2.05) node [anchor=west] {$\mu_\delta$};
	\draw (2.06, 1.65) node [anchor=south] {$\mu$};
	%
\end{tikzpicture}
\end{flushright}
\end{wrapfigure}
 Как отсюда видно, $\mu_\alpha$ является угловой скоростью по малому кругу, а значит, зависит от $\delta$. Следовательно, полное собственное движение $\mu$ можно найти, как
\begin{equation}
	\mu = \sqrt{\mu_\delta^2 + \mu_\alpha^2 \cos^2 \delta},
\end{equation}
потому что радиус малого круга, состоящего из точек со склонением~$\delta$, равен $R \cos \delta$, где $R$~--- радиус сферы, содержащей этот круг.
}

\begin{figure}[h!]
\begin{subfigure}[b]{0.47\tw}
	\begin{tikzpicture}[scale=1.05]
	\footnotesize
	
%	\foreach \x in {0, .1,...,4} {
%		\draw [line width=.1pt] (\x - 1, 0) -- (\x - 1, 4);
%	};
%	
%	\foreach \x in {0, 1,...,4} {
%		\draw [line width=.4pt] (\x - 1, 0) -- (\x - 1, 4);
%	};
%	
%	\foreach \y in {0, .1,...,4} {
%		\draw [line width=.1pt] (-1, \y) -- (4, \y);
%	};
%	
%	\foreach \y in {0, 1,...,4} {
%		\draw [line width=.4pt] (-1, \y) -- (4, \y);
%	};
	
	\draw [double] (.21, .21) arc (45:104:.3);
	\draw (-.93, 3.71) arc (-76:-35:.3);
	
	\draw (0, 0) -- (-1, 4);
	\draw (0, 0) -- (2, 2);
	\draw (-1, 4) -- (2.6, 1.6);
	
	\draw [thick, -latex] (-1, 4) -- (0, 4.25);
	\draw [thick, -latex] (-1, 4) -- (-.6, 2.4);
	
	\draw [fill=white] (-1, 4) circle (.03);
	\draw [fill=white] (0, 0) circle (.03);
	\draw [fill=white] (2, 2) circle (.03);
	
	\draw (1, 1) node [anchor=north west] {$R$};
	\draw (-.45, 2.1) node [anchor=north east] {$R_0$};
	\draw (.5, 2.95) node [anchor=south west] {$V \Delta t$};
	\draw (0, 0) node [anchor=north] {Солнце};
	\draw (-1, 4) node [anchor=south east] {Звезда};
	
	\draw (.1, .3) node [anchor=south] {$\xi$};
	\draw (-.9, 3.75) node [anchor=north west] {$\gamma$};
	
	\draw (-.5, 4.15) node [anchor=south] {$V_\tau$};
	\draw (-.75, 3.1) node [anchor=east] {$V_r$};
\end{tikzpicture}
\caption{}
\label{pic:phase-angle-1}
\end{subfigure}
\hfill
\begin{subfigure}[b]{0.47\tw}
\begin{tikzpicture}[scale=0.9]
	\footnotesize
	
	\draw (.2, 4.86) arc (-45:-135:0.28);
	\draw [double] (-1.65, 1.51) arc (5:80:0.25);
	
	\draw (0, 5) .. controls (-1.5, 4) and (-2, 2) .. (-2, 0);
	\draw (0, 5) .. controls (1.5, 4) and (2, 2) .. (2, 0);
	\draw (-2, 0) .. controls (-1, -.5) and (1, -.5) .. (2, 0);
	\draw (-1.9, 1.5) .. controls (-1, 1.5) and (1, 2) .. (1.5, 3);
	
	\draw [fill=white] (0, 5) circle (.03);
	\draw [fill=white] (-2, 0) circle (.03);
	\draw [fill=white] (2, 0) circle (.03);
	\draw [fill=white] (-1.9, 1.5) circle (.03);
	\draw [fill=white] (1.5, 3) circle (.03);
	
	\draw (-2, .2) -- (-1.8, .11) -- (-1.8, -.09);
	\draw (2, .2) -- (1.8, .11) -- (1.8, -.09);
	
	\draw (0, 5) node [anchor=south] {$P$};
	\draw (0, 1.9) node [anchor=north] {$\xi$};
	\draw (0, -.4) node [anchor=south] {$\Delta \alpha$};
	\draw (0, 4.8) node [anchor=north] {$\Delta \alpha$};
	\draw (-1, 4) node [anchor=east] {$90^\circ - \delta$};
	\draw (.9, 4.2) node [anchor=west] {$90^\circ - (\delta + \Delta \delta)$};
	\draw (0, -.4) node [anchor=north] {Небесный экватор};
\end{tikzpicture}
\caption{}
%\label{pic:phase-angle-2}
\end{subfigure}
\caption{}
\end{figure}


\change{Получим выражение для координат звезды, имеющей собственное движение $\mu = (\mu_\alpha, \mu_\delta)$, лучевую скорость $V_r$ и параллакс в начальный момент времени $\pi_0$. Найдем сначала тангенциальную скорость:
\begin{equation*}
	V_\tau = R_0 \sqrt{ \mu_\delta^2 + \mu_\alpha^2 \cos^2 \delta} = \frac{\sqrt{ \mu_\delta^2 + \mu_\alpha^2 \cos^2 \delta}}{\pi_0}.
\end{equation*}
Определим теперь угол между лучем зрения и полной скоростью звезды:
\begin{equation*}
	\gamma = \arctan \frac{V_\tau}{V_r}.
\end{equation*}
При этом полная скорость равна
\begin{equation*}
	V_0 = \sqrt{V_\tau^2 + V_r^2}.
\end{equation*}
Из теоремы косинусов можно найти расстояние для звезды через промежуток времени $\Delta t$:
\begin{equation*}
	R = \sqrt{R_0^2 + (V_0 \Delta t)^2 - 2 R_0 V_0 \Delta t \cos \gamma}.
\end{equation*}
Тогда угловое перемещение звезды равно
\begin{equation*}
	\sin \xi = \frac{V_0 \Delta t \sin \alpha}{R}.
\end{equation*}
Через компоненты собственного движения нетрудно получить угол между направлением на полюс и вектором полного собственного движения в начальный момент:
\begin{equation*}
	\tg \psi =  \frac{\mu_a \cos \delta}{\mu_\delta}.
\end{equation*}
Теперь с помощью сферической теоремы косинусов можно определить склонение звезды через время $\Delta t$:
\begin{equation*}
	\sin (\delta - \Delta \delta) = \cos \xi \sin \delta + \sin \xi \cos \delta \cos \psi.
\end{equation*}
Далее из сферической теоремы синусов получаем выражение для изменения прямого восхождения за время $\Delta t$~---
\begin{equation*}
	\sin \Delta \alpha = \frac{\sin \psi \sin \xi}{\cos (\delta - \Delta \delta)}.
\end{equation*}
}






\input{sections/cel-mech.precession.tex}
	\section{Конические сечения}
\subsection{Эллипс}

{\bfseries \term{Эллипс}}~--- плоская замкнутая кривая, сумма расстояний от любой точки которой до двух фиксированных точек, называемых фокусами, постоянна.
\begin{equation}
	|F_1 M| + |F_2 M|=\const \label{eq:ell-def}
\end{equation}
\imp{Центром} эллипса называется середина отрезка, соединяющего его фокусы.

\begin{minipage}{0.5\tw}
	\imp{Большая ось} эллипса~--- прямая, проходящая через фокусы эллипса; \imp{малая ось}~--- прямая ей перпендикулярная и проходящая через центр эллипса.
	
	Главные отрезки эллипса: \term{большая полуось} ($a$)~--- расстояние от центра эллипса до его пересечения с большой осью; \term{малая полуось} ($b$) определяется дословно также, заменив большую ось на малую; \term{фокальное расстояние} ($c$)~--- расстояние от центра эллипса до одного из фокусов, что тоже самое, половина расстояния между фокусами.
	
\end{minipage}
\begin{minipage}{0.5\tw}
	\begin{flushright}
		\vspace{-.5pc}
		\includegraphics[width = .97\tw]{Ellips}
		\captionof{figure}{Эллипс}
		\label{pic:ellipse}
	\end{flushright}
\end{minipage}


Рассмотрим крайнюю левую и крайнюю правую точки эллипса на Рис.~\ref{pic:ellipse}, назовем их $A$ и $B$ соответственно, тогда сумма расстояний $l$ от каждой из них до фокусов $F_1$ и $F_2$ равна:
\begin{equation*}
	AF_1 + AO + OF_2 = AF_1 + a + c = l = BF_2 + BO + OF_1 = BF_2 + a + c.
\end{equation*}
Откуда следует, что $A F_1 = B F_2$. Легко видеть, что $AB = 2a$, значит $l = AF_1 + AO + OF_2 = AO + OF_2 + F_2B = 2a$. Получается, сумма расстояний до фокусов от любой точки эллипса равна его удвоенной большой полуоси.

В силу равенства прямоугольных треугольников $\triangle F1 O C$ и $\triangle F_2 O C$ равны их гипотенузы $F_1C$ и $F_2C$, причем $F_1C= F_2C = l/2 = a$. Отсюда получается одно из основных соотношений в эллипсе:

\begin{equation}
	b^2 + c^2 = a^2.
\end{equation}
\term{Эксцентриситет} ($e$)~--- числовая
характеристика, показывающая степень отклонения конического сечения от окружности. Для эллипса $e$ лежит в интервале $(0, \, 1)$ и
определяется формулой
\begin{equation}
	e = \frac{c}{a}.
\end{equation}

\term{Апоцентр}~--- наиболее удаленная от заданного фокуса точка эллипса. Расстояние $Q$ до апоцентра от фокуса~--- сумма расстояний от фокуса, до центра эллипса и расстояния от центра до апоцентра, т.\,е.

\begin{equation}
	Q = c + a = a (1 + e).
\end{equation}

\term{Перицентр}~--- ближайшая точка эллипса к заданному фокусу. По аналогии с апоцентром, для расстояния $q$ от фокуса эллипса до перицентра справедливо следующее:
\begin{equation}
	q = a - c = a (1 - e).
\end{equation}

\term{Фокальный параметр}~($p$)~--- длина перпендикуляра, проведенного из фокуса до точки пересечения с эллипсом. Найдем $p$ из теоремы Пифагора для треугольника $\triangle F_1 F_2 P$:
\begin{gather}
	p^2 = (2a - p)^2 - (2c)^2, \notag\\
	p^2 =  4a^2 - 4ap + p^2 - 4a^2e^2,\notag\\
	0 = a ( 1 - e^2) - p,\notag\\
	p = a (1 - e^2) = \frac{b^2}{a} = b \sqrt{1 - e^2}.
\end{gather}

Получим теперь выражение для расстояния от произвольной точки эллипса с \term{истинной аномалией}~$\nu$~--- угол
{\slshape перицентр -- фокус -- заданная точка},
отсчитываемый в сторону движения по эллипсу. Для этого необходимо рассмотреть треугольник {\slshape перицентр -- фокус -- заданная точка} и записать для него теорему косинусов:
\begin{gather*}
	(2a - r)^2 = r^2 + (2c)^2 - 2 r \cdot 2c \cos (180^\circ - \nu),\\
	4a^2 - 4ar + r^2 = r^2 + 4 a^2 e^2 + 4 r a e \cos \nu,\\
	a - r = ae^2 + re\cos \nu,\\
	r = \frac{a(1 - e^2)}{1 + \cos \nu}.
\end{gather*}
Полученное выражение для длины радиус-вектора точки на эллипсе в зависимости от ее угла от перицентра называется \term{уравнением эллипса в полярных координатах}. Если же полюс системы координат расположить в другом фокусе, тогда полярный угол будет, очевидно, отсчитываться от точки апоцентра и в знаменателе уравнения будет знак <<$-$>>. Окончательно,
\begin{equation}
	r = \frac{a ( 1- e^2)}{1 \pm e \cos \nu}.
	\label{eq:ell-eq-pol}
\end{equation}

Перейдем теперь в декартовы координаты, в которых $r = \sqrt{x^2 + y^2}$, а $\cos \nu = x / r$, тогда:
\begin{gather*}
	\sqrt{x^2 + y^2} = \frac{a(1 - e^2)}{1 + e \cdot \dfrac{x}{\sqrt{x^2 + y^2}}} = \frac{a (1 - e^2) \sqrt{x^2 + y^2}}{\sqrt{x^2 + y^2} + e x},\\
	\sqrt{x^2 + y^2} = a(1 - e^2) - e x,\\
	x^2 + y^2 = a^2 (1 - e^2)^2 + e^2 x^2 - 2ae(1 - e^2) x,\\
	(1 - e^2) x^2 + y^2 = a^2 (1 - e^2)^2 - 2ae (1 - e^2) x,\\
	x^2 + \frac{y^2}{1 - e^2} = a^2 (1 - e^2) - 2aex.
\end{gather*}
Сдвинем систему отсчета влево так, чтобы центр эллипса оказался в начале координат, тогда новая координата по оси $x$ определяется, как $\xi = x + ae$, иначе $x = \xi - ae$. Подставим полученное выражение в преобразованное уравнение:
\begin{gather*}
	\xi^2 + a^2 e^2 - 2ae\xi + \frac{y^2}{1 - e^2} = a^2 (1 - e^2) - 2 ae\xi + 2 a^2 e^2,\\
	\xi^2 + a^2 e^2  + \frac{y^2}{1 - e^2} = a^2 - a^2 e^2 + 2 a^2 e^2,\\
	\frac{\xi^2}{a^2} + \frac{y^2}{a^2 (1 - e^2)} = 1,\\
	\frac{\xi^2}{a^2} + \frac{y^2}{b^2} = 1. \tag{\theequation} \label{eq:ell-eq-dec}
\end{gather*}
Данное уравнение называется \term{уравнением эллипса в декартовых координатах}.

Далее покажем, что любая точка плоскости, удовлетворяющая \eqref{eq:ell-eq-dec} принадлежит эллипсу с большой полуосью $a$ и малой~--- $b$, чтобы доказать равносильность предыдущих переходов. Для этого выберем произвольную точку $(x_0, y_0)$, для которой выполняется \label{eq:ell-eq-dec}, т.\,е.
\begin{equation}
	y_0 = \pm b \sqrt{1 - \frac{x_0^2}{a^2}} = \pm \sqrt{1 - e^2} \sqrt{a^2 - x_0^2}. \label{eq:ell-y0}
\end{equation}
Докажем, что множество точек $(x_0, y_0)$, для которых справедливо равенство \eqref{eq:ell-y0} являются эллипсом, а точки $(\pm a e, 0)$~--- его фокусами:
\begin{multline*}
	r_{1,2}
	= \big| (x_0, y_0) - (\pm ae, 0) \big|
	= \sqrt{(x_0 \mp ae)^2 + y_0^2} =\\
	= \sqrt{x_0^2 \mp 2 a e x_0 + a^2 e^2 + a^2(1 - e^2) - x_0^2(1 - e^2)} =\\
	= \sqrt{x_0^2 \mp 2 a e x_0 + a^2 e^2 + a^2 - a^2 e^2 - x_0^2 + e^2 x_0^2} = \\
	= \sqrt{e^2 x_0 \mp 2 a e x_0 + a^2 } = \sqrt{(e x_0 \mp a)^2} = |e x_0 \mp a| = a \mp ex_0.
\end{multline*}
Отсюда получается, что $r_1 + r_2 = 2a$, а значит рассмотренное множество точек, удовлетворяет определению эллипса. Это доказывает эквивалентность определений эллипса в виде \eqref{eq:ell-def}, \eqref{eq:ell-eq-pol} и \eqref{eq:ell-eq-dec}.

Теперь легко показать, что эллипс является образом афинного преобразования сжатия окружности с радиусом $a$. Для этого рассмотрим окружность, задаваемую уравнением $x^2 + y^2 = a^2$ и сжатие вдоль оси $y$ с коэффициентом $1/\sqrt{1 - e^2}$. Тогда $x' \hookrightarrow x$, а $y' \hookrightarrow y \sqrt{1-e^2}$. Следовательно для обратного преобразования $x \hookrightarrow x'$, а $y \hookrightarrow y'/\sqrt{1 - e^2}$. При этом прообраз сжатой окружность должен быть исходной окружность, а значит, удовлетворять исходному уравнения, то есть
\begin{gather*}
	x'^2 + \frac{y'^2}{1 - e^2} = a^2,\\
	\frac{x'^2}{a^2} + \frac{y'^2}{b^2} = 1.
\end{gather*}
Получаем, что образ окружности под действием афинного преобразования сжатия является эллипсом.

Данный факт помогает легко найти \term{площадь эллипса}~($S$)~--- площадь части
плоскости, ограниченной эллипсом. Так как под действием сжатия площади уменьшаются пропорционально коэффициенту преобразования, что есть
\begin{equation}
	S = S_\text{окр} \sqrt{1 - e^2} = \pi a^2 \sqrt{1 - e^2} = \pi a b.
\end{equation}

Также из свойств афинного преобразования и параметрического уравнения окружности следует \imp{параметрическое уравнение эллипса}, которое имеет такой вид:
\begin{equation}
	\left\{
	\begin{aligned}[lcl]
		&x=a\cos t,\\
		&y=b\sin t;\\
	\end{aligned}
	\right. \quad\quad t \in [0, \, 2\pi).
\end{equation}

Кроме этого, эллипс обладает важным \imp{оптическим свойством}, которое можно сформулировать так: свет от источника в одном из фокусов, отражается эллипсом так, что отражённые лучи пересекаются во втором фокусе или, что тоже самое, касательная к эллипсу в заданной точке образует с фокальными радиусами в данной точке равные острые углы. Для его доказательства необходимо показать равенство углов между касательной в точке к эллипсу и направлениями на фокусы. Для этого получим сначала уравнение касательной к эллипсу в произвольной точке $(x_0, y_0$, принадлежащей ему. Как было показано выше, эллипс можно представить объединением графиков двух функций \eqref{eq:ell-y0}. Найдем теперь производные от этих функций по $x_0$:
\begin{equation*}
	(y_0)'_{x_0} = \pm \sqrt{1 - e^2} \cdot \frac{-x_0}{\sqrt{a^2 - x_0^2}}.
\end{equation*}
Так как значение производной в точке равно тангенсу угла наклона касательной, то направляющий вектор касательной можно представить в виде $\vec{t} = \left(1,  (y_0)'_{x_0}\right)$, иначе:
\begin{equation*}
	\vec t =
	\begin{pmatrix}
		1\\
		\mp \sqrt{1 - e^2} \cdot \dfrac{\alpha}{\sqrt{1 - \alpha^2}}
	\end{pmatrix},~\text{где}~ \alpha \equiv \frac{x_0}{a},\quad -1 \leqslant \alpha \leqslant 1.
\end{equation*}
При этом модуль направляющего вектора $\vec t$ определяется, как
\begin{equation*}
	|\vec t| \equiv t = \sqrt{t_x^2 + t_y^2} = \sqrt{1 + (1 - e^2) \cdot \frac{\alpha^2}{1 - \alpha^2}} = \sqrt{\frac{1 - e^2 \alpha^2}{1 - \alpha^2}}.
\end{equation*}
Выпишем векторы $\vec r_1$ и $\vec r_2$, от точки $(x_0, y_0)$ до фокусов, имеющих координаты $(\pm ae, 0)$ соответственно:
\begin{equation*}
	\vec r_{1,2} =
	\begin{pmatrix}
		\pm ae - x_0\\
		-y_0
	\end{pmatrix}.
\end{equation*}
Покажем теперь, что $\cos \widehat{\vec r_1 \vec t} = - \cos \widehat{\vec r_2 \vec t}$ для верхнего полуэллипса, что завершит доказательство оптического свойства эллипса, так как для нижнего доказательство аналогично:
\begin{multline*}
	\cos \widehat{\vec r_{1, 2}  \vec t}
	= \frac{(r_{1, 2}, t)}{|\vec r_{1, 2}| | \vec t|} = \\
	= \frac{(\pm ae - x_0) \cdot 1 + \left[ -y_0 \cdot \left(- \sqrt{1 - e^2} \cdot \dfrac{\alpha}{\sqrt{1 - \alpha^2}} \right) \right]}{\sqrt{(\pm ae - x_0)^2 + y_0^2} \cdot \sqrt{\dfrac{1 - e^2 \alpha^2}{1 - \alpha^2}}} =\\
	= \frac{(\pm ae - x_0) \cdot \sqrt{1 - \alpha^2} + \alpha y_0\sqrt{1 - e^2}}{\sqrt{(\pm ae - x_0)^2 + y_0^2} \cdot \sqrt{1 - e^2 \alpha^2}}.
\end{multline*}
Из параметричекого уравнения эллипса можно получить, что $\sin t = x_0/a = \alpha$, тогда
\begin{equation*}
	y_0 = b \cos t = b \sqrt{1 - \sin^2 t} = b \sqrt{1 - \alpha^2} = a \sqrt{1 - e^2} \sqrt{1 - \alpha^2}.
\end{equation*}
Подставим данное выражение для $y_0$ в предыдущие выкладки:
\begin{multline*}
	\cos \widehat{\vec r_{1, 2}  \vec t}
	= \frac{a ( \pm e - \alpha) \sqrt{1 - \alpha^2} + \alpha a \sqrt{1 - e^2} \sqrt{1 - \alpha^2} \sqrt{1 - e^2}}{a \sqrt{(\pm e - \alpha)^2 + (1 - e^2)(1 - \alpha^2)} \cdot \sqrt{1 - e^2 \alpha^2}} = \\
	= \frac{ (\pm e - \alpha + \alpha - e^2 \alpha) \sqrt{1 - \alpha^2}}{\sqrt{e^2 \mp 2 e\alpha + \alpha^2 + 1 - \alpha^2 - e^2 + e^2 \alpha^2} \cdot \sqrt{1 - e^2 \alpha^2}} = \\
	= \frac{ (\pm e - e^2 \alpha) \sqrt{1 - \alpha^2}}{\sqrt{\mp 2 e\alpha + 1  + e^2 \alpha^2} \cdot \sqrt{1 - e^2 \alpha^2}} = \\
	= \frac{ e(\pm 1 - e \alpha) \sqrt{1 - \alpha^2}}{|1 \mp e\alpha| \sqrt{1 - e^2 \alpha^2}}
	= \frac{ e(\pm 1 - e \alpha) \sqrt{1 - \alpha^2}}{(1 \mp e\alpha) \sqrt{1 - e^2 \alpha^2}}
	= \pm \frac{e \sqrt{1 - \alpha^2}}{\sqrt{1 - e^2 \alpha^2}}.
\end{multline*}


\input{sections/conic-sec.parabola.tex}
\subsection{Гипербола}

{\bfseries \term{Гипербола}} --- геометрическое место точек евклидовой плоскости, абсолютное значение разности расстояний от которых до двух выделенных точек $F_1$ и $F_2$, называемых фокусами, постоянно и равно удвоенной действительной полуоси гиперболы.
\begin{equation}
	\bigl||F_1M|-|F_2M|\bigr| = 2a
\end{equation}

\begin{wrapfigure}[14]{r}{0.5\tw}
	\vspace{-1pc}
	\includegraphics[width = 0.5\textwidth]{Hiperbola}
	\captionof{figure}{Гипербола \label{pic:the-pic}}
\end{wrapfigure}
Ближайшие друг к другу точки двух ветвей гиперболы называются \term{вершинами} гиперболы. \term{Большая} или \term{действительная полуось}~($a$) гиперболы~--- расстояние от центра гиперболы до одной из вершин. \term{Фокальное расстояние}~($c$)~---  расстояние от центра гиперболы до одного из фокусов. \term{Эксцентриситетом} гиперболы~($e$), как и  эллипса, является отношение фокального расстояния к большой полуоси, так как большая полуось гиперболы всегда меньше её фокального расстояния, эксцентриситет гиперболы $e > 1$ и может быть найдет из определения:
\begin{equation}
	e=\frac{c}{a}.
\end{equation}

\term{Перицентрическое расстояние} ($q$) --- расстояние от фокуса до ближайшей вершины гиперболы, можно найти, как
\begin{equation}
	q = a ( e - 1).
\end{equation}
\term{Мнимая полуось}~($b$)~--- длина перпендикуляра к оси абсцисс, восставленного из вершины до пересечения с асимптотой. Равна \term{прицельному параметру}~($f$)~--- расстоянию от фокуса до асимптоты гиперболы.

\term{Фокальный параметр}~($p$)~--- длина отрезка, перпендикулярного к действительной оси, проведённого от фокуса до гиперболы. Определяется формулой
\begin{equation}
	p=\frac{b^2}{a}.
\end{equation}

\imp{Каноническое уравнение гиперболы} в прямоугольных декартовых координатах записывается следующим образом:
\begin{equation}
	\frac{x^2}{a^2}-\frac{y^2}{b^2}=1.
\end{equation}
В \imp{полярных координатах} уравнение принимает вид:
\begin{equation}
	r=\frac{p}{1-e\cos\varphi},
\end{equation}
причём полюс находится в фокусе гиперболы, а вершина гиперболы лежит на продолжении полярной оси.

\imp{Уравнение двух асимптот} является уравнением пересекающихся прямых:
\begin{equation}
	\frac{x}{a}\pm\frac{y}{b}=0.
\end{equation}

Важным соотношением для элементов гиперболы является
\begin{equation}
	c^2=a^2+b^2.
\end{equation}
Также, как и любое коническое сечение, гипербола имеет своё {\itshape оптическое свойство}: свет от источника, находящегося в одном из фокусов гиперболы, отражается второй ветвью гиперболы таким образом, что продолжения отраженных лучей пересекаются во втором фокусе.

	\newpage
\section{Астрофизика}
\subsection{Звёздные величины}
Звёздная величина~--- безразмерная числовая характеристика яркости объекта. Известно, что увеличению светового потока в 100 раз соответствует уменьшение видимой звёздной величины ровно на 5 единиц. Тогда уменьшение звёздной величины на одну единицу означает увеличение светового потока в $\sqrt[5]{100}\approx 2.512$~раз, то есть звёздные величины являются логарифмической шкалой измерения плотности потока. Зависимость, связывающая отношение освещённостей $E_1$ и $E_2$ и разность звёздных величин $m_1$ и $m_2$ двух объектов, называется \term{формулой Погсона} и имеет вид
\begin{equation}
	\frac{E_1}{E_2} = 10^{0.4(m_2 - m_1)} \quad \Longleftrightarrow \quad m_2 = m_1 + 2.5 \lg \frac{E_1}{E_2}.
	\label{eq:Pogson-law}
\end{equation}
Широко используется понятие \term{абсолютной звёздной величины} $M$~--- это видимая звёздная величина $m$ при наблюдении с установленного расстояния: для звёзд~---~10~пк, для тел Солнечной системы~---~1~\au, причем считается, что тело находится в 1~\au~и от наблюдателя и от Солнца, а фаза равна единице, то есть можно считать, что наблюдатель находится в центре Солнца, а~тело~--- в~1~\au~от него. 

Кроме этого, важно понятие \term{болометрической звёздной величины} $m_\text{bol}$~--- это звёздная величина, при расчёте которой учитывается полная мощность излучения источника во всех диапазонах электромагнитных волн. Обычная (видимая) звёздная величина учитывает излучение лишь в видимой части спектра от примерно 380~нм до примерно~780~нм. Разность между болометрической и видимой звёздными величинами называется \term{болометрической поправкой} ($BC$), которая отличается для разных спектральных классов звёзд. Из определения, болометрическая поправка может быть найдена по формуле
\begin{equation}
	BC = m_\text{bol} - m.
\end{equation}
Абсолютную звёздную величину звезды можно получить по формуле Погсона \eqref{eq:Pogson-law} из видимой звёздной величины $m$ и расстояния $r$ до неё в парсеках
\begin{equation}
	M = m + 2.5 \lg \frac{E}{E_\text{абс}} = m + 2.5 \lg \frac{(10~\text{пк})^2}{r^2} = m + 5 - 5\lg r.
	\label{eq:abs-mag}
\end{equation} 
Если принимать к рассмотрению межзвездное поглощение $A$, то формулу  \eqref{eq:abs-mag} необходимо уточнить:
\begin{equation}
	M = m + 5 - 5\lg r - Ar.
\end{equation}
\subsection{Закон Стефана-Больцмана}
\term{Закон Стефана~--- Больцмана} определяет зависимость плотности мощности излучения абсолютно чёрного тела (АЧТ) $u$ от его температуры $T$:
\begin{equation}
u = a T^4,
\end{equation} 
где $a$~--- некая универсальная константа.
Отсюда полная светимость АЧТ с площадью поверхности $S$
	\begin{equation}
	L = S \sigma T^4,
	\label{eq:steff-bol-law}
\end{equation}
константа $\sigma$ называется \term{постоянной Стефана-Больцмана}.
  
Важно отметить, что \imp{закон Стефана-Больцмана}~--- прямое следствие формулы Планка \eqref{Planck's formula}, так как
\begin{equation}
	\sigma T^4 = \int\limits^\infty_0 B(\lambda, T)\,d\lambda \int\limits_0^{\pi/2} \sin \varphi\, d\varphi \int\limits_0^{2\pi} \cos \varphi\, d\theta = \pi \int\limits^\infty_0 B(\lambda, T)\,d\lambda,
\end{equation}
откуда $\sigma = (2\pi^5k^4)/(15c^2h^3) = 5.67 \cdot 10^{-8}~\text{Вт}/(\text{м}^2\cdot \text{К}^4)$.

%Для АЧТ сферической формы с радиусом $R$ формула~\eqref{eq:steff-bol-law} принимает вид
%\begin{equation}
%L=4\pi R^2\sigma T^4.
%\end{equation}
Для звёзд главной последовательности выполняется соотношение $L \sim M^{\alpha}$, где~$\alpha$~--- коэффициент пропорциональности, который зависит от массы звезды следующим образом:
\begin{align*}
\alpha &= 2.5, \quad M < 0.43 M_\odot; & 
\alpha &= 4, \quad 0.43 M_\odot < M < 2 M_\odot;\\ 
\alpha &= 3.2, \quad 2 M_\odot < M < 20 M_\odot; & 
\alpha &= 1, \quad M > 20 M_\odot.
\end{align*}
Также существует примерная зависимость светимости звёзды от её радиуса, имеющая вид  $L\sim R^{5.2}$.
\subsection{Энергия излучения}
\term{Энергия излучения}~--- энергия, переносимая излучением ($Q_e$).\\
\term{Поток излучения}~--- физическая величина, характеризующая мощность, переносимую излучением,
\begin{equation}
 \Phi_e = \frac{d Q_e}{dt}.
\end{equation}
\imp{Теорема Гаусса}: через любую замкнутую поверхность потоки от одинаковых источников равны.

\term{Спектральная плотность потока излучения}~--- поток излучения, приходящийся на малый единичный интервал спектра,
\begin{equation}
\Phi_{e, \lambda}(\lambda) = \frac{d\Phi_e(\lambda)}{d\lambda}, \quad\quad \Phi_{e, \nu}(\nu) = \frac{d\Phi_e(\nu)}{d\nu} =  \frac{\lambda^2}{c}\Phi_{e, \lambda}(\lambda).
\end{equation}

\term{Объемная плотность энергии излучения}~--- количество энергии на единицу объема
\begin{equation}
U_e = \frac{d Q_e}{dV}.
\end{equation}

\term{Светимость}~--- величина, представляющая собой световой поток излучения, испускаемого с малого участка светящейся поверхности единичной площади,
\begin{equation}
M_e = \frac{d \Phi_e}{dS_1},
\end{equation}
здесь $S_1$~--- площадь объекта, испускающего энергию.

\term{Яркость}~--- световой поток, приходящийся на единичный телесный угол, в расчёте на единичную площадку проекции излучающей поверхности на картинную плоскость, 
\begin{equation}
L_e = \frac{d^2 \Phi_e}{d \Omega\,dS_1 \cos \varepsilon},
\end{equation}
где $\varepsilon$~--- угол между направлением потока излучения и нормалью к плоскости излучающей поверхности.

\term{Интегральная яркость}~--- интеграл яркости по видимой поверхности источника. Показывает количество энергии, пришедшее от источника за единицу времени.
\begin{equation}
\Lambda_e = \int \limits_S L_e(\vec{r})\,ds.
\end{equation}
\term{Освещенность}~--- величина, равная отношению светового потока, падающего на малый участок поверхности, к его площади~--- поверхностная плотность потока
\begin{equation}
E_e = \frac{d\Phi_e}{dS_2} \sim \frac{1}{r^2},
\end{equation}
здесь $S_2$~--- площадь поверхности приёмника, $r$~--- расстояние от источника.
\input{sections/astrophys.flux-albedo.tex}
\input{sections/astrophys.photon.tex}
\input{sections/astrophys.energy-lines.tex}
\subsection{Формула Планка}
\label{sec:planck-law}
\term{Формула Планка}~--- выражение для спектральной плотности мощности излучения абсолютно чёрного тела на интервале частот $[\nu, \nu + d \nu)$, распространяющейся с телесном угле $d\Omega$, которое было получено Максом Планком в 1900~году. Данное выражение имеет следующий вид:
\begin{equation}
B_\nu(\nu,T)=\frac{2h\nu^3}{c^2}\cdot \frac{1}{\exp\left(\frac{h\nu}{kT}\right)-1} = \left[ \frac{\text{Вт}}{\text{м}^2 \cdot \text{Гц} \cdot \text{ср}}\right],
\label{eq:plancks-law-nu}
\end{equation}
где $\nu$~--- частота излучения, $T$~--- температура АЧТ, $h$~--- постоянная Планка, $k$~--- постоянная Больцмана, $c$~--- скорость света.

Если записать закон излучения Планка \eqref{eq:plancks-law-nu} для длин волн, то
\begin{equation}
B_\lambda(\lambda,T)=\frac{2hc^2}{\lambda^5} \cdot \frac{1}{\exp\left(\frac{hc}{\lambda kT}\right)-1} = \left[ \frac{\text{Вт}}{\text{м}^3 \cdot \text{ср}}\right].
\label{eq:plancks-law-lambda}
\end{equation}
\begin{wrapfigure}[15]{l}{.6\tw}
\centering
\vspace{-.9pc}
 \begin{tikzpicture}
  \begin{axis}[
  				width 	=	.6\tw, 
				height	=	6cm, 
  				ymax	=	1e+14,
  				xmax	=	2000,
  				xmin	=	0,
  				ymin	=	0,
				xlabel	=	{Длина волны $\lambda$,~нм}, 
				ylabel 	= 	{$B_\lambda(\lambda, T)$,~$\text{Вт} \cdot \text{м}^{-3}$}
]
   \addplot+[dashed, thin, black] table[x=l, y=tl] {data/planck.txt};
   \addplot+[black] table[x=l, y=t4] {data/planck.txt} node at (axis cs:870, 1.6e+13) {\tiny{$4500$~K}};
   \addplot+[black] table[x=l, y=t5] {data/planck.txt}node at (axis cs:750, 4.2e+13) {\tiny{$5000$~K}};
   \addplot+[black] table[x=l, y=t58] {data/planck.txt}node at (axis cs:670, 8.5e+13) {\tiny{$5800$~K}};
   \addplot+[black] table[x=l, y=t7] {data/planck.txt}node at (axis cs:1350, 3.5e+13) {\tiny{$7000$~K}};
	%\addplot+[black, smooth] table[x=l, y=t15] {data/planck.txt} node at (axis cs:1670, 5.5e+13) {\tiny{$15000$~K}};
  \end{axis}
 \end{tikzpicture}
\caption{Кривые спектральной плотности мощности изотропного излучения АЧТ с разной температурой}\label{pic:wien-law}
\end{wrapfigure}
Стоит заметить, что при переходе в функции к длинам волн меняется не только частота на длину волны, но и выражение для интервала. 

Формула Планка появилась, когда стало ясно, что формула Рэлея-Джинса удовлетворительно описывает излучение только в области больших длин волн, а~с~убыванием длин волн даёт сильные расхождения с реальными данными. Однако формулу Рэлея-Джинса используют и сейчас для описания кривой Планка на больших длинах волн. 

\change{
Проделаем обратные действия: получим формулу Рэлея-Джинса из формулы Планка. Длинноволновая часть спектра характеризуется соотношением $h\nu \ll kT$, то есть 
\begin{equation*}
	\exp\left( \frac{h\nu}{kT}\right) \approx 1 + \frac{h\nu}{kT}.
\end{equation*}
Подставляя полученное выражение в знаменатель \eqref{eq:plancks-law-nu}, получим
\begin{equation*}
	B_\nu(\nu,T) \approx \frac{2h\nu^3}{c^2}\cdot \frac{1}{1 + \frac{h\nu}{kT} - 1} = \frac{2h\nu^3 }{c^2}\cdot \frac{k T}{ h \nu} = \frac{2 \nu^2 k T}{c^2}.
\end{equation*}
}
\change{
	Проделав то же самое для выражения через длину волны, получим:
}
\begin{equation}
	B(\lambda, T) \simeq \frac{2 c k T}{\lambda^4}, \quad\quad B(\nu, T) \simeq \frac{2 \nu^2 k T}{c^2}.
\label{Ray-Jean}
\end{equation}

\change{
	В коротковолновой области, наоборот, $h \nu \gg kT$, следовательно, в знаменателе формулы Планка единица много меньше стоящей там экспоненты, то есть
	\begin{equation*}
		\frac{1}{\exp\left(\frac{h\nu}{kT}\right)-1} \approx \frac{1}{\exp\left(\frac{h\nu}{kT}\right)} = \exp\left(-\frac{h\nu}{kT}\right).
	\end{equation*} 
	Отсюда получаются выражения, называемые приближением Вина:
}
\begin{equation}
B ( \lambda, T) \simeq \frac{2 h c^2}{\lambda^5} \exp \left( -\frac{h c}{\lambda k T}\right), \quad \quad B( \nu, T ) \simeq \frac{2 h \nu^3}{c^2} \exp \left( -\frac{h \nu}{k T} \right).
\end{equation}
\subsection{Закон смещения Вина}
\term{Закон смещения Вина} --- закон, устанавливающий зависимость длины волны~$\lambda_\text{макс}$, на которой спектральная плотность излучения $B_\lambda(\lambda, T)$ абсолютно чёрного тела достигает своего максимума, от температуры $T$ этого тела:
\begin{equation}
	\lambda_\text{макс} \approx \frac{b}{T} \equiv \frac{0.0029~\text{м} \cdot \text{К}}{T}.
\end{equation}
Закон является следствием исследования функции Планка (см.~\ref{sec:planck-law}) на экстремальность.
\subsection{Эффект Доплера. Красное смещение}
\term{Эффект Доплера}~--- эффект изменения частоты и длины волны электромагнитного излучения, регистрируемого приёмником, вызванный относительным движением источника и приёмника (см.~Рис.\,\ref{doppler-ef}).

При $\Delta \lambda \ll \lambda_0$ с большой точностью выполняется следующее важное соотношение:\begin{equation}
\beta \equiv \dfrac{v}{c} = \dfrac{\lambda - \lambda_0}{\lambda_0} \equiv \dfrac{\Delta \lambda}{\lambda_0},
\label{eq:dopler-ef-simple}
\end{equation}
\begin{wrapfigure}[6]{r}{0.5\tw}
\centering
\vspace{-.5pc}
\includegraphics[width=.5\tw]{doppler-ef}
\caption{Эффект Доплера}
\label{doppler-ef}
\end{wrapfigure}
где $\lambda_0$~--- лабораторная длина волны излучения источника, а $\lambda$~--- наблюдаемая. В действительности же имеет место более общий случай: \imp{релятивистский эффект Доплера}, обусловленный проявлением СТО при $v \simeq c$, для которого формула~\eqref{eq:dopler-ef-simple} усложняется и принимает вид
\begin{equation}
\nu = \nu_0 \cdot \dfrac{\sqrt{1 - \beta^2}}{1 + \beta \cdot \cos\theta},
\label{eq:dopler-ef-rel}
\end{equation}
где $\nu$~--- частота, с которой наблюдатель принимает волны, $\nu_0$~--- частота, с которой источник испускает волны, $v$~--- скорость источника, $\theta$~--- угол между направлением на источник и вектором его скорости в системе отсчёта приёмника. Если источник радиально удаляется от наблюдателя, то $\theta = 0$, если приближается, то $\theta =\pi$. Важно, что~\eqref{eq:dopler-ef-simple} напрямую следует из \eqref{eq:dopler-ef-rel} при $\beta  \ll 1$.

\term{Красное смещение}~--- явление сдвига спектральных линий химических элементов в красную (длинноволновую) сторону, обусловленное относительным движение объектов. Параметр красного смещения определяется из наблюдаемой и лабораторной длин волн как
\begin{equation}
z = \dfrac{\lambda - \lambda_0}{\lambda_0}.
\end{equation}

Доплеровское смещение длины волны в спектре источника, движущегося с лучевой скоростью $v_{r}$ и полной скоростью $v$,
\begin{equation}
z = \dfrac{1 + v_r / c}{\sqrt{1 - \beta^2}}.
\end{equation}

\term{Гравитационное красное смещение}~--- проявление эффекта изменения частоты излучения, испущенного массивным объектом, таким как звезда или чёрная дыра. Наблюдается как сдвиг спектральных линий в спектре источника в красную область спектра. Гравитационное красное смещение определяется из формулы, выведенной Эйнштейном,
\begin{equation}
z_G=\dfrac{GM}{c^2 R}-\dfrac{GM}{c^2 r},
\label{eq:grav-red-shift}
\end{equation}
где $M$~--- масса гравитирующего тела, $R$~--- радиальное расстояние от центра масс тела до точки излучения (радиус источника), $r$~---  радиальное расстояние от центра масс источника до точки наблюдения. В случае, когда наблюдатель находится от источника много дальше его радиуса, т.\,е. выполняется соотношение $r \gg R$, выражение~\eqref{eq:grav-red-shift} можно упростить до
\begin{equation}
z_G \simeq \dfrac{GM}{c^2 R}.
\end{equation}

\input{sections/astrophys.light-pressure.tex}
\input{sections/astrophys.edd.tex}
\input{sections/astrophys.grav-lens.tex}
\subsection{Закон Хаббла}
\term{Закон Хаббла}~--- эмпирический закон, связывающий скорость удаления галактик $V$ и расстояние $R$ до них линейным образом: 
\begin{equation}
	V = H R,
\end{equation}
величина $H=68~\text{км/c} \cdot \text{Мпк})$ называется \imp{постоянной Хаббла}.

При $v \ll c$ можно использовать приближение эффекта Доплера, тогда
\begin{equation}
	V = c z.
\label{eq:hubble-speed}
\end{equation}

Равенство \eqref{eq:hubble-speed} справедливо только при $z \ll 1$, а при б\'{o}льших значениях $z$ космологическое красное смещение нльзя связывать с эффектом Доплера, поэтому можно пользоваться только формулой 
\begin{equation}
	\frac{dz}{dt} = - H(z)(1+z),
\end{equation}
где постоянная Хаббла введена как функция красного смещения.
\subsection{Шкала электромагнитных волн}


\term{Гамма излучение} возникает при радиоактивных распадах ядер, при торможении электронов энергией более $10^5$~эВ и при других взаимодействиях элементарных частиц. Используются в гамма-дефектоскопии, при изучении свойств вещества.

\term{Рентгеновские лучи} излучаются при большом ускорении электронов, например при их торможении в металлах. Получают их при помощи рентгеновской трубки: электроны в вакуумной трубке ускоряются электрическим полем при высоком напряжении, достигая анода, при со­ударении резко тормозятся. При торможении электроны движут­ся с ускорением и излучают электромагнитные волны с малой длиной. 

\begin{figure}[!h]
\centering
\includegraphics[width = 1\textwidth]{scale-wave.pdf}
\caption{Шкала электромагнитных волн}
\end{figure}
\term{Ультрафиолетовые лучи}~--- излучение Солнца, ртутных ламп и т.\,п. Используются в ультрафиолетовой микроскопии, в медицине.

\term{Видимое излучение}~--- часть электромагнитного излучения, воспринимаемая глазом (от фиолетового до от красного).

\term{Инфракрасное излучение}~--- тепловое, излучается любым нагретым телом.

\term{Радиоволны} используются повсеместно в обычной жизни, это и сотовая связь, и радиолокация, и спутниковая связь, и Wi-Fi и многое другое.

\term{Низкочастотные волны}~--- диапазон, традиционно используемый в электротехнике. В промышленной электроэнергетике используется частота 50~Гц, на~которой осуществляется передача электрической энергии по линиям и преобразование напряжений трансформаторными устройствами.
\input{sections/astrophys.spec-theor-rel.tex}
\subsection{Оптическая толщина. Закон Бугера}
\term{Оптическая толщина}~--- безразмерная величина, характеризующая степень непрозрачности среды для проходящего сквозь неё излучения,
\begin{equation}
\tau = \int n(x) \sigma(x)\,dx,
\end{equation}
где $\tau$~--- оптическая толщина среды, $n$~--- концентрация частиц, $\sigma$~--- сечение их взаимодействия.

Поток $I_0$ на входе связан с потоком $I$ на выходе \term{Законом Бугера}:
\begin{equation}
I = I_0 e^{-\tau}.
\end{equation}
\input{sections/astrophys.colour.tex}
\input{sections/astrophys.mkt.tex}
\input{sections/astrophys.earth-atmosphere.tex}

	\newpage
\section{Космология}
Будем считать, что наша Вселенная не содержит каких-либо выделенных областей
 и направлений. Её глобальные характеристики одинаковы во всех точках
 пространства в фиксированный момент времени. Это составляет суть
 \term{космологического принципа}. Он подтверждается наблюдениями: распределение
 материи во Вселенной на масштабах более~$100$\,Mpc можно считать однородным
 и изотропным, а относительные флуктуации температуры реликтового фона не
 превышают $ 10^{-5}$. Также введём следующие обозначения: величины без индексов 
 или с индексом <<e>> (от англ. \textit{emit}) обозначают наблюдаемый объект, а с 
 индексами <<$0$>> или <<o>> (от англ. \textit{observe})~--- наблюдателя.
\subsection{Кинематика}
Эволюция Вселенной соответствует \imp{космологическому принципу}, если из состояния Вселенной в один момент времени, можно получить её состояние в любой другой момент путем только лишь масштабирования. Соответственно, расстояние $D$ между двумя несвязанными объектами будет меняться по следующему закону: 
\begin{equation}
D(t) = a(t)D_0,
\label{eq}
\end{equation}
 где $a(t)$~--- масштабный фактор, $D_0$~--- а расстояние между этими объектами сейчас.

\term{Масштабный фактор}~--- это безразмерная величина, характеризующая эволюцию Вселенной. Во всех точках Вселенной в заданный момент времени масштабный фактор одинаков. В момент большого взрыва масштабный фактор принимают равным нулю, а единица соответствует текущему моменту.

Продифференциируем (\ref{eq}) по времени и разделим на себя:
\begin{equation}
\frac{1}{D} \frac{dD}{dt} = \frac{1}{aD_0} \frac{D_0 \,d a}{dt} = \frac{\dot{a}}{a} \equiv H(t).
\label{eq2}
\end{equation}
Таким образом, мы получили величину, не зависящую от расстояния, и изменяющуюся только со временем. Параметр $H(t)$ называется \term{пара\-мет\-ром Хаббла-Леметра}. А его значение сейчас, т.\,е. на данный момент~$t_0$~--- \term{постоян\-ной Хаббла}, которая равна
\begin{equation}
    H_0 = H(t_0) \approx 67...\,\frac{\text{км}}{\text{c} \cdot \text{Мпк}}.
\end{equation} 
По сути, этот параметр определяет наклон касательной к графику $a(t)$ в какой-либо момент времени, то есть показывает, расширяется Вселенная, или нет. Вторая производная \imp{масштабного фактора} по времени  определяется \term{параметром замедления} $q = -\ddot{a}a \slash \dot{a}^2$:
\begin{equation}
\frac{\ddot{a}}{a} = -qH^2.
\label{eq3}
\end{equation}
Из (\ref{eq3}) видно, что при отрицательном значении $q$ Вселенная эволюционирует с ускорением, а при положительном~--- с замедлением. Судя по наблюдениям, наша Вселенная ускоренно расширяется.

Так как выделенного направления в расширении Вселенной нет, то при отсутствии пекулярных скоростей, все объекты, подверженные расширению будут удаляться от наблюдателя радиально. Напрямую из~(\ref{eq}) следует $\dot{D} \slash D = \dot{a} \slash a = H$, то есть \term{закон Хаббла}:
\begin{equation}
v_r = HD
\label{hubble}
\end{equation}
Запишем \imp{эффект Доплера}:
\begin{equation}
\frac{\lambda_\text{o} - \lambda_\text{e}}{\lambda_\text{e}} = z \simeq \frac{v_r}{c}.
\end{equation}
Здесь $\lambda_\text{o}$~--- длина волны принимаемого света, а $\lambda_\text{e}$~--- излучаемого. Стоит заметить, что \term{красное смещение} $z$ может быть больше $1$, а последнее равенство выполняется только для $v_r \ll c.$

Рассмотрим теперь объект на небольшом расстоянии $D = cdt$ от наблюдателя таком, что $v_r \slash c = d \lambda \slash \lambda \ll 1$. Используя \imp{закон Хаббла}, получаем:
\begin{equation}
\frac{v_r}{c} = \frac{d \lambda}{\lambda} = \frac{D \dot{a}}{ca} = dt \frac{\dot{a}}{a} = \frac{da}{a}.
\end{equation}
\begin{equation}
\int_{\lambda_\text{e}}^{\lambda_\text{o}} \frac{d \lambda}{\lambda} = \int_{a}^{1} \frac{da}{a} \Rightarrow \ln \lambda_\text{o} - \ln \lambda_\text{e} = \ln 1 - \ln a \Rightarrow \frac{\lambda_\text{o}}{\lambda_\text{e}} = \frac{1}{a}
\label{dopler}
\end{equation}
%\ln \frac{\lambda_\text{o}}{\lambda_\text{e}} = \ln \frac{1}{a} \Rightarrow 
Чтобы получить связь между \imp{масштабным фактором} и \imp{красным смещением}, используем \imp{эффект Доплера}. Из него следует равенство $\lambda_{\text{o}} / \lambda_{\text{e}} = z+1.$ После подстановки в \eqref{dopler} получаем зависимость $a(z)$:
\begin{equation}
a = (1+z)^{-1}
\label{zf}
\end{equation}
Последнее уравнение очень важно в современной космологии. Оно показывает зависимомть \imp{масштабного фактора}, величины, не имеющей явного физического смысла, от \imp{красного смещения}, которое можно просто измерить. \newline \linebreak
%Определим некоторые важные размерные константы:
%\begin{enumerate}
%\item \imp{Постоянная Хаббла} $H_0 \approx 67 \text{ km} \text{ s}^{-1} \text{Mpc}^{-1} \approx 2.26 \cdot 10^{-18} \text{ s}^{-1}$
%\item \imp{Хаббловское время} сейчас $t_{H_0} = H_{0}^{-1} \approx 14.0 \text{ Gyr}$
%\item \imp{Хабблосвское расстояние} сейчас $D_{H_0} = \displaystyle \frac{c}{H_0} \approx 4.28 \text{ Gpc}$
%\end{enumerate}
Теперь продифференциируем \text{(\ref{zf}) }по времени:
\begin{equation}
\frac{da}{dt} = - \frac{1}{(1+z)^{2}} \frac{dz}{dt} = - \frac{a}{1+z} \frac{dz}{dt},
\label{dadz}
\end{equation}
разделим на $a$:
\begin{equation}
H = \frac{1}{a} \frac{da}{dt} = \frac{d\ln (a)}{dt} = - \frac{1}{1+z} \frac{dz}{dt},
\end{equation}

\subsection{Динамика}
В этом разделе будут приведены \imp{иллюстрации} выводов некоторых основных уравнений в динамике эволюции плоской, однородной и изотропной Вселенной, заполненной материей, излучением (включая релятивистские нейтрино и т. п.) и тёмной энергией.
\subsubsection{Вселенная, заполненная материей.}
Выделим небольшую сферу радиусом $R(t) = R_0 a(t) \ll D_{\text{H}_0}$, массой $M$ и средней плотностью $\rho = M / \frac{4}{3} \pi R^3 $. В силу изотропности Вселенной пространство вокруг выбранной нами области гравитационно на неё не действует. Запишем выражение для ускорения свободного падения на границе этой сферы. По закону Всемирного тяготения
\begin{equation}
g = \ddot{R} = -\frac{GM}{R^2} = -\frac{4 \pi G \rho R^3}{3 R^2} = -\frac{4 \pi}{3} G \rho R.
\label{dmov}
\end{equation}
Таким образом мы получили \term{уравнение движения} для нашей Вселенной. 
%Для удобства использования перепишем его в другой форме:
%\begin{equation}
%r_0 \ddot{a} = -\frac{4 \pi G \rho_0 r_{0}^{3}}{3r_{0}^{2} a^2} = -\frac{4 \pi G \rho_0 r_{0}}{3 a^2}.
%\end{equation}
%Разделив уравнение на $r_0$, получаем:
%\begin{equation}
%\ddot{a} = -\frac{4 \pi G \rho_0}{3} a^{-2},
%\label{mov}
%\end{equation}
%где $\rho_0$~--- плотность Вселенной на данный момент. Один раз проинтегрируем (\ref{mov}). 
%$$
%\ddot{a} = \frac{d}{dt} \dot{a} = \frac{d \dot{a}}{da} \frac{da}{dt} = \frac{d \dot{a}}{da} \dot{a}
%$$
%Подставим эту замену в (\ref{mov}):
%\begin{equation}
%\dot{a} \frac{d \dot{a}}{da} = -\frac{4 \pi G \rho_0}{3} a^{-2}
%\end{equation}
%%В итоге:
%После интегррования получаем
%%$$
%%\int \dot{a} d \dot{a} = -\frac{4 \pi G \rho_0}{3} \int a^{-2} da
%%$$
%%$$
%%\frac{1}{2} \dot{a}^2 = -\frac{4 \pi G \rho_0}{3} \frac{a^{-1}}{-1} + C
%%$$
%\begin{equation}
%\frac{1}{2} \dot{a}^2 - \frac{4 \pi G \rho_0}{3} a^{-1} = C
%\end{equation}
%Это выражение по своему виду является законом сохранения энергии. Определим теперь ыид константы $C$. 
Проинетгрируем его один раз.
\begin{equation}
\ddot{R} = \frac{d}{dt} \dot{R} = \frac{d \dot{R}}{dR} \frac{dR}{dt} = \frac{d \dot{R}}{dR} \dot{R}
\label{change}
\end{equation}
Подставим эту замену в (\ref{dmov}):
\begin{equation}
\dot{R} \frac{d \dot{R}}{dR} = -\frac{GM}{R^2}
\end{equation}
%В итоге:
После интегрирования получаем
%$$
%\int \dot{a} d \dot{a} = -\frac{4 \pi G \rho_0}{3} \int a^{-2} da
%$$
%$$
%\frac{1}{2} \dot{a}^2 = -\frac{4 \pi G \rho_0}{3} \frac{a^{-1}}{-1} + C
%$$
\begin{equation}
\frac{1}{2} \dot{R}^2 - \frac{GM}{R} = C
\label{menerg}
\end{equation}
Это выражение по своему виду является законом сохранения энергии. Определим теперь вид константы $C$. Будем считать известными значения $H_0, R_0$ и $\rho_0$ в данный момент времени $t_0$. По \imp{закону Хаббла} $(dR / dt)_{t = t_0} = R_0H_0$. Также $M = \frac{4}{3} \pi R_{0}^{3} \rho_0$. Подставим эти выражения в (\ref{menerg}):
\begin{multline}
C = \frac{1}{2} \left(\frac{dR}{dt}\right)_{t = t_0} - \frac{4 \pi G \rho_0 R_{0}^{3}}{R_0}  =\\ 
= \frac{1}{2} R_{0}^{2} H_{0}^{2} - \frac{4 \pi G}{3} \rho_0 R_{0}^{2} = \frac{1}{2} R_{0}^{2}H_{0}^{2} \left( 1 - \frac{8 \pi G \rho_0}{3 H_{0}^{2}} \right)
\end{multline}
Соотношение $3H^2 / 8 \pi G = \rho_c$ называется \term{критической плотностью} Вселенной. Перепишем выражение для константы в более простом виде:
\begin{equation}
C = \frac{R_{0}^{2} H_{0}^{2}}{2} \left(1 - \frac{\rho_0}{\rho_{c,0}} \right).
\label{const}
\end{equation}
Тогда уравнение (\ref{menerg}) принимает следующий вид:
\begin{equation}
\dot{R}^2 = \frac{8 \pi G R_{0}^{3} \rho_0}{3R} + R_{0}^{2} H_{0}^{2} \left(1 - \frac{\rho_0}{\rho_{c,0}} \right).
\label{menerg1}
\end{equation}

Наиболее популярным и простым является сценарий развития Вселенной, при котором её плотность равна критической. В этом случае $C \equiv 0$ и уравнение (\ref{menerg1}) принимает вид
\begin{equation}
\frac{dR}{dt} = \sqrt{\frac{8 \pi G R_{0}^{3} \rho_0}{3R}} = H_0 \sqrt{\frac{8 \pi G R_{0}^{3} \rho_0}{3 H_{0}^{2} R}} = H_0 \sqrt{\frac{R_{0}^{3} \rho_0}{R \rho_{c,0}}} = H_0 \sqrt{\frac{R_{0}^{3}}{R}} \, .
\end{equation}
Решая его, получаем
\begin{equation}
\left(\frac{R}{R_0}\right)^{3/2} = a^{3/2} = \frac{3}{2} H_0 t; \quad t_0 = \frac{2}{3 H_0},
\end{equation}
Где $t$~--- возраст Вселенной на \imp{масштабном факторе} $a,$ а $t_0$~--- возраст Вселенной сейчас.

\imp{Постоянная Хаббла} в этом случае эволюционирует по следующему закону:
\begin{equation}
H(t) = \frac{\dot{R}}{R} = H_0 \sqrt{\frac{R_{0}^{3}}{R}} \cdot \frac{1}{R} = H_0 \left(\frac{R_{0}}{R}\right)^{3/2} = \frac{2}{3t},
\end{equation}
а плотность вещества, равная критической:
\begin{equation}
\rho(t) = \rho_c = \frac{3 H^2}{8 \pi G} = \frac{1}{6 \pi G t^2}.
\end{equation}
%\frac{3}{8 \pi G} \frac{4}{9 t^2} =
%Запишем \imp{закон сохранения энергии} в расчёте на единицу массы для этой сферы (равенство нулю достигается в выбранной нами модели):
%\begin{equation}
%\frac{\dot{r}^2}{2} - \frac{GM}{r} = E = 0
%\end{equation}
%Так как $r = r_0 a(t)$, а $M = \frac{4}{3} \pi r^3 \rho$, (2.1) принимает следующий вид:
%\begin{equation}
%\frac{\dot{a}^2}{2} - \frac{4 \pi}{3} G a^2 \rho = 0
%\end{equation}
%Умножим обе части (2.2) на $2 \slash a^2$ и выразим плотность:
%\begin{equation}
%\rho = \rho_c = \frac{3}{8 \pi G} \left(\frac{\dot{a}}{a}\right)^2 = \frac{3 H^2}{8 \pi G}
%\end{equation}
%Полученное нами выражение

\subsubsection{Учёт давления. Тёмная энергия.}
Пусть теперь помимо холодной материи в нашей Вселенной есть излучение и релятивистские частицы плотностью энергии $\varepsilon_r$. Будем рассматривать их как фотонный газ. Выполняется следующее уравнение:
\begin{equation}
\varepsilon = \rho c^2,
\label{albert}
\end{equation}
Где $\rho$~--- плотность, выраженная в $kg/m^3$ (плотность массы), а $c$~--- скорость света. Если излучение находится в термодинамическом равновесии, то $\varepsilon_r = $. Давление фотонного газа задается формулой:
$$
P = \frac{\varepsilon_r}{3} = \frac{\rho_r c^2}{3}
$$
Добавим плотность излучения к плотности вещества в (\ref{dmov}) и получим \imp{уравнение движения} для Вселенной с излучением:
\begin{equation}
\ddot{R} = -\frac{4 \pi}{3} G R \left( \rho + \frac{3P}{c^2} \right).
\label{rmov}
\end{equation}
Оно совпадает с \imp{уравнением движения} из ОТО. Здесь стоит оговориться, что излучение в силу \eqref{rmov} вносит вклад как в плотность массы, так и в давление, а \imp{холодная} материя давления не оказывает. Добавим теперь в нашу Вселенную тёмную энергию.

\term{Космологическая постоянная} $\Lambda$, отвечающая за тёмную энергию, была введена А.\,Эйнштейном в уравнения ОТО для получения статичной Вселенной при их решении. Однако после решения уравнений А.\,Фридманом оказалось, что с \imp{космологической постоянной} возможны как статические, так и нестатические решения.

Из уравнений ОТО следует, что плотность \imp{тёмной энергии} постоянна:
\begin{equation}
\varepsilon_{\Lambda} = \frac{c^4 \Lambda}{8 \pi G},
\end{equation}
а давление
\begin{equation}
P_{\Lambda} = -\varepsilon_{\Lambda} = -\frac{c^4 \Lambda}{8 \pi G}.
\end{equation}
Теперь окончательно в \eqref{rmov} имеем:
$$
\rho = \rho_m + \rho_r + \rho_{\Lambda}; \quad P = P_r + P_{\Lambda} = \frac{\varepsilon_r}{3} - \varepsilon_{\Lambda}
$$
Преобразуем скобку в \eqref{rmov}:
$$
\rho + \frac{3P}{c^2} = \rho_m + \rho_r + \rho_{\Lambda} + \frac{\varepsilon_r}{c^2} - 3\frac{\varepsilon_{\Lambda}}{c^2} = \rho_m + 2\rho_r - 2\rho_{\Lambda}
$$
После подстановки в \imp{уравнение движения} получаем:
\begin{multline}
\ddot{R} = -\frac{4 \pi}{3} G R (\rho_m + 2\rho_r - 2\rho_{\Lambda}) = \\
= -\frac{4 \pi}{3} G R \left(\rho_m + 2\rho_r - \frac{c^2 \Lambda}{4 \pi G}\right) =
\\ = -\frac{4 \pi}{3} G R \left(\rho_m + 2\rho_r\right) + \frac{\Lambda c^2}{3} R.
\label{imov}
\end{multline}
Данный вид \imp{уравнения движения} наиболее удобен для интегрирования. Обычно под $\rho$ и $P$ подразумевают вклады только материи и излучения в плотность и давленние соответственно. Учитывая, что $R(t) = a(t) R_0$, перепишем \eqref{imov} в наиболее известном виде:
\begin{equation}
\frac{\ddot{a}}{a} = -\frac{4 \pi G}{3} \left( \rho + \frac{3P}{c^2} \right) + \frac{\Lambda c^2}{3}.
\label{mov}
\end{equation}

Для интегрирования уравнения \eqref{imov} нам необходимо знать, как плотности материи и излучения эволюционируют с изменением \imp{масштабного фактора}, или размера выделенной области.

Сначала рассмотрим эволюцию плотности материи. Так как масса $M$, заключённая внутри области радиса $R$ всегда остаётся внутри него, плотность этой области будет падать как куб радиуса:
\begin{equation}
\rho_{m} = \frac{3M}{4 \pi} R^{-3} = \frac{3M}{4 \pi R_{0}^{3}} a^{-3} = a^{-3} \rho_{m,0}
\label{mden}
\end{equation}
Плотность излучения эволюционирует по следующему закону:
\begin{equation}
\rho_r = a^{-4} \rho_{r,0}.
\label{rden}
\end{equation}
Вид (\ref{rden}) можно объяснить эффектом Доплера, который испытывают фотоны, догоняя удаляющегося наблюдателя. Пространственная концентрация фотонов в выделенной области будет падать, как и плотность материи, обратно пропорционально объёму области. Также из-за эффекта Доплера (см. \ref{dopler}), энергия каждого фотона будет уменьшаться обратно пропорционально \imp{масштабному фактору}: $\epsilon = hc/\lambda \sim a^{-1}$. Совокупность этих эффектов даёт нам \eqref{rden}.

Получить это выражение более формально можно, рассматривая излучение как идеальный газ, расширяющийся адиабатически. Запишем его работу:
\begin{equation}
dA_r = d\varepsilon_r V = -P dV = -\frac{\varepsilon_r}{3} dV.
\label{work}
\end{equation}
Раскроем $d\varepsilon_r V$ как приращение произведения двух функций:
$$
d\varepsilon_r V = \varepsilon_r dV + Vd\varepsilon_r,
$$
подставим в \eqref{work}:
%$$
%udV + Vdu = -\frac{u}{3} dV
%$$
$$
Vd\varepsilon_r = -\frac{\varepsilon_r}{3} - \varepsilon _r dV = -\frac{4}{3} \varepsilon_r dV
$$
$$
-\frac{3}{4} \frac{d\varepsilon_r}{\varepsilon_r} = \frac{dV}{V}
$$
После интегрирования получаем:
$$
\ln \left(\frac{\varepsilon_r}{\varepsilon_{r,0}}\right)^{-3/4} = \ln \frac{V}{V_0}
$$
\begin{equation}
\varepsilon_r = \varepsilon_r \left( \frac{V}{V_0} \right)^{-4/3} = \varepsilon_{r,0} \left( \frac{R}{R_0} \right)^{-4} = a^{-4} \varepsilon_{r,0}
\end{equation}
Заменяя плотность энергии на плотность массы с помощью \eqref{albert}, получаем \eqref{rden}.

Итак, подставим \eqref{mden} и \eqref{rden} в \imp{уравнение движения}:
\begin{multline}
\ddot{a} = -\frac{4 \pi}{3} G a \left(a^{-3} \rho_{m,0} + 2 a^{-4} \rho_r\right) + \frac{\Lambda c^2}{3} a = \\
= -\frac{4\pi G}{3} a^{-2} \rho_{m,0} -\frac{8\pi G}{3} a^{-3} \rho_{r,0} + \frac{\Lambda c^2}{3} a.
\end{multline}
Проинтегрируем это уравнение с помощью замены $\ddot{a} = \dot{a} \displaystyle \frac{d\dot{a}}{da}$ (см. \ref{change}):
\begin{equation}
\frac{1}{2} \dot{a}^2 = \frac{4\pi G}{3} a^{-1} \rho_{m,0} + \frac{1}{2}\frac{8\pi G}{3} a^{-2} \rho_{r,0} + \frac{1}{2} \frac{\Lambda c^2}{3} a^2 + C_1
\end{equation}
Разделим на $a^2$ и умножим на $2$:
\begin{equation}
\left(\frac{\dot{a}}{a}\right)^2 = \frac{8\pi G}{3} a^{-3} \rho_{m,0} + \frac{8\pi G}{3} a^{-4} \rho_{r,0} + \frac{\Lambda c^2}{3} + \frac{C_2}{a^2}.
\label{energy}
\end{equation}
Из ОТО известно, что константа $C_2$ равна $-k c^2$, где $k$ может принимать значения $-1, 0$ и $1$ и называется кривизной пространства (при $k = 0$ пространство евклидово). Далее везде будем считать $k \equiv 0$. Вынесем $8 \pi G / 3$ за скобки:
\begin{equation}
\left(\frac{\dot{a}}{a}\right)^2 = \frac{8\pi G \rho}{3} + \frac{\Lambda c^2}{3} - \frac{k c^2}{a^2}.
\end{equation}
Это равенство называется \term{уравнением энергии}. Приведём здесь ещё одну форму этого уравнения, тоже достаточно известную. 

Преопразуем $\Lambda c^2 / 3$ (т. н. $\Lambda$-член) в \eqref{energy}:
$$
\frac{\Lambda c^2}{3} = \frac{8 \pi G}{3} \frac{\Lambda c^2}{8 \pi G} = \frac{8 \pi G}{3} \rho_{\Lambda}.
$$
В левой части \imp{уравнения энергии} стоит не что иное как \imp{параметр Хаббла}! Разделим обе части уравнения на $H_{0}^{2}$:
\begin{equation}
\left(\frac{H}{H_0}\right)^2 = \frac{8\pi G}{3H_{0}^{2}} a^{-3} \rho_{m,0} + \frac{8\pi G}{3H_{0}^{2}} a^{-4} \rho_{r,0} + \frac{8 \pi G}{3H_{0}^{2}} \rho_{\Lambda}
\end{equation}
Вспомним, что отношение $3H_{0}^{2} / 8 \pi G$~--- критическая плотность Вселенной сейчас. Также введём обозначения удельной плотности: $\Omega_{i} = \displaystyle \frac{\rho_i}{\rho_c}.$ Перепишем \imp{уравнение энергии}.
\begin{equation}
\left(\frac{H}{H_0}\right)^2 = a^{-3} \Omega_{m,0} + a^{-4} \Omega_{r,0} + \Omega_{\Lambda,0}.
\end{equation}
Впоследствие мы часто будем исползовать отношение $H / H_0,$ поэтому введём для него специальное обозначение:
\begin{equation}
\frac{H}{H_0} = E(a) = \sqrt{a^{-3} \Omega_{m,0} + a^{-4} \Omega_{r,0} + \Omega_{\Lambda,0}\,}.
\label{energa}
\end{equation}
Иногда \imp{уравнение энергии} удобнее записывать, выражая \imp{масштабный фактор} через \imp{красное смещение}. Так как $a = (1+z)^{-1},$ 
\begin{equation}
E(z) = \sqrt{\Omega_{m,0}(1+z)^{3} + \Omega_{r,0}(1+z)^{4} + \Omega_{\Lambda,0}\,}.
\end{equation}

\subsection{Расстояния в космологии}
Представим, что невдалеке от нас стоит уличный фонарь. Мы хотим узнать, на каком расстоянии он от нас находится. Как же можно это расстояние измерить?
\begin{enumerate}
\item Проще всего измерить расстояние линейкой или каким-то жестким объектом известной длины.
\item Мы можем определить расстояние по времени, которое требуется свету, чтобы дойти от фонаря до нас.
\item Зная реальный размер самого фонаря, можно измерить расстояние до него по его угловому размеру.
\item Можно измерить плотность потока излучения от фонаря и по ней определить расстояние.
\end{enumerate}
В случае с обычным уличным фонарём все эти способы дадут одинаковый результат, равный расстоянию, измеренному линейкой. Пусть теперь наш уличный фонарь~--- это далёкая галактика, не связанная с нами гравитационно и не имеющая пекулярной скорости. Попробуем определить расстояние до неё предложенными выше способами.

\subsubsection{Сопутствующие координаты}
Представим сетку координат, которая расширяется вместе с Вселенной. В таких координатах любая точка Вселенной, не имеющая пекулярной скорости, неподвижна относительно сетки. Следовательно, разность координат объекта и наблюдателя в данной системе не зависит от времени.

\subsubsection*{Сопутствующее расстояние}
По сути, эта разность координат определяется количеством узлов сетки, расположенных между наблюдателем и объектом. Предположим теперь, что в данный момент времени $t_0$ количество узлов равно $N$, а расстояние между соседними~--- $l(t_0) = l_0.$ Следовательно, расстояние между наблюдателем и объектом равно $D = N l_0.$ Поскольку наша Вселенная масштабируема, мы можем вычислить $l$ для любого другого момента времени $t$: $l(t) = l_0 a(t)$ (вспомните уравнение \eqref{eq}). Напомним, что $a(t_0) \equiv 1.$ 

Если расстояние мажду соседними узлами $l(t)$ устремить к нулю, то $D$ будет вычисляться посредством интегрирования по $dl_0.$ Так как $dl_0 = dl(t) / a(t)$, а приращение расстояния $dl(t) = c dt,$ где $c$~--- скорость света,
\begin{equation}
D \equiv D_{\text{C}} = \int dl_0 = \int \frac{c dt}{a}.
\label{com}
\end{equation}
Величина $D_{\text{C}}$ называется сопутствующим расстоянием. Приведём его к более удобному для вычислений виду. Для этого домножим и разделим подынтегральное выражение на $d \ln(a)$.
$$
D_{\text{C}} = c \int_{t}^{t_0} \frac{dt}{d \ln(a)} \frac{d \ln(a)}{a} = c \int_{a}^{1} \frac{d \ln(a)}{a H(a)} = c \int_{a}^{1} \frac{da}{a^2 H(a)}
$$
Вспомним, что $da = - dz / (1+z)^{2}$ (см. \eqref{dadz}) и $a = (1+z)^{-1}$. При замене масштабного фактора на красное смещение интеграл принимает следующий вид:
$$
-c \int_{z}^{0} \frac{dz}{(1+z)^{2}} \frac{(1+z)^{2}}{H(z)} = c \int_{0}^{z} \frac{dz}{H(z)}
$$
Воспользуемся выводами, полученными в прошлом разделе:
\begin{equation}
D_{\text{C}} = D_{\text{H}_0} \int_{0}^{z} \frac{dz}{E(z)}
\label{Dc}
\end{equation}

Поскольку сопутствующее расстояние~--- это расстояние между наблюдателем и объектом в данный момент времени, то взяв интеграл \eqref{Dc} в пределах  от $z = 0$ до $z = \infty,$ мы получим нынешний размер Вселенной. Будем считать, что относительная плотность тёмной энергии $\Omega_{\Lambda,0} \simeq 0.69,$ а вкладом излучения из-за его малости пренебрежём.
\begin{equation}
R_{0,\text{C}} = D_{\text{H}_0} \int_{0}^{\infty} \frac{dz}{E(z)} \approx 3.24D_{\text{H}_0} \approx 13.9 \text{ Gpc}.
\end{equation}

\subsubsection{Световое расстояние}

%Чтобы линейки не были подвержены расширению Вселенной, они должны быть малой длины. Обозначим её за $l.$ Теперь мы можем вычислить расстояние между соседними узлами: $l N(t) = l (N_0 / a(t))$ %При расстояних, значительно меньших размера Вселенной, оно определяется следующим соотношением:
%\begin{equation}
%D_{\text{C}} \equiv \frac{D(t)}{a(t)} = D_0.
%\end{equation}
%мы измеряем расстояние между двумя соседними узлами с помощью жестких линеек одинаковой длины. Понятно, что их количество, требуемое для измерения расстояния, в разные моменты времени будет разным.
%и, прежде, чем перейти к следующему разделу, разберемся с ещё одним уравнением, позволяющим расчитать время, которое требуется свету, излученному в момент времени $t_\text{e}$, чтобы дойти до наблюдателя. Заметим, что всё, относящееся к моменту испускания обозначается с индексом "e"\text{} (от англ. \textit{emit}), а к момету наблюдения~--- с индексом "0",или "o"\text{} (от англ. \textit{observe}). В англоязычной литературе это время называется \text{} "\imp{lookback time}". Обозначим его как $t_\text{L}$.

Пусть мы наблюдаем некоторый объект на красном смнщнении $z$. Расстояние до него равно $D_{\text{C}}$, однако из-за расширения вселенной оно не будет равно тому расстоянию, которое пройдёт свет от объекта да наблюдателя. Световое расстояние складывается из небольших отрезков $c dt$, то есть, чтобы его найти, нужно вычислить следующий интеграл:

$$
D_\text{L} = \int_{t_\text{e}}^{t_\text{o}} c dt = c \int_{a_\text{e}}^{1} \left(\frac{dt}{d \ln(a)}\right)  d \ln(a)
$$
Так как $d \ln (a) \slash dt = H$, то
\begin{equation}
D_\text{L} = c \int_{a_\text{e}}^{1} \frac{da}{aH(a)} = D_{\text{H}_0} \int_{a_\text{e}}^{1} \frac{da}{a E(a)}.
\end{equation}
Или же, выражая через $z$:
\begin{equation}
D_\text{L} = -c \int_{0}^{z_\text{e}} \left(\frac{dt}{dz}\right) dz = D_{\text{H}_0} \int_{0}^{z_\text{e}} \frac{dz}{(1+z)E(z)}.
\end{equation} 

\subsubsection{Расстояние по угловому размеру}

Предположим, что некоторый небольшой объект размера $l$ располагается перпендикулярно нашему лучу зрения на красном смещении $z$ и имеет наблюдаемый угловой размер $\theta \ll 1.$  
Для него можно ввести расстояние: $D_{\text{A}} = l \slash \theta.$ Выясним, как оно связано сосопутствующим. Пока свет шёл от объекта до нас, Вселенная успела расшириться. 
Объект сохранил свой размер, но стал занимать меньше места. Сопутствующее расстояние будет равно $l_0 \slash \theta,$ где $l_0$~--- разность координат между концами объекта на момент испускания видимого света 
(то есть, на эпоху~$z$). Получается, что $l_0 = l (1 + z),$ а расстояние по угловому размеру выражается через сопутствующее следующим образом:
\begin{equation}
D_{\text{A}} = \frac{D_{\text{C}}}{1 + z}.
\end{equation}
 
\subsubsection{Фотометрические расстояние}
Для изотропного источника с известной светимостью $L$ можно определить фотометрическое расстояние:
$$
F = \frac{L}{4 \pi D_{\text{L}}^2},
$$
где $F$~--- измеренная плотность потока излучения от источника. Однако фотометрическое расстояние не равно сопутствующему.
Если наблюдаемый объект находится на красном смещении $z,$ то длина волны каждого принятого фотона не равна длине волны излученного: $\lambda_0 = \lambda_{\text{e}} (1 + z).$ Следовательно, 
энерия фотона ($E = hc \slash \lambda$) тоже будет меняться: $E_0 = E_{\text{e}} \slash (1 + z),$ как и $F.$ Расстояние между фотонами будет увеличиваться пропорционально $1 + z,$ как и $\lambda.$ Увеличение расстояния 
между фотонами уменьшает частоту их детектирования (в сравнении с частотой испускания), а следовательно, и плотность потока в $1 + z$ раз. Поскольку плотность потока, не подверженная этим изменениям,-- это 
просто $L \slash 4 \pi D_{\text{C}}^2,$ можно связать сопутствующее и фотометрическое расстояния:
$$
\frac{L}{4 \pi D_{\text{L}}^2} = \frac{L}{4 \pi D_{\text{C}}^2} \frac{1}{(1 + z)^2},
$$
из чего следует, что
\begin{equation}
    D_{\text{L}} = D_{\text{C}} (1 + z).
\end{equation}
    
	\newpage
\section{Оптика}
\subsection{Телескоп}
\term{Телескоп}~--- устройство для наблюдения удаленных объектов. На данный момент существуют телескопы  для наблюдения во всех  диапазонах электро-магнитного излучения. По наблюдаемому диапазону телескопы делят на \imp{оптические} телескопы, \imp{радиотелескопы}, \imp{рентгеновские} телескопы и \imp{гамма-те\-ле\-скопы}. Каждый из классов в свою очередь содержит множество подклассов. Поговорим подробнее про оптические телескопы.

Оптические телескопы по своей схеме делятся на три типа: \imp{рефлекторы} (диоптрические), \imp{рефракторы} (катаптрические) и \imp{катадиоптрические}.

\vspace{-.3pc}
\begin{figure}[h!]
	\centering
	\begin{subfigure}{0.49\tw}
		\includegraphics[width = \tw]{Galiley}
		\caption{Рефрактор системы Галилея}
	\end{subfigure}
	\hfill
	\begin{subfigure}{0.49\tw}
		\includegraphics[width = \tw]{Kepler}
		\caption{Рефрактор системы Кеплера}
		\label{Kepler}
	\end{subfigure}
	\caption{Оптические схемы телескопов рефракторов}
\end{figure}
\term{Рефрактор} (линзовый телескоп)~---  оптический телескоп, в котором для собирания света используется система линз.

\vspace{-.3pc}
\begin{figure}[h!]
	\begin{subfigure}{0.49\tw}
		\includegraphics[width = \tw]{Cassigren.pdf}
		\caption{Рефлектор системы Кассегрена}
	\end{subfigure}
	\hfill
	\begin{subfigure}{0.49\tw}
		\includegraphics[width = \tw]{Gregory.pdf}
		\caption{Рефлектор системы Грегори}
		\label{Gregory}
	\end{subfigure}
	\vskip4pt
	\begin{subfigure}{0.49\tw}
		\includegraphics[width = \tw]{Newton}
		\caption{Рефлектор системы Ньютона}
	\end{subfigure}
	\hfill
	\begin{subfigure}{0.49\tw}
		\includegraphics[width = \tw]{Lomonosov.pdf}
		\caption{Рефлектор системы Ломоносова}
	\end{subfigure}
	\caption{Оптические схемы телескопов рефлекторов}
\end{figure}
\term{Рефлектор} (зеркальный телескоп)~---  оптический телескоп,  в котором светособирающими элементами являются зеркала.

\term{Катадиоптрический} (зеркально-линзовый) \term{телескоп}~--- оптический телескоп, в котором используется как система линз, так и зеркал.

\input{sections/optic.mounts.tex}
\input{sections/optic.zoom.tex}
\input{sections/optic.thin-lens.tex}
\input{sections/optic.snell-law.tex}
\subsection{Аберрации в оптике}
\paragraph{Хроматическая аберрация}
Пожалуй главным недостатком оптических схем, содержащий преломляющие оптические элементы (линзы; призмы, за исключением использования их для спектроскопии) являются \imp{хроматические аберрации}. Дело в том, что показатель преломления материала линзы зависит от длины волны падающего излучения. Это приводит к тому, что положения фокуса оптической системы зависит от длины волны излучения. При наблюдениях это проявляется как радужный ореол вокруг объектов, ухудшающий качество изображения.

\begin{wrapfigure}{r}{0.5\tw}
	\centering
	\vspace{-1pc}
	\begin{tikzpicture}
	\begin{axis}[
		height	=	4.5cm,
		width	=	6cm,
		xlabel	=	{$\lambda$, мкм},
		ylabel	=	{$n(\lambda)$},
		ylabel shift	= -1 cm,
		xmin = 0.3,
		xmax = 2.5,
		ymin = 1.48,
		ymax = 1.55,
		]
		
		\addplot[smooth, domain=0.3:2.5] table[x=l, y=n] {data/crown-dispersion.txt};
	\end{axis}
\end{tikzpicture}
\caption{}
\label{pic:crown-dispersion}	
\end{wrapfigure}
Найдем зависимость величины хроматической аберрации от величины дисперсии материала линзы. В качестве линзы рассмотрим плосковыпуклую линзу из оптического стекла~--- \imp{крона} (BK7). Зависимость $n(\lambda)$ её показателя преломления $n$ от длины волны $\lambda$ проходящего излучения представлена на графике (см.~Рис\,\ref{pic:crown-dispersion}). Важно отметить, здесь уже нельзя считать линзу тонкой, так как само по себе понятие тонкой линзы подразумевает отсутствие разного рода аберраций и условие фокусировки лучей в одной точке (фокусе), чего не происходит на практике.

\begin{wrapfigure}[9]{r}{0.63\tw}
	\centering
	\vspace{-.8pc}
	\begin{tikzpicture}
		\footnotesize
		
		\draw [decoration={snake, segment length=.7mm, amplitude=0.2mm}, decorate] (1.35, 1) arc(-31:15:0.26);
		\draw [double, line cap=butt] (.2, 1.135) arc(180:195:0.94);
		\draw  (2.7, 0) arc(180:149:0.3);
		
		\fill [lightgray] (.5, -1.5) -- (.5, 1.5) -- (1, 1.5) arc(20:-20:4.386) -- (.5, -1.5);
		
		\draw [thick] (.5, -1.5) -- (.5, 1.5);
		\draw [thick] (1, -1.5) arc(-20:20:4.386);
		\draw [semithick, dash pattern={on 5pt off 2pt on .5pt off 2pt}] (-3.8, 0) -- (3.2, 0);
%		\draw (-3.12, 0) node{$\times$};
		
		\draw [dashes] (-3.12, 0) -- (1.71, 1.29);
		
		\draw [semithick] (-3.8, 1.135) -- (1.12, 1.135) -- (3, 0);
		\draw [-latex] (-3.5, 1.135)-- (-1, 1.135);
		\draw [-latex] (1.12, 1.135) -- (2.25, 0.45);
		
		\draw [latex-latex] (-3.5, 0) -- (-3.5, 1.135);
		\draw [latex-latex] (1.27, -.5) -- (3, -.5);
		\draw [latex-latex] (1.12, -1.4) -- (3, -1.4);
		\draw [-latex] (0.5, -.5) -- (1.12, -.5);
		
		\draw (1.12, 1.135) -- (1.12, -1.5);
		\draw (1.27, 0) -- (1.27, -.6);
		\draw (3, 0) -- (3, -1.5);
		
		\draw (-3.5, 0.56) node[anchor=east] {$d$};
		\draw (-3.12, 0) node[anchor=north] {$C$};
		\draw (-.7, 0.6) node[anchor=north] {$R$};
		\draw (2.14, -.5) node[anchor=south] {$x$};
		\draw (2.14, -1.4) node[anchor=south] {$\frac{d}{\tg \gamma}$};
		
		\draw (.8, -1.5) node[anchor=south] {$n$};
		
		\draw (.2, 1) node[anchor=east] {$\alpha$};
		\draw (1.4, 1.07) node[anchor=west] {$\beta$};
		\draw (2.65, 0.15) node[anchor=east] {$\gamma$};
		
		
		\draw [fill=white] (3, 0) circle (.03); 
		\draw [fill=white] (-3.12, 0) circle (.03);
	\end{tikzpicture}
	\caption{}
	\label{pic:sphere-aberrations-lens}
\end{wrapfigure}
Итак, пусть радиус кривизны выпуклой поверхности рассматриваемой плос\-ко-вы\-пук\-лой линзы равен $R$. Рассмотрим также луч, параллельный оптической оси данной линзы, на расстоянии $d$ от этой оси (см.~Рис.\,\ref{pic:sphere-aberrations-lens}). Так как передняя поверхность линзы плоская, луч, попадая в линзу, не преломляется. Преломление происходит на выходе из линзы. Нетрудно показать, что для угла падения луча на заднюю поверхность линзы $\alpha$ справедливо, что $\sin \alpha = d/R$. По закону Снеллиуса угол преломления рассматримоего луча $\beta$ определяется соотношением $\sin \beta = n \sin \alpha$. Так как угол между нормалью к выпуклой поверхности линзы и ее оптической осью равен $\alpha$, то угол $\gamma$ между преломленным лучем и оптической осью линзы равен $\beta - \alpha$.

Расстояние до точки пересечения преломлённого луча с оптической осью линзы будем отсчитывать от вершины выпуклой поверхности линзы. Расстояние $h$ между проекцией точки преломления на оптическую ось и вершиной можно найти из теоремы Пифагора:
\begin{equation*}
	h = R - \sqrt{R^2 - d^2}.
\end{equation*}
Тогда координата фокуса для лучей на расстоянии $d$ от оптической оси равно
\begin{equation}
	x = \frac{d}{\tg \gamma} - h = \frac{d}{\tg \left( \arcsin \dfrac{n d}{R} - \arcsin \dfrac{d}{R} \right)} - \left( R - \sqrt{R^2 - d^2} \right).
	\label{eq:optics-aberr-x(d)}
\end{equation}
Введём обозначение $\delta \equiv d/R$ и разделим обе части полученного равенства на $R$, чтобы перейти к относительным единицам:
\begin{equation}
	\frac{x}{R}
	= \frac{\delta}{\tg \left( \arcsin n\delta - \arcsin \delta \right)} -  1 + \sqrt{1 - \delta^2}
	~\xrightarrow{\delta \ll 1}~  \frac{1}{n - 1} - \frac{\delta^2 n^2}{2(n-1)}.\footnote{\scriptsize Вывод приближения:
	\begin{multline*}
		\frac{\delta}{\tg \left( \arcsin n\delta - \arcsin \delta \right)} -  1 + \sqrt{1 - \delta^2} =\\
		= \frac{\delta}{\tg \left[ n \delta + \dfrac{n^3 \delta^3}{6} - \delta - \dfrac{ \delta^3}{6} + o(\delta^3) \right]} -  1 + \left(1 - \frac{\delta^2}{2} + o(\delta^2) \right) =\\
		= \frac{\delta}{\tg \left[ \delta(n-1) + \dfrac{(n^3-1) \delta^3}{6} + o(\delta^3) \right]} - \frac{\delta^2}{2} + o(\delta^2) =\\
		= \frac{\delta}{\delta(n-1) + \dfrac{(n^3-1) \delta^3}{6} + \dfrac{\delta^3}{3}(n-1)^3 + o(\delta^3)} - \frac{\delta^2}{2} + o(\delta^2) =\\
		= \frac{1/(n-1)}{1 + \dfrac{\delta^2}{6}(n^2 + n + 1) + \dfrac{\delta^2}{3}(n-1)^2 + o(\delta^2)} - \frac{\delta^2}{2} + o(\delta^2) =\\
		=\frac{1}{n-1} \left[1 - \frac{\delta^2}{6} (3n^2 -3n + 3) \right] - \frac{\delta^2}{2} + o(\delta^2)
		%	= \frac{1}{n-1} - \frac{\delta^2}{2(n-1)}(n^2 - n + 1 + n - 1) + o(\delta^2)
		\simeq \frac{1}{n-1} - \frac{\delta^2 n^2}{2(n-1)}.
	\end{multline*}
	}
\end{equation}

Найдём область определения функции $x(\delta)$. Прежде всего $\delta \geqslant 0$, потому что $d$~--- это расстояние от оптической оси, которое не может быть отрицательным. С другой стороны радиус линзы не может быть больше радиуса кривизны ее поверхности, следовательно, $\delta < 1$. Однако есть ещё одно условие, которое ограничивает $\delta$ сверху. Это эффект полного внутреннего отражения. Действительно, $\sin \beta$ не может быть больше единицы, следовательно, $\sin \alpha < \sfrac{1}{n}$, а значит, $\delta < \sfrac{1}{n}$. Для стекла, коэффициент преломления которого $n \approx 1.5$, получаем, что $\delta \in [0, \sfrac{2}{3})$.

\begin{wrapfigure}{r}{0.5\tw}
	\centering
	\vspace{-1pc}
	\begin{tikzpicture}
	\begin{axis}[
		height	=	4.5cm,
		width	=	6cm,
		xlabel	=	{$\lambda$, мкм},
		ylabel	=	{$x(\lambda)$},
		ylabel shift	= -1 cm,
		xmin = 0.3,
		xmax = 2.5,
		ymin = 1.8,
		ymax = 2.1,
		]
		
		\addplot[smooth, domain=0.3:2.5] table[x=l, y=x] {data/crown-dispersion.txt};
	\end{axis}
\end{tikzpicture}
\caption{}
\label{pic:crown-dispersion-x}	
\end{wrapfigure}
Как будет показано далее, во избежании проявления \imp{сферической аберрации}, используют линзы с маленьким относительным отверстием ($\forall \ll 1$). Следовательно, и $\delta \ll 1$, возьмем для примера значение $\delta = 0.1$. Для него, очевидно, можно использовать приближение для $x$, поэтому при заданном значении $\delta$ выражение для $x$ принимает вид:
\begin{equation*}
	x = \frac{1}{n-1} - \frac{0.01 n^2}{2(n-1)}.
\end{equation*}
Как видно из графика данной зависимости (см.~Рис.\,\ref{pic:crown-dispersion-x}), для оптического диапазона $x(\lambda)$ принимает значения от примерно 1.85 для коротковолновой (фиолетовой) части до примерно 1.95 для красного цвета. 

Чтобы компенсировать такой разбег совместно с собирающей линзой использую рассеивающую из другого материала. Объективы, где исправлена хроматическая аберрация для двух цветов и частично исправлена сферическая аберрация называют \term{ахроматами}; где хроматическая аберрация исправлена для трёх цветов, а также полностью исправлена сферическая аберрация~--- \term{апохроматами}; с более полной геометрической коррекцией~--- \term{апланатами}.

\paragraph{Сферическая аберрация}
В оптических системах, содержащих \change{сферические поверхности (линзы, зеркала)} может наблюдаться \imp{сферическая аберрация}. Суть такой аберрации состоит в том, что лучи, параллельные оптической оси, идущие на разном расстоянии от неё собираются в разных её местах. Это приводит к тому, что изображения точечных источников размываются.

\begin{wrapfigure}[12]{r}{0.55\tw}
	\centering
	\vspace{-.5pc}
	\begin{tikzpicture}
		\begin{axis}[
			height	=	5cm,
			width	=	6.5cm,
			xlabel	=	{$\delta$},
			ylabel	=	{$x(\delta)/R$},
			ylabel shift	= -1.1 cm,
			extra x ticks ={0.667},
			extra x tick labels={$\frac{1}{n}$},
			xmin=-.05,
			xmax=0.72,
			ymin=-.25,
			ymax=2.25,
			legend cell align=left,
			legend style={
			draw=none,
			fill=none,
			font=\scriptsize,
			at={(axis cs:0, 0.1)}, anchor=south west,
			row sep=.5pc,
			},
			]
			\addplot[smooth, gray] table[x=d, y=simple] {data/shere-aberrations-lens.txt};
			\addplot[smooth] table[x=d, y=x] {data/shere-aberrations-lens.txt};
			\addplot[dashes] coordinates { (0.667, -10) (0.667, 10)};
			\legend{
			$\left. \dfrac{x(\delta)}{R} \right|_{\delta \ll 1}$,
			$\dfrac{x(\delta)}{R}$,
			$\delta = \left.\dfrac{1}{n}\right|_{n=3/2}$,
			}
		\end{axis}
	\end{tikzpicture}
	\caption{}
	\label{pic:sphere-aberrations-lens-plot}
\end{wrapfigure}
Покажем наличие сферической аберрации для плосковыпуклой линзы. Рассмотрим полученное выше выражение \eqref{eq:optics-aberr-x(d)} и его приближение при $\delta \ll 1$. Зафиксируем в них $n=3/2$~--- характерное значение показателя преломления для стекла. Графики получаемых при этом зависимостей представлены на Рис.\,\ref{pic:sphere-aberrations-lens-plot}. Как видно из данных графиков, сферические аберрации проявляются уже на малых расстояниях от оптической оси. 

Чтобы показать важность сферических аберраций рассмотрим небольшой телескоп рефрактор с диаметром плосковы\-пук\-ло\-во\-го\linebreak стеклянного ($n \approx 3/2$) объектива $D = 50$~мм. Характерная точность фокусировки $\Delta x$ для таких маленьких телескопов составляет около 1~мм. Установим, при каком фокусном расстоянии такого телескопа точность фокусировки нивелирует сферическую аберрацию. 

При отсутствии сферической аберрации фокусное расстояние плосковыпуклового объектива $F = R/(n-1) = 2R$. Предельное значение $\delta$, которое нужно рассмотреть, соответствует лучам, проходящим через край объектива, следовательно, $\delta = D/(2R)$. Используя приближение, теперь можно записать выражение для требуемого $\Delta x$, чтобы найти необходимое для этого относительное отверситие $\forall$:
\begin{gather*}
	\Delta x = F - x(\delta) = F - x\left( \frac{D}{2R} \right),\\
	\Delta x = F - R\left( 2 - \frac{9 D^2}{4 \cdot 4R^2} \right),\\
	\Delta x = \frac{9D^2}{16R} = \frac{9D^2}{16F} = \frac{9}{16} D \forall;\\
	\therefore \forall = \frac{\Delta x \cdot 16}{9D} = 0.036.
\end{gather*}

\begin{wrapfigure}[9]{r}{0.5\tw}
	\centering
	\vspace{-.8pc}
	\begin{tikzpicture}
		\footnotesize
		
		\draw [decoration={snake, segment length=.7mm, amplitude=0.2mm}, decorate] (1.35, 1) arc(-31:15:0.26);
		\draw (.2, 1.135) arc(180:209:0.94);
		\draw (-2.18, 0) arc(0:15:0.94);
		
		\fill [lightgray] (1.5, -1.5) -- (1.5, 1.5) -- (1, 1.5) arc(20:-20:4.386) -- (1.5, -1.5);
		
		\draw [thick] (1, -1.5) arc(-20:20:4.386);
		\draw [semithick, dash pattern={on 5pt off 2pt on .5pt off 2pt}] (-3.8, 0) -- (1.7, 0);

		
		\draw [dashes] (-3.12, 0) -- (1.71, 1.29);
		
		\draw [semithick] (-3.8, 1.135) -- (1.12, 1.135) -- (-.85, 0);
		\draw [-latex] (-3.5, 1.135)-- (-1, 1.135);
		\draw [-latex] (1.12, 1.135) -- (-.06, 0.45);
		
		\draw [latex-latex] (-3.5, 0) -- (-3.5, 1.135);
		\draw [latex-latex] (-.85, -1.3) -- (1.27, -1.3);

		\draw (1.27, 0) -- (1.27, -1.5);
		\draw (-.85, 0) -- (-.85, -1.5);
		
		\draw (-3.5, 0.56) node[anchor=east] {$d$};
		\draw (-0.85, 0) node[anchor=north east] {$F$};
		\draw (-3.12, 0) node[anchor=south] {$C$};
		\draw (-1.2, 0.5) node[anchor=south] {$R$};
		\draw (0.27, -1.3) node[anchor=south] {$x_F(d)$};

		
		\draw (.2, 1) node[anchor=east] {$\alpha$};
		\draw (-2.2, .15) node[anchor=west] {$\alpha$};
		\draw (.3, .7) node[anchor=east] {$\alpha$};
	
		
		\draw [fill=white] (-0.85, 0) circle (.03); 
		\draw [fill=white] (-3.12, 0) circle (.03);
	\end{tikzpicture}
	\caption{}
	\label{pic:sphere-aberrations-mirrow}
\end{wrapfigure}
Найдем теперь величину аберрации сферического зеркала.\linebreak Пусть $R$~--- радиус кривизны зеркал. Рассмотрим луч, идущий параллельно оптической оси зеркала на расстоянии $d$ от неё. Он падает на зеркало под углом $\alpha$, причем $\sin \alpha = d/R$. В силу закона отражения: угол падения равен углу отражения, то есть угол отражения также равен $\alpha$. Кроме того, угол между нормалью к зеркалу в точке отражения и оптической осью зеркала также равен $\alpha$ как вертикальный. Следовательно треугольник {\slshape центр кривизны зеркала ($C$) -- точка отражения ($A$) -- точка пересечения отраженного луча с оптической осью (F)} является равнобедренным. Значит расстояние $x_F(d)$ от центра зеркала до <<фокуса>> $F$ можно найти как 
\begin{gather*}
	x_F(d) = R - \frac{R}{2} \cdot \frac{1}{\cos\alpha} = R - \frac{R}{2\sqrt{1 - \sin^2 \alpha}}  = R  - \frac{R}{2\sqrt{1 - \dfrac{d^2}{R^2}}};\\
	\left. \frac{x_F(d)}{R} \right|_{d \ll R} \simeq  1  - \frac{1}{2\left(1 - \dfrac{d^2}{2R^2} \right)} \simeq  1 - \frac{1}{2}\left(1 + \dfrac{d^2}{2R^2} \right)  = \frac{1}{2} -  \dfrac{d^2}{4R^2}.
\end{gather*}
\begin{wrapfigure}[12]{r}{0.55\tw}
	\centering
	\vspace{-.5pc}
	\begin{tikzpicture}
		\begin{axis}[
			height	=	5cm,
			width	=	6.5cm,
			xlabel	=	{$d/R$},
			ylabel	=	{$x(d)/R$},
			ylabel shift	= -1.1 cm,
			extra x ticks ={sqrt(2)/2},
			extra x tick labels={$\frac{\sqrt{2}}{2}$},
			xmin=-.05,
			xmax=0.85,
			ymin=.15,
			ymax=0.55,
			legend cell align=left,
			legend style={
			draw=none,
			fill=none,
			font=\scriptsize,
			at={(axis cs:0, .2)}, anchor=south west,
			row sep=.5pc,
			},
			]
			\addplot[smooth, gray] table[x=d, y=simple] {data/sphere-aberrations-mirrow.txt};
			\addplot[smooth] table[x=d, y=x] {data/sphere-aberrations-mirrow.txt};
			\addplot[dashes] coordinates { (sqrt(2)/2, -10) (sqrt(2)/2, 10)};
			\legend{
			$\left. \dfrac{x_F(d)}{R} \right|_{d \ll R}$,
			$\dfrac{x_F(d)}{R}$,
			$ \dfrac{d}{R} = \dfrac{\sqrt{2}}{2}$,
			}
		\end{axis}
	\end{tikzpicture}
	\caption{}
\end{wrapfigure}
Отсюда получается, что фокус сферического зеркала находится ровно между центром кривизны зеркала и центром этого зеркала. Однако в силу сферической аберрации возникает ошибка фокусировки порядка $d/R$, которая размывает изображение. Причём при $d > R\sqrt{2}/2$ лучи не <<разворачиваются>>, следовательно, не вносят вклада в изображение, так как приходят на приемник с другой стороны.

Для компенсации сферической аберрации используют различные линзы-корректоры, однако они помогают лишь частично избавиться от неё. \change{Поэтому в современных рефлекторах используются параболические зеркала, не подверженные сферическим аберрациям.}

%Напоследок нужно отметить, что сферические аберрации

\paragraph{Астигматизм} Ещё один вид аберраций оптических систем состоящий в разности радиусов кривизны оптических элементов в двух перпендикулярных направлениях. Такое возможно, например, в случае большой массы линзы или зеркала. Когда данный оптический элемент долгое время находится в вертикальном положении (оптическая ось горизонтальна), он деформируется: вдоль горизонтали радиус кривизны сохраняется, а по вертикали из-за сжатия уменьшается.

\imp{Астигматизм} проявляется в том, что пучок лучей, исходящих из какой-либо точки, после прохождения через оптическую систему собирается не в одной точке, а на двух взаимно перпендикулярных отрезках, расположенных на некотором расстоянии друг от друга. Изображения промежуточных сечений имеют форму эллипсов.

\paragraph{Кома} 

\begin{wrapfigure}[13]{l}{0.4\tw}
	\centering
	\vspace{-1pc}
	\includegraphics[width=0.4\tw]{img/optics-aberrations-coma.jpg}	
	\caption{Изображение <<хвоста>> Большой Медведицы, полученное с помощью широкоугольного объектива, страдающего ярко выраженной комой.}
	\label{pic:optics-aberrations-coma}
\end{wrapfigure}
Один из видов аберраций оптических систем~--- аберрация широкого пучка световых лучей, проходящий наклонно к оптической оси системы, как и \imp{сферическая аберрация}, обусловлена неодинаковым преломлением световых лучей различными участками линзовых компонент системы. Кома приводит к нарушению центрированности светового пучка. В результате такой аберрации изображение точки имеет вид несимметричного пятна (см.~Рис.\,\ref{pic:optics-aberrations-coma}), по форме напоминающего запятую (англ. {\itshape comma}).


































\subsection{Диффракция}

\begin{figure}[p]
	\centering
	\begin{subfigure}{\tw}
		\begin{tikzpicture}
			\begin{axis} [
				width			=	10cm,
				colormap 		= 	{GS}{rgb(0cm)=(.1, .1, .1)  rgb(1cm)	=	(1, 1, 1)},
				xlabel 			=	{$x$, $\frac{\lambda}{D}$},
				ylabel 			=	{$y$, $\frac{\lambda}{D}$},
				zlabel 			=	{$I/I_0$},
				ylabel shift 	= -.4 cm,
				xlabel shift 	= -.3 cm,
				ytick			= {-2,0,2},
				colorbar,
				colorbar style 	= {
				ytick 	= 	{0, .2, .4, .6, .8, 1.},
				}
				]
				
				\addplot3[
				samples				=	100,
				samples y			=	100,
				mesh,
				patch type			=	line,
				x filter/.code		=	\def\pgfmathresult{-5},
				smooth
				]
				table[x=x, y=y, z=I] {data/eiry-disk-x.txt};
				%
				\addplot3[
				samples			=	100,
				samples y		=	100,
				mesh,
				patch type		=	line,
				y filter/.code	=	\def\pgfmathresult{4.5},
				smooth
				]
				table[x=x, y=y, z=I] {data/eiry-disk-y.txt};
				
				\addplot3[surf] table[x=x, y=y, z=I] {data/eiry-disk.txt};
			\end{axis}
		\end{tikzpicture}
		\caption{}
		\label{}
	\end{subfigure}\\[2pc]
	\begin{subfigure}{\tw}
		\begin{tikzpicture}
			\begin{axis} [
				width			=	10cm,
				height			=	7.5cm,
				colormap 		= 	{GS}{rgb(0cm)=(.1, .1, .1)  rgb(1cm)	=	(1, 1, 1)},
				view			=	{0}{90},
				ytick 	= 	{-3, -2, ..., 3},
				colorbar,
				colorbar style 	= 	{
				ytick 	= 	{0, .2, .4, .6, .8, 1.},
				},
				xlabel 			=	{$x$, $\frac{\lambda}{D}$},
				ylabel 			=	{$y$, $\frac{\lambda}{D}$},
				]
				
				\addplot3[surf, shader=interp] table[x=x, y=y, z=I] {data/eiry-disk.txt};
			\end{axis}
		\end{tikzpicture}
		\caption{}
		\label{}
	\end{subfigure}
	\caption{}
\end{figure}

\begin{figure}[p]
	\centering
	\begin{subfigure}{\tw}
		\begin{tikzpicture}
			\begin{axis} [
				width			=	10cm,
				colormap 		= 	{GSW}{rgb(0cm)=(.1, .1, .1) rgb(.05cm)=(.99, .99, .99) rgb(1cm)	=	(1, 1, 1)},
				xlabel 			=	{$x$, $\frac{\lambda}{D}$},
				ylabel 			=	{$y$, $\frac{\lambda}{D}$},
				zlabel 			=	{$I/I_0$},
				ylabel shift 	= -.4 cm,
				xlabel shift 	= -.3 cm,
				ytick			= {-2,0,2},
				colorbar,
				colorbar style 	= {
				ytick 	= 	{0, .2, .4, .6, .8, 1.},
				}
				]
				
				\addplot3[
				samples				=	100,
				samples y			=	100,
				mesh,
				patch type			=	line,
				x filter/.code		=	\def\pgfmathresult{-5},
				smooth
				]
				table[x=x, y=y, z=I] {data/eiry-disk-x.txt};
				%
				\addplot3[
				samples			=	100,
				samples y		=	100,
				mesh,
				patch type		=	line,
				y filter/.code	=	\def\pgfmathresult{4.5},
				smooth
				]
				table[x=x, y=y, z=I] {data/eiry-disk-y.txt};
				
				\addplot3[surf] table[x=x, y=y, z=I] {data/eiry-disk.txt};
			\end{axis}
		\end{tikzpicture}
		\caption{}
		\label{}
	\end{subfigure}\\[2pc]
	\begin{subfigure}{\tw}
		\begin{tikzpicture}
			\begin{axis} [
				width			=	10cm,
				height			=	7.5cm,
				colormap 		= 	{GSW}{rgb(0cm)=(.1, .1, .1) rgb(.05cm)=(.99, .99, .99) rgb(1cm)	=	(1, 1, 1)},
				view			=	{0}{90},
				ytick 	= 	{-3, -2, ..., 3},
				colorbar,
				colorbar style 	= 	{
				ytick 	= 	{0, .2, .4, .6, .8, 1.},
				},
				xlabel 			=	{$x$, $\frac{\lambda}{D}$},
				ylabel 			=	{$y$, $\frac{\lambda}{D}$},
				]
				
				\addplot3[surf, shader=interp] table[x=x, y=y, z=I] {data/eiry-disk.txt};
			\end{axis}
		\end{tikzpicture}
		\caption{}
		\label{}
	\end{subfigure}
	\caption{}
\end{figure}

\begin{figure}[p]
	\begin{subfigure}[t]{\tw}
		\includegraphics[width=4.7cm]{eiry-disk-0}\hfill
		\begin{tikzpicture}
			\begin{axis}[
				height	=	4.125cm,
				width	=	5.5cm,
				xlabel	=	{$x$, $\frac{\lambda}{D}$},
				ylabel	=	{$I/I_0$},
				ylabel shift	= -1 cm,
				]
				
				\addplot[smooth] table[x=x, y=e0]{data/eiry-disk-profile.txt};
			\end{axis}
		\end{tikzpicture}
		\caption{Диффракционное изображение от одного источника}
	\end{subfigure}\\
	\begin{subfigure}[t]{\tw}
		\includegraphics[width=4.7cm]{eiry-disk-1}\hfill
		\begin{tikzpicture}
			\begin{axis}[
				height	=	4.125cm,
				width	=	5.5cm,
				xlabel	=	{$x$, $\frac{\lambda}{D}$},
				ylabel	=	{$I/I_0$},
				ylabel shift	= -1 cm,
				]
				
				\addplot[smooth] table[x=x, y=e1]{data/eiry-disk-profile.txt};
			\end{axis}
		\end{tikzpicture}
		\caption{Диффракционное изображение от двух источников с разделением~$1.22\lambda/D$}
	\end{subfigure}\\
	\begin{subfigure}{\tw}
		\includegraphics[width=4.7cm]{eiry-disk-2}\hfill
		\begin{tikzpicture}
			\begin{axis}[
				height	=	4.125cm,
				width	=	5.5cm,
				xlabel	=	{$x$, $\frac{\lambda}{D}$},
				ylabel	=	{$I/I_0$},
				ylabel shift	= -1 cm,
				]
				
				\addplot[smooth] table[x=x, y=e2]{data/eiry-disk-profile.txt};
			\end{axis}
		\end{tikzpicture}
		\caption{Диффракционное изображение от двух источников с разделением~$2 \cdot 1.22\lambda/D$}
	\end{subfigure}\\
	\begin{subfigure}{\tw}
		\includegraphics[width=4.7cm]{eiry-disk-3}\hfill
		\begin{tikzpicture}
			\begin{axis}[
				height	=	4.125cm,
				width	=	5.5cm,
				xlabel	=	{$x$, $\frac{\lambda}{D}$},
				ylabel	=	{$I/I_0$},
				ylabel shift	= -1 cm,
				]
				
				\addplot[smooth] table[x=x, y=e3]{data/eiry-disk-profile.txt};
			\end{axis}
		\end{tikzpicture}
		\caption{Диффракционное изображение от двух источников с разделением~$3 \cdot 1.22\lambda/D$}
	\end{subfigure}
	\caption{}
\end{figure}




	\section{Сферическая астрономия}
\input{sections/spher.astro.coordin-sys.tex}
\subsection{Суточное вращение небесной сферы}
Вследствие вращения Земли вокруг своей оси для наблюдателя на поверхности небесные объекты совершают суточное движение параллельно небесному экватору, плоскость которого совпадает с плоскостью экватора Земли. Очевидно, в ходе такого движения высота светил постоянно меняется и в некоторые моменты времени достигает своего максимального и минимального значения. 

\term{Верхняя} и \term{нижняя кульминация}~--- моменты пересечения светилом небесного меридиана, причём при верхней кульминации светило имеет наибольшую высоту, а при нижней~--- наименьшую.

Высота светила в верхней и нижней кульминации со склонением $|\delta| < |\varphi|$, соответственно:
\begin{equation}
h_{\text{в}}= 90^\circ - \varphi + \delta, \quad\quad
h_{\text{н}}= - 90^\circ + \varphi  + \delta.
\end{equation}

Если же светило имеет склонение $|\delta| > |\varphi|$, то высота в верхней и нижней кульминации вычисляется так:
\begin{equation}
h_{\text{в}}= 90^\circ + \varphi - \delta, \quad\quad
h_{\text{н}}= - 90^\circ -\varphi - \delta.
\end{equation}

Из формул для высоты в нижней кульминации вытекает условие, определяющее, пересекает ли звезда горизонт:
\begin{equation}
\begin{cases}
	h_\text{в}= +90^\circ - |\varphi + \delta| > 0^\circ,\\
	h_\text{н} = - 90^\circ + |\varphi + \delta| < 0^\circ;	
\end{cases}
\quad \Longleftrightarrow \quad~~ |\delta|< 90^{\circ} - |\varphi|.
\end{equation}

Используя формулы сферической тригонометрии (см.\,\ref{sec:spher-trig}), можно выразить зависимость часового угла светила от его зенитного расстояния:
\begin{equation}
\cos t=\frac{\cos z-\sin\varphi\sin\delta}{\cos\varphi\cos\delta}. 
\end{equation}
Отсюда следует, что для часового угла захода и восхода светила справедливо равенство:
\begin{equation}
	\cos t_{\uparrow\downarrow}=-\tg\varphi\cdot\tg\delta.
\end{equation} 

Аналогично, для вычисления азимута светила верна формула
\begin{equation}
\cos A=\frac{\cos\delta\cos t-\cos\varphi\cos z}{\sin\varphi\sin z}.
\end{equation}
Следовательно, азимуты точек восхода и захода
\begin{equation}
	A_\uparrow = \arccos \left(-\dfrac{\sin\delta}{\cos \varphi} \right)\quad\text{и}\quad A_\downarrow = - A_\uparrow.
\end{equation}

\term{Звёздное время}~$z$~--- часовой угол точки весеннего равноденствия. Из определений прямого восхождения и часового угла следует справедливость равенства\begin{equation}
z = \alpha + t.
\end{equation}
\subsection{Сферическая тригонометрия}
\label{sec:spher-trig}
\begin{wrapfigure}[10]{r}{.3\tw}
	\centering
	\vspace{-1pc}
 	\includegraphics[width=0.3\textwidth]{spher-trigonom}
 	\caption{Сферический треугольник}
\end{wrapfigure}
Для решения некоторых задач астрономии, связанных с видимыми положениями небесных тел, требуются знания о сферической тригонометрии. \imp{Сферический треугольник}~--- фигура на поверхности сферы, состоящая из трёх точек и трёх дуг больших кругов, соединяющих эти точки. Пусть $A$, $B$ и $C$~--- углы сферического треугольника, а $a$, $b$ и $c$~--- его стороны.

Сферические треугольники обладают следующими свойствами:
\begin{enumerate}
\item Два сферических треугольника равны, если они подобны.
\item Каждая сторона меньше суммы двух других сторон и больше их разности.
\item Сумма всех сторон $a+b+c$ всегда меньше $2\pi$.
\item Сумма углов сферического треугольника $\pi < A + B + C < 3\pi$.
\item Разность суммы двух углов и третьего угла меньше $\pi$
\end{enumerate}

Площадь сферического треугольника определяется по формуле:
\begin{equation}
	S = R^2( A + B + C - \pi),
\end{equation}
где $A + B + C - \pi$~--- \imp{сферический избыток}.

Рассмотрим сферический треугольник $ABC$, радиус векторы вершин соответственно $\vec{a}$, $\vec{b}$ и $\vec{c}$. причем из определения сферы $|\vec{a}| = |\vec{b}| = |\vec{c}| = r$. Пусть против вершин $A$, $B$ и $C$ лежат стороны с угловой мерой $a$, $b$ и $c$ соответсвенно. Повернем сферические координаты и нормируем так, чтобы $\vec{a} = (0, 0, 1)$, $\vec{b} = (\sin c, 0, \cos c)$, тогда $ \vec{c} = (\sin b \cos A, \sin b \sin A, \cos b)$.
	
	Теперь запишем выражение для $\scalar{b}{c}$:
	\begin{equation}
		\scalar{b}{c} = \cos a = \sin c \sin b \cos A + \cos c \cos b.
		\label{eq:spher-astro-cos-1}
	\end{equation}
	Аналогично,
	\begin{gather}
		\scalar{a}{c} = \cos b = \sin a \sin c \cos B +  \cos a \cos c,\\
		\scalar{a}{b} = \cos c = \sin a \sin b \cos C + \cos a \cos b.
		\label{eq:spher-astro-cos-1-1}
	\end{gather}
	Выразим отсюда $\cos A$:
	\begin{equation}
		\cos A = \frac{\cos a - \cos c \cos b}{\sin c \sin b}.
		\label{eq:spher-astro-cos-2}
	\end{equation}
	Формулы \eqref{eq:spher-astro-cos-1}\,--\,\eqref{eq:spher-astro-cos-2} называются \term{сферической теоремой косинусов} \imp{для стороны} \eqref{eq:spher-astro-cos-1}\,--\,\eqref{eq:spher-astro-cos-1-1} и, соответственно \imp{для угла} \eqref{eq:spher-astro-cos-2}.
	
	Из основного тригонометрического тождества имеем:
	\begin{multline*}
		\sin^2 A = 1 - \cos^2 A = 1 - \left[ \frac{\cos a - \cos c \cos b}{\sin c \sin b} \right]^2 = \\
		= \frac{\sin^2 c \sin^2 b - \cos^2 a + 2\cos a \cos c \cos b - \cos^2 c \cos^2 b}{\sin^2 c \sin^2 b}=\\
		= \frac{(1 - \cos^2 c)(1 -  \cos^2 b) - \cos^2 a + 2\cos a \cos c \cos b - \cos^2 c \cos^2 b}{\sin^2 c \sin^2 b}=\\
		= \frac{1 - \cos^2 c - \cos^2 b + \cos^2 c \cos^2 b -\cos^2 a}{\sin^2 c \sin^2 b} + \\
		+ \frac{2\cos a \cos c \cos b - \cos^2 c \cos^2 b}{\sin^2 c \sin^2 b} = \\
		= \frac{1 - \cos^2 c - \cos^2 b - \cos^2 a + 2\cos a \cos c \cos b}{\sin^2 c \sin^2 b}.
	\end{multline*}
	Извлекая квадратный корень из левой и правой части и деля их на $\sin a$ имеем
	\begin{equation*}
		\frac{\sin{A}}{\sin a} = \frac{\sqrt{1 - \cos^2 c - \cos^2 b - \cos^2 a + 2\cos a \cos c \cos b}}{\sin a \sin b \sin c}.
	\end{equation*}
	Заметим, что правая часть равенства циклична по переменным $a$, $b$ и $c$, следовательно, \term{сферическая теорема синусов} имеет вид
	\begin{equation}
		\frac{\sin A}{\sin a} = \frac{\sin B}{\sin b} = \frac{\sin C}{\sin c}.
	\end{equation}
	
	Напоследок получим \term{формулу пяти элементов}. Для этого запишем теорему косинусов в выразим в ней один из косинусов, применяя ее же:
	\begin{gather*}
		\cos a = \sin c \sin b \cos A + \cos c \cos b,\\
		\cos a = \sin c \sin b \cos A + \left( \sin a \sin b \cos C + \cos a \cos b \right)\cos b,\\
		\cos a - \cos a \cos^2 b = \sin c \sin b \cos A + \sin a \sin b \cos b \cos C,\\
		\cos a \sin^2 b = \sin c \sin b \cos A + \sin a \sin b \cos b \cos C,
	\end{gather*}
	\begin{equation}
		\cos a \sin b = \sin a \cos b \cos C + \sin c \cos A.
	\end{equation}

\term{Параллактический треугольник}~--- треугольник на небесной  сфере, образованный пересечением небесного меридиана, вертикального круга и часового круга светила. \imp{Вертикальный круг}~--- большой круг небесной сферы, проходящий через надир, зенит и светило. \imp{Часовой круг}~--- большой круг небесной сферы, проходящий через полюса мира и наблюдаемое светило.

Применяя теоремы синусов и косинусов к параллактическому треугольнику, нетрудно получить следующие соотношения:
\begin{gather}
\cos z=\sin\varphi\sin\delta+\cos\varphi\cos\delta\cos t\\
\sin z\sin A=\cos\delta\sin t\\
\sin z\cos A=-\cos\varphi\sin\delta+\sin\varphi\cos\delta\cos t
\end{gather}
\input{sections/spher.astro.sun-time.tex}
\subsection{Годичное движение Солнца}
В течение сидерического года Земля совершает полный оборот вокруг Солнца. Вследствие этого Солнце движется относительно далёких звёзд для наблюдателя на Земле. Это движение совершается по большому кругу небесной сферы, называемому \term{эклиптикой} и совпадающему с плоскостью орбиты Земли. Однако, в силу прецессии земной оси с периодом около 25765~лет, период такого движения равен \imp{тропическому году}, который длиннее сидерического года примерно на 20~мин~25~сек.

\begin{wrapfigure}[12]{r}{0.5\tw}
	\centering
	\vspace{-.9pc}
	\begin{tikzpicture}
		\begin{axis}[
			width	=	.5\tw,
			height	=	4.5cm,
			xlabel	=	{Прямое восхождение $\alpha^h$},
			ylabel	=	{Склонение $\delta^{\circ}$},
			extra y ticks	=	{23.44, -23.44},
			ytick = {-20, -10, 0, 10, 20},
			ymax	=	25,
			ymin	=	-25,
			xmax	=	24,
			xmin	=	0,
			xtick	=	{0, 4, 8, 12, 16, 20, 24},
			x dir = reverse
			]
			\addplot [domain=0:24, samples=100] {atan(sin(x*15)*tan(23.44))};
		\end{axis}
	\end{tikzpicture}
	\caption{График зависимости склонения Солнца от его прямого восхождения}
\end{wrapfigure}
В моменты, когда Солнце находится в \imp{точке весеннего равноденствия}  (20~марта, реже~21) его координаты: $\alpha=0^h$, $\delta=0^{\circ}$. Во время прохождения этой точки обе координаты Солнца растут. Так происходит до момента, пока Солнце не пройдет \imp{точку летнего солнцестояния} (21~июня, реже~20), после этого склонение Солнца начинает уменьшаться. В момент прохождения \imp{точки осеннего равноденствия} (22~или 23~сентября), координаты Солнца составляют $\alpha=12^h$, $\delta=0^{\circ}$. После прохождения \imp{точки зимнего солнцестояния} (22~или 21~декабря) склонение Солнца начинает увеличиваться.

Пренебрегая сферическими искажениями, годичный путь Солнца по небесной сфере можно считать синусоидой, откуда
\begin{equation}
	\delta=\varepsilon\cdot\sin \frac{2 \pi t}{T},
\end{equation}
где $t$~--- время, прошедшее с момента весеннего равноденствия, $T$~--- тропический год.

Более точная формула следует из сферической тригонометрии и имеет вид
\begin{equation}
	\delta=\arcsin\left(\sin\varepsilon\cdot\sin \frac{2 \pi t}{T}\right).
	\label{eq:delta-sun}
\end{equation}

Известно, что движение Солнца по эклиптике происходит неравномерно, поэтому данные формулы не являются абсолютно точными.

Прямое восхождение Солнца связано со склонением формулой
\begin{equation}
	\sin\alpha=\frac{\tg\delta}{\tg\varepsilon}.
	\label{eq:sin-alpha}
\end{equation}

Выражения \eqref{eq:delta-sun} и \eqref{eq:sin-alpha} следуют из формул перехода между экваториальной и эклиптической системами координат, получаемых из сферической тригонометрии.

\subsection{Рефракция}
\term{Рефракция}~--- явление преломления световых лучей, приходящих от небесных светил, в атмосфере планеты. Для наблюдателя на поверхности планеты с атмосферой положение светила будет отличаться от истинного на некоторый угол. Средняя величина рефракции у горизонта для земной атмосферы равна $35'$.

Для зенитного расстояния $z < 70^\circ$ величины рефракции можно определить по формуле
\begin{equation}
	\rho = 60.25'' \cdot \tg z' \cdot \frac{p}{760} \frac{273^{\circ}}{273^{\circ}+ t^{\circ}},
	\label{eq:refrac}
\end{equation}
где $t^{\circ}$~--- температура воздуха в$~^\circ$C, $p$~--- атмосферное давление в мм~рт.\,ст., $z'$~--- видимое зенитное расстояние. При н.~у.: $p = 760$ мм~рт.\,ст. и $t = 0^{\circ}$C, формула \eqref{eq:refrac} принимает вид
\begin{equation}
	\rho = 60.25'' \cdot \tg z'.
\end{equation}

\subsection{Сумерки}
\term{Сумерки}~--- часть суток, когда Солнце находится неглубоко под горизонтом. 
В зависимости от высоты Солнца под горизонтом различают \imp{гражданские}, \imp{навигационные} и \imp{астрономические} сумерки:\\
\begin{minipage}{0.54\tw}
	\begin{enumerate}
		\item Гражданские~--- от $0^{\circ}$ до $-6^{\circ}$
		\item Навигационные~--- от $-6^{\circ}$ до $-12^{\circ}$
		\item Астрономические~--- от $-12^{\circ}$ до $-18^{\circ}$
	\end{enumerate}
	Когда Солнце опускается ниже $-18^{\circ}$, наступает ночь.
\end{minipage}
\hfill
\begin{minipage}{0.44\tw}
	\centering
 	\includegraphics[width=\tw]{spher-astro-dusk.pdf}
	\captionof{figure}{Сумерки}
\end{minipage}
\newpage

	\section{Объекты космоса}
\subsection{Солнце}
\term{Солнце} --- центральное тело Солнечной системы, в нём сосредоточено 99,866\%  всей массы. Водород составляет~73\% общей массы Солнца, гелий~---~25\%. Остальные элементы: кислород, углерод, азот, магний, кремний, железо, сера, алюминий, натрий, кальций, никель и другие дают вклад всего~2\%.

По спектральной классификации Солнце~--- звезда типа G2V (жёлтый карлик на главной последовательности). Температура поверхности Солнца составляет $5 778$~К, поэтому Солнце светит почти в белом свете, но прямой свет Солнца у поверхности Земли приобретает жёлтый оттенок из-за рассеяния и поглощения коротковолновой части спектра в атмосфере.

Солнце вырабатывает энергию путём термоядерного синтеза. Каждую секунду в ядре около 4~млн.~тонн вещества превращается в лучистую энергию.\\

\term{Строение Солнца.}~~~В центре Солнца находится ядро с радиусом $150 $ -- $ 180$~тыс.~км, где идут термоядерные реакции. Плотность ядра около $1.5\times 10^5~\text{кг}/\text{м}^3$, а температура в его центре достигает $1.5\times 10^7$~К.

\begin{figure}[h!]
    \centering
    \includegraphics[width=0.8\textwidth]{sun.pdf}
    \caption{Строение Солнца. Фотография со спутника1 SOHO в фильтре $H_\alpha$ (негатив)}
\end{figure}
Над ядром, на расстояниях примерно от $0.25 R_\odot$ до $0.7R_\odot$ от его центра, находится \imp{зона лучистого переноса}. В этой зоне перенос энергии происходит главным образом с помощью излучения и поглощения фотонов. Температура в этой зоне лежит в интервале от $2\times10^6$~К сверху до $7\times10^6$~К снизу.

Над зоной лучистого  переноса (радиоактивная зона) находится \imp{конвективная зона}. Это слой толщиной примерно $2\times10^5$~км, в котором перенос энергии к поверхности совершается движением самого вещества. При приближении к поверхности конвективной зоны температура падает до $~5800$~К.

\textit{Фотосфера} --- видимая поверхность Солнца, по которой определяется размер Солнца. Эффективная температура фотосферы $T_\odot =  5778$~К.

\textit{Хромосфера} --- внешняя оболочка Солнца толщиной около 2000 км, окружающая фотосферу. Из хромосферы происходят горячие  выбросы вещества --- \textit{спикулы}. Температура хромосферы увеличивается с высотой до $2\times10^4$~К.

\textit{Солнечная корона} --- последний внешний слой Солнца, который состоит из протуберанцев и энергетических  извержений, образующих солнечный ветер. Средняя температура короны $2 \times 10^6$~К, а в некоторых частях достигает  и $20\times10^6$~К. Столь высокая температура обусловлена процессами, происходящими в магнитном поле звезды. Однако, несмотря на столь высокую температуру, корона видна лишь во время солнечных затмений, так как плотность её очень мала.

\imp{Вращение Солнца} происходит не твердотельно~--- угловая скорость на разных широтах отличается, при удалении от экватора она уменьшается. Период вращения Солнца на разных широтах можно найти, наблюдая за солнечными пятнами и другими образованиями в фотосфере звезды. На экваторе период вращения составляет 25.05 суток, к полюсу он увеличивается до 34 суток. По наблюдениям за пятнами в течение длительного периода при помощи метода наименьших квадратов можно найти зависимость углового перемещения пятна за сутки от гелиографической широты:
\begin{equation}
    \Delta\lambda=14.37^{\circ}-2.7^{\circ}\sin^2\varphi,
\end{equation}
где $\Delta\lambda$~--- угловое перемещение пятна, $\varphi$~--- гелиографическая широта. Данная зависимость верна только для широт $\varphi < 40^\circ$.

\subsection{Спектральная классификация звёзд}
Звёзды в зависимости от формы своего спектра делятся на \imp{спектральные классы}, основные из них представлены в Таблице \ref{tab:spectr-types}. Данная классификация называется \term{Гарвардской} и основывается на линиях,  чётко связанных с температурой звезды. Масса, радиус и светимость здесь приведены для средних представителей спектрального класса, лежащих на главной последовательности (V).

\begin{table}[h!]
    \centering
    \footnotesize
    \renewcommand{\arraystretch}{1.4}
    \renewcommand{\tabcolsep}{0pt}
    \begin{tabularx}{\tw}{|C{0.1}|C{0.3}|C{0.23}|C{0.13}|C{0.13}|C{0.13}|}
        \hline
        {\bfseries Класс} & {$\mathbf{T}$, К} & {\bfseries Цвет} & {$\mathbf{M}$, $M_{\odot}$} & {$\mathbf{R}$, $R_{\odot}$} & {$\mathbf{L}$, $L_{\odot}$}\\
        \hline
        O & $3 \times 10^4$ --- $6 \times 10^4$ & Голубой & 60 & 15 & $1.4 \times 10^6$\\

        B & $1 \times 10^4$ --- $3 \times 10^4$ & Бело-голубой & 18 & 7 & $2 \times 10^4$\\

        A & $7.5 \times 10^3$ --- $1 \times 10^4$ & Белый & 3.1 & 2.1 & 80\\

        F & $6 \times 10^3$ --- $7.5 \times 10^3$ & Жёлто-белый & 1.7 & 1.3 & 6\\

        G & $5 \times 10^3$ --- $6 \times 10^3$ & Жёлтый & 1.1 & 1.1 & 1.2\\

        K & $3.5 \times 10^3$ --- $5 \times 10^3$ & Оранжевый & 0.8 & 0.9 & 0.4\\

        M & $2 \times 10^3$ --- $3.5 \times 10^3$ & Красный & 0.3 & 0.4 & 0.04\\
        \hline
    \end{tabularx}
    \caption{Гарвардская спектральная классификация звёзд}
    \label{tab:spectr-types}
\end{table}

Внутри каждого спектрального класса выделяют~10 подклассов от~0 до~9, иногда используется десятичная запись (например, K4.5). Подклассы в начале последовательности называются ранними, а те, что в конце — поздними. Данная подклассификация говорит только о положении на прямой спектральных классов, она связана с температурой звезды сложным образом — функция перехода от номера подкласса к температуре не является ни строго линейной, ни строго логарифмической, и отличается для разных спектральных классов.

Ниже приведён график зависимости интенсивности линий от спектрального класса,\lookPicRef{pic:lines}, а также список основных линий для каждого из них:
\begin{itemize}
	\item \term{Класс O}. Спектр в основном состоит из линий  многократно ионизованных атомов, видны линии гелия, линии водорода слабые.
	\item \term{Класс B}. Отсутствуют линии He$\,$II, линии He$\,$I (4030$\,\mathring{\text{A}}$) наиболее интенсивны в B2 и пропадают к B9. Сильнее проявляются линии водорода, а также становится заметным характерный горб в районе бальмеровского скачка.
	\item \term{Класс A}. Очень сильные линии водорода, полностью пропадают линии гелия.
	\item \term{Класс F}. Ослабевают линии водорода, начинают появляться линии  металлов.
	\item \term{Класс G}. Продолжают слабеть линии водорода. Линии натрия (3934$\,\mathring{\text{A}}$ и 3968$\,\mathring{\text{A}}$) достигают максимума интенсивности в G0.
	\item \term{Класс K}. Доминируют линии металлов. Линии водорода уже незаметны, линии натрия хорошо видны. В K5 и далее начинают наблюдаться линии оксида титана.
	\item \term{Класс M}. Становятся сильнее линии оксида титана, очень много линий нейтральных металлов.
\end{itemize}

\begin{figure}[h!]
	\centering
	\tikzsetnextfilename{lines}
	\begin{tikzpicture}
		\begin{axis} [
                width   =    \tw,
                height  =    6cm,
                xmax    =    1,
                xmin    =    0,
                ymax    =    0.7,
                ymin    =    0,
                xtick={0.106, 0.236, 0.394, 0.547, 0.705, 0.855},
                ytick=\empty,
                scaled ticks=false,
                xticklabels={B0, A0, F0, G0, K0, M0},
%                extra x ticks={0.106, 0.236, 0.394, 0.547, 0.705, 0.855}, 
%    			extra x tick labels={30000, 10000, 7500, 6000, 5000, 3500}, 
%    			extra x tick style={xticklabel pos=top},
    			ylabel  =	{Относительная интенсивность},
    			grid		=	none,
    			ylabel style={at={(rel axis cs:-0.02,0.5)}}
            ]
				\addplot[black, smooth, thin, dashed] table[col sep=comma] {data/lines-He-II.csv} node at (rel axis cs:0.03, 0.7) {\scriptsize{He II}};
				\addplot[black, smooth, thin, dashed] table[col sep=comma] {data/lines-Si-IV.csv} node at (rel axis cs:0.07, 0.38) {\scriptsize{Si IV}};
				\addplot[black, smooth, thin] table[col sep=comma] {data/lines-He-I.csv} node at (rel axis cs:0.11, 0.7) {\scriptsize{He I}};
				\addplot[black, smooth, thin, dashed] table[col sep=comma] {data/lines-Si-III.csv} node at (rel axis cs:0.12, 0.3) {\scriptsize{Si III}};
				\addplot[black, smooth, thin, dashed] table[col sep=comma] {data/lines-Si-II.csv} node at (rel axis cs:0.24, 0.26) {\scriptsize{Si II}};
				\addplot[black, smooth, thin, dashed] table[col sep=comma] {data/lines-Mg-II.csv} node at (rel axis cs:0.25, 0.42) {\scriptsize{Mg II}};
				\addplot[black, smooth] table[col sep=comma] {data/lines-H-I.csv} node at (rel axis cs:0.27, 0.87) {\scriptsize{H I}};
				\addplot[black, smooth, thin] table[col sep=comma] {data/lines-Fe-II.csv} node at (rel axis cs:0.6, 0.43) {\scriptsize{Fe II}};
				\addplot[black, smooth, dashed] table[col sep=comma] {data/lines-Ca-II.csv} node at (rel axis cs:0.73, 0.77) {\scriptsize{Ca II}};
				\addplot[black, smooth, thin, dashed] table[col sep=comma] {data/lines-Fe-I.csv} node at (rel axis cs:0.77, 0.42) {\scriptsize{Fe I}};
				\addplot[black, smooth, thin] table[col sep=comma] {data/lines-Ca-I.csv} node at (rel axis cs:0.85, 0.5) {\scriptsize{Ca I}};
				\addplot[black, smooth, dashed] table[col sep=comma] {data/lines-Ti-O.csv} node at (rel axis cs:0.93, 0.77) {\scriptsize{Ti O}};
            \end{axis}
	\end{tikzpicture}
	\caption{Относительная интенсивность основных линий}
    \label{pic:lines}
\end{figure}

Помимо основных спектральных классов звёзд существуют дополнительные: W~--- звёзды Вольфа-Райе, очень тяжёлые яркие звёзды с температурой порядка 70000~К и интенсивными эмиссионными линиями спектра; L~--- звёзды или коричневые карлики с температурой 1500\,--\,2000~К и соединениями металлов в атмосфере; T~--- метановые коричневые карлики с температурой 700\,--\,1500~К; Y~--- очень холодные (метано-аммиачные) коричневые карлики с температурой ниже 700~К; C~--- углеродные звёзды, гиганты с повышенным содержанием углерода. Ранее относились к классам R и N.

Однако гарвардская классификация учитывает лишь влияние температуры на спектр, для более точного разбиения вместе с температурой требуется учитывать и светимость звезды. Классификация звёзд по светимости называется \term{Йеркской}, и основывается она на линиях, которые напрямую связаны с силой гравитации на поверхности звезды. Классификация сводится к разбиению на 8 классов:
\begin{itemize}
	\item \term{Класс Ia}. Наиболее яркие свехргиганты;
	\item \term{Класс Ib}. Менее яркие сверхгиганты;
	\item \term{Класс II}. Яркие гиганты;
	\item \term{Класс III}. Гиганты;
	\item \term{Класс IV}. Субгиганты;
	\item \term{Класс V}. Звёзды главной последовательности (карлики);
	\item \term{Класс VI}. Субкарлики;
	\item \term{Класс VII}. Белые карлики.
\end{itemize}
Запись спектрального класса представляет собой латинскую букву, арабское число и римское число, например, спектральный класс Солнца~---~G2V. Спектральный класс (показатель цвета) и абсолютная звёздная величина задают положение звезды на \imp{Диаграмме Герцшпрунга-Рассела}. Ниже, на~\picRef{pic:stellar-spectra}, приведены примеры спектров звёзд разных спектральных классов внутри класса светимости V.
\begin{figure}[p]
	\centering
	\foreach \x/\lege in {spectrum_O9V/O9V, spectrum_B8V/B8V, spectrum_A0V/A0V, spectrum_F0V/F0V, spectrum_G0V/G0V, spectrum_K3V/K3V} {
    \begin{subcaptionblock}{\tw}
        \tikzsetnextfilename{\x}
        \begin{tikzpicture}
            \begin{axis} [
                width   =    1.05\tw,
                height  =    3.5cm,
                xmax    =    10000,
                xmin    =    1000,
                scaled ticks=false,
                xticklabel=\empty,
                grid=both,
                legend cell align=left,
            	legend style={
                row sep = 0.8pc,
                draw=none,
                fill=none,
                font=\scriptsize,
                at={(axis cs:9000, 1)}, anchor=north west,
            	}
            ]
                \addplot[black, smooth] table[col sep=comma, x=wavelength, y=spectrum] {data/\x.csv};
            \end{axis}
            \node[anchor=north east] at (rel axis cs: 0.99, 1) {\scriptsize\lege};
    \end{tikzpicture}
    \end{subcaptionblock}
    \hspace{-0.5cm}
  	\hfill
    }
    \begin{subcaptionblock}{\tw}
        \tikzsetnextfilename{spectrum-M0V}
        \begin{tikzpicture}
            \begin{axis} [
                width   =    1.05\tw,
                height  =    3.5cm,
                xmax    =    10000,
                xmin    =    1000,
                scaled ticks=false,
                xtick={1000, 2000, 3000, 4000, 5000, 6000, 7000, 8000, 9000, 10000},
                xlabel  =    {Длина волны $\lambda,~\,\mathring{\text{A}}$},
                grid=both,
                legend cell align=left,
            	legend style={
                row sep = 0.8pc,
                draw=none,
                fill=none,
                font=\scriptsize,
                at={(axis cs:9000, 0.25)}, anchor=north west,
            	}
            ]
                \addplot[black, smooth] table[col sep=comma, x=wavelength, y=spectrum] {data/spectrum_M0V.csv};
            \end{axis}
            \node[anchor=north east] at (rel axis cs: 0.99, 0.37) {\scriptsize M0V};
    \end{tikzpicture}
    \end{subcaptionblock}
    \caption{Спектры звёзд}
    \label{pic:stellar-spectra}
\end{figure}


%    \includegraphics[width=10cm]{gr}

\term{Диаграмма Герцшпрунга-Рассела} демонстрирует зависимость светимости или абсолютной звёздной величины от спектрального класса, показателя цвета $(B-V)$ или эффективной температуры фотосферы звезды, \lookPicRef{pic:hr-diagram}.

Была предложена примерно в 1910 году независимо Эйнаром Герцшпрунгом и Генри Расселом. Диаграмма используется для классификации звёзд и соответствует современным представлениям о звёздной эволюции.

Около $90 \%$ звёзд находятся на главной последовательности. Их светимость обусловлена термоядерными реакциями превращения водорода в гелий. Выделяется также несколько ветвей проэволюционировавших звёзд-гигантов, в которых происходит горение гелия и более тяжёлых элементов. В левой нижней части диаграммы находятся полностью проэволюционировавшие белые карлики.

Мнемонические правила для запоминания спектральных классов: <<\textbf{O}h \textbf{B}e \textbf{A} \textbf{F}ine \textbf{G}irl, \textbf{K}iss \textbf{M}e \textbf{R}ight \textbf{N}ow \textbf{S}weetheart.>> и <<\textbf{В}ообразите: \textbf{О}дин \textbf{Б}ритый \textbf{А}нгличанин \textbf{Ф}иники \textbf{Ж}евал \textbf{К}ак \textbf{М}орковь --- \textbf{Р}азве \textbf{Н}е \textbf{С}мешно?>>

\begin{figure}[h!]
    \centering
    \vspace{-1pc}
    \tikzsetnextfilename{hr-diagram}
    \begin{tikzpicture}
         \begin{axis}[
                         height    =    10cm,
                         width    =    10cm,
                         ymax    =    14.,
                         ymin    =    -6.,
                         y dir    =    reverse,
                         xmax    =    2.,
                         xmin    =    -.5,
                         axis x line* = bottom,
                         axis y line* = right,
                         xlabel  =   $B-V$,
                         y label style = {at={(axis description cs: 1.07, 0.5)}, rotate=180},
                         ylabel    =    {Абсолютная звёздная величина $M$, $\!~^m$}
                     ]
            \ifthenelse{\boolean{useLightPlotVersion}}{}{
                \addplot+[only marks, mark = o, mark options={scale=0.2, darkgray}] table[x=BV, y=M, col sep = comma]{data/gr-plot.csv};
            }
         \end{axis}
         \begin{semilogyaxis}[
                         height    =    10cm,
                         width    =    10cm,
                         ymax    =    2.088e4,
                         ymin    =    2.088e-4,
                         xmax    =    2.,
                         xmin    =    -.5,
                         minor x tick num = 0,
                         minor y tick num = 01,
                         xtick = {-0.264, 0, 0.3, 0.58, 0.791, 1.57},
                         xticklabels = {B0, A0, F0, G0, K0, M0},
                         axis x line* = top,
                         axis y line* = left,
                         xlabel    =    {Спектральный класс},
                         x label style = {at={(axis description cs: 0.5, 1.03)}, rotate=0},
                         ylabel    =    {Светимость $L$, $L_\odot$},
                         ymajorgrids     =    false,
                         xmajorgrids     =    false
                    ]
        \end{semilogyaxis}
     \end{tikzpicture}
     \caption{Диаграмма Герцшпрунга--Рассела}
    \label{pic:hr-diagram}
\end{figure}

\begin{wrapfigure}[11]{r}{0.5\tw}
    \vspace{-1pc}
    \centering
    \footnotesize
    \renewcommand{\arraystretch}{1.4}
    \renewcommand{\tabcolsep}{0pt}
    \begin{tabularx}{0.5\tw}{|C{0.36}|C{0.22}|C{0.42}|}
        \hline
        {\bfseries Обозначение} & {\bfseries Элемент} & {\bfseries Длина волны, $\mathring{\text{A}}$} \\
        \hline
        A & $\text{O}_2$ & 7594\\

        B & $\text{O}_2$ & 6867\\

        C & $\text{H}_\alpha$ & 6563\\

        $\text{D}_{12}$ & Na & 5896, 5890\\

        F & $\text{H}_{\beta}$ & 4958\\

        G' & $\text{H}_{\gamma}$ & 4340\\

        H & Na & 3968\\
        
        K & Na & 3934\\
        \hline
    \end{tabularx}
    \caption{Линии Фраунгофера}
    \label{tab:fraunhofer-lines}
\end{wrapfigure}
\term{Линии Фраунгофера} — линии поглощения в спектре Солнца подробно описанные Йозефом Фраунгофером в 19 веке. Наиболее яркие среди них называются латинскими буквами от A до K и упорядочены в порядке \imp{уменьшения} длины волны, см. Таблицу\,\ref{tab:fraunhofer-lines}.











\subsection{Переменные звёзды}
\term{Переменные звёзды}~--- звёзды, у которых наблюдаются колебания блеска.   Для отнесения звезды к разряду переменных достаточно, чтобы блеск звезды хотя бы однажды претерпел изменение.

Переменные звёзды делятся на две большие группы: \imp{затменные} и \imp{физические}, причём физические подразделяются на \imp{пульсирующие} и \imp{эруптивные}.

\begin{wrapfigure}[11]{l}{0.47\tw}
    \centering
    \vspace{-1.2pc}
    \tikzsetnextfilename{light-curve-d-cep}
    \begin{tikzpicture}
        \begin{axis}[
            height    =    4.5cm,
            width    =    .5\tw,
            xlabel    =    {Фаза},
            ylabel    =    {Блеск $m$, $~^m$},
            ymax    =    8.,
            ymin    =    6.5,
            y dir    =    reverse,
            xmax    =    1,
            xmin    =    0
        ]
            \addplot+[only marks, mark = o, mark options={scale=0.2, black}] table[x=f, y=m, col sep = comma]{data/light-curve-D-Cep.csv};
        \end{axis}
    \end{tikzpicture}
    \caption{Кривая блеска переменной типа $\delta$\,Cep}
    \label{pic:d-cep}
\end{wrapfigure}
К \term{пульсирующим} переменным  относят те звёзды, переменность которых вызвана процессами, происходящими в их недрах. Эти процессы приводят к периодическому изменению температуры поверхности и радиуса фотосферы, а вместе с тем и блеска звезды. Период переменности варьируется в пределе от долей суток до~нескольких~лет в зависимости от типа переменной.

Классический пример пульсирующих переменных звёзд~--- \imp{цефеиды}, названные в честь первой открытой переменной данного типа~--- $\delta$\,Cep. Абсолютную звёздную величину $M$ и период $T$ (в сутках) цефеид связывает соотношение
\begin{equation}
    M = -1.43^m - 2.81\lg T.
\end{equation}

\begin{wrapfigure}[11]{r}{0.47\tw}
    \centering
    \vspace{-1.2pc}
    \tikzsetnextfilename{light-curve-rr-lyr}
    \begin{tikzpicture}
        \begin{axis}[
            height    =    4.5cm,
            width    =    .5\tw,
            xlabel    =    {Фаза},
            ylabel    =    {Блеск $m$, $~^m$},
            ymax    =    12.5,
            ymin    =    11.5,
            y dir    =    reverse,
            xmax    =    1,
            xmin    =    0
        ]
            \addplot+[only marks, mark = o, mark options={scale=0.2, black}] table[x=f, y=m, col sep = comma]{data/rr-lyr.csv};
        \end{axis}
    \end{tikzpicture}
    \caption{Кривая блеска переменной типа RR~Lyr} % http://www.astrouw.edu.pl/asas/?psect=acvs&page=details&id=000405-1659.8
    \label{pic:light-curve-rr-lyr}
\end{wrapfigure}
Ещё один класс пульсирующих переменных звёзд~--- \imp{переменные типа RR~Lyr}, прототипом которого стала звезда RR~Lyr. Такие звёзды довольно стары и маломассивны. Они являются гигантами спектрального класса А, лежащими на горизонтальной ветви диаграммы Герцшпрунга\,--\,Рассела. Светимости этих звёзд различаются слабо и составляют порядка $40L_\odot$. Поэтому они, как и цефеиды, используются в качестве стандартных свеч.

К \term{эруптивным} переменным звёздам относятся звёзды, меняющие свой блеск нерегулярно или единожды за время наблюдений. Все изменения блеска эруптивных звёзд связывают с бурными процессами и вспышками в их хромосферах и коронах. К таким, например, относятся \imp{новые} и \imp{сверхновые}.

\term{Затменно-переменные} звёзды --- системы из двух звёзд, суммарный блеск которых периодически изменяется с течением времени. Причиной изменения блеска могут быть затмения звёзд друг другом, или изменение их формы взаимной гравитацией в тесных системах. На \picRef{pic:w-uma}\,--\,\ref{pic:b-lyr}  представлены кривые блеска затменно-переменных звёзд трёх основных типов.

\begin{figure}[h!]
    \centering
    \begin{minipage}[c]{0.49\tw}
        \tikzsetnextfilename{light-curve-w-uma}
        \begin{tikzpicture}
            \begin{axis}[
                height    =    4.5cm,
                width    =    \tw,
                xlabel    =    {Фаза},
                ylabel    =    {Блеск $m$, $~^m$},
                ymax    =    1.1,
                ymin    =    -.1,
                y dir    =    reverse,
                xmax    =    1,
                xmin    =    0
                ]

                \addplot[smooth] table[x=t, y=m, col sep = comma]{data/light-curve-W-UMa.csv};
            \end{axis}
        \end{tikzpicture}
    \end{minipage}
    \hfill
    \begin{minipage}[c]{0.49\tw}
        \centering
        \includegraphics[width = .9\tw]{w-uma}
    \end{minipage}
    \caption{Кривая блеска переменной типа W\,UMa}
    \label{pic:w-uma}
\end{figure}

\begin{figure}[h!]
    \centering
    \begin{minipage}[c]{0.49\tw}
        \tikzsetnextfilename{light-curve-b-per}
        \begin{tikzpicture}
            \begin{axis}[
                height    =    4.5cm,
                width    =    \tw,
                xlabel    =    {Фаза},
                ylabel    =    {Блеск $m$, $~^m$},
                ymax    =    .7,
                ymin    =    -.1,
                y dir    =    reverse,
                xmax    =    1.,
                xmin    =    .0
                ]
                \addplot[smooth] table[x=t, y=m, col sep = comma]{data/light-curve-B-Per.csv};
            \end{axis}
        \end{tikzpicture}
    \end{minipage}
    \hfill
    \begin{minipage}[c]{0.49\tw}
        \centering
        \includegraphics[width=.9\tw]{b-per}
    \end{minipage}
    \caption{Кривая блеска переменной типа $\beta$\,Per}
\end{figure}

\begin{figure}[h!]
    \centering
    \begin{minipage}[c]{0.49\tw}
        \tikzsetnextfilename{light-curve-b-lyr}
        \begin{tikzpicture}
            \begin{axis}[
                height    =    4.5cm,
                width    =    \tw,
                xlabel    =    {Фаза},
                ylabel    =    {Блеск $m$, $~^m$},
                ymax    =    .7,
                ymin    =    -.1,
                y dir    =    reverse,
                xmax    =    1,
                xmin    =    .0
                ]
                \addplot[smooth] table[x=t, y=m, col sep = comma]{data/light-curve-B-Lyr.csv};
            \end{axis}
        \end{tikzpicture}
    \end{minipage}
    \hfill
    \begin{minipage}[c]{0.49\tw}
        \centering
        \includegraphics[width = .9\tw]{b-lyr}
    \end{minipage}
    \caption{Кривая блеска переменной типа $\beta$\,Lyr}
    \label{pic:b-lyr}
    \vspace{-.8pc}
\end{figure}

\subsection{Вырожденные звёзды}
\term{Вырожденные звезды}~--- звезды, в которых силам гравитации противостоят силы давление вырожденного газа. К таким относятся \imp{белые карлики} и \imp{нейтронные звезды}.

\begin{figure}[!h]
    \centering
    \begin{minipage}[c]{0.49\tw}
        \tikzsetnextfilename{light-curve-b-lyr}
        \begin{tikzpicture}
            \begin{axis}[
                height    =    4.5cm,
                width    =    \tw,
                xlabel    =    {Фаза},
                ylabel    =    {Блеск $m$, $~^m$},
                ymax    =    .7,
                ymin    =    -.1,
                y dir    =    reverse,
                xmax    =    1,
                xmin    =    .0
                ]
                \addplot[smooth] table[x=t, y=m, col sep = comma]{data/light-curve-B-Lyr.csv};
            \end{axis}
        \end{tikzpicture}
    \end{minipage}
    \hfill
    \begin{minipage}[c]{0.49\tw}
        \centering
        \includegraphics[width = .9\tw]{b-lyr}
    \end{minipage}
    \caption{Кривая блеска переменной типа $\beta$\,Lyr}
    \label{pic:b-lyr}
    \vspace{-.8pc}
\end{figure}
\term{Белые карлики}~--- проэволюционировавшие звёзды лишённые собственных источников термоядерной энергии и светящие за счёт остывания. Масса белого карлика находится в диапазоне от $0.6M_{\odot}$ до $1.44 M_{\odot}$. Верхняя границы массы белого карлика называется пределом Чандрасекара, звезда с массой больше данного предела не может существовать как белый карлик. Радиус белых карликов примерно в $10^2$ раз меньше солнечного, т.е. можно считать, что $R_\text{БК} \simeq R_\oplus$. Плотность белых карликов лежит в диапазоне $10^7$\,--\,$10^{10}$~$\text{кг}/\text{м}^3$.

\term{Нейтронная звезда}~--- сверхплотная звезда, образующаяся в результате взрыва Сверхновой. Вещество нейтронной звезды состоит в основном из нейтронов. Масса нейтронной звезды лежит в пределах от $0.1M_{\odot}$ до $2$\,--\,$2.8M_{\odot}$ (предел Оппенгеймера-Волкова). Размер данной звезды составляет лишь $10$\,--\,$20$~км, а плотность составляет $10^{16}$\,--\,$10^{18}$ $\text{кг}/\text{м}^3$.  Дальнейшему гравитационному сжатию нейтронной звезды препятствует давление ядерной материи, возникающее за счёт взаимодействия нейтронов. Так как нейтронные звёзды образуются в результате  коллапса массивных звёзд, то из-за сохранения момента импульса скорость их вращения может достигать $10^5$~км/с. При наличии сильного магнитного поля и быстром вращении нейтронная звзеда может наблюдаться с Земли как \term{пульсар}.

\subsection{Чёрные дыры}
\term{Чёрная дыра}~(ЧД)~--- область пространства-времени с массой $M$, гравитационное притяжение которой настолько велико, что покинуть её не могут даже объекты, движущиеся со скоростью света $c$. Граница этой области называется \imp{горизонтом событий}, а её характерный размер~$R_G$~--- \imp{гравитационным радиусом}, для величины которого справедливо равенство
\begin{equation}
    R_G = \frac{2 G M}{c^2}.
\end{equation}

Минимальная масса ЧД составляет около $2.5M_{\odot}$. А плотность ЧД определяется отношением ее массы~$M$ к~объему~$V$, следовательно
\begin{equation}
    \rho = \frac{M}{V} = \frac{3c^6}{32\pi M^2G^3}.
\end{equation}

\term{Эффект излучения} (испарения) \term{Хокинга}~--- эффект, при котором гравитационное поле черной дыры поляризует вакуум, в результате чего возможно образование не только виртуальных, но и реальных пар частица~--античастица. Одна из частиц, оказавшаяся чуть ниже горизонта событий, падает внутрь чёрной дыры, а другая, оказавшаяся чуть выше горизонта, улетает, унося энергию (то есть часть массы) чёрной дыры. Для мощности излучения ЧД справедлива формула
\begin{equation}
    L = \frac{h c^6}{30720 \pi^2 G^2 M^2},
\end{equation}
где $h$ --- постоянная Планка. Спектр хокинговского излучения для безмассовых полей оказался строго совпадающим с излучением абсолютно чёрного тела, что позволило приписать ЧД температуру, равную
\begin{equation}
    T = \frac{h c^3}{16 \pi^2 k G M},
\end{equation}
где $k$ --- постоянная Больцмана.

\subsection{Галактики}
\term{Морфологическая классификация галактик}~--- система разделения галактик на группы по визуальным признакам, используемая в астрономии. Наиболее известной является классификация, разработанная Хабблом и дополненная другими учеными.~\cite{hubble_fork}
\begin{figure}[h!]
    \centering
    \vspace{-.9pc}
    \includegraphics[width=0.65\tw]{hubble-fork.pdf}
    \caption{<<Вилка Хаббла>>}
\end{figure}

Согласно данной классфикации галактики делятся на 4 типа:
\begin{enumerate}[itemsep=3pt, label={\arabic*.}, leftmargin=1pc]
    \item{\term{Эллиптические галактики} имеют гладкую эллиптическую форму без отличительных деталей с равномерным уменьшением яркости от центра к периферии. Обозначаются буквой E с индексом. Индекс можно рассчитать по формуле
    \begin{equation}
        i = 10 \cdot \left(1 - \frac{b}{a}\right),
    \end{equation}\nopagebreak
    где $a$ и $b$~--- большая и малая полуоси видимого эллипса.

    К эллиптическим галактикам с абсолютной звёздной величиной меньше $-18^m$ применимо соотношение Фабер-Джексона~\cite{faber_jackson}\cite{faber_jackson_corrected}:
    \begin{equation}
        \begin{cases}
            \begin{aligned}
                &L \propto \sigma^{3.1} &~\text{при}~M > -21^m,\\
                &L \propto \sigma^{15.0} &~\text{при}~M < -21^m,   
            \end{aligned} 
       \end{cases}
    \end{equation}
    где $\sigma$~--- дисперсия скоростей вещества в галактике, а $M$~--- абсолютная звездная величина галактики}
    \item{\term{Спиральные галактики} состоят из уплощенного диска из звёзд и газа, в центре которого находится сферическое уплотнение, называемое балджем, а также обширного сферического гало. Спиральные галактики обозначаются SB при наличии бара (перемычки между рукавами) или S при отсутствии бара. В зависимости от размеров ядра и балджа галактики делят на 3 группы: a, b и c. Для галактик Sa характерен большой балдж, для галактик Sc~--- маленький. Галактики Sb представляют собой нечто среднее между галактиками Sa и Sc.

    Светимость спиральных галактик $L$ связана с  максимальной скоростью вращения $v_\text{макс}$ вещества в них эмпирическим \term{соотношением Талли-Фишера}~\cite{tully_fisher}\cite{tully_fisher_correction}:
    \begin{equation}
        L \propto v_\text{макс}^4.
    \end{equation}
    Абсолютная звёздная величина Млечного пути $M_\text{MW} \simeq -21^m$.}
    \item{\term{Неправильные или иррегулярные галактики}~--- галактики, лишенные как вращательной симметрии, так и значительного ядра. Обозначение: Irr.}
    \item{\term{Линзовидные галактики}~--- галактики, являющиеся переходными между спиральными и эллиптическими. Обозначения: S0, SB0.}
\end{enumerate}


\subsection{Другие объекты}
\begin{minipage}{0.63\tw}
    \term{Шаровое звёздное скопление}~--- скопление звёзд, состоящее из нескольких сотен тысяч светил, тесно связанных гравитацией. Млечный путь насчитывает около 160 шаровых звёздных скоплений. Диаметры шаровых скоплений составляют 20\,--\,60~пк, массы~--- $10^4$\,--\,$10^6$~солнечных.\\

    \term{Планетарная туманность}~--- система из звезды, называемой ядром туманности, и симметрично окружающей ее светящейся газовой оболочки. Планетарные туманности образуются при сбросе внешних слоёв (оболочек) красных гигантов и сверхгигантов с массой от $0.8M_\odot$ до $8M_\odot$ на завершающей стадии их эволюции. Характерный размер~--- 1\,--\,2~св.\,лет.\\

    \term{Рассеянное звёздное скопление}~--- слабо связанная группа из сотен или тысяч звёзд, сформировавшихся из одного гигантского \imp{молекулярного облака} и имеющих одинаковый возраст. Рассеянные звёздные скопления встречаются только в тонком диске Галактики, их типичный диаметр~--- несколько парсек.
\end{minipage}
\hfill
\begin{minipage}{0.32\tw}
    \centering
    \vspace{-1.2pc}
    \includegraphics[width = .7\tw]{m13.pdf}
    \captionof{figure}{Шаровое скопление M13 (негатив)}
    \vspace{1pc}
    \includegraphics[width = .7\tw]{m57.pdf}
    \captionof{figure}{Пла\-не\-тар\-ная туманность M57 (негатив)}
    \vspace{1.2pc}
    \includegraphics[width = .7\tw]{m45.pdf}
    \captionof{figure}{Рассеянное звёздное скопление M45 (негатив)}
\end{minipage}


	\newpage
\section{Математика}
\subsection{Вектор}
\imp{Вектором} называется \imp{направленный отрезок} или упорядоченная пара точек. Уточним это понятие: \term{вектором} будем называть класс эквивалентности направленных отрезков, то есть все такие направленные отрезки, коллинеарные и равные по длине друг другу.

    Векторы $\vec{x}_1, \ldots, \vec{x}_n$ называются \term{линейно зависимыми}, если найдётся такой набор констант $\alpha_1, \ldots, \alpha_n$, что $|\alpha_1| + \ldots + |\alpha_n| \not= 0$ и $\alpha_1 \vec{x}_1 + \ldots + \alpha_n \vec{x}_n = \vec{0}$. Иными словами векторы линейно зависимы, когда существует их нетривиальная линейная комбинация, равная нулевому вектору.

    С другой стороны, векторы $\vec{x}_1, \ldots, \vec{x}_n$ называются \term{линейно независимыми}, если из условия $\alpha_1 \vec{x}_1 + \ldots + \alpha_n \vec{x}_n = \vec{0}$ следует, что $\forall i : \alpha_i = 0$.

    Нетрудно догадаться, что размер линейно независимой системы векторов, ограничен. \imp{Максимальной линейно независимой системой векторов} $\{\vec{x}_1, \ldots, \vec{x}_n\}$ принято называть такую линейно независимую систему векторов $\LL$, что любой вектор $\vec{v} \not\in \LL$ можно представить в виде $\vec{v} = \alpha_1 \vec{x_1} + \ldots + \alpha_n \vec{x}_n$, а сами векторы $\vec{x}_1, \ldots, \vec{x}_n$, очевидно, линейно независимы.

    Здесь важно, что такое разложение единственно. Действительно, предположим, что также верно представление $\vec{v} = \beta_1 \vec{x_1} + \ldots + \beta_n \vec{x}_n$, но тогда $\vec{0} = (\alpha_1 - \beta_1) \vec{x}_1 + \ldots + (\alpha_n - \beta_n) \vec{x}_n$, а так как векторы $\vec{x}_1, \ldots, \vec{x}_n$ линейно независимы, то $\forall i: a_i = b_i$, следовательно разложение единственно.

    Упорядоченную максимальную линейно независимую комбинацию векторов $\{\vec{e}_1, \ldots, \vec{e}_n \}$ называют \term{базисом} или, что очевидно тоже самое, упорядоченная система линейно независимых векторов, что любой другой вектор пространства есть их линейная комбинация.

    Так как разложение вектора по линейно независимой системе единственно, а базисные векторы $\vec{e}_1, \ldots, \vec{e}_n$ упорядочены, то коэффициенты разложения $\alpha_1, \ldots, \alpha_n$ однозначно определяют вектор и называются его \imp{компонентами} или \imp{координатами}.

    Углубимся в математику. \term{Полем} $F$ называется такое множество с введенными на нем операциями сложения $+:F \times F  \rightarrow F$ и умножения $\cdot: F \times F \rightarrow F$, что выполнены следующие свойства:
    \begin{enumerate}
        \item  $\forall a, b \in F: a+ b = b + a$;
        \item $\forall a, b, c \in F : (a + b) + c = a + ( b + c)$;
        \item  $\exists 0 \in F~~\forall a \in F: a  + 0 = 0 + a = a$;
        \item  $\forall a \in F ~~ \exists (-a) \in F: a + (-a) = 0$;
        \item  $\forall a, b \in F: a \cdot b = b \cdot a$;
        \item  $\forall a, b, c \in F : (a \cdot b) \cdot c = a \cdot ( b \cdot c)$;
        \item $\exists 1 \in F~~\forall a \in F: a \cdot 1 = 1 \cdot a = a$;
        \item  $\forall a \in F, a \not = 0 \quad \exists a^{-1} \in F: a \cdot a^{-1} = a^{-1} \cdot a = 1$;
        \item $\forall a, b, c \in F: (a + b) \cdot c = a \cdot c + b \cdot c$.
    \end{enumerate}

    \term{Векторное (линейное) пространство} $V(F)$ над полем $F$~--- упорядоченная четверка $(V, F, + , \cdot)$, где определены операции сложения векторов $+ : V \times V \rightarrow V$ и умножения на скаляр $F \times V \rightarrow V$ такие, что
    \begin{enumerate}
        \item $\forall \vec{x}, \vec{y} \in V: \vec{x} + \vec{y} = \vec{y} + \vec{x}$;
        \item $ \forall \vec{x}, \vec{y}, \vec{z} \in V: (\vec{x} + \vec{y}) + \vec{z} =  \vec{x} + (\vec{y} + \vec{z})$;
        \item $\exists \vec{0} \in V~~\forall \vec{x} \in F: \vec{x} + \vec{0} = \vec{x}$;
        \item $\forall \vec{x} \in V~~\exists (-\vec{x}) \in V: \vec{x} + (-\vec{x}) = \vec{0}$;
        \item $\forall \alpha, \beta \in F~~\forall \vec{x} \in V: \alpha(\beta \vec{x}) =  (\alpha \beta) \vec{x}$;
        \item $\forall \vec{x} \in V: 1 \cdot \vec{x} = \vec{x}$;
        \item $\forall \alpha, \beta \in F~~\forall \vec{x} \in V: (\alpha + \beta) \vec{x} = \alpha \vec{x} + \beta \vec{x}$;
        \item $\forall \alpha \in F~~\forall \vec{x}, \vec{y} \in V: \alpha \vec{x} + \alpha \vec{y} = \alpha ( \vec{x} + \vec{y})$.
    \end{enumerate}

    \term{Размерность} векторного пространства $\dim V$ равна размеру базиса в нем. Важно, что нет различия в том, размер какого именно базиса принять за размерность пространства. Действительно, если размер базиса $\LL_1$ строго меньше размера базиса $\LL_2$, значит базис $\LL_1$ не является максимальной линейно независимой системой векторов, либо векторы из $\LL_2$ линейно зависимы, так как размер $\LL_2$ больше размера базиса $\LL_1$. Полученное противоречие доказывает, что размеры всех базисов векторного пространства равны между собой, а значит, определение корректно.

\subsection{Системы координат}
\label{sec:coord-systems}

\begin{wrapfigure}{r}{0.27\tw}
    \centering
    \vspace{-1pc}
    \tikzsetnextfilename{math-coord-sys-decart}
    \begin{tikzpicture}[scale=1]
        \footnotesize
        \coordinate (O) at (0, 0);
        \coordinate (E1) at (2, 0);
        \coordinate (E2) at (1, 0.5);
        \coordinate (E3) at (0.5, 1.5);
        \coordinate (M) at (2, 2);

        \draw[semithick, -latex] (O) -- (E1);
        \draw[semithick, -latex] (O) -- (E2);
        \draw[semithick, -latex] (O) -- (E3);
        \draw[thick, -latex] (O) -- (M);

        \point (O);
        \point (M);

        \draw (1, 0) node[anchor=north]{$\vec{e}_1$};
        \draw (0.66, 0.33) node[anchor=north]{$\vec{e}_2$};
        \draw (0.25, 0.75) node[anchor=east]{$\vec{e}_3$};
        \draw (1, 1.05) node[anchor=south]{$\vec{r}$};
        \draw (M) node[anchor=north west]{$M$};
        \draw (O) node[anchor=north east]{$O$};
    \end{tikzpicture}
    \caption{}
    \vspace{-1pc}
    \label{pic:math-coord-sys-decart}
\end{wrapfigure}
Зафиксируем точку $O$ в пространстве и рассмотрим произвольную точку $M$. Вектор $\vec{r} = \overrightarrow{OM}$ называется радиус-вектором точки $M$. Пусть в рассматриваемом пространстве также выбран базис $\{\vec{e}_1, \ldots, \vec{e}_n \}$, тогда совокупность точка $O$ и базиса называется \term{декартовой системой координат}. Причем точка $O$~--- начало координат, а базисные векторы $\vec{e}_1, \ldots, \vec{e}_n$ задают координатные оси.

\begin{wrapfigure}{r}{0.27\tw}
    \centering
    \vspace{-1pc}
    \tikzsetnextfilename{math-coord-sys-ord-basis}
    \begin{tikzpicture}[scale=1]
        \footnotesize
        \coordinate (O) at (0, 0);
        \coordinate (E1) at (1.5, 0);
        \coordinate (E2) at (-1, -0.5);
        \coordinate (E3) at (0, 1.5);
%        \coordinate (M) at (2, 2);

        \draw[semithick, -latex] (O) -- (E1);
        \draw[semithick, -latex] (O) -- (E2);
        \draw[semithick, -latex] (O) -- (E3);
%        \draw[thick, -latex] (O) -- (M);

        \point (O);
%        \point (M);

        \draw (-0.5, -0.25) node[anchor=south]{$\vec{e}_1$};
        \draw (0.75, 0) node[anchor=south]{$\vec{e}_2$};
        \draw (0, 0.75) node[anchor=east]{$\vec{e}_3$};
%        \draw (1, 1.05) node[anchor=south]{$\vec{r}$};
%        \draw (M) node[anchor=south]{$M$};
        \draw (0, -0.1) node[anchor=north]{$O$};

        \draw (0, 0.25) -- (0.25, 0.25) -- (0.25, 0);
        \draw (0.25, 0) -- (0.027, -0.112) -- (-0.223, -0.112);
        \draw (0, 0.25) -- (-0.223, 0.138) -- (-0.223, -0.112);
    \end{tikzpicture}
    \caption{}
    \vspace{-1pc}
    \label{pic:math-coord-sys-ord-basis}
\end{wrapfigure}
Однако пользоваться представлением векторов в произвольном базисе довольно сложно, поэтому рассмотрим специальный тип~--- \term{орто\-нор\-ми\-рованный базис}~--- это такой базис, базисные векторы которого попарно ортогональны, и длина каждого равна единице.

\term{Прямоугольной декартовой системой координат} (ПДСК) называют декартову систему координат с ортонормированным базисом. В практических задачах использовать ПДСК не всегда удобно, поэтому также рассмотрим другие системы координат. 

\begin{wrapfigure}{r}{0.27\tw}
    \centering
    \vspace{-1pc}
    \tikzsetnextfilename{math-coord-sys-polar}
    \begin{tikzpicture}[scale=1]
        \clip(-1.05, -1.55) rectangle (2.1, 1.55);

        \footnotesize
        \coordinate (O) at (0, 0);
        \coordinate (E1) at (2, 0);
%        \coordinate (E2) at (1, 0.5);
%        \coordinate (E3) at (0.5, 1.5);
        \coordinate (M) at (1, 1);

        \draw[semithick, -latex] (O) -- (E1);
%        \draw[semithick, -latex] (O) -- (E2);
%        \draw[semithick, -latex] (O) -- (E3);
        \draw[thick, -latex] (O) -- (M);
        \draw[-latex] (0.65, 0) arc(0:45:0.65);

        \foreach \r in {0.5, 1, 1.5} {
            \draw (O) circle(\r);
        };

        \point (O);
        \point (M);

        \draw (1.2, 0) node[anchor=north]{$\vec{r}_0$};
%        \draw (0.66, 0.33) node[anchor=north]{$\vec{e}_2$};
%        \draw (0.25, 0.75) node[anchor=east]{$\vec{e}_3$};
        \draw (0.5, 0.5) node[anchor=south]{$\vec{r}$};
        \draw (0.55, 0.3) node[anchor=west]{$\varphi$};
        \draw (1, 0.9) node[anchor=south east]{$M$};
        \draw (O) node[anchor=north east]{$O$};
    \end{tikzpicture}
    \caption{}
    \label{pic:math-coord-sys-polar}
\end{wrapfigure}
На плоскости, то есть в пространстве $\R^2$, имеющем размерность два, часто применяется \term{полярная система координат}. В ней координатами вектора является его длина $r$~--- расстояние до точки от начала отсчёта, и угол $\varphi$ радиус-вектора с начальной ось.

Пусть $(x,y)$~--- координаты некоторого вектора в ПДСК на $\R^2$, тогда не сложно получить, что его координаты в полярной системе координат (начальная ось совпадает с осью $Ox$)  удовлетворяют следующим соотношениям:
\begin{equation}
    \begin{cases}
        r = \sqrt{x^2 + y^2},\\
        \sin \varphi = y/r,\\
        \cos \varphi = x/r.
    \end{cases}
    \quad \Leftrightarrow \quad
    \begin{cases}
        x = r \cos \varphi,\\
        y = r \sin \varphi.
    \end{cases}
\end{equation}

\begin{wrapfigure}{r}{0.35\tw}
    \centering
%    \vspace{-1pc}
    \tikzsetnextfilename{math-coord-sys-cyl}
    \begin{tikzpicture}[scale=1]
        \clip(-1.3, -1.1) rectangle (2.5, 2.5);

        \footnotesize
        \coordinate (O) at (0, 0);
        \coordinate (E1) at (2.3, 0);
        \coordinate (E2) at (0, 2.3);
        \coordinate (E3) at (-1, -1);
        \coordinate (M) at (0.8, 1.3);
        \coordinate (M') at (0.8, -0.48);
        \coordinate (M'') at (1, -0.6);

        \draw[semithick, -latex] (O) -- (E1);
        \draw[semithick, -latex] (O) -- (E2);
        \draw[semithick, -latex] (O) -- (E3);
        \draw[thick, -latex] (O) -- (M);

        \draw [dashes] (0, 1.78) -- (M) -- (M');
        \draw [dashes] (M'') -- (O);
        \draw [dashes] (1.28, 0) -- (M') -- (-0.48, -0.48);

        \draw (O) ellipse(2 and 0.7);
        \draw (0, 1.78) ellipse(2 and 0.7);

        \draw [-latex](-0.2, -0.2) arc(-99.4:-75.76:1.22);

        \foreach \p in {O, M, M', M''} {
            \point (\p);
        };

        \draw (0.5, 0.65) node[anchor=south east]{$\vec{r}$};
        \draw (0.05, -0.15) node[anchor=north]{$\varphi$};
        \draw (0.42, -0.2) node[anchor=west]{$r$};
        \draw (0.8, 0.4) node[anchor=west]{$h$};
        \draw (M) node[anchor=south west]{$M$};
        \draw (O) node[anchor=south east]{$O$};

        \draw (E1) node[anchor=north]{$y$};
        \draw (E2) node[anchor=east]{$z$};
        \draw (E3) node[anchor=east]{$x$};
    \end{tikzpicture}
    \caption{}
    \label{pic:math-coord-sys-cyl}
    \vspace{-2pc}
\end{wrapfigure}
Теперь пусть $(x, y, z)$~--- координаты некоторого вектора в ПДСК на~$\R^3$. Обозначим за $h$~--- длину проекции этого вектора на ось $z$, $r$~--- длину его проекции на плоскость $Oxy$, $\varphi$~--- угол между проекцией на плоскость $Oxy$ и осью $Ox$. Тогда тройка $(r, \varphi, h)$~--- координаты рассматриваемого векторы в \term{цилиндрической системе координат}, и верно представление
\begin{equation}
    \begin{cases}
        r = \sqrt{x^2 + y^2},\\
        \sin \varphi = y/r,\\
        \cos \varphi = x/r,\\
        h = z.
    \end{cases}
    \quad \Leftrightarrow \quad
    \begin{cases}
        x = r \cos \varphi,\\
        y = r \sin \varphi,\\
        z = h.
    \end{cases}
\end{equation}

\begin{wrapfigure}{r}{0.35\tw}
    \centering
    \vspace{-2.5pc}
    \tikzsetnextfilename{math-coord-sys-sphere}
    \begin{tikzpicture}[scale=1]
        \clip(-1.3, -1.1) rectangle (2.5, 2.5);

        \footnotesize
        \coordinate (O) at (0, 0);
        \coordinate (E1) at (2.3, 0);
        \coordinate (E2) at (0, 2.3);
        \coordinate (E3) at (-1, -1);
        \coordinate (M) at (0.8, 1.3);
        \coordinate (M') at (0.8, -0.48);
        \coordinate (M'') at (1, -0.6);

        \draw[semithick, -latex] (O) -- (E1);
        \draw[semithick, -latex] (O) -- (E2);
        \draw[semithick, -latex] (O) -- (E3);
        \draw[thick, -latex] (O) -- (M);

        \draw [dashes] (0, 1.78) -- (M) -- (M');
        \draw [dashes] (M'') -- (O);
        \draw [dashes] (0, 2) arc(90:-90:1.05 and 2);
        \draw [dashes] (M') -- (-0.48, -0.48);
        \draw [dashes] (M') -- (1.28, 0);
%        \draw[-latex] (1.02, 0) arc(0:45:1 and 0.4);

        \draw (O) circle(2);
        \draw (0, 2) arc(90:270:0.7 and 2);
        \draw (2, 0) arc(360:180:2 and 0.7);

        \draw [-latex](-0.2, -0.2) arc(-99.4:-75.76:1.22);
        \draw [-latex](0.29, -0.18) arc(-9.44:18:1.22);
%        \draw (O) ellipse(0.6 and 0.21);
%        \draw (-1, 0) circle(1.3);


        \foreach \p in {O, M, M', M''} {
            \point (\p);
        };


        \draw (0.5, 0.6) node[anchor=south east]{$\vec{r}$};
%        \draw (1.05, -0.25) node[anchor=west]{$x$};
%        \draw (0.2, -0.45) node[anchor=north]{$y$};
%        \draw (0.75, 0.3) node[anchor=west]{$z$};
        \draw (0.3, 0.15) node[anchor=west]{$\varphi$};
        \draw (0.05, -0.15) node[anchor=north]{$\theta$};
        \draw (M) node[anchor=west]{$M$};
        \draw (O) node[anchor=south east]{$O$};

        \draw (E1) node[anchor=north]{$y$};
        \draw (E2) node[anchor=east]{$z$};
        \draw (E3) node[anchor=east]{$x$};
    \end{tikzpicture}
    \caption{}
    \label{pic:math-coord-sys-sphere}
    \vspace{-1.5pc}
\end{wrapfigure}
Остается рассмотреть \term{сферическую систему координат}. Здесь координатами точки будет длина $r$ радиус-вектора $\vec{r}$ и два угла: $\theta$~--- угол между радиус-вектором и плоскостью $Oxy$ и $\varphi$~--- угол между проекцией радиус-вектора на плоскость $Oxy$ и осью $Ox$. Верны формулы перехода:
\begin{equation}
    \begin{cases}
        r = \sqrt{x^2 + y^2 + z^2},\\
        \theta = \arcsin{z/r},\\
        \sin \varphi = y/r,\\
        \cos \varphi = x/r.
    \end{cases}
    \quad \Leftrightarrow \quad
    \begin{cases}
        x = r \cos \theta \cos \varphi    ,\\
        y = r \cos \theta \sin \varphi,\\
        z = r \sin \theta.
    \end{cases}
\end{equation}

\subsection{Скалярное произведение}
\term{Скалярным произведением} двух векторов называется билинейная операция над ними, зависящая только от длин этих векторов и угла между ними, результатом которой является скаляр. Скалярное произведение векторов $\vec{a}$ и  $\vec{b}$ выражается следующим образом:
\begin{equation}
    \scalar{a}{b} = |\vec{a}||\vec{b}| \cos \widehat{\vec{a}\vec{b}}. \label{eq:scalar-prod1}
\end{equation}
\begin{equation}
    \scalar{a}{b} = \vec{a}^{\T} \vec{b} =
    \begin{pmatrix}
        a_1 & \cdots & a_n
    \end{pmatrix}
    \begin{pmatrix}
        b_1\\
        \vdots\\
        b_n
    \end{pmatrix}
    = a_1 b_1 + \ldots + a_n b_n.
    \label{eq:scalar-prod2}
\end{equation}

Докажем эквивалентность \eqref{eq:scalar-prod1} и \eqref{eq:scalar-prod2}. Пусть в ортонормированном базисе $\{\vec{e}_1, \ldots, \vec{e}_2\}$ векторы имеют следующие представления:
\begin{equation}
    \vec{a} = \sum\limits_{i = 1}^n a_i \vec{e}_i, \qquad \vec{b} = \sum\limits_{i = 1}^n b_i \vec{e}_i.
\end{equation}
Заметим, что $(\vec{a} \cdot \vec{e}_i) = |\vec{a}||\vec{e}_i| \cos \theta_i = |\vec{a}| \cos \theta_i = a_i$, где $\theta_i$~--- угол вектора $\vec{a}$ с $i$-м базисным вектором $\vec{e}_i$. Тогда
\begin{equation}
    \scalar{a}{b} = \left( \vec{a} \cdot \sum\limits_{i=1}^n b_i\vec{e}_i \right) = \sum\limits_{i=1}^n b_i(\vec{a} \cdot \vec{e}_i) = \sum\limits_{i=1}^n a_i b_i = a^{\T}b.
\end{equation}

Из \eqref{eq:scalar-prod1} и \eqref{eq:scalar-prod2} очевидна билинейность и симметричность скалярного произведение, то есть
\begin{equation}
    \scalar{(a + b)}{(c + d)} = \scalar{(c + d)}{(a + b)} = \scalar{a}{c} + \scalar{a}{d} + \scalar{b}{c} + \scalar{b}{d}.
\end{equation}

Практическое пременение скалярное произведение находит в вопросах проверки ортогональности векторов (как частный случай нахождения угла между векторами), потому что $\vec{a} \perp \vec{b}$ тогда и только тогда, когда $\scalar{a}{b} = 0$, так как
\begin{equation}
    \cos \widehat{\vec{a}\vec{b}} = \frac{\scalar{a}{b}}{|\vec{a}||\vec{b}|}.
\end{equation}

Отсюда также получается выражение для проекции вектора $\vec{a}$ на прямую с направляющим вектором $\vec{l}$:
\begin{equation}
    \pr_\vec{l} \vec{a} = \frac{\scalar{a}{l}}{|\vec{l}|^2} \vec{l}.
\end{equation}

Используя скалярное произведение, получим важную утверждение теоремы косинусов из планиметрии: пусть $\vec{c} = \vec{b} - \vec{a}$, то есть имеем треугольник со сторонами $|\vec{a}|$, $|\vec{b}|$ и $|\vec{c}|$. Рассмотрим скалярное произведение вектора $\vec{c}$ самого на себя:
\begin{multline}
    \scalar{c}{c} = \scalar{(b - a)}{(b - a)} = \scalar{b}{(b-a)} - \scalar{a}{(b - a)} = \\
    = \scalar{b}{b} - 2\scalar{b}{a} + \scalar{a}{a}
\end{multline}
\begin{equation}
    c^2 = b^2 + a^2 - 2ab\cos \widehat{\vec{a}\vec{b}}
\end{equation}

\subsection{Матрица}
На практике часто оказывается полезным матричное исчисление. \term{Матрицей} размера $m \times n$ над полем $F$ называется $nm$ элементов из $F$, которые удобно представлять в виде 
    \begin{equation}
        \underset{m \times n}{A} =
        \begin{pmatrix}
            a_{11} & a_{12} & \cdots & a_{1n}\\
            a_{21} & a_{22} & \cdots & a_{2n}\\
            \vdots & \vdots & \ddots & \vdots\\
            a_{m1} & a_{m2} & \cdots & a_{mn}
        \end{pmatrix}.
    \end{equation}
    К матрицам применимы следующие простейшие операции:
    \begin{equation}
    \underset{m \times n}{A} + \underset{m \times n}{B} =
    \begin{pmatrix}
        a_{11} + b_{11} & a_{12} + b_{12} & \cdots & a_{1n} +  b_{1n}\\
        a_{21} + b_{21} & a_{22} + b_{22}& \cdots & a_{2n} + b_{2n}\\
        \vdots & \vdots & \ddots & \vdots\\
        a_{m1} + b_{m1} & a_{m2} + b_{m2} & \cdots & a_{mn} +  b_{mn}
    \end{pmatrix};
    \end{equation}
    \begin{multline}
        \underset{m \times n}{A} \cdot \underset{n \times m}{B} =
        \begin{pmatrix}
            a_{11} & a_{12} & \cdots & a_{1n}\\
            a_{21} & a_{22} & \cdots & a_{2n}\\
            \vdots & \vdots & \ddots & \vdots\\
            a_{m1} & a_{m2} & \cdots & a_{mn}
        \end{pmatrix}
        \begin{pmatrix}
            b_{11} & b_{12} & \cdots & b_{1m}\\
            b_{21} & b_{22} & \cdots & b_{2m}\\
            \vdots & \vdots & \ddots & \vdots\\
            b_{n1} & b_{n2} & \cdots & b_{nm}
        \end{pmatrix} = \\
        = \begin{pmatrix}
            \sum\limits_{i=1}^{n} a_{1i} b_{i1} & \sum\limits_{i=1}^{n} a_{1i} b_{i2} & \cdots & \sum\limits_{i=1}^{n} a_{1i} b_{im}\\
            \sum\limits_{i=1}^{n} a_{2i} b_{i1} & \sum\limits_{i=1}^{n} a_{2i} b_{i2} & \cdots & \sum\limits_{i=1}^{n} a_{2i} b_{im}\\
            \vdots & \vdots & \ddots & \vdots\\
            \sum\limits_{i=1}^{n} a_{mi} b_{i1} & \sum\limits_{i=1}^{n} a_{mi} b_{i2} & \cdots & \sum\limits_{i=1}^{n} a_{m    i} b_{im}\\
        \end{pmatrix},
    \end{multline}
    легко видеть, $A B \not = B A$.

    \term{Определитель}~--- функция $\det(X):\underset{n \times n}{\text{Mat}} \rightarrow \R$, вычисляемая по формуле:
    \begin{equation}
        \det A = \det
        \begin{pmatrix}
            a_{11} & a_{12} & \cdots & a_{1n}\\
            a_{21} & a_{22} & \cdots & a_{2n}\\
            \vdots & \vdots & \ddots & \vdots\\
            a_{m1} & a_{m2} & \cdots & a_{mn}
        \end{pmatrix} =
        \sum\limits_{k = 1}^n (-1)^{k+1} a_{hk} \det M_{hk},
        \label{eq:det-def1}
    \end{equation}
    где $M_{hk}$~--- дополнительный минор~--- матрица, полученная из $A$ вычеркиванием $h$-й строки и $k$-го столбца. Данное рекурсивное выражение для определителя матрица $A$ называется разложением по $h$-й строке, аналогично определяется разложение по столбцу. Для формальной полноты определения скажем, что определитель матрицы порядка единицы равен единственному элементу матрицы.

    Раскрывая выражение \eqref{eq:det-def1}, получим эквивалентное ему:
    \begin{equation}
        \det A = \sum\limits_{(\alpha_1, \ldots, \alpha_n)} (-1)^{\pi(\alpha_1, \ldots, \alpha_n)} a_{1\alpha_1} a_{2\alpha_2} \ldots a_{(n-1)\alpha_{n-1}} a_{n\alpha_n},
        \label{eq:det_def2}
    \end{equation}
    где суммирование производится по всевозможным перестановкам индексов $(\alpha_1, \ldots, \alpha_n)$, а $\pi(\alpha_1, \ldots, \alpha_n)$~--- число инверсий в перестановке. Отсюда следует полилинейность определителя. Перегруппировкой членов получим, что суммирование может производиться не только по перестановкам строк, но и по перестановкам столбцов.

    Матрицу можно представить, как
    \begin{equation}
        A = \begin{pmatrix}
            \vec{a}_1 & \cdots & \vec{a}_n
        \end{pmatrix},
    \end{equation}
    где $\vec{a}_1, \ldots, \vec{a}_n$~--- вектора размерности $m$. А также можно считать, что
    \begin{equation}
        A = \begin{pmatrix}
            \vec{a}_1^{\T}\\
            \vdots\\
            \vec{a}_m^{\T}
        \end{pmatrix}
    \end{equation}
    где $\vec{a}_1, \ldots, \vec{a}_n$~--- вектора размерности $n$.

    Из выражения \eqref{eq:det_def2} для определителя следует, что
    \begin{equation}
        \det \begin{pmatrix}
            \vec{a}_1 & \cdots & \vec{a}_i & \vec{a}_{i+1} & \cdots & \vec{a}_n
        \end{pmatrix} =
    -\det \begin{pmatrix}
            \vec{a}_1 & \cdots & \vec{a}_{i+1} & \vec{a}_i & \cdots & \vec{a}_n
        \end{pmatrix},
    \end{equation}
    так как число инверсий для соответствующей перестановки изменяется на единицу изменяется на единицу. Если же поменять местами не соседние строки, а две произвольные, то знак определителя также изменится на противоположны, так как <<перетаскиваение>> $i$-ой строки по $i+1, \ldots, j - 1$ строчкам дает столько же инверсий, сколько <<перетаскивание>> $j$-ой по этим же строчкам, плюс еще одна инверсия при смене порядка $i$-ой и $j$-ой строк.

    Отсюда вытекает еще одно свойство: если две строки (столбца) матрицы равны, то определитель равен нулю. Проверим это, при перестановке равных строк местами матрица не изменяется, но про предыдущему свойству изменяется знак определителя, то есть $\det A = - \det A$, значит, $\det A = 0$.

    Из линейности по каждой строке и каждому столбцу вытекает, что общий множитель элементов какой-либо строки (столбца) определителя можно вынести за знак определителя. Следовательно, если хотя бы одна строка (столбец) матрицы нулевая, то определитель равен нулю.

    Если две (или несколько) строки (столбца) матрицы линейно зависимы, то её определитель равен нулю. Достаточно разложить одну из строк (столбцов) по остальным, воспользоваться линейность, а потом вынести множитель, получим две одинаковые строки, то есть определитель каждой такой матрицы будет равен нулю.

    Из последнего вытекает утверждение, что при добавлении к любой строке (столбцу) линейной комбинации других строк (столбцов) определитель не изменится.

    Встает вопрос, как можно применить полученные знания об определителе квадратной матрицы? Очень просто! Составив матрицу из $n$ векторов в пространстве с размерностью $n$ и посчитав ее определитель, можно выяснить, является ли система из данных векторов линейно зависимой.

\subsection{Векторное произведение}
\label{sec:cross-product}

Тройку векторов будем называть \imp{правой}, если для наблюдателя, находящегося в конце третьего вектора, кратчайший поворот от первого вектора ко второму осуществляется против часовой стрелки, иначе левой.

Рассмотрим еще одну операцию над векторами~--- \term{векторное произведение} $\cross{a}{b}: \R^3 \times \R^3 \rightarrow \R^3$~--- антисимметричную и билинейную, задаваемую по правилу:
\begin{equation}
    \cross{a}{b} = |\vec{a}||\vec{b}| \sin \widehat{\vec{a}\vec{b}} \cdot \vec{n},
\end{equation}
\begin{wrapfigure}{r}{0.25\tw}
    \centering
    \vspace{.1pc}
    \tikzsetnextfilename{math-cross}
    \begin{tikzpicture}[scale=1]
        \footnotesize
        \coordinate (O) at (0, 0);
        \coordinate (A) at (1.5, 0);
        \coordinate (B) at (1, 0.5);
        \coordinate (AB) at (2.5, 0.5);
        \coordinate (N) at (0, 1.5);

        \draw[thick, -latex] (O) -- (A);
        \draw[thick, -latex] (O) -- (B);
        \draw[semithick, -latex] (O) -- (N);

        \draw[dashes] (A) -- (AB);
        \draw[dashes] (B) -- (AB);

        \draw (0.75, 0) node[anchor=north]{$\vec{a}$};
        \draw (0.5, 0.25) node[anchor=south]{$\vec{b}$};
        \draw (0, 0.75) node[anchor=east]{$\vec{n}$};
    \end{tikzpicture}
    \caption{}
    \vspace{-1pc}
    \label{pic:math-cross}
\end{wrapfigure}
где $\vec{n}$~--- вектор нормали к плоскости, построенной на векторах $\vec{a}$ и $\vec{b}$, направление которой определяется таким образом, чтобы тройка векторов $\{\vec{a}, \vec{b}, \vec{n} \}$ была правой. Из определения понятно, что модуль векторного произведения равен площади параллелограмма, построенного на векторах $\vec{a}$~и~$\vec{b}$.

Так как площадь параллелограмма, построенного на векторах $\vec{a}$~и~$\vec{b}$ равна удвоенной площади треугольнока, построенного на этих же векторах, то
\begin{equation}
    |\cross{a}{b}| = |\cross{a}{(a - b)} | = |\cross{b}{(a-b)}|
\end{equation}
\begin{equation}
    a b \sin C = a c \sin B = b c \sin A \quad \Rightarrow \quad \frac{\sin C}{c} = \frac{\sin B}{b} = \frac{\sin A}{a}.
\end{equation}
Последнее двойное равенство называется теоремой синусов.

Рассмотрим выражение для векторного произведения в координатной форме:
\begin{multline}
    \cross{a}{b} =
    \cross{
    \begin{pmatrix}
        a_1\\
        a_2\\
        a_3
    \end{pmatrix}}
    {\begin{pmatrix}
    b_1\\
    b_2\\
    b_3
\end{pmatrix}}
= \\ =
[( a_1 \vec{e}_1 + a_2 \vec{e_2} + a_3 \vec{e_3}) \times ( b_1 \vec{e}_1 + b_2 \vec{e_2} + b_3 \vec{e_3})] = \\
= a_1 b_1 \vec{0} + a_1 b_2 \vec{e}_3 - a_1 b_3 \vec{e}_2 - a_2 b_1 \vec{e}_3 + a_2 b_2 \vec{0} + a_2 b_3 \vec{e}_1 + a_3 b_1 \vec{e}_2  - a_3 b_2 \vec{e}_1 + a_3 b_3 \vec{0} = \\
= (a_2 b_3 - a_3 b_2) \vec{e}_1 + (a_3 b_1 - a_1 b_3) \vec{e}_2 + (a_1 b_1 - a_2 b_1) \vec{e}_3 = \\
= \begin{vmatrix}
a_2 & a_3\\
b_2 & b_3
\end{vmatrix} \vec{e}_1 -
\begin{vmatrix}
a_1 & a_3\\
b_1 & b_3
\end{vmatrix}\vec{e}_2 +
\begin{vmatrix}
a_1 & a_2\\
b_1 & b_2
\end{vmatrix} \vec{e}_3 = \begin{vmatrix}
\vec{e}_1 & \vec{e}_2 & \vec{e}_3\\
a_1 & a_2 & a_3\\
b_1 & b_2 & b_3
\end{vmatrix}
\end{multline}

\subsection{Смешанное произведение}
\label{sec:triple}
\begin{wrapfigure}{r}{0.35\tw}
    \centering
    \vspace{-1pc}
    \tikzsetnextfilename{math-triple}
    \begin{tikzpicture}[scale=1]
        \footnotesize
        \coordinate (O) at (0, 0);
        \coordinate (A) at (1.5, 0);
        \coordinate (B) at (1, 0.5);
        \coordinate (C) at (0.5, 1.5);
        \coordinate (AB) at (2.5, 0.5);
        \coordinate (AC) at (2, 1.5);
        \coordinate (BC) at (1.5, 2);
        \coordinate (ABC) at (3, 2);
        \coordinate (N) at (0, 2);

        \draw[thick, -latex] (O) -- (A);
        \draw[thick, -latex] (O) -- (B);
        \draw[thick, -latex] (O) -- (C);
        \draw[semithick, -latex] (O) -- (N);

        \draw[dashes] (A) -- (AB);
        \draw[dashes] (A) -- (AC);
        \draw[dashes] (B) -- (AB);
        \draw[dashes] (B) -- (BC);
        \draw[dashes] (C) -- (AC);
        \draw[dashes] (C) -- (BC);
        \draw[dashes] (AB) -- (ABC);
        \draw[dashes] (AC) -- (ABC);
        \draw[dashes] (BC) -- (ABC);

        \draw (0.75, 0) node[anchor=north]{$\vec{b}$};
        \draw (0.5, 0.25) node[anchor=south]{$\vec{c}$};
        \draw (0.25, 0.82) node[anchor=west]{$\vec{a}$};
        \draw (0, 1) node[anchor=east]{$\vec{n}$};
    \end{tikzpicture}
    \caption{}
    \label{pic:math-triple}
\end{wrapfigure}

Векторное произведение определяет вектор площади параллелограмма, построенного на двух векторах, а скалярное произведение~--- величину проекции одного вектора на другой. Рассмотрим такую операцию:
\begin{equation}
    \triple{a}{b}{c} = \scalar{a}{\cross{b}{c}}.
\end{equation}

Разберем, что является результатом данной операции: $\cross{b}{c} = \vec{n}$~--- вектор нормали к плоскости векторов $\vec{b}$ и $\vec{c}$ такой, что $|\vec{n}| = |b||c| \sin \widehat{\vec{b}\vec{c}} \equiv S$~--- площадь параллелограмма.

Идём дальше, $\scalar{a}{n} = |a||n|\cos \widehat{\vec{a} \vec{n}}$~--- произведение длины вектора $\vec{n}$ на длину проекции $\pr_{\vec{n}} \vec{a}$. Значит величина смешанного произведения есть объем параллелепипеда, построенного на них.

В матричной форме смешанное произведение можно записать, как
\begin{equation}
    \triple{a}{b}{c} = \det
    \begin{pmatrix}
        \vec{a}^{\T}\\
        \vec{b}^{\T}\\
        \vec{c}^{\T}
    \end{pmatrix} =
    \begin{vmatrix}
        a_1 & a_2 & a_3\\
        b_1 & b_2 & b_3\\
        c_1 & c_2 & c_3
    \end{vmatrix},
\end{equation}
то есть определитель матрицы $n \times n$~--- ориентированные объем $n$-мерного параллелепипеда, построенного на $n$ векторах.

Практическое значение смешанного произведения основано на его свойстве: если проекция вектора $\vec{a}$ на вектор $\vec{n}$~--- $\pr_\vec{n} \vec{a} = 0$, значит, векторы $\vec{a}$, $\vec{b}$ и $\vec{c}$ лежат в одной плоскости, либо хотя бы один из них нулевой, что также означает, что эти три вектора лежат в одной плоскости.

Отсюда можно сделать вывод, что равенство нулю смешанного произведения трех векторов означает их компланарность или, что тоже самое, равенство нулю объема параллелепипеда, построенного на них.

\subsection{Прямая}

\begin{wrapfigure}{r}{0.35\tw}
    \centering
    \vspace{-.8pc}
    \tikzsetnextfilename{math-line}
    \begin{tikzpicture}[scale=1.1]
        \footnotesize
        \coordinate (O) at (0, 0);
        \coordinate (A) at (1, 1.5);
        \coordinate (B) at (2, 1);
        \coordinate (C) at (1.5, 1.25);
        \coordinate (L1) at (0, 2);
        \coordinate (L2) at (3, 0.5);


        \draw[-latex] (O) -- (A);
        \draw[-latex] (O) -- (B);
        \draw[thick, -latex] (A) -- (C);

        \draw[semithick] (L1) -- (L2);

        \point (O);
        \point (A);
        \point (B);

        \draw (O) node[anchor=north]{$O'$};
        \draw (A) node[anchor=south]{$O$};
        \draw (B) node[anchor=south]{$P$};

        \draw (1, 0.5) node[anchor=north]{$\vec{r}$};
        \draw (0.5, 0.75) node[anchor=east]{$\vec{r}_0$};
        \draw (1.3, 1.35) node[anchor=south]{$\vec{a}$};
        \draw (3, 0.6) node[anchor=south east]{$l$};
    \end{tikzpicture}
    \caption{}
    \label{pic:math-line}
\end{wrapfigure}
Рассмотрим необходимое условие, чтобы произвольная точка~$P$ с радиус-вектором~$\vec{r}$ лежала на прямой~$l$, проходящей через точку~$O$ с радиус вектором $\vec{r}_0$ (\lookPicRef{pic:math-line}). Пусть $ \vec{a} = \begin{pmatrix} a_1 & \cdots & a_n\end{pmatrix}^{\T}$~--- направляющий вектор прямой $l$, то есть $l$ параллельна прямой, содержащей вектор $\vec{a}$, тогда формально данное условие можно записать так:
\begin{equation}
    \vec{r} = \vec{r}_0 + \lambda \vec{a},\quad \lambda \in \R.
\end{equation}
Конкретизируем для случая векторов из $\R^2$:
\begin{gather*}
    \vec{r} - \vec{r}_0 = \lambda \vec{a} \quad \Rightarrow \quad
    \begin{cases}
        x - x_0 = \lambda x_a,\\
        y - y_0 = \lambda y_a.
    \end{cases}
\end{gather*}
Решив данную систему уравнений, получим, что
\begin{equation*}
    \lambda = \frac{x - x_0}{x_a} = \frac{y - y_0}{y_a}.
\end{equation*}
преобразуем второе равенство:
\begin{gather*}
    \frac{1}{x_a} \cdot x + \left( - \frac{1}{y_a} \right) \cdot y + \left( \frac{y_0}{y_a} - \frac{x_0}{x_a} \right) = 0,\\
    y_a x + (-x_a)y + (x_a y_0 - y_a x_0) = 0,
\end{gather*}
Сделав замену $y_a \equiv A$, $-x_a \equiv B$, $x_a y_0 - y_a x_0 \equiv C$, получим \imp{каноническое уравнение прямой на плоскости в декартовых координатах}:
\begin{equation}
    Ax + By + C = 0.
\end{equation}
Заметим, что
\begin{equation*}
    \scalar{a}{n} \equiv \scalar{a}{
    \begin{pmatrix}
        A\\
        B
    \end{pmatrix}} = x_a y_a - y_a x_a = 0,
\end{equation*}
значит, $\vec{n} \perp \vec{a}$, то есть вектор $\vec{n}$ есть вектор нормали к прямой с направляющим вектором $\vec{a}$, так как коэффициенты $A$ и $B$ не зависят от фиксированной точки $O$.

\subsection{Плоскость}

\begin{wrapfigure}{r}{0.35\tw}
    \centering
    \vspace{-1pc}
    \tikzsetnextfilename{math-plane}
    \begin{tikzpicture}[scale=1.1]
        \footnotesize

        \draw[semithick, fill=lightgray] (1, -1) -- (3.5, -1) -- (2.5, -2) -- (0, -2) -- cycle;

        \coordinate (O) at (0.5, 0);
        \coordinate (A) at (1, -1.75);
        \coordinate (B) at (2.25, -1.25);
        \coordinate (C) at (1.75, -1.75);
        \coordinate (D) at (1.5, -1.25);


        \draw[-latex] (O) -- (A);
        \draw[-latex] (O) -- (B);

        \draw[thick, -latex] (A) -- (C);
        \draw[thick, -latex] (A) -- (D);

        \point (O);
        \point (A);
        \point (B);

        \draw (O) node[anchor=south]{$O'$};
        \draw (A) node[anchor=east]{$O$};
        \draw (B) node[anchor=north]{$P$};

        \draw (1.375, -0.625) node[anchor=south]{$\vec{r}$};
        \draw (0.75, -0.875) node[anchor=east]{$\vec{r}_0$};
        \draw (1.375, -1.75) node[anchor=south]{$\vec{a}$};
        \draw (1.3, -1.55) node[anchor=south east]{$\vec{b}$};
        \draw (3.2, -1) node[anchor=north east]{$\Pi$};

    \end{tikzpicture}
    \caption{}
    \label{pic:math-plane}
\end{wrapfigure}
Аналогично предыдущему разделу, рассмотрим условие принадлежности точки $P$ с радиус-вектором $\vec{r}$ плоскости $\Pi$ в $\R^3$. Пусть неколлинеарные ($ \cross{a}{b} \not = 0$) векторы  $\vec{a}$ и $\vec{b}$~--- направляющие векторы плоскости $\Pi$, а точка $O$ с радиус-вектором $\vec r_0$ такова, что $\vec{r}_0 \in \Pi$. Тогда
\begin{equation}
    \vec{r} = \vec{r}_0 + \lambda\vec{a} + \mu\vec{b} \quad \Rightarrow \quad \begin{cases}
    x - x_0 = \lambda x_a + \mu x_b,\\
    y - y_0 = \lambda y_a + \mu y_b,\\
    z - z_0 = \lambda z_a + \mu z_b;
\end{cases}\quad \lambda, \mu \in \R.
\end{equation}
Преобразуем полученную систему уравнений:
\begin{align*}
& \left\{
\begin{aligned}
    \lambda &= \dfrac{x - x_0 - \mu x_b}{x_a},\\
    y - y_0 &= \dfrac{x - x_0 - \mu x_b}{x_a} \cdot y_a + \mu y_b,\\
    z - z_0 &= \lambda z_a + \mu z_b.
\end{aligned}\right.\\
& \left\{
\begin{aligned}
    \lambda &= \dfrac{x - x_0 - \mu x_b}{x_a},\\
    y - y_0 &= (x - x_0) \cdot \dfrac{y_a}{x_a} + \mu \left(y_b - \dfrac{x_b y_a}{x_a} \right),\\
    z - z_0 &= \lambda z_a + \mu z_b.
\end{aligned}\right.\\
&\left\{
\begin{aligned}
    \lambda &= \dfrac{x - x_0 - \mu x_b}{x_a},\\
    \mu &= \dfrac{x_a y - x_a y_0 - (x - x_0) \cdot y_a}{x_a y_b - x_b y_a},\\
    z - z_0 &= \lambda z_a + \mu z_b.
\end{aligned}\right.
\end{align*}
Подставим выражения для $\lambda$ и $\mu$ в третье уравнение:
\begin{multline*}
    z - z_0 = \dfrac{x - x_0 - \mu x_b}{x_a} \cdot z_a + \mu z_b = (x - x_0) \cdot \dfrac{z_a}{x_a} + \mu \left( z_b - \dfrac{x_b z_a}{x_a} \right) = \\
    = (x - x_0) \cdot \dfrac{z_a}{x_a} + \dfrac{x_a y - x_a y_0 - (x - x_0) \cdot y_a}{x_a y_b - x_b y_a} \cdot \left( z_b - \dfrac{x_b z_a}{x_a} \right)
\end{multline*}
Приведя подобные слагаемые с $x$, $y$ и $z$, получим:
\begin{multline*}
z = x \cdot \underbrace{\left( \dfrac{z_a}{x_a} - \dfrac{y_a}{x_a y_b - x_b y_a} \left( z_b - \dfrac{x_b z_a}{x_a} \right) \right)}_A +\\
+ y \cdot \underbrace{\dfrac{x_a}{x_a y_b - x_b y_a} \cdot \left( z_b - \dfrac{x_b z_a}{x_a} \right)}_B +\\
+ \underbrace{z_0 - \dfrac{x_0 z_a}{x_a} - \dfrac{x_a y_0 - x_0 y_a}{x_a y_b - x_b y_a} \cdot \left( z_b - \dfrac{x_b z_a}{x_a} \right)}_D.
\end{multline*}
Сделаем замену:
\begin{align*}
&
\begin{aligned}
    A \equiv  \dfrac{z_a}{x_a} &- \dfrac{y_a}{x_a y_b - x_b y_a} \left( z_b - \dfrac{x_b z_a}{x_a} \right) = \\
    &\quad\quad= \frac{x_a y_b z_a - x_b y_a z_a}{x_a (x_a y_b - x_b y_a)} - \frac{x_a y_a z_b - x_b y_a z_a}{x_a (x_a y_b - x_b y_a)} = \frac{y_b z_a - y_a z_b}{x_a y_b - x_b y_a},
\end{aligned}\\
&
B \equiv \dfrac{x_a}{x_a y_b - x_b y_a} \cdot \left( z_b - \dfrac{x_b z_a}{x_a} \right) 
%= \frac{x_a (x_a z_b - x_b z_a)}{x_a (x_a y_b - x_b y_a)} = 
\frac{x_a z_b - x_b z_a}{x_a y_b - x_b y_a},\\
&
D \equiv z_0 - \dfrac{x_0 z_a}{x_a} - \dfrac{x_a y_0 - x_0 y_a}{x_a y_b - x_b y_a} \cdot \left( z_b - \dfrac{x_b z_a}{x_a} \right).
\end{align*}
В результате такой замены получим уравнение
\begin{equation}
Ax + By - z + D = 0.
\end{equation}
Домножим в нём обе части на $\xi \equiv x_a y_b - x_b y_a$ и сделаем ещё одну замену:
\begin{equation*}
A'x + B'y + C'z + D' = 0,
\end{equation*}
где $D' = \xi D$, а
\begin{equation*}
\begin{cases}
    A' = y_b z_a - y_a z_b = \det
    \begin{pmatrix}
        y_b & z_b\\
        y_a & z_a
    \end{pmatrix}\\[1pc]
    B' = x_a z_b - x_b z_a = \det
    \begin{pmatrix}
        x_a & z_a\\
        x_b & z_b
    \end{pmatrix}\\[1pc]
    C' = x_b y_a - x_a y_b = \det
    \begin{pmatrix}
        x_b & y_b\\
        x_a & y_a
    \end{pmatrix}
\end{cases}
\end{equation*}
Легко видеть, что
\begin{equation}
\begin{pmatrix}
    A'\\
    B'\\
    C'
\end{pmatrix} =
\cross{b}{a} \equiv \vec{n}.
\end{equation}
Из свойств векторного произведения вектор $\vec n$ является \imp{вектором нормали} к плоскости $\Pi$.

Теперь  самое первое условие можно записать так:
\begin{equation}
\scalar{r}{n} = \triple{r}{b}{a} = -D' = (\vec{r}_0, \vec{b}, \vec{a}).
\end{equation}

\subsection{Производная}

\begin{wrapfigure}[11]{r}{0.46\tw}
	\centering
	\vspace{-1.1pc}
	\centering
	\begin{tikzpicture}[scale=1.1]
		\footnotesize
		
%		\foreach \x in {-0.5, -0.4,...,3} {
%			\draw [line width=.1pt] (\x, -0.5) -- (\x, 3);
%		};
%		
%		\foreach \x in {0, 1,...,3} {
%			\draw [line width=.4pt] (\x , -0.5) -- (\x , 3);
%		};
%		
%		\foreach \y in {0, 1,...,3} {
%			\draw [line width=.4pt] (-0.5, \y) -- (3, \y);
%		};
%		
%		\foreach \y in {-0.5, -0.4,...,3} {
%			\draw [line width=.1pt] (-0.5, \y) -- (3, \y);
%		};
		
		\draw[semithick, -latex] (-0.5, 0) -- (3, 0); 		
		\draw[semithick, -latex] (0, -0.5) -- (0, 3); 	
		
		\draw [thick] (-0.3, 0.5) .. controls (2, 0.5) and (2, 3) .. (3, 3);	
		
		\draw (0.5, 0.28) -- (2.5, 2.28);
		\draw [dashes] (0, 1.2) -- (2.6, 1.2);
		\draw [dashes] (0, 2.5) -- (2.6, 2.5);
		\draw [dashes] (1.4, 0) -- (1.4, 1.6);
		\draw [dashes] (2.3, 0) -- (2.3, 2.9);
		\draw [latex-latex] (2.3, 1.2) -- (2.3, 2.08);
		
		\draw (1.8, 1.2) arc(0:43:0.4);
		
		\draw (1.8, 1.4) node[anchor=west]{$\alpha$};
		\draw (3, 0) node[anchor=south]{$x$};
		\draw (0, 3) node[anchor=west]{$f(x)$};
		\draw (1.4, -0.06) node[anchor=north]{$x_0$};
		\draw (2.3, 0) node[anchor=north]{$x_0 + \Delta x$};
		\draw (0, 1.2) node[anchor=east]{$f(x_0)$};
		\draw (0, 2.5) node[anchor=east]{$f(x + \Delta x)$};
		\draw (2.3, 1.6) node[anchor=west]{$df(x_0)$};
		
		
		\draw [fill=white] (1.4, 1.2) circle(0.03);
		\draw [fill=white] (2.3, 1.2) circle(0.03);
		\draw [fill=white] (2.3, 2.5) circle(0.03);
		\draw [fill=white] (1.4, 0) circle(0.03);
		\draw [fill=white] (2.3, 0) circle(0.03);
		\draw [fill=white] (2.3, 2.08) circle(0.03);
		\draw [fill=white] (0, 1.2) circle(0.03);
		\draw [fill=white] (0, 2.5) circle(0.03);
	\end{tikzpicture}
	\caption{}
	\label{pic:math-div}
\end{wrapfigure}
\term{Производная в точке}~--- предел отношения приращения функции к приращению её аргумента при стремлении приращения аргумента к нулю, если такой предел существует. \imp{Геометрический смысл производной}: значение производной в точке численно равно тангенсу угла наклона касательной к графику функции в данной точке. Следовательно, точки, где производная обнуляется, соответствуют локальным минимумам и максимумам функции.
\begin{equation}
	f^\prime(x_0) = \lim_{\Delta x \to 0}\frac{f(x_0 + \Delta x) - f(x_0)}{\Delta x}
\end{equation}
Общепринятые обозначения для производной функции $f(x)$ в точке $x_0$:
\begin{equation}
	f^\prime(x_0) = f^\prime_x(x_0) = D f(x_0) = \frac{d f}{d x}(x_0) = \dot{f} (x_0).
\end{equation}
Правила дифференцирования:\\[-0.5pc]
\begin{minipage}{0.5\textwidth}
	\begin{align*}
		(f+g)^\prime &= f^\prime + g^\prime;\\
		(Cf)^\prime &= Cf^\prime;\\
		(fg)^\prime &= f^\prime g + f g^\prime;
	\end{align*}
\end{minipage}
\begin{minipage}{0.5\textwidth}
	\begin{align*}
		\left(\dfrac{f}{g}\right)^\prime &= \dfrac{f^\prime g - f g^\prime}{g^2};\\
		\dfrac{d}{dx}f\bigl(g(x)\bigr) &= \dfrac{df(g)}{dg}\dfrac{dg(x)}{dx}.
	\end{align*}
\end{minipage}\\[0.5pc]
Таблица производных:
\begin{align*}
	(x^a)^\prime &= a x^{a-1};
	& (\cos x)^\prime &= - \sin x;
	& (\arccos x)^\prime &= - \dfrac{1}{\sqrt{1 - x^2}};\\
	(a^x)^\prime &= a^x \ln a;
	& (\log_a x)^\prime &= \dfrac{1}{x \ln a};
	& (\arctg x)^\prime &= \dfrac{1}{1 + x^2};\\
	(\sin x)^\prime &= \cos x;
	& (\arcsin x)^\prime &=
	\dfrac{1}{\sqrt{1 - x^2}};
	&  (\arcctg x)^\prime &= - \dfrac{1}{1 + x^2}.
\end{align*}

\subsection{Интеграл}
\term{Неопределенным интегралом} функции $f(x)$ называется такая функция $F(x)$, производная которой равна $f(x)$.
\begin{equation}
	F(x) = \int f(x)dx,\quad F^\prime(x)=f(x).
\end{equation}

\begin{wrapfigure}{r}{0.46\tw}
	\centering
	\vspace{-1.1pc}
	\centering
	\begin{tikzpicture}[scale=1.1, yscale=0.7]
		\footnotesize
		\clip(-0.5, -1.05) rectangle (3.6, 2.1);
		
%		\foreach \x in {-5, -4.9,...,5} {
%			\draw [line width=.1pt] (\x, -5) -- (\x, 5);
%		};
%		
%		\foreach \x in {-5, -4,..., 5} {
%			\draw [line width=.4pt] (\x , -5) -- (\x , 5);
%		};
%		
%		\foreach \y in {-5, -4,..., 5} {
%			\draw [line width=.4pt] (-5, \y) -- (5, \y);
%		};
%		
%		\foreach \y in {-5, -4.9,..., 5} {
%			\draw [line width=.1pt] (-5, \y) -- (5, \y);
%		};
		

		\fill [lightgray] (0.4, 0) -- (0.4, 1.7) .. controls (0.9, 1.65) and (1.1, 0.85) .. (1.55, 0) -- cycle;	
		\fill [lightgray] (1.55, 0) .. controls (1.9, -0.9) and (2.3, -1.2) .. (2.5, -0.8) -- (2.5, 0) -- cycle;	
		\draw [thick] (-0.3, 0.5) .. controls (1, 5) and (2, -5) .. (3, 1);	
		
		\draw[semithick, -latex] (-0.5, 0) -- (3.5, 0); 		
		\draw[semithick, -latex] (0, -0.5) -- (0, 2); 
		
		\draw [dashes] (0.4, 0) -- (0.4, 1.7);
		\draw [dashes] (2.5, 0) -- (2.5, -0.8);

		\draw (0.9, 0.5) node{$S_+$};
		\draw (2.15, -0.45) node{$S_-$};
		\draw (3.5, 0) node[anchor=south]{$x$};
		\draw (0, 2) node[anchor=west]{$y$};
		\draw (0.4, 0) node[anchor=north]{$a$};
		\draw (2.5, 0) node[anchor=south]{$b$};
		\draw (2.95, 0.8) node[anchor=west]{$f(x)$};

		\point (0.4, 1.7);
		\point (0.4, 0);
		\point (2.5, 0);
		\point (2.5, -0.8);
		
	\end{tikzpicture}
	\caption{}
	\label{pic:math-int}
\end{wrapfigure}
\term{Определенный интеграл} характеризуется верхним и нижним пределом интегрирования. Значение определенного интеграла численно равно площади под графиком функции на данном промежутке и вычисляется по формуле \imp{Ньютона--Лейбница}:
\begin{equation}
	\int\limits^b_a f(x) \,d x = F(x) \biggr|^b_a = F(b) - F(a)
\end{equation}
Правила интегрирования:
\begin{align*}
	&\int c f(x) \,d x = c \int f(x) \,d x;\quad &&  \int f(ax + b) \,d x = \dfrac{1}{a}F(ax + b) + C;\\
	&\int f \,d g = fg - \int g \,d f; && \int \bigl[f(x) + g(x)\bigr] \,d x = \int f(x) \,d x + \int g(x) \,d x;
\end{align*}
Таблица интегралов:
\begin{align*}
	&\int  x^a \,d x = \dfrac{x^{a+1}}{a+1} + C,\quad a \neq -1; \quad
	&&\int \dfrac{dx}{\sqrt{a^2 - x^2}} = \arcsin\dfrac{x}{a} + C;\\
	&\int \frac{dx}{x} = \ln x + C;
	&&\int \dfrac{dx}{-\sqrt{a^2 - x^2}} = \arccos\dfrac{x}{a} + C;\\
	&\int a^x \,d x = \dfrac{a^x}{\ln a} + C;
	&&\int \dfrac{dx}{x^2 + a^2} = \dfrac{1}{a} \arctg \dfrac{x}{a} + C; \\
	&\int \cos x \,d x = \sin x + C;
	&&\int \dfrac{dx}{x^2 - a^2} = \dfrac{1}{2a} \ln \dfrac{|x - a|}{|x + a|} + C;\\
	&\int \sin x \,d x = -\cos x + C;
	&&\int \dfrac{dx}{\sqrt{x^2 + a}} = \ln \left| x + \sqrt{x^2 + a} \right| + C.
\end{align*}

Рассмотрим пример использования интеграла~--- получим формулу для нахождения объема шара с радиусом~$R$. Сначала получим вспомогательную формулу площади круга радиуса~$R$. Для этого рассмотрим тонкое кольцо радиуса~$r$~и ширины~$dr$, его площадь составляет~$2\pi r \,d r$. Следовательно, интегрируя площадь таких колец по радиусу~$r$ от~$0$ до~$R$, получим площадь круга
\begin{equation*}
    S = \int_0^R 2 \pi r \,d r = 2 \pi \cdot \left.\frac{r^2}{2} \right|_0^R = \pi R^2.
\end{equation*}
Теперь, используя полученное выражение для площади круга, легко найти объем шара. Рассмотрим тонкий слой толщины~$dx$ на расстоянии~$x$ от центра шара, его радиус равен $\sqrt{R^2 - x^2}$. Проинтегрируем объем таких слоев~по~$x$ от~$-R$ до~$R$:
\begin{equation*}
    V 
        = \int_{-R}^{R} \pi (R^2 - x^2) \,d x 
        = \pi \left.\left(xR^2 - \frac{x^3}{3}\right)\right|_{-R}^{R}
        = \pi \left(2R^3 - \frac{2R^3}{3} \right) 
        = \frac{4}{3} \pi R^3.
\end{equation*}
\subsection{Телесный угол}
\label{subsec:solid-angle}

\begin{wrapfigure}{r}{0.46\tw}
	\centering
	\vspace{-1.1pc}
	\centering
	\begin{tikzpicture}[scale=1.1]
		\footnotesize
%		\clip(-2.1, -2.1) rectangle (2.1, 2.1);
%		
%		\foreach \x in {-5, -4.9,...,5} {
%			\draw [line width=.1pt] (\x, -5) -- (\x, 5);
%		};
%		
%		\foreach \x in {-5, -4,..., 5} {
%			\draw [line width=.4pt] (\x , -5) -- (\x , 5);
%		};
%		
%		\foreach \y in {-5, -4,..., 5} {
%			\draw [line width=.4pt] (-5, \y) -- (5, \y);
%		};
%		
%		\foreach \y in {-5, -4.9,..., 5} {
%			\draw [line width=.1pt] (-5, \y) -- (5, \y);
%		};

		
		\draw [semithick] (0, 0) circle (2);
		\draw [semithick] (-2, 0) arc(-180:0:2 and 0.5);
		\draw [semithick, dashes] (-2, 0) arc(180:0:2 and 0.5);
		
		\draw [top color=white!80!black, bottom color=white!50!black] (0, 0.8) .. controls +(270:0.1) 
		and ++(0:-0.1) .. (0.2, 0.6) .. controls ++(0:0.2) 
		and ++(0:-0.2) .. (0.6, 0.8) .. controls ++(0:0.1) 
		and ++(0:-0.1) .. (0.9, 0.6) .. controls +(0:0.1) 
		and ++(270:0.2) .. (1.1, 0.8) .. controls +(90:0.2) 
		and ++(0:0.1) .. (0.9, 1.2) .. controls +(180:0.1) 
		and ++(0:0.1) .. (0.6, 1.1) .. controls +(180:0.1) 
		and ++(0:0.1) .. (0.3, 1.4) .. controls +(180:0.2) 
		and ++(90:0.2) .. (0, 0.8) -- cycle;

		
		\draw [-latex] (0, 0) -- (-2, 0);
		\draw [top color=white!80!black, bottom color=white!50!black] (0, 0.8) .. controls +(270:0.1) 
		and ++(0:-0.1) .. (0.2, 0.6) .. controls ++(0:0.2) 
		and ++(0:-0.2) .. (0.6, 0.8) .. controls ++(0:0.1) 
		and ++(0:-0.1) .. (0.9, 0.6) .. controls ++(0:0.1)
		and ++(225:0.01) ..(1.03, 0.62) -- (0, 0) -- cycle;

		
		\draw (0.7, 0.95) node{$S$};
		\draw (-1, 0) node[anchor=south]{$R$};
		\draw (0, 0) node[anchor=north west]{$O$};
		\draw [fill=white] (0, 0) circle(0.03);
		
	\end{tikzpicture}
	\caption{}
	\label{pic:math-solid-angle}
\end{wrapfigure}
\term{Телесный угол}~--- часть пространства, которая является объединением всех лучей, выходящих из данной точки (вершины угла) и пересекающих некоторую поверхность (которая называется поверхностью, стягивающей данный телесный угол). Телесный угол измеряется в стерадианах и равен отношению площади сферы, которую вырезает данный угол, к квадрату радиуса данной сферы.
\begin{equation}
	\Omega = \frac{S}{R^2}
\end{equation}
Телесный угол полной сферы равен $4\pi$. Величину телесного угла, образованного конусом с углом раствора $\alpha$ можно определить по формуле
\begin{equation}
	\Omega = 2 \pi \left(1 - \cos \frac{\alpha}{2}\right).
\end{equation}

Телесный угол, соответствующий сегменту сферы радиуса $R$ и высоты $h$, равен
\begin{equation}
	\Omega = 2 \pi \left(1 - \dfrac{h}{\sqrt{R^2 + h^2}}\right).
\end{equation}

Для сферического треугольника с углами $\alpha$, $\beta$ и $\gamma$ справедливо соотношение
\begin{equation}
	\Omega = \alpha + \beta + \gamma - \pi
\end{equation}

\subsection{Формулы приближенного вычисления}
Формулы приближенного вычисления при $x \ll 1$:
\begin{align*}
	\sin x &\approx x - \frac{x^3}{6} \approx x & \cos x &\approx 1 - \frac{x^2}{2} \\
	\tg x &\approx x & \ln(1+x) &\approx x \\
	(1 + x)^\alpha &\approx 1 + \alpha x & e^x &\approx 1 + x \\
	\sin (\theta + x) &\approx \sin \theta + x \cos \theta & \cos(\theta + x) &\approx \cos \theta - x \sin \theta
\end{align*}

\begin{wrapfigure}{r}{0.5\tw}
	\centering
	\vspace{-.9pc}
	\begin{tikzpicture}
		\begin{axis}[
			width	=	.5\tw,
			height	=	4.5cm,
			xlabel	=	{$x$},
			ylabel	=	{$y$},
%			extra x ticks	=	{1.571},
%			ytick = {-20, -10, 0, 10, 20},
			ymax	=	1,
			ymin	=	0,
			xmax	=	1.571,
			xmin	=	0,
%			xtick	=	{0, 4, 8, 12, 16, 20, 24},
%			x dir = reverse
			legend cell align=left,
				legend style={
				row sep = 1mm,
				draw=none,
				fill=none,
				font=\scriptsize,
				at={(axis cs:1.5, 0.1)}, anchor=south east,
				},
			]
			\addplot [domain=0:2, samples=100] {sin(deg(x))};
			\addplot [domain=0:2, samples=100, dashed] {x};
			\addplot [domain=0:2, samples=100, dotted] {x - (x^3)/6};
			\legend{
				$\sin(x)$,
				$x$,
				$x - x^3\!/6$
			}
		\end{axis}
	\end{tikzpicture}
	\caption{}
\end{wrapfigure}
Также существует равенство для нескольких малых переменных:
\begin{equation*}
	(1 + a)^\alpha (1 + b)^\beta \ldots \approx 1 + \alpha a + \beta b + \ldots
\end{equation*}

\subsection{Дифференциальные уравнения}
Под \term{дифференциальными уравнениями} понимаются любые алгебраические или трансцендентные нетривиальные равенства, содержащие дифференциалы или производные. Существует большое число различных видов дифференциальных уравнений, изучение которых занимает не один семестр университетской программы. В рамках текущего повествования рассмотрим только два простейших из них для создания общего представления о дифференциальных уравнениях.

\subsubsection{Уравнения с разделяющимися переменным}

Рассмотрим уравнение $ P(t,x)\,dt + Q(t,x)\,dx = 0$, где $P$ и $Q$~--- некоторые непрерывные в некоторой области $\Omega \subseteq \R^2$ функции. Если их можно представить в виде $P(t,x) = T_p(t)X_p(x)$ и $Q(t,x) = T_q(t)X_q(x)$, то это уравнение с \term{разделяющимися переменными}. Его можно записать как
\begin{equation*}
    T_p(t)X_p(x)\,dt + T_q(t)X_q(x)\,dx = 0.
\end{equation*}

Отдельно необходимо рассмотреть случаи, когда $X_p(x) = 0$ или $T_q(t) = 0$, а общее решение можно найти, разделив обе части уравнения на $X_p(x)T_q(t)$. В таком случае уравнение примет вид
\begin{equation*}
    \frac{T_p(t)}{T_q(t)} \, dt = - \frac{X_q(x)}{X_p(x)} \, dx,
\end{equation*}
интегрируя левую и правую часть, получим,
\begin{equation*}
     \int \frac{T_p(t)}{T_q(t)} \, dt = - \int \frac{X_q(x)}{X_p(x)} \, dx + C,
\end{equation*}
где константа интегрирования $C$ определяется из начальных условий. Примеры использования и решения таких уравнений можно найти в разделах~\ref{sec:gravity-law} и ~\ref{sec:earth-atmosphere}.

\subsubsection{Уравнения гармонического консервативного осциллятора}

\term{Гармонические колебания} материальной точки совершаются под действием сил, пропорциональных смещению колеблющейся точки от положения равновесия и направленной противоположно этому смещению. Примерами гармонических колебаний могут служить математический и пружинный маятники.

Рассмотрим материальную точку массы $m$ и возвращающую силу $F = -kx$, где $x$~--- смещение материальной точки от точки равновесия, где сила $F$ на неё не действует, а $k$~--- некая постоянная, определяющая величину силы $F$.

Тогда по второму закону Ньютона ускорение $a$ материальной точки можно записать как
\begin{equation*}
    \frac{d^2 x}{dt^2} \equiv a = - \frac{k}{m} x.
\end{equation*}
Обозначим $k/m$ как $\omega^2$, а вторую производную смещения по времени как $\ddot x$, получим уравнение
\begin{equation}
    \ddot x + \omega^2 x = 0.
\end{equation}

Это уравнение описывает поведение консервативного гармонического осциллятора (\imp{гармонических колебаний} без затухания). Его общее решение имеет вид
\begin{equation}
	x(t) = C_1 \sin \omega t + C_2 \cos \omega t = C_0 \sin (\omega t + \varphi_0),
	\label{eq:harmonic-oscillation}
\end{equation}
где~$C_0$, $C_1$~и~$C_2$~--- некоторые константы, определяющие амплитуду колебаний,~а~$\varphi_0$~--- начальная фаза колебаний. Все параметры колебаний устанавливаются из начальных условий. 

Из~\eqref{eq:harmonic-oscillation} видно, что для периода колебаний справедливо выражение
\begin{equation}
	T = \frac{2 \pi}{\omega}.
\end{equation}
Величину $\omega$ называют круговой частотой, так как измеряется она в радианах в секунду. Частота $f$ в Гц связана в круговой частотой соотношением $\omega = 2 \pi f$.






	\newpage
\section{Практическая астрономия}
\input{sections/prac-astro.arc-radius.tex}
\input{sections/prac-astro.ols.tex}
\input{sections/prac-astro.error.tex}
	\newpage

\section*{Приложение}
\addcontentsline{toc}{section}{Приложение}
\renewcommand{\leftmark}[1]{Приложение}

\begin{table}[h!]
    \scriptsize
    \renewcommand{\arraystretch}{1.5}
    \renewcommand{\tabcolsep}{1pt}
    \centering
    \begin{tabularx}{\tw}{|C{.13}|C{.125}|C{.15}|C{.125}|C{.115}|C{.16}|C{.15}|}
        \hline
        \quad\quad\quad Объект &  Большая полуось  $\mathbf{a}$,~а.\,е. & Сиде\-ри\-чес\-кий пе\-риод $\mathbf{T}$,~год & Эксцен\-триситет $\mathbf{e}$ & Накло\-нение $\mathbf{i}$,~$~^\circ$ & Долгота восходящего узла $\mathbf{\Omega}$,~$~^\circ$ & Аргумент перицентра $\boldsymbol{\omega} $, $~^\circ$\\
        \hline
        Меркурий & 0.387 & 0.241 & 0.206 & 7.00  & 48.3  & 29.1\\

        Венера     & 0.723 & 0.615 & 0.007 & 3.39  & 76.7  & 54.9\\

        Земля    & 1.000 & 1.000 & 0.017 & 0.00    & 348.7 & 114.2\\

        Марс     & 1.524 & 1.881  & 0.093 & 1.85  & 49.6  & 286.5\\

        Церера   & 2.765 & 4.601  & 0.079 & 10.6 & 80.4  & 2.83\\

        Юпитер   & 5.204 & 11.86 & 0.048 & 1.31  & 100.6 & 275.1\\

        Сатурн   & 9.582 & 29.46 & 0.056 & 2.49  & 113.6 & 336.0\\

        Уран     & 19.23 & 84.01 & 0.044 & 0.77  & 74.0  & 96.5\\

        Нептун   & 30.06  & 164.8 & 0.011 & 1.77  & 131.8 & 265.6\\

        Плутон   & 39.48 & 247.9 & 0.249 & 17.1 & 110.2 & 113.8\\
        \hline
    \end{tabularx}
    \caption{Параметры орбит больших тел Солнечной системы}
\end{table}
\begin{table}[h!]
    \footnotesize
    \renewcommand{\arraystretch}{1.5}
    \renewcommand{\tabcolsep}{2pt}
    \centering
    \begin{tabularx}{\tw}{|C{0.14}|C{0.16}|C{0.17}|C{0.17}|C{0.17}|C{0.2}|}
        \hline
        Объект & Символ & Масса $\mathbf M$,~кг & Радиус $\mathbf R$,~м & Период $\mathbf T$,~ч & Наклон оси $\mathbf i$,~$~^\circ$\\
        \hline
        Солнце & $\odot$ & $1.99 \times 10^{30}$ & $6.97 \times 10^8$ & 609.1 &7.25\\

        Меркурий & $\mercury$ & $3.33 \times 10^{23}$ & $2.44 \times 10^6$ & 1408. & 0.035\\

        Венера   &  $\venus$  & $4.87 \times 10^{24}$ & $6.05 \times 10^6$ & 5833. & 177.4\\

        Земля    & $\oplus$   & $5.97 \times 10^{24}$ & $6.37 \times 10^6$ & 23.93 & 23.44 \\
        Луна    &    $\rightmoon$ & $7.35 \times 10^{22}$ & $1.74 \times 10^6$ &  655.7 & 1.54\\

        Марс    & $\mars$ & $6.42 \times 10^{23} $ & $3.39 \times 10^6 $  & 24.62 & 25.19 \\

        Церера  &  & $9.39 \times 10^{20}$ & $4.63 \times 10^{5}$  & 9.077 & 3 \\

        Юпитер   &$\jupiter$ & $1.90 \times 10^{27}$ & $7.00 \times 10^{7}$ & 9.925 & 3.13 \\

        Сатурн   &$\saturn$& $5.68 \times 10^{26}$ & $5.82 \times 10^{7}$ & 10.53 & 26.73  \\

        Уран    & $\uranus$& $8.68 \times 10^{25}$ & $2.54 \times 10^7$ & 17.24 & 97.77  \\

        Нептун   &$\neptune$& $1.02 \times 10^{26}$  & $2.46 \times 10^7$ & 15.97 & 28.32  \\

        Плутон   &$\pluto$& $1.30 \times 10^{22}$ & $1.19 \times 10^6 $ & 153.3 & 119.6 \\
        \hline
    \end{tabularx}
    \caption{Физические характеристики больших тел Солнечной системы}
\end{table}
\noindent Светимость Солнца $L_\odot$ \hfill $3.828 \times 10^{26}$~Вт\\
Видимая звёздная величина Солнца $m_\odot$ \hfill $-26.74^m$\\
Абсолютная звёздная величина Солнца $M_\odot$ \hfill $+4.83^m$\\
Показатель цвета Солнца $(B - V)_\odot$ \hfill $+0.67^m$\\
Эффективная температура Солнца $T_\odot$ \hfill $5778$~К\\
Большая полуось орбиты Луны $a_{\moon}$ \hfill $384399$~км\\
Эксцентриситет орбиты Луны $e_{\moon}$ \hfill $0.055$\\
Наклонение плоскости орбиты Луны к эклиптике $i_{\moon}$ \hfill $5.15^\circ$\\
Сидерический период Луны $T_{\moon}$ \hfill $27.3217$~сут\\
Геометрическое альбедо Луны $A_{\moon}$ \hfill $0.12$\\
Видимая звёздная величина Луны в полнолунии $m_{\odot}$ \hfill $-12.7^m$\\
Геометрическое Альбедо Земли $A_\oplus$ \hfill $0.37$\\[-5pt]
\rule{\tw}{.7pt}\\
Заряд электрона $e$ \hfill $-1.6 \times 10^{-19}$~Кл\\
Постоянная Планка $h$ \hfill $6.626 \times 10^{-34}~\text{Дж}\cdot\text{с}$\\
Постоянная Стефана-Больцмана $\sigma$ \hfill $5.670 \times 10^{-8}~\text{Вт} \cdot \text{м}^{-2} \cdot \text{К}^{-4}$
Гравитационная постоянная $G$ \hfill $6.672 \times 10^{-11}~\text{м}^3 \cdot \text{с}^{-2} \cdot \text{кг}^{-1}$\\
Постоянная Больцмана $k$ \hfill $1.381 \times 10^{-23}~\text{Дж} \cdot \text{К}^{-1}$\\
Постоянная Хаббла $H$ \hfill $67~\text{км} \cdot \text{с}^{-1} \cdot \text{Мпк}^{-1}$


\begin{figure}[h!]
    \centering
    \vspace{-.5pc}
    \includegraphics[width=\tw]{sky-map.pdf}
    \caption{Карта звёздного неба до $-45^\circ$ по склонению}
    \vspace{-1cm}
\end{figure}

	\newpage
\thispagestyle{empty}
\begin{center}
	{ \bfseries \itshape Для заметок}
\end{center}
	
	\bibliographystyle{ieeetr}
	\bibliography{lit}
	%\newpage
%\thispagestyle{empty}
~
\vfill
\centering
\begin{minipage}{0.8\tw}
	\small
	\centering
	Подписано в печать \signed.\\
	Формат 84$\times$108/32. Бумага офсетная. Печать офсетная.\\
	Тираж 500 экз.\\ 
	Отпечатано в ГУП МО <<Коломенская типография>>.\\ 140400, г.~Коломна, ул.~III Интернационала, д.~2a
\end{minipage}
\label{pg:last-page}


\end{document}
