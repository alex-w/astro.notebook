\section*{Предисловие}

Приветствую тебя, дорогой читатель! Я, Шепелев Алексей Сергеевич, один из авторов данной книги, хочу познакомить тебя с Астрадью поближе и дать несколько рекомендаций по работе с ней. Да-да, именно работе, а не прочтению, так как материал, собранный в этой книге необходимо тщательно прорабатывать, чтобы извлечь из него максимальную пользу.

\subsection*{Исторический экскурс}

Позволю себе отвлечься и рассказать о себе. В школе я увлекался астрономией, как и многие мои друзья, кто, собственно, стал ими благодаря этой науке.  С кем-то мы вместе ходили в один кружок им.~Е.\,П.~Левитана под руководством Кузнецова Михаила Владимировича в небольшом подмосковном городе Жуковский, с кем-то познакомились на областных сборах, пройдя отбор на заключительный этап Всероссийской олимпиады школьников~(всеросс), а с кем-то уже на учебно-трениро\-вочных сборах для призёров и победителей всеросса~--- кандидатов в сборную РФ по астрономии. А кто-то поддерживал меня перед наблюдательным туром Международной олимпиады школьников~(IAO).

Вероятно, у тебя, читатель, возник вопрос: получается, ты, автор, добился всего, чего может пожелать олимпиадник в школьные годы? Возможно\dots~ Скажу одно, я точно был счастлив, видимо, поэтому вспоминаю те годы с таким трепетом и теплом.

Однако я так и не завоевал ни одной золотой медали на международных олимпиадах. Почему? Самое очевидное~--- недоработал. Жалею? Возможно. Но сейчас речь не об этом. Парадокс~--- я боялся читать книжки. Нет-нет, не шучу. Я боялся даже начать читать книжку, если она должна была стать сложной. Это и привело меня к написанию данного предисловия.

Впервые Астрадь была напечатана на обычном принтере уже в далеком 2013 году Борисовым Святославом Борисовичем тогда учеником 11 класса Гимназии~№2 города Раменское и упомянутого выше астрономического кружка. Он сверстал её в MS~Word, собрав в брошюре на 40 страниц самые важные и известные законы и формулы, знание которых могло пригодиться участнику олимпиады по астрономии. Он сделал это для друзей, которых подарила ему, как и мне, астрономия.

Спустя год, 8 апреля 2014 года, на всероссе в Великом Новгороде за завтраком я листал электронный вариант той, самой первой, Астради. Взгляд остановился на неизвестной мне тогда формуле относительного отверстия. Ровно через сутки стало известно, что я получио полный балл за решение задач теоретического тура, что произошло впервые за всю историю олимпиады. Кстати, там была задача про относительное отверстие.

В 2017 году появилась мысль сверстать Астрадь в \LaTeX, дополнить, но оставить оригинальный формат небольшой брошюры, которая не испугает даже самого юного читателя. Весной 2018 года перед всероссом было напечатано первое официальное издание Астради: в типографии, с ISBN кодом, тиражом 500~экземпляров. Участники всеросса 2018 года в Волгограде по достоинству оценили это достижение. Позже были исправлены найденные опечатки, и в декабре 2020 года было напечатано второе издание чуть меньшим тиражом.

Тогда помимо учебы в университете, я часто преподавал на учебных сборах по астрономии, в том числе на областных в Московской области. Где и узнал неприятный для себя факт: недобросовестные школьники не пытаются разобраться и понять формулы и соотношения, приведенные в Астради с весьма немногословными объяснениями, а лишь пробуют найти в книжке буквы, а правильнее сказать переменные, из условия задачи.\footnote{Например, орбита с эксцентриситетом $i$ и наклонением $e$ из условия задачи имела бы в решении совершенно другую форму и ориентацию в пространстве.}

Это и побудило к пересмотру самой идеи Астради. Было решено добавить в неё вывод формул и полезных фактов, существенно увеличить объём, а повествование вести в более общих терминах, сместив фокус с частных случаев на разбор фактов в общем виде. Кроме этого хотелось приоткрыть дверь в науку, показать подходы, идеи и методы. Так появилась на свет книжка, которую ты держишь перед собой. 

\subsection*{Инструкция}
Как было сказано выше, это предисловие~--- попытка помочь тебе не испугаться объема книги, математических преобразований и функций, физических законов, моделей и приближений, встречающих с первой страницы.

Не бойся \textbf{преобразований}. Мы постарались изложить все настолько подробно, чтобы любое <<очевидно, что>> было действительно очевидно, конечно, только при условии внимательного изучения предыдущего материала. Не пропускай подробные выкладки, только проделав их самостоятельно, ты действительно прочувствуешь полученный результат.

Не бойся \textbf{сложных формул}. Любая сложная, может быть где-то страшная, запись~--- это лишь краткое представление какого-то факта. Читай и разбирайся в определениях, это приведёт к лучшему пониманию материала. Любой факт имеет частные случаи, особые формы при некоторых приближениях. Начинай с них, простых случаев, это поможет сэкономить время в критических ситуациях. Едва ли кто-то решает задачи в максимально общем виде, разве что суперкомпьютеры.

Не бойся \textbf{искать}. В книге приведены определения основных математических и физических понятий, но все-таки это книга по астрономии. Так что не бойся разбираться с каждым непонятным словом или математическим оператором, выходя за пределы Астради.

Не бойся \textbf{исследовать}. Астрадь стала больше, но она не стала учебником. Мы постарались упорядочить разделы в порядке зависимости одного от другого, но часто оказывается, что связь разделов имеет различное направление в зависимости от глубины погружения в тему. Поэтому вряд ли имеет смысл читать Астрадь по порядку от корки до корки. Советую тебе шаг за шагом открывать для себя новые разделы, проходя каждый не один и даже не два раза. Первое знакомство может быть поверхностным, лишь узнать ключевую формулу. Второе, например, поможет разобраться с выводом, получить понимание причинности используемой формулы. А, к примеру, в третий раз получится разобраться со следствиями уже знакомого закона.

Не бойся не запомнить~--- гораздо важнее \textbf{понять} и \textbf{уловить идею}. Не все выводы и факты необходимо знать наизусть, только самые важные.  Цель авторов этого издания не только снабдить читателя необходимыми формулами и фундаментальными фактами, но и доступно показать возможные подходы к формализации законов природы порой далеко не школьным методами. Некоторые выводы и факты можно воспринимать, как решение некой абстрактной задачи.

Не бойся \textbf{изучать иллюстрации}. Не побоюсь громких слов, в иллюстрации вложена душа. И сотни часов работы, конечно. Иллюстраций много, они разнообразны: схемы, графики, диаграммы, модели, фотографии. Внимательно изучи каждую. Визуальная память~--- самая сильная среди всех видов памяти. Кроме этого иллюстрации помогают создать визуальный образ, помогающий в решении задач и дальнейшем изучении астрономии, который удается хорошо и надолго запомнить.

Не бойся \textbf{читать статьи} и \textbf{другие книги}. Мы постарались наполнить Астрадь ссылками на всевозможные источники, где либо приведен оригинальный материал по теме, либо тема разобрана существенно глубже, чем это может позволить формат данной книги. Для твоего удобства список литературы снабжен QR кодами со ссылками на печатные версии упоминаемых источников. Часть статей на английском, большинство книг~--- университетские учебники. Но это не должно отталкивать тебя: нет никакой необходимости читать их полностью, попробуй разобраться в соответствующем разделе. Уверен, у тебя получится!

\begin{flushright}
    Удачи!\\
    \it Алексей Шепелев
\end{flushright}
