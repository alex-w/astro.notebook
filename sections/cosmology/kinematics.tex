\subsection{Кинематика}
Эволюция Вселенной соответствует \imp{космологическому принципу}, если из состояния Вселенной в один момент времени, можно получить её состояние в любой другой момент путем только лишь масштабирования. Соответственно, расстояние $D$ между двумя несвязанными объектами будет меняться по следующему закону: 
\begin{equation}
D(t) = a(t)D_0,
\label{eq}
\end{equation}
 где $a(t)$~--- масштабный фактор, $D_0$~--- а расстояние между этими объектами сейчас.

\term{Масштабный фактор}~--- это безразмерная величина, характеризующая эволюцию Вселенной. Во всех точках Вселенной в заданный момент времени масштабный фактор одинаков. В момент большого взрыва масштабный фактор принимают равным нулю, а единица соответствует текущему моменту.

Продифференциируем (\ref{eq}) по времени и разделим на себя:
\begin{equation}
\frac{1}{D} \frac{dD}{dt} = \frac{1}{aD_0} \frac{D_0 \,d a}{dt} = \frac{\dot{a}}{a} \equiv H(t).
\label{eq2}
\end{equation}
Таким образом, мы получили величину, не зависящую от расстояния, и изменяющуюся только со временем. Параметр $H(t)$ называется \term{пара\-мет\-ром Хаббла-Леметра}. А его значение сейчас, т.\,е. на данный момент~$t_0$~--- \term{постоян\-ной Хаббла}, которая равна
\begin{equation}
    H_0 = H(t_0) \approx 67...\,\frac{\text{км}}{\text{c} \cdot \text{Мпк}}.
\end{equation} 
По сути, этот параметр определяет наклон касательной к графику $a(t)$ в какой-либо момент времени, то есть показывает, расширяется Вселенная, или нет. Вторая производная \imp{масштабного фактора} по времени  определяется \term{параметром замедления} $q = -\ddot{a}a \slash \dot{a}^2$:
\begin{equation}
\frac{\ddot{a}}{a} = -qH^2.
\label{eq3}
\end{equation}
Из (\ref{eq3}) видно, что при отрицательном значении $q$ Вселенная эволюционирует с ускорением, а при положительном~--- с замедлением. Судя по наблюдениям, наша Вселенная ускоренно расширяется.

Так как выделенного направления в расширении Вселенной нет, то при отсутствии пекулярных скоростей, все объекты, подверженные расширению будут удаляться от наблюдателя радиально. Напрямую из~(\ref{eq}) следует $\dot{D} \slash D = \dot{a} \slash a = H$, то есть \term{закон Хаббла}:
\begin{equation}
v_r = HD
\label{hubble}
\end{equation}
Запишем \imp{эффект Доплера}:
\begin{equation}
\frac{\lambda_\text{o} - \lambda_\text{e}}{\lambda_\text{e}} = z \simeq \frac{v_r}{c}.
\end{equation}
Здесь $\lambda_\text{o}$~--- длина волны принимаемого света, а $\lambda_\text{e}$~--- излучаемого. Стоит заметить, что \term{красное смещение} $z$ может быть больше $1$, а последнее равенство выполняется только для $v_r \ll c.$

Рассмотрим теперь объект на небольшом расстоянии $D = cdt$ от наблюдателя таком, что $v_r \slash c = d \lambda \slash \lambda \ll 1$. Используя \imp{закон Хаббла}, получаем:
\begin{equation}
\frac{v_r}{c} = \frac{d \lambda}{\lambda} = \frac{D \dot{a}}{ca} = dt \frac{\dot{a}}{a} = \frac{da}{a}.
\end{equation}
\begin{equation}
\int_{\lambda_\text{e}}^{\lambda_\text{o}} \frac{d \lambda}{\lambda} = \int_{a}^{1} \frac{da}{a} \Rightarrow \ln \lambda_\text{o} - \ln \lambda_\text{e} = \ln 1 - \ln a \Rightarrow \frac{\lambda_\text{o}}{\lambda_\text{e}} = \frac{1}{a}
\label{dopler}
\end{equation}
%\ln \frac{\lambda_\text{o}}{\lambda_\text{e}} = \ln \frac{1}{a} \Rightarrow 
Чтобы получить связь между \imp{масштабным фактором} и \imp{красным смещением}, используем \imp{эффект Доплера}. Из него следует равенство $\lambda_{\text{o}} / \lambda_{\text{e}} = z+1.$ После подстановки в \eqref{dopler} получаем зависимость $a(z)$:
\begin{equation}
a = (1+z)^{-1}
\label{zf}
\end{equation}
Последнее уравнение очень важно в современной космологии. Оно показывает зависимомть \imp{масштабного фактора}, величины, не имеющей явного физического смысла, от \imp{красного смещения}, которое можно просто измерить. \newline \linebreak
%Определим некоторые важные размерные константы:
%\begin{enumerate}
%\item \imp{Постоянная Хаббла} $H_0 \approx 67 \text{ km} \text{ s}^{-1} \text{Mpc}^{-1} \approx 2.26 \cdot 10^{-18} \text{ s}^{-1}$
%\item \imp{Хаббловское время} сейчас $t_{H_0} = H_{0}^{-1} \approx 14.0 \text{ Gyr}$
%\item \imp{Хабблосвское расстояние} сейчас $D_{H_0} = \displaystyle \frac{c}{H_0} \approx 4.28 \text{ Gpc}$
%\end{enumerate}
Теперь продифференциируем \text{(\ref{zf}) }по времени:
\begin{equation}
\frac{da}{dt} = - \frac{1}{(1+z)^{2}} \frac{dz}{dt} = - \frac{a}{1+z} \frac{dz}{dt},
\label{dadz}
\end{equation}
разделим на $a$:
\begin{equation}
H = \frac{1}{a} \frac{da}{dt} = \frac{d\ln (a)}{dt} = - \frac{1}{1+z} \frac{dz}{dt},
\end{equation}