\subsection{Аберрации в оптике}
\paragraph{Сферическая аберрация}
В оптических системах, содержащих сферические поверхности (линзы, зеркала) может наблюдаться \imp{сферическая аберрация}. Суть такой аберрации в том, что лучи, параллельные оптической оси, идущие на разном расстоянии от нее собираются в разных её местах. Это приводит к тому, что на краях получаемого изображения, например, на ПЗС матрице, теряется резкость.

Покажем наличие сферической аберрации для плоско-выпуклой линзы. Важно отметить, здесь уже нельзя считать линзу тонкой, так как само по себе понятие тонкой линзы включает в себя условие фокусировки лучей в одной точке (фокусе), чего не происходит на практике. 

Итак, рассмотрим плоско-выпуклую линзу с радиусом выпуклой стороны $R$. Рассмотрим также луч, параллельный оптической оси этой линзы, на расстоянии $d$ от неё (см.~Рис.\,\ref{??}). Так как передняя поверхность линзы плоская, луч, попадая в линзу, не преломляется. Преломление происходит на выходе из линзы. Нетрудно показать, что для угла падения луча на заднюю поверхность линзы $\alpha$ справедливо, что $\sin \alpha = d/R$. По закону Снеллиуса угол преломления рассматримоего луча $\beta$ определяется соотношением $\sin \beta = n \sin \alpha$. Так как угол между нормалью к выпуклой поверхности линзы и ее оптической осью равен $\alpha$, то угол $\gamma$ между преломленным лучем и оптической осью линзы составляет $\beta - \alpha$.

Расстояние до точки пересечения преломленного луча с оптической осью линзы будем отсчитывать от вершины выпуклой поверхности. Расстояние $h(d)$ между проекцией точки преломления на оптическую ось и вершиной можно найти из теоремы Пифагора:
\begin{equation*}
	h(d) = R - \sqrt{R^2 - d^2}.
\end{equation*}
Тогда координата фокуса для лучей на расстоянии $d$ от оптической оси равно
\begin{equation*}
	x = \frac{d}{\tg \gamma} - h(d) = \frac{d}{\tg \left( \arcsin \dfrac{n d}{R} - \arcsin \dfrac{d}{R} \right)} - \left( R - \sqrt{R^2 - d^2} \right).
\end{equation*}
Введём обозначение $\delta \equiv d/R$ и разделим обе части полученного равенства на $R$, чтобы перейти к относительным единицам:
\begin{equation}
	\frac{x}{R} 
	= \frac{\delta}{\tg \left( \arcsin n\delta - \arcsin \delta \right)} -  1 + \sqrt{1 - \delta^2} 
	~\overset{\delta \ll 1}{\longrightarrow}~  \frac{1}{n - 1} - \frac{8\delta^2}{3}.\footnote{\scriptsize Вывод приближения:
\begin{multline*}
	\frac{\delta}{\tg \left( \arcsin n\delta - \arcsin \delta \right)} -  1 + \sqrt{1 - \delta^2} =\\
	= \frac{\delta}{\tg \left[ n \delta + \dfrac{n^3 \delta^3}{6} - \delta - \dfrac{ \delta^3}{6} + o(\delta^3) \right]} -  1 + \left(1 - \frac{\delta^2}{2} + o(\delta^2) \right) =\\
	= \frac{\delta}{\tg \left[ \delta(n-1) + \dfrac{(n^3-1) \delta^3}{6} + o(\delta^3) \right]} - \frac{\delta^2}{2} + o(\delta^2) =\\
	= \frac{\delta}{\delta(n-1) + \dfrac{(n^3-1) \delta^3}{6} + \dfrac{\delta^3}{3}(n-1)^3 + o(\delta^3)} - \frac{\delta^2}{2} + o(\delta^2) =\\
	= \frac{1/(n-1)}{1 + \dfrac{\delta^2}{6}(n^2 + n + 1) + \dfrac{\delta^2}{3}(n-1)^2 + o(\delta^2)} - \frac{\delta^2}{2} + o(\delta^2) =\\
	=\frac{1}{n-1} \left[1 - \frac{\delta^2}{6} (3n^2 -3n + 3) \right] - \frac{\delta^2}{2} + o(\delta^2) 
%	= \frac{1}{n-1} - \frac{\delta^2}{2(n-1)}(n^2 - n + 1 + n - 1) + o(\delta^2) 
	\simeq \frac{1}{n-1} - \frac{\delta^2 n^2}{2(n-1)}.
\end{multline*} 
}
\end{equation}
 

\begin{figure}[h!]
	\centering
	\begin{tikzpicture}
	\begin{axis}[
		height	=	6cm,
		width	=	8cm,
		xlabel	=	{$\delta$},
		ylabel	=	{$\dfrac{x(d)}{R}$},
		ylabel shift	= -1.1 cm,
		extra x ticks ={0.75},
    	extra x tick labels={$\frac{1}{n}$},
    	xmin=-.05,
    	xmax=0.8,
    	ymin=0,
    	ymax=3.5
		]
		\addplot[smooth] table[x=d, y=x] {data/shere-aberrations-lens.txt};
		\addplot[smooth] table[x=d, y=simple] {data/shere-aberrations-lens.txt};
		\addplot[dashes] coordinates { (0.75, -10) (0.75, 10)};
	\end{axis}
\end{tikzpicture}
\caption{}
\end{figure}

Найдём область определения функции $x(\delta)$. Прежде всего $\delta \geqslant 0$, потому что $d$~--- это расстояние от оптической оси, которое не может быть отрицательным. С другой стороны радиус линзы не может быть больше радиуса кривизны ее поверхности, следовательно, $\delta < 1$. Однако есть ещё одно условие, которое ограничивает $\delta$ сверху. Это эффект полного внутреннего отражения. Действительно, $\sin \beta$ не может быть больше единицы, следовательно, $\sin \alpha < \sfrac{1}{n}$, а значит, $\delta < \sfrac{1}{n}$. Для стекла, коэффициент преломления которого $n = \sfrac{4}{3}$, получаем, что $\delta \in [0, \sfrac{3}{4})$.

Как видно из графиков, приближение работает