\subsection{Аберрации в оптике}
\paragraph{Хроматическая аберрация}
Пожалуй главным недостатком оптических схем, содержащий преломляющие оптические элементы (линзы; призмы, за исключением использования их для спектроскопии) является \imp{хроматические аберрации}. Дело в том, что показатель преломления материала линзы зависит от длины волны падающего излучения. Это приводит к тому, что положения фокуса оптической системы зависит от длины волны излучения. При наблюдениях это проявляется как радужный ореол вокруг объектов, ухудшающий качество изображения.

\begin{wrapfigure}{r}{0.5\tw}
    \centering
    \vspace{-1pc}
    \tikzsetnextfilename{crown-dispersion}
    \begin{tikzpicture}
        \begin{axis} [
            height  =   4.5cm,
            width   =   6cm,
            xlabel  =   {$\lambda$, мкм},
            ylabel  =   {$n(\lambda)$},
            xmin    =   0.3,
            xmax    =   2.5,
            ymin    =   1.48,
            ymax    =   1.55,
        ]
            \addplot[smooth, domain=0.3:2.5] table[x=l, y=n] {data/crown-dispersion.txt};
        \end{axis}
    \end{tikzpicture}
    \caption{}
    \label{pic:crown-dispersion}
\end{wrapfigure}
Найдем зависимость величины хроматической аберрации от величины дисперсии материала линзы. В качестве линзы рассмотрим плосковыпуклую линзу из оптического стекла~--- \imp{крона} (BK7). Зависимость $n(\lambda)$ её показателя преломления $n$ от длины волны $\lambda$ проходящего излучения представлена на графике (см.~Рис\,\ref{pic:crown-dispersion}). Важно отметить, здесь нельзя считать линзу тонкой, так как само по себе понятие тонкой линзы подразумевает отсутствие разного рода аберраций и условие фокусировки лучей в одной точке (фокусе), чего не происходит на практике.

\begin{wrapfigure}[9]{r}{0.63\tw}
    \centering
    \vspace{-.8pc}
    \tikzsetnextfilename{sphere-aberrations-lens}
    \begin{tikzpicture}
        \footnotesize

        \draw [decoration={snake, segment length=.7mm, amplitude=0.2mm}, decorate] (1.35, 1) arc(-31:15:0.26);
        \draw [double, line cap=butt] (.2, 1.135) arc(180:195:0.94);
        \draw  (2.7, 0) arc(180:149:0.3);

        \fill [lightgray] (.5, -1.5) -- (.5, 1.5) -- (1, 1.5) arc(20:-20:4.386) -- (.5, -1.5);

        \draw [thick] (.5, -1.5) -- (.5, 1.5);
        \draw [thick] (1, -1.5) arc(-20:20:4.386);
        \draw [semithick, dash pattern={on 5pt off 2pt on .5pt off 2pt}] (-3.8, 0) -- (3.2, 0);
        %        \draw (-3.12, 0) node{$\times$};

        \draw [dashes] (-3.12, 0) -- (1.71, 1.29);

        \draw [semithick] (-3.8, 1.135) -- (1.12, 1.135) -- (3, 0);
        \draw [-latex] (-3.5, 1.135)-- (-1, 1.135);
        \draw [-latex] (1.12, 1.135) -- (2.25, 0.45);

        \draw [latex-latex] (-3.5, 0) -- (-3.5, 1.135);
        \draw [latex-latex] (1.27, -.5) -- (3, -.5);
        \draw [latex-latex] (1.12, -1.4) -- (3, -1.4);
        \draw [-latex] (0.5, -.5) -- (1.12, -.5);

        \draw (1.12, 1.135) -- (1.12, -1.5);
        \draw (1.27, 0) -- (1.27, -.6);
        \draw (3, 0) -- (3, -1.5);

        \draw (-3.5, 0.56) node[anchor=east] {$d$};
        \draw (-3.12, 0) node[anchor=north] {$C$};
        \draw (-.7, 0.6) node[anchor=north] {$R$};
        \draw (2.14, -.5) node[anchor=south] {$x$};
        \draw (2.14, -1.4) node[anchor=south] {$\frac{d}{\tg \gamma}$};

        \draw (.8, -1.5) node[anchor=south] {$n$};

        \draw (.2, 1) node[anchor=east] {$\alpha$};
        \draw (1.4, 1.07) node[anchor=west] {$\beta$};
        \draw (2.65, 0.15) node[anchor=east] {$\gamma$};

        \draw [fill=white] (3, 0) circle (.03);
        \draw [fill=white] (-3.12, 0) circle (.03);
    \end{tikzpicture}
    \caption{}
    \label{pic:sphere-aberrations-lens}
\end{wrapfigure}
Итак, пусть радиус кривизны выпуклой поверхности рассматриваемой плос\-ко-вы\-пук\-лой линзы равен $R$. Рассмотрим луч, параллельный оптической оси линзы, идущий на расстоянии~$d$ от неё (см.~Рис.\,\ref{pic:sphere-aberrations-lens}). Так как передняя поверхность линзы плоская, луч, попадая в линзу, не преломляется. Преломление происходит на выходе из линзы. Нетрудно показать, что для угла падения луча на заднюю поверхность линзы $\alpha$ справедливо равенство $\sin \alpha = d/R$. По закону Снеллиуса угол преломления $\beta$ рассматримоего луча  определяется соотношением $\sin \beta = n \sin \alpha$. Так как угол между нормалью к выпуклой поверхности линзы и её оптической осью равен $\alpha$, то угол $\gamma$ между преломленным лучем и оптической осью линзы равен $\beta - \alpha$.

Расстояние до точки пересечения преломлённого луча с оптической осью линзы будем отсчитывать от вершины выпуклой поверхности линзы. Расстояние $h$ между проекцией точки преломления на оптическую ось и вершиной найдём из теоремы Пифагора:
\begin{equation*}
    h = R - \sqrt{R^2 - d^2}.
\end{equation*}
Получим, координата фокуса для лучей на расстоянии $d$ от оптической оси равна
\begin{equation}
    x = \frac{d}{\tg \gamma} - h = \frac{d}{\tg \left( \arcsin \dfrac{n d}{R} - \arcsin \dfrac{d}{R} \right)} - \left( R - \sqrt{R^2 - d^2} \right).
    \label{eq:optics-aberr-x(d)}
\end{equation}
Введём обозначение $\delta \equiv d/R$ и разделим обе части полученного равенства на $R$, чтобы перейти к относительным единицам:
\begin{equation}
    \frac{x}{R}
    = \frac{\delta}{\tg \left( \arcsin n\delta - \arcsin \delta \right)} -  1 + \sqrt{1 - \delta^2}
    ~\xrightarrow{\delta \ll 1}~  \frac{1}{n - 1} - \frac{\delta^2 n^2}{2(n-1)}.\footnote{\scriptsize Вывод приближения:
    \begin{multline*}
        \frac{\delta}{\tg \left( \arcsin n\delta - \arcsin \delta \right)} -  1 + \sqrt{1 - \delta^2} =\\
        = \frac{\delta}{\tg \left[ n \delta + \dfrac{n^3 \delta^3}{6} - \delta - \dfrac{ \delta^3}{6} + o(\delta^3) \right]} -  1 + \left(1 - \frac{\delta^2}{2} + o(\delta^2) \right) =\\
        = \frac{\delta}{\tg \left[ \delta(n-1) + \dfrac{(n^3-1) \delta^3}{6} + o(\delta^3) \right]} - \frac{\delta^2}{2} + o(\delta^2) =\\
        = \frac{\delta}{\delta(n-1) + \dfrac{(n^3-1) \delta^3}{6} + \dfrac{\delta^3}{3}(n-1)^3 + o(\delta^3)} - \frac{\delta^2}{2} + o(\delta^2) =\\
        = \frac{1/(n-1)}{1 + \dfrac{\delta^2}{6}(n^2 + n + 1) + \dfrac{\delta^2}{3}(n-1)^2 + o(\delta^2)} - \frac{\delta^2}{2} + o(\delta^2) =\\
        =\frac{1}{n-1} \left[1 - \frac{\delta^2}{6} (3n^2 -3n + 3) \right] - \frac{\delta^2}{2} + o(\delta^2)
        %    = \frac{1}{n-1} - \frac{\delta^2}{2(n-1)}(n^2 - n + 1 + n - 1) + o(\delta^2)
        \simeq \frac{1}{n-1} - \frac{\delta^2 n^2}{2(n-1)}.
    \end{multline*}
    }
\end{equation}

Найдём область определения функции $x(\delta)$. Прежде всего учтём, что~$\delta \geqslant 0$, потому что $d$~--- это расстояние от оптической оси, которое не может быть отрицательным. С другой стороны, радиус линзы не может быть больше радиуса кривизны ее поверхности, следовательно, $\delta < 1$. 

Однако есть ещё одно условие, ограничивающее $\delta$ сверху. Это эффект полного внутреннего отражения. Действительно, $\sin \beta$ не может быть больше единицы, следовательно, $\sin \alpha < \sfrac{1}{n}$, а значит, $\delta < \sfrac{1}{n}$. Для стекла, коэффициент преломления которого $n \approx 1.5$, имеем, $\delta \in [0, \sfrac{2}{3})$.

\begin{wrapfigure}{r}{0.5\tw}
    \centering
    \vspace{-1pc}
    \tikzsetnextfilename{crown-dispersion-x}
    \begin{tikzpicture}
        \begin{axis}[
            height    =    4.5cm,
            width    =    6cm,
            xlabel    =    {$\lambda$, мкм},
            ylabel    =    {$x(\lambda)$},
%            ylabel shift    = -1 cm,
            xmin = 0.3,
            xmax = 2.5,
            ymin = 1.8,
            ymax = 2.1,
            ]

            \addplot[smooth, domain=0.3:2.5] table[x=l, y=x] {data/crown-dispersion.txt};
        \end{axis}
    \end{tikzpicture}
    \caption{}
    \label{pic:crown-dispersion-x}
\end{wrapfigure}
Как будет показано ниже, во избежании проявления \imp{сферической аберрации}, используют линзы с маленьким относительным отверстием ($\forall \ll 1$). Следовательно, и $\delta \ll 1$, возьмем для примера значение $\delta = 0.1$. Для него, очевидно, можно использовать приближение для $x$, поэтому выражение для $x$ принимает вид:
\begin{equation*}
    x(\lambda)|_{\delta = 0.1} = \frac{1}{n-1} - \frac{0.01 n^2}{2(n-1)}.
\end{equation*}
Как видно из графика данной зависимости (см.~Рис.\,\ref{pic:crown-dispersion-x}), для оптического диапазона $x(\lambda)$ принимает значения от примерно 1.85 для коротковолновой (фиолетовой) части до примерно 1.95 для красного цвета.

Чтобы компенсировать такой разбег совместно с собирающей линзой используют рассеивающую~--- из другого материала. Объективы, где исправлена хроматическая аберрация для двух цветов и частично исправлена сферическая аберрация называют \term{ахроматами}; где хроматическая аберрация исправлена для трёх цветов, а также полностью исправлена сферическая аберрация~--- \term{апохроматами}; с более полной геометрической коррекцией~--- \term{апланатами}.

\paragraph{Сферическая аберрация}
В оптических системах, содержащих сферические поверхности (линзы, зеркала) может наблюдаться \imp{сферическая аберрация}. Суть такой аберрации состоит в том, что лучи, параллельные оптической оси, идущие на разном расстоянии от неё, собираются в разных её точках. Это приводит к размытию изображения точечных источников.

\begin{wrapfigure}[12]{r}{0.55\tw}
    \centering
    \vspace{-.5pc}
    \tikzsetnextfilename{sphere-aberrations-lens-plot}
    \begin{tikzpicture}
        \begin{axis}[
            height    =    5cm,
            width    =    6.5cm,
            xlabel    =    {$\delta$},
            ylabel    =    {$x(\delta)/R$},
%            ylabel shift    = -1.1 cm,
            extra x ticks ={0.667},
            extra x tick labels={$\frac{1}{n}$},
            xmin=-.05,
            xmax=0.72,
            ymin=-.25,
            ymax=2.25,
            legend cell align=left,
            legend style={
                 draw=none,
                 fill=none,
                 font=\scriptsize,
                 at={(axis cs:0, 0.1)},
                 anchor=south west,
                 row sep=.5pc,
            },
        ]
            \addplot[smooth, gray] table[x=d, y=simple] {data/shere-aberrations-lens.txt};
            \addplot[smooth] table[x=d, y=x] {data/shere-aberrations-lens.txt};
            \addplot[dashes] coordinates { (0.667, -10) (0.667, 10)};
            \legend{
                 $\left. \dfrac{x(\delta)}{R} \right|_{\delta \ll 1}$,
                 $\dfrac{x(\delta)}{R}$,
                 $\delta = \left.\dfrac{1}{n}\right|_{n=3/2}$,
            }
        \end{axis}
    \end{tikzpicture}
    \caption{}
    \label{pic:sphere-aberrations-lens-plot}
\end{wrapfigure}
Покажем наличие сферической аберрации для плосковыпуклой линзы. Для этого рассмотрим полученное выше выражение \eqref{eq:optics-aberr-x(d)} и его приближение при $\delta \ll 1$. Зафиксируем в них $n=3/2$~--- характерное значение показателя преломления для стекла. Графики получаемых при этом зависимостей представлены на Рис.\,\ref{pic:sphere-aberrations-lens-plot}. Как видно из данных графиков, сферические аберрации проявляются уже на малых расстояниях от оптической оси.

Чтобы показать важность сферических аберраций, рассмотрим небольшой телескоп-рефрактор с диаметром плос\-ко\-вы\-пук\-ло\-во\-го стеклянного ($n \approx 3/2$) объектива $D = 50$~мм. Характерная точность фокусировки~$\Delta x$ для таких маленьких телескопов составляет около 1~мм. Установим, при каком фокусном расстоянии такого телескопа точность фокусировки нивелирует сферическую аберрацию.

При отсутствии сферической аберрации фокусное расстояние плосковыпуклового объектива $F = R/(n-1) = 2R$. Максимальное значение~$\delta$, для объектива, достигается для лучей, проходящих через край объектива. Следовательно, $\delta \leqslant D/(2R)$. Теперь, используя приближение, можно записать выражение для требуемого $\Delta x$, чтобы найти необходимое для этого относительное отверситие $\forall$:
\begin{gather*}
    \Delta x = F - x(\delta) = F - x\left( \frac{D}{2R} \right),\\
    \Delta x = F - R\left( 2 - \frac{D^2}{4R^2} \cdot \frac{9}{4} \right),\\
    \Delta x = \frac{9D^2}{16R} = \frac{9D^2}{8F} = \frac{9}{8} D \forall;\\
    \therefore \forall = \frac{\Delta x \cdot 8}{9D} = 0.019 \approx \frac{1}{50}.
\end{gather*}

\begin{wrapfigure}[9]{r}{0.5\tw}
    \centering
    \vspace{-.8pc}
    \tikzsetnextfilename{spherical-aberration-mirrow-scheme}
    \begin{tikzpicture}
        \footnotesize

        \draw [decoration={snake, segment length=.7mm, amplitude=0.2mm}, decorate] (1.35, 1) arc(-31:15:0.26);
        \draw (.2, 1.135) arc(180:209:0.94);
        \draw (-2.18, 0) arc(0:15:0.94);

        \fill [lightgray] (1.5, -1.5) -- (1.5, 1.5) -- (1, 1.5) arc(20:-20:4.386) -- (1.5, -1.5);

        \draw [thick] (1, -1.5) arc(-20:20:4.386);
        \draw [semithick, dash pattern={on 5pt off 2pt on .5pt off 2pt}] (-3.8, 0) -- (1.7, 0);

        \draw [dashes] (-3.12, 0) -- (1.71, 1.29);

        \draw [semithick] (-3.8, 1.135) -- (1.12, 1.135) -- (-.85, 0);
        \draw [-latex] (-3.5, 1.135)-- (-1, 1.135);
        \draw [-latex] (1.12, 1.135) -- (-.06, 0.45);

        \draw [latex-latex] (-3.5, 0) -- (-3.5, 1.135);
        \draw [latex-latex] (-.85, -1.3) -- (1.27, -1.3);

        \draw (1.27, 0) -- (1.27, -1.5);
        \draw (-.85, 0) -- (-.85, -1.5);

        \draw (-3.5, 0.56) node[anchor=east] {$d$};
        \draw (-0.85, 0) node[anchor=north east] {$F$};
        \draw (-3.12, 0) node[anchor=south] {$C$};
        \draw (-1.2, 0.5) node[anchor=south] {$R$};
        \draw (0.27, -1.3) node[anchor=south] {$x_F(d)$};

        \draw (.2, 1) node[anchor=east] {$\alpha$};
        \draw (-2.2, .15) node[anchor=west] {$\alpha$};
        \draw (.3, .7) node[anchor=east] {$\alpha$};

        \draw [fill=white] (-0.85, 0) circle (.03);
        \draw [fill=white] (-3.12, 0) circle (.03);
    \end{tikzpicture}
    \caption{}
    \label{pic:sphere-aberrations-mirrow}
\end{wrapfigure}
Найдем теперь величину аберрации сферического зеркала.\linebreak Пусть $R$~--- радиус кривизны зеркала. Рассмотрим луч, идущий параллельно оптической оси зеркала на расстоянии $d$ от неё. Он падает на зеркало под углом $\alpha$, где $\sin \alpha = d/R$. В силу закона отражения: угол падения равен углу отражения, то есть угол отражения также равен $\alpha$. Кроме того, угол между нормалью к зеркалу в точке отражения и оптической осью зеркала тоже равен $\alpha$, как вертикальный. Следовательно, треугольник {\slshape центр кривизны зеркала ($C$) -- точка отражения ($A$) -- точка пересечения отраженного луча с оптической осью (F)} является равнобедренным. Значит расстояние $x_F(d)$ от центра зеркала до <<фокуса>> $F$ можно найти как
\begin{gather*}
    x_F(d) = R - \frac{R}{2} \cdot \frac{1}{\cos\alpha} = R - \frac{R}{2\sqrt{1 - \sin^2 \alpha}}  = R  - \frac{R}{2\sqrt{1 - \dfrac{d^2}{R^2}}};\\
    \left. \frac{x_F(d)}{R} \right|_{d \ll R} \simeq  1  - \frac{1}{2\left(1 - \dfrac{d^2}{2R^2} \right)} \simeq  1 - \frac{1}{2}\left(1 + \dfrac{d^2}{2R^2} \right)  = \frac{1}{2} -  \dfrac{d^2}{4R^2}.
\end{gather*}
\begin{wrapfigure}[12]{r}{0.55\tw}
    \centering
    \vspace{-.5pc}
    \tikzsetnextfilename{spherical-aberration-mirrow-result}
    \begin{tikzpicture}
        \begin{axis}[
            height  =   5cm,
            width   =   6.5cm,
            xlabel  =   {$d/R$},
            ylabel  =   {$x(d)/R$},
            extra x ticks = {sqrt(2)/2},
            extra x tick labels = {$\frac{\sqrt{2}}{2}$},
            xmin    =   -.05,
            xmax    =   0.85,
            ymin    =   .15,
            ymax    =   0.55,
            legend cell align = left,
            legend style = {
                draw    =   none,
                fill    =   none,
                font    =   \scriptsize,
                at      =   {(axis cs:0, .2)}, anchor=south west,
                row sep =   .5pc,
            },
        ]
            \addplot[smooth, gray] table[x=d, y=simple] {data/sphere-aberrations-mirrow.txt};
            \addplot[smooth] table[x=d, y=x] {data/sphere-aberrations-mirrow.txt};
            \addplot[dashes] coordinates { (sqrt(2)/2, -10) (sqrt(2)/2, 10)};
            \legend{
                $\left. \dfrac{x_F(d)}{R} \right|_{d \ll R}$,
                $\dfrac{x_F(d)}{R}$,
                $ \dfrac{d}{R} = \dfrac{\sqrt{2}}{2}$,
            }
        \end{axis}
    \end{tikzpicture}
    \caption{}
\end{wrapfigure}
Отсюда получается, что фокус сферического зеркала находится ровно между центром зеркала и центром его кривизны. Однако в силу сферической аберрации возникает ошибка фокусировки порядка $d^2/R^2$, которая размывает изображение. Причём при $d > R/\sqrt{2}$ лучи не <<разворачиваются>>, следовательно, не вносят вклада в изображение, так как приходят на приемник с другой стороны.

Для компенсации сферической аберрации используют различные линзы-коррек\-торы, однако они помогают лишь частично избавиться от неё. Поэтому в современных рефлекторах используются параболические зеркала, не подверженные сферическим аберрациям.

%Напоследок нужно отметить, что сферические аберрации

\paragraph{Астигматизм} Ещё один вид аберраций оптических систем, состоящий в разности радиусов кривизны оптических элементов в двух перпендикулярных направлениях. Такое возможно, например, в случае большой массы линзы или зеркала. Когда такой оптический элемент долгое время находится в вертикальном положении (оптическая ось горизонтальна), он деформируется: вдоль горизонтали радиус кривизны сохраняется, а по вертикали, из-за сжатия,~--- уменьшается.

\begin{wrapfigure}[13]{r}{0.4\tw}
    \vspace{-0.8pc}
    \centering
    \tikzsetnextfilename{astigmatism}
    \begin{tikzpicture}
        \footnotesize

        % def sizes
        \def\R{4}
        \def\h{0.7}
        \def\L{3.5}
        \def\S{0.3}

        % def lens arc angles
        \def\a{-asin(\h/\R)}
        \def\b{asin(\h/\R)}
        \def\c{180-asin(\h/\R)}
        \def\d{180+asin(\h/\R)}

        % lens center
        \tkzDefPoint(0,0){O}

        % lens nodes
        \tkzDefShiftPoint[O](0,-\h){L1}
        \tkzDefShiftPoint[O](0,\h){L2}

        % draw lens
        \fill[lightgray] (L1) arc(\a:\b:\R) arc(\c:\d:\R);
        \draw[semithick, line cap=round] (L1) arc(\a:\b:\R);
        \draw[semithick, line cap=round] (L2) arc(\c:\d:\R);

        % image
        \tkzDefShiftPoint[O](\L, 0){O1}
        \tkzDefShiftPoint[O1](0,\S){I1}
        \tkzDefShiftPoint[O1](0,-\S){I2}

        % light paths
        \tkzDrawLines[add = 0 and 0.1](L1,I1 L2,I2)

        % light paths intersection
        \tkzInterLL(L1,I1)(L2,I2) \tkzGetPoint{I}

        %%%%%%%

        % lens center
        \tkzDefPoint(0,2){O'}

        % lens nodes
        \tkzDefShiftPoint[O'](0,-\h){L1'}
        \tkzDefShiftPoint[O'](0,\h){L2'}

        % draw lens
        \fill[lightgray] (L1') arc(\a:\b:\R) arc(\c:\d:\R);
        \draw[semithick, line cap=round] (L1') arc(\a:\b:\R);
        \draw[semithick, line cap=round] (L2') arc(\c:\d:\R);

        % image
        \tkzDefShiftPoint[O'](\L, 0){O1'}

        % light paths
        \tkzDrawLines[add = 0 and 0.1](L1',O1' L2',O1')

        %%%%%%%%

        % intersect line
        \tkzDefLine[parallel=through I](I1,I2) \tkzGetPoint{I'}

        \tkzInterLL(I,I')(L1',O1') \tkzGetPoint{I1'}
        \tkzInterLL(I,I')(L2',O1') \tkzGetPoint{I2'}

        % draw images
        \tkzDrawSegments[thick](I1,I2 I1',I2')

        % draw parallels

        \tkzDefLine[parallel=through I](I1,I2) \tkzGetPoint{I'}


        \tkzGetPointCoord(I){i}
        \tkzGetPointCoord(L1){l}
        \tkzGetPointCoord(L2'){ll}

        \tkzDefPoint(\ix, \ly){Ii}
        \tkzDefPoint(\ix, \lly){Ii'}

        \tkzGetPointCoord(O1){o}
        \tkzDefPoint(\ox, \ly){Oo}
        \tkzDefPoint(\ox, \lly){Oo'}

        \tkzDrawSegments[dashed](L1',L2 Ii,Ii' Oo,Oo')

        \tkzDrawLines[dash dot, add=0.1 and 0.1](O,O1 O',O1')
	\end{tikzpicture}
    \caption{Схема формирования изображения при астигматизме, вид в двух перпендикулярных плоскостях. Пунктиром отмечено (слева направо): плоскость линзы, плоскости формирования изображения по двум разным осям}
    \label{pic:astigmatism}
\end{wrapfigure}
Астигматизм проявляется так, что параллельный пучок лучей после прохождения через оптическую систему собирается не в одной точке, а на двух взаимно перпендикулярных отрезках, расположенных на некотором удалении друг от друга. Промежуточные сечения в этом случае имеют форму эллипсов.

\paragraph{Кома}

\begin{wrapfigure}[13]{l}{0.4\tw}
    \centering
    \vspace{-1pc}
    \includegraphics[width=0.4\tw]{optical-aberrations-coma.pdf}
    \caption{Изображение <<хвоста>> Большой Медведицы, полученное с помощью широкоугольного объектива, страдающего ярко выраженной комой.}
    \label{pic:optics-aberrations-coma}
\end{wrapfigure}
Один из видов аберраций оптических систем~--- аберрация широкого пучка световых лучей, проходящий наклонно к оптической оси системы, как и \imp{сферическая аберрация}, обусловлена неодинаковым преломлением световых лучей различными участками линзовых компонент системы. Кома приводит к нарушению центрированности светового пучка. В результате такой аберрации изображение точки имеет вид несимметричного пятна (см.~Рис.\,\ref{pic:optics-aberrations-coma}), по форме напоминающего запятую (англ. {\itshape comma}).

\begin{figure}[h]
    \centering
    \tikzsetnextfilename{thick-lens}
    \begin{tikzpicture}
        \tkzInit[
            xmax=5,
            xmin=-2.7,
            ymin=-2.5,
            ymax=2.5
        ]
        \tkzClip
            
        \def\width{1.5}
        \def\alph{23.5}
        \def\n{2}
        \def\bet{asin(sin(\alph) / \n)}
        \def\h{2}
        \def\r{4.5}
        \def\d{2}

        \def\earthR{0.25}
        \def\moonR{0.7}
        \def\sunR{1}
        
        \tkzDefPoint(0,0){O}
        \tkzDefPoint(-\d,0){C}

        \tkzDefShiftPoint[O](0,\width){A}
        \tkzDefShiftPoint[O](0,-\width){B}
        \tkzDefShiftPoint[C](\r,0){R}

        \tkzDefPointBy[rotation = center A angle -90 + \alph](O)
        \tkzGetPoint{A'}
        \tkzDefPointBy[rotation = center A angle 90 + \bet](O)
        \tkzGetPoint{a'}
        \tkzInterLC(A,a')(C,R) \tkzGetFirstPoint{A''}
        \tkzGetVectxy(C,A''){Va}
        \def\a{atan(\Vay/\Vax)}
        \def\aa{asin(\n * sin(\a - \bet))}
        \tkzDefPointBy[rotation = center A'' angle \bet - \a + 180 -\aa](A)
        \tkzGetPoint{A'''}

        \tkzDefPointBy[rotation = center B angle 90 + \alph](O) 
        \tkzGetPoint{B'}
        \tkzDefPointBy[rotation = center B angle -90 + \bet](O) 
        \tkzGetPoint{b}
        \tkzInterLC(B,b)(C,R) \tkzGetSecondPoint{B''}
        \tkzGetVectxy(C,B''){Vb}
        \def\b{atan(-\Vby/\Vbx)}
        \def\bb{asin(\n * sin(\b + \bet))}
        \tkzDefPointBy[rotation = center B'' angle -\bet - \b + 180 + \bb](B)
        \tkzGetPoint{B'''}

        \tkzInterLL(A'',A''')(B'',B''') \tkzGetPoint{I}
        \tkzDefPointBy[homothety=center A'' ratio 1.7](I) \tkzGetPoint{I'}
        \tkzDefPointBy[homothety=center C ratio 3](O) \tkzGetPoint{C'}
        \tkzInterLL(C,C')(B'',B''') \tkzGetPoint{W}
        \tkzDefPointBy[homothety=center C ratio 1.5](A) \tkzGetPoint{AC}
        \tkzDefPointBy[homothety=center C ratio 1.3](B) \tkzGetPoint{BC}
        \tkzDefPointBy[homothety=center C ratio 1.3](A'') \tkzGetPoint{AC''}
        \tkzDefPointBy[homothety=center C ratio 1.3](B'') \tkzGetPoint{BC''}        
    
        \def\margin{2}
        \tkzDefShiftPoint[A](0, \margin){A2}
        \tkzDefShiftPoint[B](0, -\margin){B2}
        
        \tkzDrawSector[R, fill=gray!50, draw=none](C,\r)(-90,90)
        \tkzDrawArc[R, thick](C,\r)(-90,90)
        \tkzDrawSector[R, fill=white, draw=none](O,\r)(90,-90)
        \tkzDrawLine[dash dot](C,C')
        \tkzDrawSegments[thick](A2,B2 A',A B',B B,B'' A,A'' A'',I B'',I)
        
        \def\normalLength{1.9}
        \tkzDefShiftPoint[A](-1.5*\normalLength , 0){A1}
        \tkzDefShiftPoint[A](\normalLength, 0){A2}
        \tkzDefShiftPoint[B](-\normalLength, 0){B1}
        \tkzDefShiftPoint[B](\normalLength, 0){B2}
        
        
        \tkzDrawSegments(C,AC C,BC)
        \tkzDrawSegments[dashed](C,BC'' C,AC'' A1,A2 B1,B2)
        \tkzDrawPoints(A, B, C, O)
        
        \tkzSetUpArc[line width=.2pt]
        \tkzMarkAngle[line width = .3pt, size=0](C,O,A) % костыль для расстояния между дугами
        
        \tkzMarkAngle[size=0.8, arc=ll, mark=o, mksize=2pt](A1,A,A')
        \tkzLabelAngle[font=\scriptsize, pos=1.05](A1,A,A'){$\alpha$}
        
        \tkzMarkAngle[size=0.8, arc=ll, mark=o, mksize=2pt](B1,B,B')
        \tkzLabelAngle[font=\scriptsize, pos=1.05](B1,B,B'){$\alpha$}
        
        \tkzMarkAngle[size=0.6, arc=ll, mark=x, mksize=2.5pt](C,B,B1)
        \tkzLabelAngle[font=\scriptsize, pos=0.8](C,B,B1){$\xi$}
        
        \tkzMarkAngle[size=1.7](A2,A,A'')
        \tkzLabelAngle[font=\scriptsize, pos=1.9](A2,A,A''){$\beta$}
        
        \tkzMarkAngle[size=1.5](B2,B,B'')
        \tkzLabelAngle[font=\scriptsize, pos=1.7](B2,B,B''){$\beta$}
        
        \tkzMarkAngle[size=0.6, arc=lll, mark=o, mksize=2pt](I,A'',AC'')
        \tkzLabelAngle[font=\scriptsize, pos=0.9](I,A'',AC''){$\delta_1$}
        
        \tkzMarkAngle[size=0.7, arc=lll, mark=s,  mksize=2.5pt](BC'',B'',I)
        \tkzLabelAngle[font=\scriptsize, pos=1](BC'',B'',I){$\delta_2$}
        
        \tkzMarkAngle[size=0.5, arc=lll, mark=||, mksize=2pt](C,W,B'')
        \tkzLabelAngle[font=\scriptsize, pos=0.8](C,W,B''){$\omega_2$}
        
        \tkzMarkAngle[size=0.7, arc=lll](A'',A,AC)
        \tkzLabelAngle[font=\scriptsize, pos=1.2](A'',A,AC){\adjustbox{raise=-3ex}{$\xi - \beta$}}
        
        \tkzMarkAngle[size=0.6, arc=ll, mark=x, mksize=2.5pt](C,B'',B)
        \tkzLabelAngle[font=\scriptsize, pos=0.9](C,B'',B){$\gamma_2$}
        
        \tkzMarkAngle[size=1.6, arc=ll](A,A'',C)
        \tkzLabelAngle[font=\scriptsize, pos=1.8](A,A'',C){$\gamma_1$}
        
        
        \tkzMarkAngle[size=0.7, arc=lll, mark=x, mksize=3pt](O,C,A'')
        \tkzLabelAngle[font=\scriptsize](O,C,A''){$\eta_1$}
        
        \tkzMarkAngle[size=0.8, arc=lll, mark=|, mksize=3pt](B'',C,O)
        \tkzLabelAngle[font=\scriptsize, pos=1.2](B'',C,O){$\eta_2$}
        
        \tkzDrawPoints(B'', A'', I)
        
        \tkzLabelPoint[above](B){\adjustbox{right=8mm}{$A_2$}}
        \tkzLabelPoint[below](B''){\adjustbox{right=7mm}{$B_2$}}
        \tkzLabelPoint[above](A){\adjustbox{left=8mm}{$A_1$}}
        \tkzLabelPoint[above](A''){\adjustbox{right=6.5mm}{$B_1$}}
        \tkzLabelPoint[below](C){$C$}
        \tkzLabelPoint[above right](O){$O$}
        \tkzLabelPoint[above](I){\adjustbox{right=10mm}{$(x,y)$}}
        
        \tkzLabelSegments[above](O,R){$h$}
        
        \tkzDefShiftPoint[C](-0.5, 0){C'}
        \tkzDefShiftPoint[A](-\d -0.5,0){A'}
        \tkzDrawSegment[latex-latex](C',A')
        \tkzLabelSegment[right](C',A'){$d$}
        
    \end{tikzpicture}
    \caption{}
    \label{pic:optical-aberrations-coma1}
\end{figure}

Найдем положение фокуса светового пучка, идущего под углом $\alpha$ к оптической оси. Для этого рассмотрим два луча из него, вместе с оптической осью лежащих в одной плоскости, таких, что точки преломления их передней (плоской) поверхностью линзы $A_1$ и $A_2$ лежат на расстоянии $d$ от оптической оси линзы (см.~Рис.\,\ref{pic:optical-aberrations-coma1}).

Пусть толщина линзы~--- расстояние вдоль оптической оси от вершины выпуклой поверхности, до центра плоской поверхности $O$, равна $h$. А радиус кривизны выпуклой поверхности равен $R$. Тогда расстояние $l$ от центра кривизны задней (выпуклой) поверхности линзы до точек $A_1$ и $A_2$ можно найти из теоремы Пифагора для треугольника $\triangle COA_{1,2}$:
\begin{equation*}
    l = \sqrt{(R - h)^2 + d^2}.
\end{equation*}
Равные углы $\angle A_{1,2} C O$ обозначим за $\xi$, где
\begin{equation*}
    \xi = \arcsin \frac{d}{l}.
\end{equation*}

Угол падения рассматриваемых лучей на плоскую поверхность линзы равен $\alpha$, следовательно, по закону Снеллиуса угол преломления $\beta$ определяется выражением
\begin{equation*}
    \beta = \arcsin \frac{\sin \alpha}{n},
\end{equation*}
где $n$~--- показатель преломления линзы. Обозначим углы падения верхнего и нижнего луча на заднюю поверхность линзы $\gamma_{1,2}$ соответственно. А точки преломления лучей задней поверхностью соответственно $B_{1,2}$. Запишем теорему синусов для треугольников $\triangle C A_{1,2} B_{1,2}$, чтобы найти углы $\gamma_{1,2}$ соответственно:
\begin{gather*}
    \frac{R}{\sin (180^\circ - (\xi \mp \beta))} = \frac{l}{\sin \gamma_{1,2}},\\
    \therefore \gamma_1 = \arcsin \frac{l\sin (\xi \mp \beta)}{R}.
\end{gather*}

Чтобы найти координаты точек $B_{1,2}$, определим сначала углы $ \eta_{1,2} \equiv \angle B_{1,2} C O$:
\begin{equation*}
    \eta_{1,2} = \xi - \Big[180^\circ - \gamma_{1,2} - \big(180^\circ - \left(\xi \mp \beta\right)\big)\Big] = \gamma_{1,2} \pm \beta.
\end{equation*}

Введем теперь декартову систему координат, за начало отсчета примем центр кривизны выпуклой поверхности линзы, ось $x$ направим вдоль оптической оси вправо, ось $y$ перпендикулярно вверх. В такой системе координаты точки $B_{1,2}$ задаются векторами
\begin{equation*}
    \begin{pmatrix}
        x_{1,2}\\
        y_{1,2}
    \end{pmatrix}
    = R
    \begin{pmatrix}
        \cos \eta_{1,2}\\
        \pm \sin \eta_{1,2}
    \end{pmatrix}.
\end{equation*}
Остается воспользоваться законом Снеллиуса для определения углов $\delta_{1,2}$:
\begin{equation*}
    \delta_{1,2} = \arcsin \left( n \sin \gamma_{1,2} \right).
\end{equation*}

Поиск координат  $(x,y)$ точки пересечения лучей, вышедших из линзы начнём с поиска абсциссы. Можно заметить, что она должно удовлетворять уравнению
\begin{gather*}
    (x - x_1) \tg \omega_1 + (x - x_2) \tg \omega_2 = y_1 - y_2,\\
    \therefore x = \frac{y_1 - y_2 + x_1\tg \omega_1 + x_2 \tg \omega_2}{\tg \omega_1 + \tg \omega_2};
\end{gather*}
где $\omega_1 = \delta_1 - \eta_1$, а $\omega_2 = \delta_2 - \eta_2$. Теперь не сложно найти координату $y$ <<фокуса>>:
\begin{equation*}
    y = y_1 - (x - x_1) \tg \omega_1 = y_2 + (x - x_2) \tg \omega_2.
\end{equation*}

Продемонстрируем полученные зависимости (см.~Рис.\,\ref{pic:coma}), при следующих параметрах: $R = 1$, $h = 0.3$, $\alpha = 50^\circ$, $n=1.5$.

\begin{figure}[h]
    \begin{subcaptionblock}{0.49\tw}
    \tikzsetnextfilename{thick-lens-coma-1}
        \begin{tikzpicture}
            \begin{axis} [
                height  =   5cm,
                width   =   6cm,
                xlabel  =   {$d$},
                ylabel  =   {},
                ylabel shift = 2mm,
                xmin    =   0,
                xmax    =   0.4,
                ymin    =   .6,
                ymax    =   2,
                legend cell align = left,
                legend style = {
                    draw    =   none,
                    fill    =   none,
                    font    =   \scriptsize,
                    at      =   {(axis cs:.4, 2)}, anchor=north east,
                },
            ]
                \addplot[smooth, gray] table[x=d, y=x] {data/coma.txt};
                \addplot[smooth] table[x=d, y=y] {data/coma.txt};
                \addplot[dashes] coordinates { (0.359606, -10) (0.359606, 1.6)};
                \legend{
                    $x(d)$,
                    $y(d)$,
                }
            \end{axis}
        \end{tikzpicture}
%        \caption{}
    \end{subcaptionblock}
    \hfill
    \begin{subcaptionblock}{0.49\tw}
        \tikzsetnextfilename{thick-lens-coma-2}
        \begin{tikzpicture}
            \begin{axis} [
                height  =   5cm,
                width   =   6cm,
                xlabel  =   {$x$},
                ylabel  =   {$y$},
                xmin    =   1.2,
                xmax    =   2,
                ymin    =   .6,
                ymax    =   1.1,
            ]
                \addplot[only marks, mark = o, mark options={scale=0.2, black}] table[x=x, y=y] {data/coma.txt};
            \end{axis}
        \end{tikzpicture}
%        \caption{}
    \end{subcaptionblock}
    \caption{}
    \label{pic:coma}
\end{figure}

Понятно, что смещение положение фокуса вдоль оси $x$ обусловлено \imp{сферической аберрацией}, однако смещение вдоль ось $y$ есть проявление \imp{аберрации кома}.

\paragraph{Дисторсия и кривизна фокальной поверхности}
Для тонкой линзы выполняется правило, что лучи, проходящие через оптический центр не преломляются. А световые пучки, идущие под различными углами фокусируются на фокальной плоскости, перпендикулярной оптической оси. Отсюда следует, что параллельные пучки, идущие под углом $\alpha$ к оптической оси и проходящие через оптический центр тонкой линзы, фокусируются на расстоянии $F \tg \alpha$ от оптической оси. 

\begin{figure}[h]
    \begin{subcaptionblock}{0.49\tw}
        \tikzsetnextfilename{distorsion-xy}
        \begin{tikzpicture}
            \begin{axis}
                [
                height    =    5cm,
                width    =    6cm,
                xlabel    =    {$x$},
                ylabel    =    {$y$},
                xmin=1.25,
                xmax=3.25,
                ymin=-.2,
                ymax=1.2,
                ]
                \addplot[only marks, mark = o, mark options={scale=0.2, black}] table[x=x, y=y] {data/distorsion-xy.txt};
            \end{axis}
        \end{tikzpicture}
        \caption{}
        \label{pic:pitzval}
    \end{subcaptionblock}
    \hfill
    \begin{subcaptionblock}{0.49\tw}
        \tikzsetnextfilename{distorsion-y}
        \begin{tikzpicture}
            \begin{axis} [
                height    =    5cm,
                width    =    6cm,
                xlabel    =    {$\alpha$},
                ylabel    =    {$y(\alpha)$},
                xmin    =    0,
                xmax    =    90,
                ymin    =    0,
                ymax    =    10,
                legend cell align = left,
                legend style = {
                    draw    =    none,
                    fill    =    none,
                    font    =    \scriptsize,
                    at        =    {(axis cs:5, 9)}, anchor=north west,
                },
            ]
                \addplot[smooth, gray] table[x=alpha, y=tan] {data/distorsion-y.txt};
                \addplot[smooth] table[x=alpha, y=y] {data/distorsion-y.txt};
                \legend{
                    $2R\tg \alpha$,
                    $y(\alpha)$
                }
            \end{axis}
        \end{tikzpicture}
        \caption{}
        \label{pic:distorsion-y}
    \end{subcaptionblock}
    \caption{}
\end{figure}

Легко догадаться, что для толстой линзы это не выполняется. На рисунке Рис.\,\ref{pic:pitzval} показаны координаты точек фокусировки бесконечно узких пучков, падающих под углом $\alpha \in [0^\circ, 90^\circ]$ к оптической оси на центр передней поверхности рассмотренной выше плосковыпуклой линзы: ($R = 1$, $h = 0.3$, $n=1.5$, шаг составляет $1^\circ$). Отсюда ясно, что фокальная <<плоскость>> является таковой только в очень узком диапазоне направлений вокруг оптической оси. В остальной области фокальная поверхность является выпуклой в сторону от линзы.

А на рисунке Рис.\,\ref{pic:distorsion-y} показана зависимость $y$-координаты точки фокуса от угла падения на линзу в сравнение с функцией $F \tg \alpha = 2R \tg \alpha$. Нелинейность данной зависимости от $R \tg \alpha$ проявляется в неравномерности увеличения изображения по полю зрения: чем дальше от центра, тем меньше увеличение. Такой эффект называется \term{дисторсией}, в данном случае~--- \imp{бочкообразной}, от схожести с видом деревянных бочек. Также возможна ситуация, когда $y(\alpha)$ растет быстрее $F \tg \alpha$, тогда дисторсия имеет противоположный знак и называется \imp{подушкообразной} (достаточно вспомнить, как выглядит плотно набитая подушка).

\begin{figure}[h]
    \centering
    \begin{subcaptionblock}{0.31\tw}
        \centering
        \tikzsetnextfilename{distorsion-flat}
        \begin{tikzpicture}
            \begin{axis}[
                height  =    4cm,
                width      =    4cm,
                xmin    =    -1.25,
                xmax    =    1.25,
                ymin    =    -1.25,
                ymax    =    1.25,
                hide axis
            ]
                \addplot[only marks, mark = o, mark options={scale=0.2, black}] table[x=x, y=y] {data/distorsion-types.txt};
            \end{axis}
        \end{tikzpicture}
        \caption{Без дисторсии}
        \label{pic:distorsion-flat}
    \end{subcaptionblock}
    \hfill
    \begin{subcaptionblock}{0.31\tw}
        \centering
        \tikzsetnextfilename{distorsion-barrel}
        \begin{tikzpicture}
            \begin{axis}[
                height  =    4cm,
                width      =    4cm,
                xmin    =    -1.25,
                xmax    =    1.25,
                ymin    =    -1.25,
                ymax    =    1.25,
                hide axis
            ]
                \addplot[only marks, mark = o, mark options={scale=0.2, black}] table[x=x_barrel, y=y_barrel] {data/distorsion-types.txt};
            \end{axis}
        \end{tikzpicture}
        \caption{Дисторсия <<бочка>>}
        \label{pic:distorsion-barrel}
    \end{subcaptionblock}
    \hfill
    \begin{subcaptionblock}{0.31\tw}
        \centering
        \tikzsetnextfilename{distorsion-pillow}
        \begin{tikzpicture}
            \begin{axis}[
                height    =    4cm,
                width    =    4cm,
                xmin    =    -1.25,
                xmax    =    1.25,
                ymin    =    -1.25,
                ymax    =    1.25,
                hide axis
            ]
                \addplot[only marks, mark = o, mark options={scale=0.2, black}] table[x=x_pillow, y=y_pillow] {data/distorsion-types.txt};
            \end{axis}
        \end{tikzpicture}
        \caption{Дисторсия <<подушка>>}
        \label{pic:distorsion-pillow}
    \end{subcaptionblock}
    \caption{Виды дисторсии}
\end{figure}


