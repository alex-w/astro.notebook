\subsection{Импульс}
По определению импульс с силой связывает уравнение:
\begin{equation*}
	F = \frac{dp}{dt}.
\end{equation*}
Рассмотрим малое изменение импульса в лабораторной СО, напомним что $F = ma'$:
\begin{equation*}
	dp = F dt = m a' dt = m a \gamma^{3} dt = m \gamma^{3} dv.
\end{equation*}
Тогда полный импульс тела $p$ может быть получен как интеграл от 0 до $v$:
\begin{equation*}
	p = \int\limits_0^{p} dp = \int\limits_0^{v} m \gamma^{3} \,dv = \int\limits_0^{v} \frac{m \, dv}{\left(1 - v^2 / c^2\right)^{3/2}}.
\end{equation*}
Для того чтобы найти данный интеграл, применим тригонометрическую подстановку:
\begin{equation}
	\frac{v}{c} = \sin x, \quad dv = c \cdot \cos x \, dx, \quad \sqrt{1-\frac{v^2}{c^2}} = \cos x.
	\label{eq:sr-imp-trigsub}
\end{equation}
Тогда, подставляя в интеграл получим:
\begin{equation*}
	\int\limits \frac{mc \cdot \cos x \, dx}{\cos^{3} x} = mc \tg x = \frac{mv}{\sqrt{1-v^2/c^2}}.
\end{equation*}
Заметим что для начала мы взяли неопределенный интеграл, а далее по формуле Ньютона-Лейбница:
\begin{equation*}
	p=\left.\frac{m v}{\sqrt{1-v^2 / c^2}}\right|_0 ^v = \frac{mv}{\sqrt{1-v^2/c^2}} = \gamma m v.
\end{equation*}