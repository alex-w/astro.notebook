\subsection{Быстрота}
Рассмотрим подстановку:
\begin{equation}
	\frac{u}{c} = \th x, \quad \frac{v}{c} = \th y.
\end{equation}
И распишем выражение $c \cdot \th (x+y)$:
\begin{equation*}
	c \cdot \th (x+y) 
	   = \frac{u+v}{1+\frac{u v}{c^2}} 
	   = v'.
\end{equation*}
Мы получили закон сложения скоростей. Тогда преобразуем выражение далее как:
\begin{equation*}
    \arcth \frac{v'}{c} 
        = x + y 
        = \arcth \frac{u}{c} +\arcth \frac{v}{c}.
\end{equation*}
Мы получили величину, которую можно просто складывать вместо использования громоздкой формулы для релятивистского сложения скоростей. В случае, когда нужно несколько раз переводить скорости из одной СО в другую, это бывает полезно. Такая величина называется быстротой, и обычно обозначается как
\begin{equation*}
	\vartheta \equiv \arcth \frac{v}{c}.
\end{equation*}
Выразим лоренц-фактор через быстроту:
\begin{multline}
	\gamma
	   = \frac{1}{\sqrt{1 - v^2 / c^2}}
	   = \frac{1}{\sqrt{1 - \th^2 \vartheta}}
	   = \frac{1}{\sqrt{1-\frac{\sh^2 \vartheta}{\ch^2 \vartheta}}} = \\
	   = \frac{1}{\sqrt{\frac{\ch^2 \vartheta - \sh^2 \vartheta}{\ch^2 \vartheta}}}
	   = \frac{1}{\sqrt{\frac{1}{\ch^2 \vartheta}}}
	   =\ch \vartheta.
\end{multline}
Выразим импульс тела через его массу и быстроту:
\begin{equation}
    p 
        = \gamma m v
        = \ch \vartheta \cdot m v 
        = \frac{v}{c} \cdot \ch \vartheta \cdot m c
        = \th \vartheta \cdot \ch \vartheta \cdot m c
        = m c \sh \vartheta.
\end{equation}
Запишем выражения для преобразований Лоренца через быстроту
\begin{gather*}
    x' 
        = \frac{x - v t}{\sqrt{1 - v^2 / c^2}}
        = (x - v t) \ch \vartheta
        = x \ch \vartheta - c t \sh \vartheta; \\
    t' 
        = \frac{t - v x / c^2}{\sqrt{1 - v^2 / c^2}}
        = \left( t - v x / c^2 \right) \ch \vartheta
        = \frac{1}{c}(c t \ch \vartheta - x \sh \vartheta).
\end{gather*}
Для удобства запишем эти выражения через переменную $\tau$:
\begin{align*}
    x' & = x \ch \vartheta - \tau \sh \vartheta; \\
    \tau' & = \tau \ch \vartheta - x \sh \vartheta.
\end{align*}
Заметим что последнюю систему можно выразить через произведение матриц:
\begin{equation}
	\begin{pmatrix}
		x' \\
		\tau'
	\end{pmatrix} = 
	\begin{pmatrix}
		\ch \vartheta & -\sh \vartheta \\
		-\sh \vartheta & \ch \vartheta \\ 
	\end{pmatrix}
	\begin{pmatrix}
		x \\
		\tau
	\end{pmatrix}.
\end{equation}