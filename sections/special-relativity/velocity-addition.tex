\subsection{Закон сложения скоростей}
Еще один эффект СТО связан с изменением скоростей при переходе из одной СО в другую. Скорости нельзя просто складывать, так как скорость~--- это производная координаты по времени, а время в каждой системе отсчета свое. Однако можно найти закон, по которому скорости будут преобразовываться при переходе между системами отсчета. Пусть система отсчета $F$ движется относительно СО $F'$ со скоростью $v$ вдоль оси $Ox$, скорость некоторого тела в СО $F$ равна $u$, будем считать, что она направлена вдоль этой же оси $Ox$, найдем скорость этого же тела в СО $F'$.

Запишем преобразования Лоренца, приняв $x = ut$:
\begin{gather*}
    x' 
        = \frac{x + v t}{\sqrt{1 - v^2 / c^2}} 
        = \frac{t (v + u)}{\sqrt{1 - v^2 / c^2}}, \\
    t' 
        = \frac{t + v x / c^2}{\sqrt{1 - v^2 / c^2}} 
        = \frac{t \left( 1 + v u / c^2 \right)}{\sqrt{1 - v^2 / c^2}},
\end{gather*}
$v'$ будет по определению равно производной $x'$ по времени $t'$:
\begin{equation}
	v' 
	   = \frac{d x'}{d t'} 
	   = \frac{u + v}{1 + u v / c^2}.
\end{equation}
Это и есть релятивистский закон сложения скоростей. Можно заметить, как просто он сводится к классическому если принять $u v \ll c^2$.