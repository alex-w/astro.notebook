\subsection{Интервал}
Проведём мысленный эксперимент. Есть два наблюдателя в космическом пространстве, вдали от всех тел и не взаимодействующие с ними. Назовем их Алиса и Боб. Боб движется на своем космическом корабле относительно корабля Алисы с постоянной скоростью $v$; в тот момент, когда Боб пролетает мимо Алисы, они синхронизируют свои часы, то есть, настраивают их так, чтобы они шли с одинаковой скоростью, и запускают в этот момент отсчет времени. Алиса посылает Бобу световой сигнал через время $t$ после начала отсчета времени, который дойдёт до Боба за время $\Delta t$ по часам Алисы, при этом для расстояния, пройденного светом, можно записать соотношение: 
 \begin{equation}
 	v(t + \Delta t) = c \Delta t \,\rightarrow\, \Delta t = \frac{vt}{c-v}.
 \label{eq:sr-1}
 \end{equation}
 Приемник на корабле Боба регистрирует поступивший сигнал и сразу же испускает ответный. В системе отсчета Алисы сигнал будет идти одинаковое время в обе стороны, так как расстояние, пройденное светом одинаковое, в момент прихода ответного сигнала часы Алисы будут показывать время $t+2\Delta t$. Пусть в момент приема сигнала часы Боба показывают некое время $t'$. Рассматривая передачу ответного сигнала в СО Боба, заметим, что она ничем не отличается от передачи первого сигнала при рассмотрении из СО Алиса. Тогда время испускания сигнала Бобом относится к времени приёма Алисой так же, как время испускания Алисой к времени приёма Бобом, иначе
 \begin{equation}
 	\frac{t+2\Delta t}{t'}=\frac{t'}{t} \,\rightarrow\, (t')^2 = t(t+2\Delta t).
 	 \label{eq:sr-2}
 \end{equation}
 Из \eqref{eq:sr-1} можно получить что
\begin{equation*}
	t(t+2\Delta t)=\frac{c+v}{c-v}t^2.
\end{equation*}
Отсюда получим соотношение между $t'$ и $t$
\begin{equation*}
	t' = t \sqrt{\frac{c+v}{c-v}}.
\end{equation*}
Координата $x$ Боба в момент приема сигнала в СО Алисы есть $v(t+\Delta t)$. Из \eqref{eq:sr-1} получим:
\begin{equation*}
	t + \Delta t = \frac{ct}{c-v}, \quad x = \frac{cvt}{c-v}.
\end{equation*}
Рассмотрим величину $c^2(t+\Delta t)^2-x^2$:
\begin{equation}
c^2(t+\Delta t)^2-x^2=\frac{c^4 t^2}{(c-v)^2}-\frac{c^2 v^2 t^2}{(c-v)^2}=c^2 t^2 \frac{c+v}{c-v}=c^2(t')^2.
\end{equation}
Данное соотношение справедливо независимо от относительной скорости движения систем отсчета. Поэтому для двух различных систем отсчета можно записать: 
\begin{equation*}
	c^2t_1^2-x_1^2=c^2t_2^2-x_2^2=c^2(t')^2.
\end{equation*}
Здесь $t_1$ и $x_1$~--- время и расстояние между событиями в первой системе отсчета, а $t_2$ и $x_2$~--- во второй. Но в силу однородности и изотропности пространства не важно, где выбирается начало отсчета и куда направлены координатные оси. Отсюда следует, что величина $c^2\Delta t^2 - \Delta r^2$, где $\Delta t$~--- промежуток времени между двумя событиями в заданной системе отсчета, а $\Delta r$~--- расстояние в пространстве между точками, в которых эти события наступили, одинакова во всех системах отсчета.

Величина $s^2=c^2 \Delta t^2-\Delta r^2$ в СТО называется квадратом \term{интервала} между двумя событиями. При определении интервала используется разность квадратов координат (время в СТО также считается координатой). Поэтому пространство-время называется псевдоевклидовым пространством (в отличие от евклидова пространства, где квадрат расстояния между двумя точками равен сумме квадратов разностей координат). Также его называют пространством Минковского в честь Германа Минковского, который ввел его в рассмотрение.

