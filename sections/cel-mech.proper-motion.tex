\subsection{Собственное движение звёзд}
\term{Собственным движением} $(\mu)$ называется изменение координат звёзд на небесной сфере, вызванное относительным движением звёзд и Солнца, обычно измеряется в mas/год.
\begin{equation}
	\mu = \frac{V_\tau}{D},
\end{equation}
где $V_\tau$~--- тангенциальная относительная скорость звезды, $D$~--- расстояние до неё.

\change{Разделяют также собственное движение по склонению~--- $\mu_\delta$ и собственное движение по прямому восхождению~--- $\mu_\alpha$, которые определяются следующими выражениями:}
\begin{equation}
  \mu_\delta = \frac{\delta(t_2) - \delta(t_1)}{t_2 - t_1}, \quad \quad \mu_\alpha = \frac{\alpha(t_2) - \alpha(t_1)}{t_2 - t_1}.
\end{equation}
\change{
\begin{wrapfigure}{r}{.4\tw}
\begin{flushright}
	\vspace{-1pc}
	\begin{tikzpicture}
	\footnotesize
	\draw [dashes] (0, 4) arc(90:0:3 and 4);
	\draw [dashes] (0, 4) arc(90:0:2 and 4); 
	%
	\draw [dashes] (3.47, 2) arc(0:-70:3.47 and 1.16);	
	\draw [dashes] (2.64, 3) arc(0:-70:2.64 and 0.88);
	%
	\draw [thick, -latex] (2.3, 2.55) arc(-34:-56:2.64 and 0.88);
	\draw [thick, -latex] (2.3, 2.55) arc(53:29:2 and 4);
	\draw [thick, -latex] (2.3, 2.55) .. controls (2.3, 1.9) and (2.1, 1.4) .. (1.93, 1.03);
	%
	\draw (.9, 2.2) node [anchor=south] {$\delta(t_1)$};
	\draw (1.2, .9) node [anchor=south] {$\delta(t_2)$};
	%
	\draw (2, 0) node [anchor=north] {$\alpha(t_2)$};
	\draw (3, 0) node [anchor=north] {$\alpha(t_1)$};
	%
	\draw [fill=white] (2.3, 2.55) circle (0.03);
	\draw [fill=white] (1.93, 1.03) circle (0.03);
	\draw [fill=white] (0, 4) circle (0.03);
	%
	\draw (0, 4) node [anchor=north] {$P$};
	%
	\draw (1.9, 2.4) node [anchor=south] {$\mu_\alpha$};
	\draw (2.6, 2.05) node [anchor=west] {$\mu_\delta$};
	\draw (2.06, 1.65) node [anchor=south] {$\mu$};
	%
\end{tikzpicture}
\end{flushright}
\end{wrapfigure}
 Как отсюда видно, $\mu_\alpha$ является угловой скоростью по малому кругу, а значит, зависит от $\delta$. Следовательно, полное собственное движение $\mu$ можно найти, как
\begin{equation}
	\mu = \sqrt{\mu_\delta^2 + \mu_\alpha^2 \cos^2 \delta},
\end{equation}
потому что радиус малого круга, состоящего из точек со склонением~$\delta$, равен $R \cos \delta$, где $R$~--- радиус сферы, содержащей этот круг.
}

\begin{figure}[h!]
\begin{subfigure}[b]{0.47\tw}
	\begin{tikzpicture}[scale=1.05]
	\footnotesize
	
%	\foreach \x in {0, .1,...,4} {
%		\draw [line width=.1pt] (\x - 1, 0) -- (\x - 1, 4);
%	};
%	
%	\foreach \x in {0, 1,...,4} {
%		\draw [line width=.4pt] (\x - 1, 0) -- (\x - 1, 4);
%	};
%	
%	\foreach \y in {0, .1,...,4} {
%		\draw [line width=.1pt] (-1, \y) -- (4, \y);
%	};
%	
%	\foreach \y in {0, 1,...,4} {
%		\draw [line width=.4pt] (-1, \y) -- (4, \y);
%	};
	
	\draw [double] (.21, .21) arc (45:104:.3);
	\draw (-.93, 3.71) arc (-76:-35:.3);
	
	\draw (0, 0) -- (-1, 4);
	\draw (0, 0) -- (2, 2);
	\draw (-1, 4) -- (2.6, 1.6);
	
	\draw [thick, -latex] (-1, 4) -- (0, 4.25);
	\draw [thick, -latex] (-1, 4) -- (-.6, 2.4);
	
	\draw [fill=white] (-1, 4) circle (.03);
	\draw [fill=white] (0, 0) circle (.03);
	\draw [fill=white] (2, 2) circle (.03);
	
	\draw (1, 1) node [anchor=north west] {$R$};
	\draw (-.45, 2.1) node [anchor=north east] {$R_0$};
	\draw (.5, 2.95) node [anchor=south west] {$V \Delta t$};
	\draw (0, 0) node [anchor=north] {Солнце};
	\draw (-1, 4) node [anchor=south east] {Звезда};
	
	\draw (.1, .3) node [anchor=south] {$\xi$};
	\draw (-.9, 3.75) node [anchor=north west] {$\gamma$};
	
	\draw (-.5, 4.15) node [anchor=south] {$V_\tau$};
	\draw (-.75, 3.1) node [anchor=east] {$V_r$};
\end{tikzpicture}
\caption{}
\label{pic:phase-angle-1}
\end{subfigure}
\hfill
\begin{subfigure}[b]{0.47\tw}
\begin{tikzpicture}[scale=0.9]
	\footnotesize
	
	\draw (.2, 4.86) arc (-45:-135:0.28);
	\draw [double] (-1.65, 1.51) arc (5:80:0.25);
	
	\draw (0, 5) .. controls (-1.5, 4) and (-2, 2) .. (-2, 0);
	\draw (0, 5) .. controls (1.5, 4) and (2, 2) .. (2, 0);
	\draw (-2, 0) .. controls (-1, -.5) and (1, -.5) .. (2, 0);
	\draw (-1.9, 1.5) .. controls (-1, 1.5) and (1, 2) .. (1.5, 3);
	
	\draw [fill=white] (0, 5) circle (.03);
	\draw [fill=white] (-2, 0) circle (.03);
	\draw [fill=white] (2, 0) circle (.03);
	\draw [fill=white] (-1.9, 1.5) circle (.03);
	\draw [fill=white] (1.5, 3) circle (.03);
	
	\draw (-2, .2) -- (-1.8, .11) -- (-1.8, -.09);
	\draw (2, .2) -- (1.8, .11) -- (1.8, -.09);
	
	\draw (0, 5) node [anchor=south] {$P$};
	\draw (0, 1.9) node [anchor=north] {$\xi$};
	\draw (0, -.4) node [anchor=south] {$\Delta \alpha$};
	\draw (0, 4.8) node [anchor=north] {$\Delta \alpha$};
	\draw (-1, 4) node [anchor=east] {$90^\circ - \delta$};
	\draw (.9, 4.2) node [anchor=west] {$90^\circ - (\delta + \Delta \delta)$};
	\draw (0, -.4) node [anchor=north] {Небесный экватор};
\end{tikzpicture}
\caption{}
%\label{pic:phase-angle-2}
\end{subfigure}
\caption{}
\end{figure}


\change{Получим выражение для координат звезды, имеющей собственное движение $\mu = (\mu_\alpha, \mu_\delta)$, лучевую скорость $V_r$ и параллакс в начальный момент времени $\pi_0$. Найдем сначала тангенциальную скорость:
\begin{equation*}
	V_\tau = R_0 \sqrt{ \mu_\delta^2 + \mu_\alpha^2 \cos^2 \delta} = \frac{\sqrt{ \mu_\delta^2 + \mu_\alpha^2 \cos^2 \delta}}{\pi_0}.
\end{equation*}
Определим теперь угол между лучем зрения и полной скоростью звезды:
\begin{equation*}
	\gamma = \arctan \frac{V_\tau}{V_r}.
\end{equation*}
При этом полная скорость равна
\begin{equation*}
	V_0 = \sqrt{V_\tau^2 + V_r^2}.
\end{equation*}
Из теоремы косинусов можно найти расстояние для звезды через промежуток времени $\Delta t$:
\begin{equation*}
	R = \sqrt{R_0^2 + (V_0 \Delta t)^2 - 2 R_0 V_0 \Delta t \cos \gamma}.
\end{equation*}
Тогда угловое перемещение звезды равно
\begin{equation*}
	\sin \xi = \frac{V_0 \Delta t \sin \alpha}{R}.
\end{equation*}
Через компоненты собственного движения нетрудно получить угол между направлением на полюс и вектором полного собственного движения в начальный момент:
\begin{equation*}
	\tg \psi =  \frac{\mu_a \cos \delta}{\mu_\delta}.
\end{equation*}
Теперь с помощью сферической теоремы косинусов можно определить склонение звезды через время $\Delta t$:
\begin{equation*}
	\sin (\delta - \Delta \delta) = \cos \xi \sin \delta + \sin \xi \cos \delta \cos \psi.
\end{equation*}
Далее из сферической теоремы синусов получаем выражение для изменения прямого восхождения за время $\Delta t$~---
\begin{equation*}
	\sin \Delta \alpha = \frac{\sin \psi \sin \xi}{\cos (\delta - \Delta \delta)}.
\end{equation*}
}





