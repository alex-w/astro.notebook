\subsection{Чёрные дыры}
\term{Чёрная дыра}~(ЧД)~--- область пространства-времени с массой $M$, гравитационное притяжение которой настолько велико, что покинуть её не могут даже объекты, движущиеся со скоростью света $c$. Граница этой области называется \imp{горизонтом событий}, а её характерный размер~$R_G$~--- \imp{гравитационным радиусом}, для величины которого справедливо равенство
\begin{equation}
    R_G = \frac{2 G M}{c^2}.
\end{equation}

Минимальная масса ЧД составляет около $2.5M_{\odot}$. А плотность ЧД определяется отношением ее массы~$M$ к~объему~$V$, следовательно
\begin{equation}
    \rho = \frac{M}{V} = \frac{3c^6}{32\pi M^2G^3}.
\end{equation}

\term{Эффект излучения} (испарения) \term{Хокинга}~--- эффект, при котором гравитационное поле черной дыры поляризует вакуум, в результате чего возможно образование не только виртуальных, но и реальных пар частица~--античастица. Одна из частиц, оказавшаяся чуть ниже горизонта событий, падает внутрь чёрной дыры, а другая, оказавшаяся чуть выше горизонта, улетает, унося энергию (то есть часть массы) чёрной дыры. Для мощности излучения ЧД справедлива формула
\begin{equation}
    L = \frac{h c^6}{30720 \pi^2 G^2 M^2},
\end{equation}
где $h$ --- постоянная Планка. Спектр хокинговского излучения для безмассовых полей оказался строго совпадающим с излучением абсолютно чёрного тела, что позволило приписать ЧД температуру, равную
\begin{equation}
    T = \frac{h c^3}{16 \pi^2 k G M},
\end{equation}
где $k$ --- постоянная Больцмана.

\term{Квазар}~— центральная область галактики с активным ядром, содержащая свермассивную черную дыру. Квазары представляют собой очень яркие и, в то же время, далекие и древние источники, поэтому ввиду их малого параллакса их зачастую используют для калибровки точных астрометрических инструментов. На~\picRef{pic:spectrum-QSO} представлен характерный спектр квазара.

\begin{figure}[h!]
    \centering
    \tikzsetnextfilename{spectrum-QSO}
    \begin{tikzpicture}
        \footnotesize
        \begin{axis} [
            height = 4cm,
            width  = \tw,
            xlabel = {Длина волны $\lambda,~\mathring{\text{A}}$},
            ylabel style={
                text width=3cm,
                align=center
            },
            ylabel = Относительная\\интенсивность,
            xmax   = 7000,
            xmin   = 4000,
%            xticklabels={4000, 5000, 6000, 7000, 8000, 9000},
%            xtick  ={4000, 5000, 6000, 7000, 8000, 9000},
            ytick  = {0, 1},
        ]
            \addplot[smooth] table[x=wavelength, y=flux, col sep=comma]{data/spectrum-QSO.csv};
            \draw[thin] (axis cs:6500,0.3) -- (axis cs:6400,0.5) node[anchor=south] {$\text{H}_\alpha$};
            \draw[thin] (axis cs:4861,0.43) -- (axis cs:4800,0.6) node[anchor=south] {$\text{H}_\beta$};
            \draw[thin] (axis cs:4340,0.19) -- (axis cs:4340,0.4) node[anchor=south] {$\text{H}_\gamma$};

            \draw[thin] (axis cs:5010,0.5) -- (axis cs:4980,0.8);
            \draw[thin] (axis cs:4959,0.27) -- (axis cs:4980,0.8) node[anchor=south] {O\,III};
        \end{axis}
    \end{tikzpicture}
    \caption{Приведённый спектр квазара}
    \label{pic:spectrum-QSO}
\end{figure}
