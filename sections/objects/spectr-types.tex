\subsection{Спектральные классы звёзд}
\begin{table}[h!]
	\centering
	\footnotesize
	\renewcommand{\arraystretch}{1.4}
	\renewcommand{\tabcolsep}{0pt}
	\begin{tabularx}{\tw}{|C{0.1}|C{0.3}|C{0.23}|C{0.13}|C{0.13}|C{0.13}|}
		\hline
		{\bfseries Класс} & {$\mathbf{T}$, К} & {\bfseries Цвет} & {$\mathbf{M}$, $M_{\odot}$} & {$\mathbf{R}$, $R_{\odot}$} & {$\mathbf{L}$, $L_{\odot}$}\\
		\hline
		O & $3 \times 10^4$ --- $6 \times 10^4$ & Голубой & 60 & 15 & $1.4 \times 10^6$\\
		
		B & $1 \times 10^4$ --- $3 \times 10^4$ & Бело-голубой & 18 & 7 & $2 \times 10^4$\\
		
		A & $7.5 \times 10^3$ --- $1 \times 10^4$ & Белый & 3.1 & 2.1 & 80\\
		
		F & $6 \times 10^3$ --- $7.5 \times 10^3$ & Жёлто-белый & 1.7 & 1.3 & 6\\
		
		G & $5 \times 10^3$ --- $6 \times 10^3$ & Жёлтый & 1.1 & 1.1 & 1.2\\
		
		K & $3.5 \times 10^3$ --- $5 \times 10^3$ & Оранжевый & 0.8 & 0.9 & 0.4\\
		
		M & $2 \times 10^3$ --- $3.5 \times 10^3$ & Красный & 0.3 & 0.4 & 0.04\\
		\hline
	\end{tabularx}
	\caption{Современная спектральная классификация звёзд}
	\label{tab:spectr-types}
\end{table}
Звёзды в зависимости от своего цвета делятся на \imp{спектральные классы}, основные из них представлены в Таблице\,\ref{tab:spectr-types}. Масса, радиус и светимость приведены средних представителей спектрального класса, лежащих на главной последовательности (V).

Запись спектрального класса представляет собой латинскую букву, арабское число и римское число, например, спектральный класс Солнца~--- G2V. арабское число показывает к какой именно части спектрального класса относится звезда: к более синей (число меньше) или к красной (число больше). Так,~G10V~--- это тоже самое, что K0V. Спектральный класс (показатель цвета) и абсолютная звёздная величина задают положение звезды на \imp{Диаграмме Герцшпрунга-Рассела}.


%	\includegraphics[width=10cm]{gr}
	
\term{Диаграмма Герцшпрунга-Рассела} показывает зависимость светимости или абсолютной звёздной величины от спектрального класса, показателя цвета $(B-V)$ или эффективной температуры фотосферы звезды.

Была предложена примерно в 1910 году независимо Эйнаром Герцшпрунгом и Генри Расселом. Диаграмма используется для классификации звёзд и соответствует современным представлениям о звёздной эволюции.

Около $90 \%$ звёзд находятся на главной последовательности. Их светимость обусловлена термоядерными реакциями превращения водорода в гелий. Выделяется также несколько ветвей проэволюционировавших звёзд-гигантов, в которых происходит горение гелия и более тяжёлых элементов. В левой нижней части диаграммы находятся полностью проэволюционировавшие белые карлики.

Мнемонические правила для запоминания спектральных классов: <<\textbf{O}h \textbf{B}e \textbf{A} \textbf{F}ine \textbf{G}irl, \textbf{K}iss \textbf{M}e \textbf{R}ight \textbf{N}ow \textbf{S}weetheart.>> и <<\textbf{В}ообразите: \textbf{О}дин \textbf{Б}ритый \textbf{А}нгличанин \textbf{Ф}иники \textbf{Ж}евал \textbf{К}ак \textbf{М}орковь --- \textbf{Р}азве \textbf{Н}е \textbf{С}мешно?>>
\begin{figure}[h!]
	\centering
	\vspace{-1pc}
	\tikzsetnextfilename{hr-diagram}
	\begin{tikzpicture}
 		\begin{axis}[
 						height	=	10cm,
 						width	=	10cm,
 						ymax	=	14.,
 						ymin	=	-6.,
 						y dir	=	reverse,
 						xmax	=	2.,
 						xmin	=	-.5,
 						axis x line* = bottom,
 						axis y line* = right,
 						xlabel  =   $B-V$,
 						y label style = {at={(axis description cs: 1.07, 0.5)}, rotate=180},
 						ylabel	=	{Абсолютная звёздная величина $M$, $\!~^m$}
 					]
			\ifthenelse{\boolean{useLightPlotVersion}}{}{
			    \addplot+[only marks, mark = o, mark options={scale=0.2, darkgray}] table[x=BV, y=M]{data/gr-plot.txt};
			}
 		\end{axis}
 		\begin{semilogyaxis}[
 						height	=	10cm,
 						width	=	10cm,
 						ymax	=	2.088e4,
 						ymin	=	2.088e-4,
 						xmax	=	2.,
 						xmin	=	-.5,
 						minor x tick num = 0,
 						minor y tick num = 01,
 						xtick = {-0.264, 0, 0.3, 0.58, 0.791, 1.57},
 						xticklabels = {B0, A0, F0, G0, K0, M0},
 						axis x line* = top,
 						axis y line* = left,
 						xlabel	=	{Спектральный класс},
 						x label style = {at={(axis description cs: 0.5, 1.03)}, rotate=0},
 						ylabel	=	{Светимость $L$, $L_\odot$},
 						ymajorgrids	 =	false,
 						xmajorgrids	 =	false
    				]
		\end{semilogyaxis}
 	\end{tikzpicture}
 	\caption{Диаграмма Герцшпрунга--Рассела}
	\label{}
\end{figure}
Помимо основных спектральных классов звёзд существуют дополнительные: W~--- звёзды Вольфа-Райе, очень тяжёлые яркие звёзды с температурой порядка $70000$~К и интенсивными эмиссиоными линиями спектра; L~--- звёзды или коричневые карлики с температурой 1500\,--\,2000~К и соединениями металлов в атмосфере; T~--- метановые коричневые карлики с температурой $700 - 1500$~К; Y~---  очень холодные (метано-аммиачные) коричневые карлики с температурой ниже $700$~К; C~--- углеродные звёзды, гиганты с повышенным содержанием углерода. Ранее относились к классам R и N.
