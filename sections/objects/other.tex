\subsection{Другие объекты}
\begin{minipage}{0.63\tw}
    \term{Шаровое звёздное скопление}~--- скопление звёзд, состоящее из нескольких сотен тысяч светил, тесно связанных гравитацией. Млечный путь насчитывает около 160 шаровых звёздных скоплений. Диаметры шаровых скоплений составляют 20\,--\,60~пк, массы~--- $10^4$\,--\,$10^6$~солнечных.\\

    \term{Планетарная туманность}~--- система из звезды, называемой ядром туманности, и симметрично окружающей ее светящейся газовой оболочки. Планетарные туманности образуются при сбросе внешних слоёв (оболочек) красных гигантов и сверхгигантов с массой от $0.8M_\odot$ до $8M_\odot$ на завершающей стадии их эволюции. Характерный размер~--- 1\,--\,2~св.\,лет.\\

    \term{Рассеянное звёздное скопление}~--- слабо связанная группа из сотен или тысяч звёзд, сформировавшихся из одного гигантского \imp{молекулярного облака} и имеющих одинаковый возраст. Рассеянные звёздные скопления встречаются только в тонком диске Галактики, их типичный диаметр~--- несколько парсек.
\end{minipage}
\hfill
\begin{minipage}{0.32\tw}
    \centering
    \vspace{-1.2pc}
    \includegraphics[width = .7\tw]{m13.pdf}
    \captionof{figure}{Шаровое скопление M13 (негатив)}
    \vspace{1pc}
    \includegraphics[width = .7\tw]{m57.pdf}
    \captionof{figure}{Пла\-не\-тар\-ная туманность M57 (негатив)}
    \vspace{1.2pc}
    \includegraphics[width = .7\tw]{m45.pdf}
    \captionof{figure}{Рассеянное звёздное скопление M45 (негатив)}
\end{minipage}

