\subsection{Вырожденные звёзды}
\term{Вырожденные звезды}~--- звезды, в которых силам гравитации противостоят силы давление вырожденного газа. К таким относятся \imp{белые карлики} и \imp{нейтронные звезды}.

\begin{figure}[!h]
    \centering
    \begin{minipage}[c]{0.49\tw}
        \tikzsetnextfilename{light-curve-b-lyr}
        \begin{tikzpicture}
            \begin{axis}[
                height    =    4.5cm,
                width    =    \tw,
                xlabel    =    {Фаза},
                ylabel    =    {Блеск $m$, $~^m$},
                ymax    =    .7,
                ymin    =    -.1,
                y dir    =    reverse,
                xmax    =    1,
                xmin    =    .0
                ]
                \addplot[smooth] table[x=t, y=m, col sep = comma]{data/light-curve-B-Lyr.csv};
            \end{axis}
        \end{tikzpicture}
    \end{minipage}
    \hfill
    \begin{minipage}[c]{0.49\tw}
        \centering
        \includegraphics[width = .9\tw]{b-lyr}
    \end{minipage}
    \caption{Кривая блеска переменной типа $\beta$\,Lyr}
    \label{pic:b-lyr}
    \vspace{-.8pc}
\end{figure}
\term{Белые карлики}~--- проэволюционировавшие звёзды лишённые собственных источников термоядерной энергии и светящие за счёт остывания. Масса белого карлика находится в диапазоне от $0.6M_{\odot}$ до $1.44 M_{\odot}$. Верхняя границы массы белого карлика называется пределом Чандрасекара, звезда с массой больше данного предела не может существовать как белый карлик. Радиус белых карликов примерно в $10^2$ раз меньше солнечного, т.е. можно считать, что $R_\text{БК} \simeq R_\oplus$. Плотность белых карликов лежит в диапазоне $10^7$\,--\,$10^{10}$~$\text{кг}/\text{м}^3$.

\term{Нейтронная звезда}~--- сверхплотная звезда, образующаяся в результате взрыва Сверхновой. Вещество нейтронной звезды состоит в основном из нейтронов. Масса нейтронной звезды лежит в пределах от $0.1M_{\odot}$ до $2$\,--\,$2.8M_{\odot}$ (предел Оппенгеймера-Волкова). Размер данной звезды составляет лишь $10$\,--\,$20$~км, а плотность составляет $10^{16}$\,--\,$10^{18}$ $\text{кг}/\text{м}^3$.  Дальнейшему гравитационному сжатию нейтронной звезды препятствует давление ядерной материи, возникающее за счёт взаимодействия нейтронов. Так как нейтронные звёзды образуются в результате  коллапса массивных звёзд, то из-за сохранения момента импульса скорость их вращения может достигать $10^5$~км/с. При наличии сильного магнитного поля и быстром вращении нейтронная звзеда может наблюдаться с Земли как \term{пульсар}.
