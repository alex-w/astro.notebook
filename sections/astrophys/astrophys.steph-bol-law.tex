\subsection{Закон Стефана-Больцмана}
\term{Закон Стефана~--- Больцмана} определяет зависимость плотности мощности излучения абсолютно чёрного тела (АЧТ) $u$ от его температуры $T$:
\begin{equation}
u = a T^4,
\end{equation} 
где $a$~--- некая универсальная константа.
Отсюда полная светимость АЧТ с площадью поверхности $S$
	\begin{equation}
	L = S \sigma T^4,
	\label{eq:steff-bol-law}
\end{equation}
константа $\sigma$ называется \term{постоянной Стефана-Больцмана}.
  
Важно отметить, что \imp{закон Стефана-Больцмана}~--- прямое следствие формулы Планка \eqref{Planck's formula}, так как
\begin{equation}
	\sigma T^4 = \int\limits^\infty_0 B(\lambda, T)\,d\lambda \int\limits_0^{\pi/2} \sin \varphi\, d\varphi \int\limits_0^{2\pi} \cos \varphi\, d\theta = \pi \int\limits^\infty_0 B(\lambda, T)\,d\lambda,
\end{equation}
откуда $\sigma = (2\pi^5k^4)/(15c^2h^3) = 5.67 \cdot 10^{-8}~\text{Вт}/(\text{м}^2\cdot \text{К}^4)$.

%Для АЧТ сферической формы с радиусом $R$ формула~\eqref{eq:steff-bol-law} принимает вид
%\begin{equation}
%L=4\pi R^2\sigma T^4.
%\end{equation}
Для звёзд главной последовательности выполняется соотношение $L \sim M^{\alpha}$, где~$\alpha$~--- коэффициент пропорциональности, который зависит от массы звезды следующим образом:
\begin{align*}
\alpha &= 2.5, \quad M < 0.43 M_\odot; & 
\alpha &= 4, \quad 0.43 M_\odot < M < 2 M_\odot;\\ 
\alpha &= 3.2, \quad 2 M_\odot < M < 20 M_\odot; & 
\alpha &= 1, \quad M > 20 M_\odot.
\end{align*}
Также существует примерная зависимость светимости звёзды от её радиуса, имеющая вид  $L\sim R^{5.2}$.