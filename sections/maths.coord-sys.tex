\subsection{Системы координат}
	
	Зафиксируем точку $O$ в пространстве и рассмотрим произвольную точку $M$. Вектор $\vec{r} = \overrightarrow{OM}$ называется радиус-вектором точки $M$. Пусть в рассматриваемом пространстве также выбран базис, тогда совокупность точка $O$ и базиса называется \term{декартовой системой координат}. Причем точка $O$~--- начало координат, а базисные векторы $\vec{e}_1, \ldots, \vec{e}_n$ задают координатные оси.
	
	Однако пользоваться представлением векторов в произвольном базисе довольно сложно, поэтому рассмотрим специальный тип~--- \term{ортонормированный базис}~--- это такой базис, в котором базисные векторы попарноортогональны и длина каждого равна единице.
	
	Тогда \term{прямоугольной декартовой системой координат} (ПДСК) называют декартову систему координат с ортонормированным базисом.
	
	В практических задачах использовать ПДСК не всегда удобно, поэтому рассмотрим также другие системы координат. На плоскости, то есть в пространстве $\R^2$, имеющем размерность два, часто применяется полярная система координат. В ней координатами вектора является его длина $r$~--- расстояние точки от начала отсчета, и угол $\varphi$ с начальной ось. 
	
	Пусть $(x,y)$~--- координаты некоторого вектора в ПДСК на $\R^2$, тогда не сложно получить, что его координаты в полярной системе координат (начальная ось совпадает с осью $Ox$)  удовлетворяют следующим соотношениям:
	\begin{equation}
		\begin{cases}
			r = \sqrt{x^2 + y^2},\\
			\sin \varphi = y/r,\\
			\cos \varphi = x/r.
		\end{cases}
		\quad \Leftrightarrow \quad 
		\begin{cases}
			x = r \cos \varphi,\\
			y = r \sin \varphi.
		\end{cases}
	\end{equation}
	
	Теперь пусть $(x, y, z)$~--- координаты некоторого вектора в ПДСК на~$\R^3$. Обозначим за $h$~--- длину проекции этого вектора на ось $z$, $r$~--- длину его проекции на плоскость $Oxy$, $\varphi$~--- угол между проекцией на плоскость $Oxy$ и осью $Ox$. Тогда тройка $(r, \varphi, h)$~--- координаты рассматриваемого векторы в цилиндрической системе координат и верно представление
	\begin{equation}
		\begin{cases}
			r = \sqrt{x^2 + y^2},\\
			\sin \varphi = y/r,\\
			\cos \varphi = x/r,\\
			h = z.
		\end{cases}
		\quad \Leftrightarrow \quad 
		\begin{cases}
			x = r \cos \varphi,\\
			y = r \sin \varphi,\\
			z = h.
		\end{cases}
	\end{equation}
	
	Остается рассмотреть сферическую систему координат. Здесь координатами точки будет длина радиус-вектора $r$ и два угла: $\theta$~--- угол между радиус-вектором и плоскостью $Oxy$ и $\varphi$~--- угол между проекцией радиус-вектора на плоскость $Oxy$ и осью $Ox$. Верны формулы перехода:
	\begin{equation}
		\begin{cases}
			r = \sqrt{x^2 + y^2 + z^2},\\
			\theta = \arcsin{z/r},\\
			\sin \varphi = y/r,\\
			\cos \varphi = x/r.
		\end{cases}
		\quad \Leftrightarrow \quad
		\begin{cases}
			x = r \cos \theta \cos \varphi	,\\
			y = r \cos \theta \sin \varphi,\\
			z = r \sin \theta.
		\end{cases}
	\end{equation}