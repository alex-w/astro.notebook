\subsection{Интеграл}
\term{Первообразной} функции $f(x)$ называется такая функция $F(x)$, что $F'(x) = f(x)$ на всей области определения $f(x)$. Из определения легко понять, что первообразные функции $f(x)$ образуют целое семейство первообразных $G(x) = F(x) + C$, где $C$~--- константа, определяющая конкретную первообразную. Это семейство называется \term{неопределенным интегралом} функции $f(x)$,
\begin{equation}
    F(x) = \int f(x)\,dx + C,\quad F'(x)=f(x).
\end{equation}

\begin{wrapfigure}{r}{0.4\tw}
    \centering
    \vspace{-1pc}
    \tikzsetnextfilename{math-int}
    \begin{tikzpicture}
        \footnotesize
        \tkzDefPoint(0,0){O};
        \tkzDefShiftPoint[O](3.5,0){X};
        \tkzDefShiftPoint[O](0,1.7){Y};

        \tkzInit[
          ymin    =   -0.7,
          ymax    =   1.8,
          xmin    =   -0.4,
          xmax    =   3.7
        ]
        \tkzClip

        \tkzDrawLine[add = 0.1 and 0, semithick, -latex](O,X);
        \tkzDrawLines[add = 0.2 and 0, semithick, -latex](O,Y);

        \tkzDefPoint(3, 0.8){F};

        \begin{scope}[yscale=0.7]
            \fill [lightgray] (0.4, 0) -- (0.4, 1.7) .. controls (0.9, 1.65) and (1.1, 0.85) .. (1.55, 0) -- cycle;
            \fill [lightgray] (1.55, 0) .. controls (1.9, -0.9) and (2.3, -1.2) .. (2.5, -0.8) -- (2.5, 0) -- cycle;
            \draw [thick] (-0.3, 0.5) .. controls (1, 5) and (2, -5) .. (3, 1);
        \end{scope}

        \tkzDefPoint(0.4,0){A};
        \tkzDefPoint(0.4,1.19){A'};
        \tkzDefPoint(2.5,0){B};
        \tkzDefPoint(2.5,-0.56){B'};

        \tkzDrawSegments[dashes](A,A' B,B');

        \tkzLabelSegment[pos=0.43,right=4pt](A,A'){$S_+$};
        \tkzLabelSegment[pos=0.55,left](B,B'){$S_-$};

        \tkzLabelPoint[above](X){$x$};
        \tkzLabelPoint[right](Y){$y$};
        \tkzLabelPoint[below](A){$a$};
        \tkzLabelPoint[above](B){$b$};

        \tkzLabelPoint[right](F){$f(x)$};

        \tkzDrawPoints(A,A',B,B');
    \end{tikzpicture}
    \caption{}
    \label{pic:math-int}
\end{wrapfigure}
\term{Определенный интеграл} характеризуется верхним и нижним пределом интегрирования. Значение определенного интеграла численно равно площади под графиком функции на данном промежутке и вычисляется по формуле \imp{Ньютона--Лейбница}:
\begin{equation}
    \int\limits^b_a f(x) \,d x = F(x) \biggr|^b_a = F(b) - F(a)
\end{equation}
Правила интегрирования:
\begin{align*}
    &\int c f(x) \,d x = c \int f(x) \,d x;\quad &&  \int f(ax + b) \,d x = \dfrac{1}{a}F(ax + b) + C;\\
    &\int f \,d g = fg - \int g \,d f; && \int \bigl[f(x) + g(x)\bigr] \,d x = \int f(x) \,d x + \int g(x) \,d x;
\end{align*}
Таблица интегралов:
\begin{align*}
    &\int  x^a \,d x = \dfrac{x^{a+1}}{a+1} + C,\quad a \neq -1; \quad
    &&\int \dfrac{dx}{\sqrt{a^2 - x^2}} = \arcsin\dfrac{x}{a} + C;\\
    &\int \frac{dx}{x} = \ln x + C;
    &&\int \dfrac{dx}{-\sqrt{a^2 - x^2}} = \arccos\dfrac{x}{a} + C;\\
    &\int a^x \,d x = \dfrac{a^x}{\ln a} + C;
    &&\int \dfrac{dx}{x^2 + a^2} = \dfrac{1}{a} \arctg \dfrac{x}{a} + C; \\
    &\int \cos x \,d x = \sin x + C;
    &&\int \dfrac{dx}{x^2 - a^2} = \dfrac{1}{2a} \ln \dfrac{|x - a|}{|x + a|} + C;\\
    &\int \sin x \,d x = -\cos x + C;
    &&\int \dfrac{dx}{\sqrt{x^2 + a}} = \ln \left| x + \sqrt{x^2 + a} \right| + C.
\end{align*}

Рассмотрим пример использования интеграла~--- получим формулу для нахождения объема шара с радиусом~$R$. Сначала получим вспомогательную формулу площади круга радиуса~$R$. Для этого рассмотрим тонкое кольцо радиуса~$r$~и ширины~$dr$, его площадь составляет~$2\pi r \,d r$. Следовательно, интегрируя площадь таких колец по радиусу~$r$ от~$0$ до~$R$, получим площадь круга
\begin{equation*}
    S = \int_0^R 2 \pi r \,d r = 2 \pi \cdot \left.\frac{r^2}{2} \right|_0^R = \pi R^2.
\end{equation*}
Теперь, используя полученное выражение для площади круга, легко найти объем шара. Рассмотрим тонкий слой толщины~$dx$ на расстоянии~$x$ от центра шара, его радиус равен $\sqrt{R^2 - x^2}$. Проинтегрируем объем таких слоев~по~$x$ от~$-R$ до~$R$:
\begin{equation*}
    V 
        = \int_{-R}^{R} \pi (R^2 - x^2) \,d x 
        = \pi \left.\left(xR^2 - \frac{x^3}{3}\right)\right|_{-R}^{R}
        = \pi \left(2R^3 - \frac{2R^3}{3} \right) 
        = \frac{4}{3} \pi R^3.
\end{equation*}

В заключение нужно отметить, что интегрирование может вестись не только вдоль координатных осей той или иной системы координат, но и вдоль произвольной кривой в пространстве, по произвольной поверхности в пространстве или, например, произвольной области пространства. В таком случае вместо пределов интегрирования указывают кривую, поверхность или область, по которой ведется интегрирование. Однако за сложной записью в любом случае кроется кратный интеграл, кратность которого определяется размерностью пространства, в котором определена область интегрирования.
