\subsection{Интеграл}
\term{Неопределенным интегралом} функции $f(x)$ называется такая функция $F(x)$, производная которой равна $f(x)$.
\begin{equation}
	F(x) = \int f(x)dx,\quad F^\prime(x)=f(x).
\end{equation}

\begin{wrapfigure}{r}{0.46\tw}
	\centering
	\vspace{-1.1pc}
	\centering
	\begin{tikzpicture}[scale=1.1, yscale=0.7]
		\footnotesize
		\clip(-0.5, -1.05) rectangle (3.6, 2.1);
		
%		\foreach \x in {-5, -4.9,...,5} {
%			\draw [line width=.1pt] (\x, -5) -- (\x, 5);
%		};
%		
%		\foreach \x in {-5, -4,..., 5} {
%			\draw [line width=.4pt] (\x , -5) -- (\x , 5);
%		};
%		
%		\foreach \y in {-5, -4,..., 5} {
%			\draw [line width=.4pt] (-5, \y) -- (5, \y);
%		};
%		
%		\foreach \y in {-5, -4.9,..., 5} {
%			\draw [line width=.1pt] (-5, \y) -- (5, \y);
%		};
		

		\fill [lightgray] (0.4, 0) -- (0.4, 1.7) .. controls (0.9, 1.65) and (1.1, 0.85) .. (1.55, 0) -- cycle;	
		\fill [lightgray] (1.55, 0) .. controls (1.9, -0.9) and (2.3, -1.2) .. (2.5, -0.8) -- (2.5, 0) -- cycle;	
		\draw [thick] (-0.3, 0.5) .. controls (1, 5) and (2, -5) .. (3, 1);	
		
		\draw[semithick, -latex] (-0.5, 0) -- (3.5, 0); 		
		\draw[semithick, -latex] (0, -0.5) -- (0, 2); 
		
		\draw [dashes] (0.4, 0) -- (0.4, 1.7);
		\draw [dashes] (2.5, 0) -- (2.5, -0.8);

		\draw (0.9, 0.5) node{$S_+$};
		\draw (2.15, -0.45) node{$S_-$};
		\draw (3.5, 0) node[anchor=south]{$x$};
		\draw (0, 2) node[anchor=west]{$y$};
		\draw (0.4, 0) node[anchor=north]{$a$};
		\draw (2.5, 0) node[anchor=south]{$b$};
		\draw (2.95, 0.8) node[anchor=west]{$f(x)$};

		\draw [fill=white] (0.4, 1.7) circle(0.03);
		\draw [fill=white] (0.4, 0) circle(0.03);
		\draw [fill=white] (2.5, 0) circle(0.03);
		\draw [fill=white] (2.5, -0.8) circle(0.03);
		
	\end{tikzpicture}
	\caption{}
	\label{pic:math-int}
\end{wrapfigure}
\term{Определенный интеграл} характеризуется верхним и нижним пределом интегрирования. Значение определенного интеграла численно равно площади под графиком функции на данном промежутке и вычисляется по формуле \imp{Ньютона--Лейбница}:
\begin{equation}
	\int\limits^b_a f(x)\,dx = F(x) \biggr|^b_a = F(b) - F(a)
\end{equation}
Правила интегрирования:
\begin{align*}
	&\int c f(x) \,dx = c \int f(x) \,dx;\quad &&  \int f(ax + b) \,dx = \dfrac{1}{a}F(ax + b) + C;\\
	&\int f \,dg = fg - \int g \,df; && \int \bigl[f(x) + g(x)\bigr] \,dx = \int f(x) \,dx + \int g(x) \,dx;
\end{align*}
Таблица интегралов:
\begin{align*}
	&\int  x^a \,dx = \dfrac{x^{a+1}}{a+1} + C,\quad a \neq -1; \quad
	&&\int \dfrac{dx}{\sqrt{a^2 - x^2}} = \arcsin\dfrac{x}{a} + C;\\
	&\int \frac{dx}{x} = \ln x + C;
	&&\int \dfrac{dx}{-\sqrt{a^2 - x^2}} = \arccos\dfrac{x}{a} + C;\\
	&\int a^x \,dx = \dfrac{a^x}{\ln a} + C;
	&&\int \dfrac{dx}{x^2 + a^2} = \dfrac{1}{a} \arctg \dfrac{x}{a} + C; \\
	&\int \cos x \,dx = \sin x + C;
	&&\int \dfrac{dx}{x^2 - a^2} = \dfrac{1}{2a} \ln \dfrac{|x - a|}{|x + a|} + C;\\
	&\int \sin x \,dx = -\cos x + C;
	&&\int \dfrac{dx}{\sqrt{x^2 + a}} = \ln \left| x + \sqrt{x^2 + a} \right| + C.
\end{align*}
