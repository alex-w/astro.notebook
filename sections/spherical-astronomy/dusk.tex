\subsection{Сумерки}
\begin{wrapfigure}[10]{r}{0.53\tw}
    \centering
    \vspace{-1pc}
    \tikzsetnextfilename{dusk}
    \begin{tikzpicture}
        \def\R{4}
        \def\r{2.5}
        \def\eps{0.1}
        
        \tkzDefPoint(0,0){C}
        \tkzDefShiftPoint[C](\R,0){R}
        \tkzDefShiftPoint[C](\r,0){r}
        \tkzDefPoint(0,\r){G}
        \tkzDefPoint({-sqrt(\R^2 - \r^2)},\r){S}

        \tkzDefPointBy[homothety=center G ratio -0.5](S) \tkzGetPoint{Sun}
        \tkzDrawSegment[-latex](G,Sun)
        \tkzLabelPoint[right](Sun){$\odot$}

        \def\a{12}

        \begin{scope}

            \tkzInit[
                xmin=-\R - \eps,
                xmax={0.7*\r},
                ymin=-0.1*\R,
                ymax=\R + \eps
            ]
            \tkzClip

            \newcommand{\drawZone}[2]{
                \begin{scope}[rotate={#1}]
                    \tkzInit[
                        xmin=-\R,
                        xmax=0,
                        ymin=0,
                        ymax=\r
                    ]
                    \tkzClip

                    \tkzDrawCircle[fill=#2](C,R)
                \end{scope}
            }

            \drawZone{0*\a}{black!10}
            \drawZone{1*\a}{black!20}
            \drawZone{2*\a}{black!30}
            \drawZone{3*\a}{black!40}
        \end{scope}

        \begin{scope}
            \tkzInit[
                xmin=-\R - \eps,
                xmax={0.7*\r},
                ymin=-0.1*\R - \eps,
                ymax=\R + \eps
            ]
            \tkzClip

            \tkzDrawCircle[fill=white, draw=none](C,r)
        \end{scope}

        \begin{scope}
            \tkzInit[
                xmin=-\R - \eps,
                xmax={0.7*\r},
                ymin=-0.1*\R,
                ymax=\R + \eps
            ]
            \tkzClip
            \tkzDrawCircle[black, semithick](C,R)
            \tkzDrawCircle[fill=white, draw=black, semithick](C,r)
        \end{scope}
        
        \tkzDrawSegments(G,C S,G)
        \tkzMarkRightAngle[size=0.2](S,G,C)
        \tkzDrawPoints(S,G)
        \tkzDefShiftPoint[G](0,0){g}
        \foreach \x in {1,...,3} {
            \tkzDefPointBy[rotation=center C angle {\x*\a}](G) \tkzGetPoint{G'}
            \tkzDefPointBy[rotation=center C angle {\x*\a}](S) \tkzGetPoint{S'}

            \tkzMarkAngle[size={\x*0.05 + 1.3}](g,C,G')
            \tkzLabelAngle[pos={\x*0.05 + 1.5}](g,C,G'){\adjustbox{right=9pt}{\scriptsize$6^{\circ}$}}

            \tkzDrawSegments(G',C S',G')
            \tkzMarkRightAngle[size=0.2](S',G',C)
            \tkzDrawPoints(S',G')

            \tkzDefShiftPoint[G'](0,0){g}
        }

        \foreach[count=\i from 0] \t in {{День}, {Граж. сум.}, {Нав. сум.}, {Астр. сум.}, {Ночь}} {
            \tkzDefPointBy[rotation=center C angle {\i * \a}](G) \tkzGetPoint{G'}
            \tkzDefPointBy[rotation=center C angle {\i * \a}](S) \tkzGetPoint{S'}
            \tkzDrawSegment[
                decorate,
                decoration={
                    text along path,
                    text={|\footnotesize|\t},
                    text align={
                        left indent=12pt
                    },
                    raise=4pt
                }
            ](S',G')
        }

        \tkzDrawPoint(C)
    \end{tikzpicture}
    \caption{Сумерки}
    \label{pic:dusk}
\end{wrapfigure}

\term{Сумерки}~--- время суток, когда Солнце находится неглубоко под горизонтом. В зависимости от высоты Солнца под горизонтом различают \imp{граж\-дан\-ские} (от~$0^\circ$ до~$-6^\circ$), \imp{нави\-га\-ци\-онные} (от~$-6^\circ$ до~$-12^\circ$) и \imp{астро\-но\-ми\-ческие} (от~$-12^\circ$ до~$-18^\circ$) сумерки. Когда Солнце опускается ниже~$-18^\circ$, наступает ночь.
