\subsection{Суточное вращение небесной сферы}
Вследствие вращения Земли вокруг своей оси для наблюдателя на поверхности небесные объекты совершают суточное движение параллельно небесному экватору, плоскость которого совпадает с плоскостью экватора Земли. Очевидно, в ходе такого движения высота светил постоянно меняется и в некоторые моменты времени достигает своего максимального и минимального значения.

\begin{figure}[h!]
    \centering
    \begin{subcaptionblock}{0.49\tw}
        \tikzsetnextfilename{culmination-1}
        \begin{tikzpicture}
            \footnotesize
            \tkzDefPoint(0,0){O}
            \def\R{2.1}
            \def\PHI{55}
            \def\DELTA{17}

            \tkzDefShiftPoint[O](\R,0){N}
            \tkzInterLC[common=N](O,N)(O,N) \tkzGetFirstPoint{S}

            \tkzDefPointBy[rotation=center O angle \PHI](N) \tkzGetPoint{P}
            \tkzInterLC[common=P](O,P)(O,N) \tkzGetFirstPoint{P'}

            \tkzDefPointBy[rotation=center O angle 90](N) \tkzGetPoint{Z}
            \tkzInterLC[common=Z](O,Z)(O,N) \tkzGetFirstPoint{Z'}

            \tkzDefPointBy[rotation=center O angle -90+\PHI](S) \tkzGetPoint{Q}
            \tkzInterLC[common=Q](O,Q)(O,N) \tkzGetFirstPoint{Q'}

            \tkzDefPointBy[rotation=center O angle \DELTA](Q') \tkzGetPoint{D'}
            \tkzDefPointBy[rotation=center O angle -\DELTA](Q) \tkzGetPoint{D}

            \tkzDrawCircle[black,semithick](O,N)

            \tkzMarkAngles[arc=lll, size=1](Q',O,D' D,O,Q)
            \tkzLabelAngles[pos=1.2](Q',O,D' D,O,Q){\footnotesize$\delta$}

            \tkzMarkAngle[arc=ll, size=0.3](N,O,P)
            \tkzLabelAngles[pos=0.5](N,O,P){\footnotesize$\varphi$}

            \tkzMarkAngle[arc=ll, size=0.4, mark=|, mksize=2pt](Q,O,S)
            \tkzLabelAngles[pos=0.4, left](Q,O,S){\raisebox{6pt}{\scalebox{0.7}{$90^\circ - \varphi$}}}

            \tkzMarkAngle[arc=l, size=1.4, mark=x, mksize=1.5pt](D,O,S)
            \tkzLabelAngles[pos=1.7](D,O,S){\footnotesize$h_\text{в}$}

            \tkzMarkAngle[arc=l, size=0.9, mark=o, mksize=1pt](D',O,N)
            \tkzLabelAngles[pos=1.2](D',O,N){\footnotesize$h_\text{н}$}

            \tkzInterLL(D,D')(P,P') \tkzGetPoint{M}

            \tkzMarkRightAngles[size=0.18](P',O,Q' Z,O,S D',M,P)

            \tkzDrawSegments(N,S P,P' Z,Z' Q,Q' D,D' O,D O,D')
            \tkzDrawPoints(O, N, S, P, P', Z, Z', Q, Q', D, D')

            \tkzLabelPoints[below](Z')
            \tkzLabelPoints[above](Z)
            \tkzLabelPoints[left](S)
            \tkzLabelPoints[right](N)
            \tkzLabelPoints[above right](P)
            \tkzLabelPoints[above left](Q)
            \tkzLabelPoints[below left](P')
            \tkzLabelPoints[below right](Q')
        \end{tikzpicture}
        \caption{$|\delta| < |\varphi|$}
        \label{pic:culmination-less}
    \end{subcaptionblock}
    \hfill
    \begin{subcaptionblock}{0.49\tw}
        \tikzsetnextfilename{culminations-2}
        \begin{tikzpicture}
            \footnotesize
            \tkzDefPoint(0,0){O}
            \def\R{2.1}
            \def\PHI{27}
            \def\DELTA{40}

            \tkzDefShiftPoint[O](\R,0){N}
            \tkzInterLC[common=N](O,N)(O,N) \tkzGetFirstPoint{S}

            \tkzDefPointBy[rotation=center O angle \PHI](N) \tkzGetPoint{P}
            \tkzInterLC[common=P](O,P)(O,N) \tkzGetFirstPoint{P'}

            \tkzDefPointBy[rotation=center O angle 90](N) \tkzGetPoint{Z}
            \tkzInterLC[common=Z](O,Z)(O,N) \tkzGetFirstPoint{Z'}

            \tkzDefPointBy[rotation=center O angle -90+\PHI](S) \tkzGetPoint{Q}
            \tkzInterLC[common=Q](O,Q)(O,N) \tkzGetFirstPoint{Q'}

            \tkzDefPointBy[rotation=center O angle \DELTA](Q') \tkzGetPoint{D'}
            \tkzDefPointBy[rotation=center O angle -\DELTA](Q) \tkzGetPoint{D}

            \tkzDrawSegments(N,S P,P' Z,Z' Q,Q' D,D' O,D O,D')

            \tkzDrawCircle[black,semithick](O,N)

            \tkzMarkAngles[arc=lll, size=1](Q',O,D' D,O,Q)
            \tkzLabelAngles[pos=1.2](Q',O,D' D,O,Q){\footnotesize$\delta$}

            \tkzMarkAngle[arc=ll, size=0.5](N,O,P)
            \tkzLabelAngles[pos=0.7](N,O,P){\footnotesize$\varphi$}

            \tkzMarkAngle[arc=ll, size=0.4, mark=|, mksize=2pt](Q,O,S)
            \tkzLabelAngles[pos=0.4, left](Q,O,S){\raisebox{6pt}{\scalebox{0.7}{$90^\circ - \varphi$}}}

            \tkzMarkAngle[arc=l, size=1.1, mark=x, mksize=1.5pt](N,O,D)
            \tkzLabelAngles[pos=1.4, fill=white, inner sep=1pt](N,O,D){\footnotesize$h_\text{в}$}

            \tkzMarkAngle[arc=l, size=0.9, mark=o, mksize=1pt](D',O,N)
            \tkzLabelAngles[pos=1.2](D',O,N){\footnotesize$h_\text{н}$}

            \tkzInterLL(D,D')(P,P') \tkzGetPoint{M}

            \tkzMarkRightAngles[size=0.18](P',O,Q' Z,O,S D',M,P)


            \tkzDrawPoints(O, N, S, P, P', Z, Z', Q, Q', D, D')

            \tkzLabelPoints[below](Z')
            \tkzLabelPoints[above](Z)
            \tkzLabelPoints[left](S)
            \tkzLabelPoints[right](N)
            \tkzLabelPoints[above right](P)
            \tkzLabelPoints[above left](Q)
            \tkzLabelPoints[below left](P')
            \tkzLabelPoints[below right](Q')
        \end{tikzpicture}
        \caption{$|\delta| > |\varphi|$}
        \label{pic:culmination-greater}
    \end{subcaptionblock}
    \caption{Схемы кульминаций в проекции на плоскость небесного меридиана}
    \label{}
\end{figure}

\term[верхняя кульминация]{Верхняя} и \term{нижняя кульминация}~--- моменты пересечения светилом небесного меридиана, причём при верхней кульминации светило имеет наибольшую высоту, а при нижней~--- наименьшую.

Высота светила в верхней и нижней кульминации со склонением $|\delta| < |\varphi|$ (\picRef{pic:culmination-less}), соответственно:
\begin{equation}
    h_{\text{в}}= 90^\circ - \varphi + \delta, \quad\quad
    h_{\text{н}}= - 90^\circ + \varphi  + \delta.
\end{equation}

Если же светило имеет склонение $|\delta| > |\varphi|$ (\picRef{pic:culmination-greater}), то высота в верхней и нижней кульминации вычисляется так:
\begin{equation}
    h_{\text{в}}= 90^\circ + \varphi - \delta, \quad\quad
    h_{\text{н}}= - 90^\circ -\varphi - \delta.
\end{equation}

Из формул для высоты в нижней кульминации вытекает условие, определяющее, пересекает ли звезда горизонт:
\begin{equation}
    \begin{cases}
        h_\text{в} = +90^\circ - |\varphi - \delta| > 0^\circ,\\
        h_\text{н} = - 90^\circ + |\varphi + \delta| < 0^\circ;
    \end{cases}
    \quad \Longleftrightarrow \quad~~ |\delta|< 90^{\circ} - |\varphi|.
\end{equation}

Используя формулы сферической тригонометрии (см.\,\ref{sec:spher-trig}), можно выразить зависимость часового угла светила от его зенитного расстояния:
\begin{equation}
    \cos t=\frac{\cos z-\sin\varphi\sin\delta}{\cos\varphi\cos\delta}.
\end{equation}
Отсюда следует, что для часового угла захода и восхода светила справедливо равенство:
\begin{equation}
    \cos t_{\uparrow\downarrow}=-\tg\varphi\cdot\tg\delta.
\end{equation}

Аналогично, для вычисления азимута светила верна формула
\begin{equation}
    \cos A=\frac{\cos\delta\cos t-\cos\varphi\cos z}{\sin\varphi\sin z}.
\end{equation}
Следовательно, азимуты точек восхода и захода
\begin{equation}
    A_\uparrow = \arccos \left(-\dfrac{\sin\delta}{\cos \varphi} \right)\quad\text{и}\quad A_\downarrow = - A_\uparrow.
\end{equation}

\term{Звёздное время}~$z$~--- часовой угол точки весеннего равноденствия. Из определений прямого восхождения и часового угла следует справедливость равенства\begin{equation}
z = \alpha + t.
\end{equation}
