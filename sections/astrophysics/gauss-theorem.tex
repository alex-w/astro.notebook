\subsection{Теорема Гаусса}
\label{sec:gauss}

Пусть элементарная площадка имеет площадь $dS$, нормаль $\vec n$, определим вектор этой площадки, как $\vec{dS} = \vec n \,d S$. Пусть напряженность гравитационного поля (ускорение свободного падения) постоянно в малой окрестности площадки, тогда поток напряженности поля через площадку равен $d\Phi = \scalar{g}{dS} = g \,d S \cos \theta$, где $\theta$~--- угол поворота от вектора $\vec g$ к вектору $\vec {dS}$.  

\paragraph{Теорема Гаусса} Пусть внутри замкнутой поверхности $s$ заключена масса $M$, тогда полный поток напряженности гравитационного поля через поверхность $s$ равен
\begin{equation}
	\Phi = \oint\limits_s \scalar{g(r)}{dS} = -4\pi G M.
\end{equation}    
\imp{Доказательство}\,\cite{кириченко2011электричество}. Начнем с самого простого случая~--- точечной массы $M$, а в качестве замкнутой поверхности $s$ возьмём сферу, центр которой совпадает с положение точечной массы, радиусом $R$. Тогда в любой точке поверхности модуль вектора напряженности гравитационного поля будет равен $g = GM/R^2$, а направление $\vec g$ будет противоположным направлению нормалей к сфере (считаем нормали внешними). Следовательно,
\begin{equation*}
	\Phi = \oint\limits_s \scalar{g(r)}{dS} = -\oint\limits_s \frac{GM}{R^2} \,d S = -\int\limits_{4\pi} \frac{GM}{R^2} R^2 \,d \Omega = -4\pi GM.\footnote{см. раздел \ref{subsec:solid-angle}}
\end{equation*}
Этот случай легко обобщается на несферические поверхности, потому что, как можно заметить, подынтегральное выражение не зависит от расстояния элементарной площадки от точечной массы.

Пусть теперь точечная масса $M$ находится вне поверхности $s$. В таком случае суммарный поток через поверхность должен быть равен нуль, потому как внутри поверхности нет никакой массы. Действительно, пустим луч из точечной массы через поверхность, ясно, луч пересечет поверхность дважды, причем первый раз нормаль к поверхности с вектором ускорения свободного падения будет образовывать острый угол ($\cos \theta > 0$), а во втором~--- тупой ($\cos \theta < 0$). Рассмотрим телесный угол $d \Omega$ вокруг этого направления и найдём вклад в суммарный поток от потоков через элементарные площадки в данном телесном углу:
\begin{equation*}
	d\Phi = d\Phi_+ + d \Phi_- = \frac{GM}{R_1^2} R_1^2 \,d \Omega -  \frac{GM}{R_2^2} R_2^2 \,d \Omega = 0.
\end{equation*}
Так как вклад не зависит от выбора направления, то теорема в этом случае доказана.

До этого момента в расчет брались только выпуклые относительно точечной массы поверхности. Поступая аналогично предыдущему случаю, можно показать, что если масса находится внутри поверхности, то луч, пересекающий поверхность не один раз, обязательно пересекает её нечетное число раз. А значит, элементарные площадки, находящиеся в телесном углу $d\Omega$ дадут вклад $d\Phi = - GM \,d \Omega$. Следовательно, общий поток будет равен $\Phi = - 4 \pi GM$.

Остается обобщить полученные результаты на случай произвольного распределения масс. В силу принципа суперпозиции полей поток напряженности суммарного поля является суммой потоков полей элементарных масс. Пусть $\rho(\vec r)$~--- распределение плотности внутри поверхности. Тогда суммарный поток
\begin{equation*}
	\Phi = \int\limits_V -4\pi G \rho(\vec r) \,d V = -4 \pi GM.
\end{equation*}

Остается сказать, что приведенные выше выкладки не зависят от природы поля и подходят для любого поля, напряженность которого подчиняется закону $C\vec{r}/r^3$. Так, для электрического поля $\Phi = 4\pi Q$, где $Q$~--- суммарный заряд внутри поверхности. На самом деле теорема Гаусса выполняется не только для полей, но и для функций, являющихся зависимостями вида $C/r^2$, так, например, суммарный поток излучения проходящий через замкнутую поверхность равен $L$, где $L$~--- суммарная светимость источников внутри поверхности.
