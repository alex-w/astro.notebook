\subsection{Формула Планка}
\label{sec:planck-law}
\term{Формула Планка}~--- выражение для спектральной плотности мощности излучения
 абсолютно чёрного тела на интервале частот $[\nu, \nu + d \nu)$,
 распространяющегося в телесном угле $d\Omega$, которое было получено Максом
 Планком в 1900~году. Данное выражение имеет следующий вид:
\begin{equation}
    B_\nu(\nu,T)
    = \frac{2h\nu^3}{c^2}\cdot \frac{1}{\exp\left(\frac{h\nu}{kT}\right)-1}
    = \left[ \frac{\text{Вт}}{\text{м}^2 \cdot \text{Гц} \cdot \text{ср}}\right],
    \label{eq:plancks-law-nu}
\end{equation}
где $\nu$~--- частота излучения, $T$~--- температура АЧТ, $h$~--- постоянная
 Планка, $k$~--- постоянная Больцмана, $c$~--- скорость света. Если записать закон
 излучения Планка \eqref{eq:plancks-law-nu} для длин волн, получится
\begin{equation}
    B_\lambda(\lambda,T)
    = \frac{2hc^2}{\lambda^5} \cdot \frac{1}{\exp\left(\frac{hc}{\lambda kT}\right)-1}
    = \left[ \frac{\text{Вт}}{\text{м}^3 \cdot \text{ср}}\right].
    \label{eq:plancks-law-lambda}
\end{equation}

Стоит заметить, что при переходе к выражению формулы Планка через длину
 волны также меняется выражение для интервала, поэтому в результате 
 изменяется степень переменной в знаменателе.

Формула Планка появилась, когда стало ясно, что формула Рэлея-Джинса
 удовлетворительно описывает излучение только в области больших длин волн,
 а~с~убыванием длин волн сильно расходится с реальными данными. Однако формулу
 Рэлея-Джинса используют и сейчас для описания спектра абсолютно чёрного тела 
 в длинноволновой области.


Проделаем обратные действия: получим формулу Рэлея-Джинса из формулы Планка. Длинноволновая часть спектра характеризуется соотношением $h\nu \ll kT$, то есть
\begin{equation*}
    \exp\left( \frac{h\nu}{kT}\right) \approx 1 + \frac{h\nu}{kT}.
\end{equation*}
Подставляя данное выражение в знаменатель \eqref{eq:plancks-law-nu}, получим
\begin{equation*}
    B_\nu(\nu,T)
    \approx \frac{2h\nu^3}{c^2}\cdot \frac{1}{1 + \frac{h\nu}{kT} - 1}
    = \frac{2h\nu^3 }{c^2}\cdot \frac{k T}{ h \nu}
    = \frac{2 \nu^2 k T}{c^2}.
\end{equation*}

\begin{figure}[t]
    \centering
    \vspace{-.9pc}
    \tikzsetnextfilename{wien-law}
    \begin{tikzpicture}
        \begin{axis}[
            width     =    \tw,
            height    =    8cm,
            ymax    =    1e+14,
            xmax    =    2000,
            xmin    =    0,
            ymin    =    0,
            xlabel    =    {Длина волны $\lambda$,~нм},
            ylabel     =     {$B_\lambda(\lambda, T)$,~$\text{Вт} \cdot \text{м}^{-3}$},
            restrict y to domain        =    0:1e+15
            ]
            \addplot+[dashed, thin, black] table[x=l, y=tl] {data/planck.txt};
            \addplot[black] table[x=l, y=t3800] {data/planck.txt} node at (axis cs:870, 1.4e+13) {\scriptsize{$3800$~K}};
            \addplot+[black, smooth, solid] table[x=l, y=t4500] {data/planck.txt} node at (axis cs:750, 2.7e+13) {\scriptsize{$4500$~K}};
            \addplot+[black, smooth, solid] table[x=l, y=t5000] {data/planck.txt}node at (axis cs:710, 4.4e+13) {\scriptsize{$5000$~K}};
            \addplot+[black, smooth, solid] table[x=l, y=t5800] {data/planck.txt}node at (axis cs:640, 8.7e+13) {\scriptsize{$5800$~K}};
            \addplot+[black, smooth, solid] table[x=l, y=t7000] {data/planck.txt}node at (axis cs:1340, 3.3e+13) {\scriptsize{$7000$~K}};
            \addplot+[black, smooth, solid] table[x=l, y=t10000] {data/planck.txt} node at (axis cs:1430, 5.5e+13) {\scriptsize{$10000$~K}};
            \addplot+[black, smooth, solid] table[x=l, y=t20000] {data/planck.txt} node at (axis cs:1670, 8.0e+13) {\scriptsize{$20000$~K}};
        \end{axis}
    \end{tikzpicture}
    \caption{Кривые спектральной плотности мощности изотропного излучения АЧТ с разной температурой}
    \label{pic:wien-law}
\end{figure}

Проделав те же действия для формулы Планка через длину волны, получим:
\begin{equation}
    B(\lambda, T)
    \simeq \frac{2 c k T}{\lambda^4}, \quad\quad B(\nu, T) \simeq \frac{2 \nu^2 k T}{c^2}.
    \label{Ray-Jean}
\end{equation}

В коротковолновой области, наоборот, $h \nu \gg kT$, следовательно, единица в знаменателе формулы Планка много меньше стоящей там экспоненты, то есть
\begin{equation*}
    \frac{1}{\exp\left(\frac{h\nu}{kT}\right)-1}
    \approx \frac{1}{\exp\left(\frac{h\nu}{kT}\right)}
    = \exp\left(-\frac{h\nu}{kT}\right).
\end{equation*}
Отсюда получаются приближения, называемые приближениями Вина:
\begin{equation}
    B ( \lambda, T)
    \simeq \frac{2 h c^2}{\lambda^5} \exp\left(-\frac{h c}{\lambda k T}\right), \quad B( \nu, T )
    \simeq \frac{2 h \nu^3}{c^2} \exp\left(-\frac{h \nu}{k T}\right).
\end{equation}
