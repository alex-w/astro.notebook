\subsubsection{Гало}
\label{sec:halo}

Как было сказано выше, различные виды \imp{гало} наблюдаются в результате рефракции солнечного света в кристаллах льда, составляющих перистые облака. Эти облака в среднем находятся на высоте от 5 до 10~км, имеют температуру около $-40^\circ$C, почему состоят исключительно из кристаллов льда.


\begin{wrapfigure}[6]{r}{0.3\tw}
    \centering
    \vspace{-0.7pc}
    \newcommand{\drawPrizm}[2]{
        \tkzDefPoint(0,0){C}
	    
	    \def\R{#1}
	    \def\h{#2}
	    \def\f{0}
	    
	    \tkzDefShiftPoint[C](\R,0){x}
	    \tkzDefPointBy[rotation=center C angle \f](x)  \tkzGetPoint{R1}
	    \tkzDefPointBy[rotation=center C angle 60](R1) \tkzGetPoint{R2}
	    \tkzDefPointBy[rotation=center C angle 60](R2) \tkzGetPoint{R3}
	    \tkzDefPointBy[rotation=center C angle 60](R3) \tkzGetPoint{R4}
	    \tkzDefPointBy[rotation=center C angle 60](R4) \tkzGetPoint{R5}
	    \tkzDefPointBy[rotation=center C angle 60](R5) \tkzGetPoint{R6}
	    
	    \tkzGetPointCoord(R1){r}
	    \tkzGetPointCoord(R2){R}
	    \tkzGetPointCoord(R3){rr}
	    \tkzGetPointCoord(R4){rR}
	    \tkzGetPointCoord(R5){Rr}
	    \tkzGetPointCoord(R6){RR}
	    
	    \draw[dashed] (\rx,\ry,0) -- (\Rx,\Ry,0) -- (\rrx,\rry,0) -- (\rRx,\rRy,0);
	    \draw (\rRx,\rRy,0) -- (\Rrx,\Rry,0) -- (\RRx,\RRy,0) -- (\rx,\ry,0);
	    
	    \draw (\rx,\ry,\h) -- (\Rx,\Ry,\h) -- (\rrx,\rry,\h) -- (\rRx,\rRy,\h) -- (\Rrx,\Rry,\h) -- (\RRx,\RRy,\h) -- cycle; 
	    
	    \draw (\rx,\ry,0) -- (\rx,\ry,\h);
	    \draw[dashed] (\Rx,\Ry,0) -- (\Rx,\Ry,\h);
	    \draw[dashed] (\rrx,\rry,0) -- (\rrx,\rry,\h);
	    \draw (\rRx,\rRy,0) -- (\rRx,\rRy,\h);
	    \draw (\Rrx,\Rry,0) -- (\Rrx,\Rry,\h);
	    \draw (\RRx,\RRy,0) -- (\RRx,\RRy,\h);
    }

    \tdplotsetmaincoords{70}{165}
    \begin{tikzpicture}[tdplot_main_coords]
        \drawPrizm{0.4}{1.5}
	\end{tikzpicture}
	\tdplotsetmaincoords{70}{185}
    \begin{tikzpicture}[tdplot_main_coords]
        \drawPrizm{0.9}{0.3}
	\end{tikzpicture}
	\caption{}
	\label{}    
\end{wrapfigure}

В условия формирования перистых облаков кристаллы льда имеют шестиугольную симметрию и являются прямоугольными шестиугольными призмами. Их боковые грани могут иметь разные пропорции, но угол между соседними всегда составляет $60^\circ$. В дальнейшем будем разделять кристаллы на два вида: пластинчатые~--- размер оснований много больше высоты призмы, и колончатые~--- наоборот, высота призмы существенно больше размеров основания.

\paragraph{$\mathbf{22^\circ}$ гало}
\begin{wrapfigure}[11]{r}{0.25\tw}
    \vspace{-1pc}
    \centering
    \begin{tikzpicture}[
        arrow/.style 2 args={
            postaction=decorate,
                decoration={
                    markings, 
                    mark=at position #2 with {\arrow{#1}}
                } 
            }
    ]
    
	    \tkzDefPoint(0,0){C}
	    
	    \def\R{2}
	    \def\f{0}
	    \def\n{1.31}
	    
	    \tkzInit[
	       xmin={-0.2*\R},
	       xmax={1.1*\R},
	       ymin={-1.1*\R},
	       ymax={1.3*\R},
	    ]
	    \tkzClip
	    
	    
	    \tkzDefShiftPoint[C](\R,0){x}
	    \tkzDefPointBy[rotation=center C angle \f](x)  \tkzGetPoint{R1}
	    \tkzDefPointBy[rotation=center C angle 60](R1) \tkzGetPoint{R2}
	    \tkzDefPointBy[rotation=center C angle 60](R2) \tkzGetPoint{R3}
	    \tkzDefPointBy[rotation=center C angle 60](R3) \tkzGetPoint{R4}
	    \tkzDefPointBy[rotation=center C angle 60](R4) \tkzGetPoint{R5}
	    \tkzDefPointBy[rotation=center C angle 60](R5) \tkzGetPoint{R6}
	    
	    \tkzDrawPolygon(R1,R2,R3,R4,R5,R6)
	    
	    \def\al{30}
	    \def\x{0.4}
	    
	    \tkzDefPointBy[homothety=center R2 ratio \x](R3) \tkzGetPoint{I}
	    \tkzDefLine[perpendicular=through I](R3,R2) \tkzGetPoint{I'}
	    
	    \def\bet{asin(sin(\al / 180 * pi) / \n)}
	    \def\b{\R * (sqrt(3) / 2  + (0.5 - \x) * tan(pi / 2 - \bet))}
	    \def\xo{(\b + sqrt(3) * \R) / (tan(pi / 2 - \bet) + sqrt(3))}
	    \def\yo{sqrt(3) * \xo - sqrt(3) * \R}
	    
	    \tkzDefPoint(\xo,\yo){O}
	    \tkzDefLine[perpendicular=through O](R1,R6) \tkzGetPoint{O'}
	    
	    \tkzInterLL(I,I')(O,O') \tkzGetPoint{T}
	    
	    \tkzDefPointBy[rotation=center I angle {-180 + \al}](T) \tkzGetPoint{x}
	    \tkzDefPointBy[homothety=center I ratio 0.4](x) \tkzGetPoint{A}
	    
	    \tkzFindAngle(I,O,T) \tkzGetAngle{gam}
	    \def\del{asin(\n * sin(\gam / 180 * pi)) * 180 / pi}
	    
	    \tkzDefPointBy[rotation=center O angle {180 - \del}](T) \tkzGetPoint{x}
	    \tkzDefPointBy[homothety=center O ratio 0.5](x) \tkzGetPoint{D}
	    
	    \tkzDrawSegments[semithick, arrow={latex}{0.3}](A,I)
	    \tkzDrawSegments[semithick, arrow={latex}{0.5}](I,O)
	    \tkzDrawSegments[semithick, arrow={latex}{0.9}](O,D)
	     
	    \tkzDrawLines[dashed, add=0.5cm and 0.5cm](I,T O,T)
	    
	    \tkzMarkRightAngles[size=0.2](R3,I,T T,O,R6)
	    
	    \tkzMarkAngle[arc=lll, size=0.5](T,I,O)
	    \tkzLabelAngle[pos=0.75](T,I,O){\footnotesize$\beta$}
	    
	    \tkzMarkAngle[arc=lll, mark=|, size=0.4, mksize=2](I',I,A)
	    \tkzLabelAngle[pos=0.6](I',I,A){\footnotesize$\alpha$}
	    
	    \tkzMarkAngle[line width = .3pt, size=0.01](I,O,T)
	    \tkzMarkAngle[arc=ll, size=0.3](I,O,T)
	    \tkzLabelAngle[pos=0.5](I,O,T){\footnotesize$\gamma$}
	    
	    \tkzMarkAngle[arc=ll, mark=|, size=0.3, mksize=2](D,O,O')
	    \tkzLabelAngle[pos=0.5](D,O,O'){\footnotesize$\delta$}
	    
	    \tkzMarkAngles[size=0.2](O,T,I R2,R1,R6 R3,R2,R1 R1,R6,R5)
	    \tkzLabelAngle[pos=0.5](O,T,I){\scriptsize$120^\circ$}
	   	    
	    \tkzDrawPoints(I, T, O, R1, R2, R6)
	\end{tikzpicture}
	\caption{}
	\label{}
\end{wrapfigure}
Яркая окружность вокруг Солнца, угловой радиус которой оставляет около $22^\circ$ в зависимости от длины волны излучения. Формируется в произвольно расположенных кристаллах в результате рефракции света на двух, следующих через одну, боковых гранях.

В зависимости от угла падения света на первую грань меняется угол итогового преломления, минимальная величина которого~--- примерно $22^\circ$. В силу экстремальности этого значения, наибольшее число кристаллом преломляют солнечный свет именно под этим углом.

Рассмотрим детальнее геометрию формирования данного гало.




\paragraph{Паргелический круг}

Когда пластинчатые кристаллы дрейфуют вниз под действием силы тяжести, из-за сопротивления воздуха и своей формы они ориентируются определённым образом: их основания почти горизонтальны. Такая конфигурация стабильна, то есть небольшие отклонения создают корректирующие силы, возвращающие силы. Таким образом у пластинчатых кристаллом только одна степень свободы~--- вращение в горизонтальное плоскости.

\paragraph{Паргелий}

% todo: Как пересечение 22 гало и паргелического круга

