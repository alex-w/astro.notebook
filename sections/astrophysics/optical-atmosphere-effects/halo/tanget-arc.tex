\paragraph{Тангенциальная дуга}

\begin{figure}[h!]
    \foreach \h in {0,15,30,45} {
        \begin{subcaptionblock}{0.48\tw}
            \tikzsetnextfilename{tanget-arc-h-\h}
            \begin{tikzpicture}
                \begin{axis}[
                    height  =   6.5cm,
                    width   =   6.5cm,
                    xmin    =   -2.01,
                    xmax    =   2.01,
                    ymin    =   -2.01,
                    ymax    =   2.01,
                    grid    =   none,
                    axis line style = {draw=none},
                    every tick/.style = {draw=none},
                    yticklabels = {,,},
                    xticklabels = {,,},
                    legend cell align = left,
                    legend style = {
                         draw       =   none,
                         fill       =   none,
                         font       =   \scriptsize,
                         at         =   {(axis cs:2.2, -1)}, 
                         anchor     =   south west,
                         row sep    =   .5pc,
                    }
                ]
                    \addplot[only marks, mark = o, mark options={scale=0.2}, black] table[x=x, y=y] {data/tanget_arc_h\h.txt};
%                    
                    \foreach \hh in {-80,-70,...,-10,10,20,...,80} {
                        \addplot+[smooth, dashes, gray] table[x=x, y=y] {data/tanget_arc_grid\hh.txt};
                    }
%                    
                    \addplot+[smooth, gray, solid] table[x=x, y=y] {data/tanget_arc_grid0.txt};
                    \addplot+[smooth, black, solid] table[x=x, y=y] {data/tanget_arc_grid_border.txt};
                \end{axis}
            \end{tikzpicture}
            \caption{$h = \h^\circ$}
        \end{subcaptionblock}
        \ifthenelse{\isodd{\h}}{\\}{\hfill}
    }
    \caption{Результат компьютерного моделирования тангециальной дуги при разных высотах Солнца над горизонтом в стереографической проекции}
    \label{pic:tanget-arc}
\end{figure}

В то время как пластинчатые кристаллы ориентируются горизонтальная торцевыми гранями, колончатые ориентируются горизонтально главной осью. В силу чего также имеют две степени свободы для вращения: вертикальную и вокруг своей оси.   

\imp{Тангенциальная дуга} формируется в ходе различных преломлений солнченого света в горизонтально ориентированных колончатых кристаллах. Легко понять, что колончатые кристаллы, главная ось которых лежат в картинной плоскости~--- формируют верхнюю и нижнюю точки $22^\circ$~гало. Поэтому \imp{тангенциальная дуга} или две её части всегда касаются $22^\circ$~гало в этих точках, а в момент, когда Солнца находится в зените~--- совпадает в ним. 

%Результаты компьютерного моделирования \imp{тангециальных дуг} при разной высоте Солнца можно увидеть на \picRef{pic:tanget-arc}.
