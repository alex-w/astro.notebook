\subsection{Предел Эддингтона}
\term{Предел Эддингтона}~--- величина мощности электромагнитного излучения, исходящего из недр звезды, при которой его давления достаточно для компенсации веса оболочек звезды, которые окружают зону термоядерных реакций, то есть звезда находится в состоянии равновесия: не сжимается и не расширяется.\\
Сила тяжести $F_g$, действующая со стороны тела массы $M$ на протон, находящийся на расстоянии $r$ от него, равна
\begin{equation}
    F_g = \frac{G M m_p}{r^2}.
\end{equation}
Поток излучения $I$  от тела со светимостью $L$ на расстоянии $r$ выражается выражается, как
\begin{equation}
    I=\frac{L}{4\pi r^2}.
\end{equation}
Тогда сила $F_r$, действующая на электрон вследствие томсоновского рассеяния фотонов на электронах,
\begin{equation}
    F_r = \frac{I \sigma_T}{c},
\end{equation}
где $\sigma_T$~--- томсоновское сечение рассеяния фотона на электроне, равное
\begin{equation}
    \sigma_T = \left(\frac{8\pi}{3}\right)\left(\frac{e^2}{m_e c^2}\right)^2 = 6.65 \times 10^{-29}~\text{м}^2.
\end{equation}
Таким образом, так как $F_g = F_r$, то
\begin{equation}
    L_\text{Эдд} = \frac{4\pi G M m_p c}{\sigma_T} =  \frac{M}{M_\odot} \times 10^{31}~\text{Вт}.
\end{equation}
