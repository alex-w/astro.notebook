\subsection{Ретроградное движение и точки стояния}
Рассмотрим характер движения планеты по небесной сфере в течение синодического периода. Пусть \imp{наблюдатель}~--- условная точка, обращающаяся вокруг Солнца по некоторой круговой орбите, согласно законам Кеплера. Назовем \imp{лучем зрения} прямую, проходящую через наблюдателя и наблюдаемую планету. Тогда угловая скорость планеты на небесной сфере наблюдателя $\boldsymbol \omega = \cross{(r_\text{п} - r_\text{н})}{(u - v)}$, где $\vec{u}$ и $\vec{v}$~--- векторы орбитальных скоростей планеты и наблюдателя соответсвенно. 

Пусть планета является внешней для наблюдателя, случай внутренней рассматривается аналогично. Найдём угловую скорость планеты в противостоянии и соединении:
\begin{align*}
    \omega_\text{п} &= \frac{u - v}{a_\text{п} - a_\text{н}} < 0,\\ 
    \omega_\text{с} &= \frac{u + v}{a_\text{п} + a_\text{н}} > 0,
\end{align*}
где в качестве положительного направления выбрано направления движения Солнца на небе наблюдателя. 

Из геометрического смысла и определения угловой скорости очевидно, что это непрерывная функция, а значит, существуют такие моменты, когда планета меняет направление своего видимого движения по небесной сфере наблюдателя, такие точки называются \term{точками стояния}. А движение планеты в противоположную движению Солнца сторону~--- \term{ретроградным движением}. Причём внешние планеты находятся в ретроградном движении в окрестности противостояния, а внутренние~--- в окрестности нижнего соединения.

\begin{wrapfigure}[13]{r}{0.6\tw}
    \centering
    \vspace{-1pc}
    \tikzsetnextfilename{retrograde-movement-schema}
    \begin{tikzpicture}
        \footnotesize
        
        % The Sun
        \tkzDefPoint(0,0){S}
        
        % Radius of base planet's orbit 
        \def\r{3}
        
        % Ration of orbits' radiuses
        \def\b{1.5}
        
        % Radius of external planet's orbit
        \def\R{\r * \b}
        
        \tkzInit[
           xmin={-0.5},
           xmax={\R + 1},
           ymin={-0.6},
           ymax={\R + 0.1},
        ]
        \tkzClip
        
        % Angle alpha from derivation (Base - Exteranal - Sun)
        \pgfmathsetmacro\ALPHA{acos((\b + sqrt(\b)) / (\b * sqrt(\b) + 1))}
        
        % Angle psi from derivation (External at the begin - Sun - external at the end)
        \pgfmathsetmacro\PSI{2 * \ALPHA / (\b * sqrt(\b) - 1)}
        
        % Position of base planet at the begin
        \tkzDefPoint(\r,0){E}
        
        % Position of external planet at the begin
        \tkzDefPointBy[homothety=center S ratio \b](E) \tkzGetPoint{m}
        \tkzDefPointBy[rotation=center S angle \ALPHA](m) \tkzGetPoint{M}
        
        % Position of base planet at the end
        \tkzDefPointBy[rotation=center S angle (2 * \ALPHA + \PSI)](E) \tkzGetPoint{E'}
        
        % Position of external planet at the end
        \tkzDefPointBy[homothety=center S ratio \b](E') \tkzGetPoint{m}
        \tkzDefPointBy[rotation=center S angle -\ALPHA](m) \tkzGetPoint{M'}
        
        % Intersection on lines (base - external) at the begin and at the end
        \tkzInterLL(E,M)(E',M') \tkzGetPoint{I}
        
        % Orbital speed vector of the external planet at the begin
        \tkzDefLine[tangent at=M](S) \tkzGetPoint{x}
        \tkzDefPointBy[homothety=center M ratio (1.3/sqrt(\b))](x) \tkzGetPoint{vM}
        
        % Orbital speed vector of the base planet at the begin
        \tkzDefLine[tangent at=E](S) \tkzGetPoint{x}
        \tkzDefPointBy[homothety=center E ratio 1.3](x) \tkzGetPoint{vE}
        
        % Line passing through external planet and orthogonal to the line (base - external) at the begin
        \tkzDefLine[tangent at=M](E) \tkzGetPoint{x}
        \tkzDefPointBy[homothety=center M ratio 0.5](x) \tkzGetPoint{M1}
        \tkzDefPointBy[homothety=center M ratio -0.5](x) \tkzGetPoint{M2}
        
        % Line passing through base planet and orthogonal to the line (base - external) at the begin
        \tkzDefLine[tangent at=E](M) \tkzGetPoint{x}
        \tkzDefPointBy[homothety=center E ratio 0.4](x) \tkzGetPoint{E1}
        \tkzDefPointBy[homothety=center E ratio -1.2](x) \tkzGetPoint{E2}
        
    
        % Intersection of the line passing through base planet and orthogonal to the line (base - external) at the begin and line Sun - external planet at the begin
        \tkzInterLL(E,E1)(S,M) \tkzGetPoint{D}
        
        
        \tkzDrawSegments[thick](S,E E,M M,S S,E' E',M' M',S) 
        \tkzDrawSegments[dashed](M,I M',I)
        \tkzDrawSegments[semithick,-latex](E,vE M,vM)
        
        \tkzDrawLines[semithick](M1,M2 E1,E2)
        
        \tkzDrawCircles[black,semithick](S,E S,M)
        
        \tkzMarkAngles[arc=l, size=0.7](E,S,M M',S,E')
        \tkzLabelAngles[pos=0.9](E,S,M M',S,E'){\footnotesize$\alpha$}
        
        \tkzMarkAngles[arc=ll, mark=|, mksize=2pt, size=0.4](S,M,E E',M',S)
        \tkzLabelAngles[pos=0.6](S,M,E E',M',S){\footnotesize$\delta$}
        
        \tkzMarkAngles[arc=ll, size=0.4](M,S,M')
        \tkzLabelAngles[pos=0.6](M,S,M'){\footnotesize$\psi$}
        
        \tkzMarkAngles[arc=lll, size=0.4](M,E,vE D,E,S)
        \tkzLabelAngles[pos=0.4, left](D,E,S){\raisebox{5pt}{~\!\!\scalebox{0.75}{$90^\circ\!\!-\!\delta\!-\!\alpha$}}}
        
        \tkzMarkAngles[arc=l, mark=x, mksize=2pt, size=1.1](vE,E,D)
        \tkzLabelAngles[pos=1.2, above, fill=white, inner sep=0.5](vE,E,D){\footnotesize$\alpha + \delta$}
        
        \tkzMarkAngles[arc=ll, mark=||, mksize=3pt, size=0.4](E2,D,S)
        \tkzLabelAngles[pos=0.45, left](E2,D,S){\footnotesize$90^\circ - \delta$}
        
        \tkzMarkAngles[arc=ll, size=9.5](M',I,M)
        \tkzLabelAngles[pos=9.8](M',I,M){\footnotesize$\Delta\lambda$}
        
        \tkzMarkRightAngles[size=0.2](vE,E,S M,E,E1 E,M,M2 vM,M,S)
        
        \tkzDrawPoints(S, E, M, E', M', I)
        
        \tkzLabelPoints[above](E', I)
        \tkzLabelPoint[above](M'){$~\,M'$}
        \tkzLabelPoint[below left](E){$E$} 
        \tkzLabelPoint[right](M){$M$} 
        \tkzLabelPoint[left](S){$S$}
        \tkzLabelPoint[right](vE){$\vec{v}_\text{н}$}
        \tkzLabelPoint[right](vM){$\vec{v}_\text{п}$}        
    \end{tikzpicture}
    \caption{}
    \label{pic:retrograde-movement-schema}
\end{wrapfigure}

Получим выражение для размера петли ретроградного движения планеты. Для простоты изложения положим, что планета является внешней. В конце покажем, что полученное выражение справедливо и для внутренних планет. 

Итак, пусть обиты планеты и наблюдателя круговые с радиусами $a_\text{п}$ и  $a_\text{н}$ соответственно, пусть также $\alpha$~--- разность эклиптических долгот наблюдателя и планеты, \lookPicRef{pic:retrograde-movement-schema}. Тогда расстояние $r$ между ними можно найти из теоремы косинусов:
\begin{equation}
    r = \sqrt{a_\text{п}^2 + a_\text{н}^2 - 2 a_\text{п} a_\text{н} \cos \alpha}.
    \label{eq:retrograde-movement-distance}
\end{equation}

Пусть $\delta$~--- угол Солнце~-- планета~-- наблюдатель, запишем теорему синусов:
\begin{equation}
    \frac{a_\text{н}}{\sin \delta} = \frac{r}{\sin \alpha}.
    \label{eq:retrograde-movement-sinuses-theorem}
\end{equation}
Остается сказать, что обнуление видимой угловой скорости планеты происходит при равенстве проекций скоростей планеты и наблюдателя на нормаль к лучу зрения, что эквивалентно
\begin{equation}
    v_\text{н} \cos (\alpha + \delta) = v_\text{п} \cos \delta.
    \label{eq:retrograde-movement-velocity-projectione-equivalence}
\end{equation}

Найдём теперь $\alpha$ из системы уравнений \eqref{eq:retrograde-movement-distance}\,--\,\eqref{eq:retrograde-movement-velocity-projectione-equivalence}. Для удобства изложения введём обозначение $b \equiv a_\text{п} / a_\text{н} > 1$. И начнём с \eqref{eq:retrograde-movement-sinuses-theorem}, откуда выразим $\sin \delta$:
\begin{multline*}
    \sin \delta 
        = \frac{a_\text{н} \sin \alpha}{r} 
        = \frac{a_\text{н} \sin \alpha}{\sqrt{a_\text{п}^2 + a_\text{н}^2 - 2 a_\text{п} a_\text{н} \cos \alpha}} = \\
        =  \frac{a_\text{н} \sin \alpha}{a_\text{н} \sqrt{1 + \dfrac{a_\text{п}^2}{a_\text{н}^2} - 2 \dfrac{a_\text{п}}{a_\text{н}} \cos \alpha}}
        = \frac{\sin \alpha}{\sqrt{1 + b^2 - 2 b \cos \alpha}}.
\end{multline*}
Далее получим выражение для $\cos \delta$ из основного тригонометрического тождества:
\begin{multline*}
    0 < \cos \delta 
        = \sqrt{1 - \sin^2 \delta} 
        = \sqrt{1 - \frac{\sin^2 \alpha}{1 + b^2 - 2 b \cos \alpha}} = \\
        = \sqrt{\frac{1 + b^2 - 2 b \cos \alpha - 1 + \cos^2 \alpha}{1 + b^2 - 2 b \cos \alpha}}
        = \frac{b - \cos \alpha}{\sqrt{1 + b^2 - 2 b \cos \alpha}}
\end{multline*}
Также вспомним, что
\begin{equation*}
    \frac{v_\text{п}}{v_\text{н}} = \frac{\sqrt{\dfrac{GM_\odot}{a_\text{п}}}}{\sqrt{\dfrac{GM_\odot}{a_\text{н}}}} = \sqrt{\frac{a_\text{н}}{a_\text{п}}} \equiv \sqrt{\frac{1}{b}}.
\end{equation*}
Остаётся раскрыть косинус суммы в \eqref{eq:retrograde-movement-velocity-projectione-equivalence}:
\begin{gather}
    v_\text{н} (\cos \alpha \cos \delta - \sin \alpha \sin \delta) = v_\text{п} \cos \delta,\nonumber\\
    \cos \alpha \cos \delta - \sin \alpha \sin \delta = \sqrt{\frac{1}{b}} \cos \delta,\nonumber\\
    \cos \delta \left( \cos \alpha - \sqrt{\frac{1}{b}} \right) = \sin \alpha \sin \delta,\nonumber\\
    \frac{b - \cos \alpha}{\sqrt{1 + b^2 - 2 b \cos \alpha}}\left( \cos \alpha - \sqrt{\frac{1}{b}} \right) = \sin \alpha \frac{\sin \alpha}{\sqrt{1 + b^2 - 2 b \cos \alpha}},\nonumber\\
    b \cos \alpha - \sqrt{b} - \cos^2 \alpha + \sqrt{\frac{1}{b}} \cos \alpha = 1 - \cos^2 \alpha,\nonumber\\
    \cos \alpha = \frac{b + \sqrt{b}}{b \sqrt{b} + 1}.
    \label{eq:retrograde-movement-alpha}
\end{gather}
Вторая точка стояния определяется той же разницей долгот, что видно из \picRef{pic:retrograde-movement-schema}, следовательно, ретроградное движение планеты происходит на протяжении времени
\begin{equation*}
    t 
        = \frac{2 \alpha}{|\omega_\text{н} - \omega_\text{п}|} 
        = \frac{2 \alpha}{\left|\dfrac{2\pi}{T_\text{н}} - \dfrac{2\pi}{T_\text{п}} \right|} 
%        = \frac{\alpha}{\pi} \cdot \frac{T_\text{п} T_\text{н}}{T_\text{п} - T_\text{н}} = \\
        = \frac{\alpha}{\pi}\cdot \frac{T_\text{п}}{\left|\dfrac{T_\text{п}}{T_\text{н}} - 1\right|} 
        = \frac{\alpha}{\pi}\cdot \frac{T_\text{п}}{\left| b \sqrt{b} - 1 \right|}.
\end{equation*}
А планета за это время проходит по орбите угол
\begin{equation*}
    \psi = t \omega_\text{п} = \frac{\alpha}{\pi} \cdot \frac{T_\text{п}}{b \sqrt{b} - 1} \cdot \frac{2\pi}{T_\text{п}} = \frac{2\alpha}{b \sqrt{b} - 1}.
\end{equation*}
Остается отметить, размер петли попятного движения $\Delta \lambda = 2\delta - \psi$, что следует из \picRef{pic:retrograde-movement-schema}. Чтобы получить аналитический вид данного выражения, найдем
\begin{multline*}
    \cos 2 \delta 
        = 1 - 2 \sin^2 \delta 
        = 1 - \frac{2(1 - \cos^2 \alpha)}{1 + b^2 - 2b \cos \alpha} = \\
        =\frac{1 + b^2 - 2 b \cos \alpha - 2 + 2 \cos^2 \alpha}{1 + b^2 - 2b\cos \alpha} =\\
        = \frac{-1 + b^2 - 2b \cdot \dfrac{b + \sqrt{b}}{b \sqrt{b} + 1} + 2 \cdot \dfrac{b^2 + b +2b\sqrt{b}}{b^3 + 1 + 2 b \sqrt{b}}}{1 + b^2 - 2b \cdot \dfrac{b + \sqrt{b}}{b \sqrt{b} + 1}}\\
%        = \frac{\dfrac{\scriptstyle{-b^3 - 1 - 2b\sqrt{b} + b^5 + b^2 + 2b^3\sqrt{b} - 2b^3 \sqrt{b} - 2b^3 - 2b^2 - 2b\sqrt{b} + 2b^2 + 2b + 4b\sqrt{b}}}{\scriptstyle{\left(b \sqrt{b} +1\right)^2}}}{\dfrac{\scriptstyle{b\sqrt{b} + 1 + b^3 \sqrt{b} + b^2 - 2b^2 - 2b\sqrt{b}}}{\scriptstyle{b \sqrt{b} + 1}}} = \\
        = \frac{-1 + 2b + b^2 - 3b^3 + b^5}{\left(b \sqrt{b} + 1\right) \left(1 - b \sqrt{b} - b^2 + b^3 \sqrt{b}\right)} = \\
        = \frac{(b^2 - 1) - 2b(b^2 - 1) + b^3(b^2 - 1)}{(1 - b^3)(1 - b^2)} = \\
        = \frac{1 - 2b + b^3}{b^3 - 1} = \frac{b^2(b-1) + b(b-1) - (b - 1)}{\left(b-1\right)\left(b^2 + b + 1\right)} = \frac{b^2 + b - 1}{b^2 + b + 1}.
\end{multline*} 
Отсюда, учитывая \eqref{eq:retrograde-movement-alpha}, размер петли
\begin{equation}
    \Delta\lambda = \arccos \frac{b^2 + b - 1}{b^2 + b + 1} - \frac{2}{\left|b\sqrt{b} - 1 \right|} \cdot \arccos \frac{b + \sqrt{b}}{b \sqrt{b} + 1}.
\end{equation}

\begin{wrapfigure}[9]{r}{0.47\tw}
    \centering
    \vspace{-1pc}
    \tikzsetnextfilename{retrograde-movement-loop-size-plot}
    \begin{tikzpicture}
        \begin{axis}[
            width    =    .5\tw,
            height    =    4.5cm,
            xlabel    =    {$b$},
            ylabel    =    {$\Delta \lambda, ~^\circ$},
%            ymax    =    1,
            ymin    =    0,
            xmax    =    10,
            xmin    =    0,
            legend cell align=left,
            legend style={
                row sep = 1mm,
                draw=none,
                fill=none,
                font=\scriptsize,
                at={(axis cs:1.2, 2.5)}, anchor=south west,
            },
        ]
            \addplot [domain=0:10, samples=37, smooth] {acos((x^2 + x - 1)/(x^2 + x + 1)) - 2 / abs(x*sqrt(x) - 1) * acos((x + sqrt(x))/(x * sqrt(x) + 1))};
            \addplot[mark=o, only marks, black, mark options={
        scale=0.75, fill=white}]  coordinates {(0.42, 14.2) (0.72,16.15) (1.52,15.93) (5.2,9.95) (9.6,6.76)};
            \legend{
                $\Delta \lambda(b)$,
                Планеты Сол.\,сист.
            }
        \end{axis}
    \end{tikzpicture}
    \caption{}
\end{wrapfigure}
Легко заметить, что для внутренней планеты результат будет тот же, стоит только поменять местами индексы наблюдателя и планеты в выкладках выше. 








