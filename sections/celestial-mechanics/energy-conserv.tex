\subsection{Закон сохранения энергии}

Эмпирически установлено, что в замкнутой системе энергия не берётся из ниоткуда и не исчезает в никуда, а лишь переходит из одной формы в другую. Именно этот принцип и называется \term{законом сохранения энергии}. Так для замкнутой системы двух тел сохраняется полная механическая энергия системы $E_0$~--- сумма потенциальной ($\Pi$) и кинетической ($K$) энергий.

Отсюда следует, что для движения тела c массой $m$ в гравитационном  в поле тела
с массой $M\gg m$ со скоростью $v$ на расстоянии $r$ от
гравитационного центра справедливо следующее соотношение\footnote{здесь не рассматривается вращательное движение тел}:
\begin{equation}
	\frac{m v^2}{2}-\frac{GM m }{r}=E_0,
\end{equation}
данное равенство принято называть \imp{законом сохранения энергии} тела, движущегося в поле консервативных (потенциальных) сил.

Определим, как знак полной механической энергии связан с характером движения пробного тела в гравитационном поле массивного. Пусть $E_0 < 0$, так как $v^2 \geqslant 0$, то $1/r > 0$, следовательно $r < \infty$, то есть движение ограничено (финитно).

При инфинитном движении в некоторый момент пробное тело удаляется бесконечно далеко от массивного, другими словами $r \rightarrow \infty$. В этом случае, согласно закону сохранения энергии, $E_0 = K \geqslant 0$, а значит, минимальное значение полной механической энергии, при котором движение неограниченно, равно нулю. При этом на бесконечном удалении от гравитирующего центра скорость пробного тела также будет равна нулю. Если же $E_0 > 0$, минимальная скорость пробного тела всегда больше нуля, а получение выражения для вычисления ее величины оставим читателю.

Найдем минимальную скорость $v_2$, необходимую пробному телу, чтобы удалиться от массивного тела бесконечно далеко. Как было показано выше, в этом случае $E_0 = 0$, следовательно,
\begin{gather}
    \frac{mv_2^2}{2} = \frac{GMm}{r}, \nonumber \\
    v_2 = \sqrt{\frac{2GM}{r}}.
\end{gather}
Полученная скорость $v_2$ называется \term{второй космической} или \term{пара\-бо\-ли\-ческой}\footnote{причина такого названия ясна из раздела \ref{sec:orbit-types}} скоростью и является минимальной необходимой скоростью, чтобы покинуть зону влияния гравитации тела с массой $M$, находясь от него расстоянии $r$.


%%%%%%%%%%%%%%%%%%%%%%%%%%%%%%%%%%%%%%%%%
%                                       %
%       Не удалять, еще пригодится      %
%                                       %
%%%%%%%%%%%%%%%%%%%%%%%%%%%%%%%%%%%%%%%%%

%\begin{wrapfigure}[10]{l}{.5\tw}
%	\centering
%	\vspace{-1pc}
%	\includegraphics[width = 0.45\tw]{speeds}
%	\caption{Возможные траектории тела \label{pic:orbits}}
%\end{wrapfigure}
%Если $E_0>0$, то траектория тела~--- \imp{гипербола},
%ветви которой асимптотически приближаются к двум прямым. Стоит заметить,
%что на бесконечно большом удалении малого \linebreak тела от массивного
%его скорость остается положительной, так как суммарная энергия $E_0$
%больше нуля.
%
%Если $E_0=0$, то траектория тела~--- \imp{парабола}. При стремлении
%расстояния $r$ между телами к бесконечности скорость тела с стремится к нулю.
%
%Отсюда становится очевидно, что при параболической и гиперболической
%траекториях движение тела не ограничено (инфинитно).
%
%Если $E_0<0$, то траектория тела~--- \imp{эллипс}. При
%эллиптической траектории движение ограничено (финитно), так как малое тело
%не может бесконечно удаляться по причине того,
%что суммарная энергия меньше нуля.
%
%На Рис.\,\ref{pic:orbits} представлены примеры возможных траекторий малого тела
%относительно центрального (точка C). При $v_0 > v_{2}$ тело движется
%по гиперболе, при $v_0 = v_{2}$~--- по параболе,
%а при $v_0 < v_{2}$~--- по эллипсу.
%
%\term{Первая космическая скорость} --- минимальная скорость, необходимая для
%того, чтобы маломассивное тело стало искусственным спутником центрального тела.
%\begin{equation}v_1=\sqrt{\frac{GM}{R}},
%\end{equation}
%где $M$ --- масса центрального тела, а $R$~--- радиус орбиты. Отсюда несложно получить выражение для
%скорости искусственного небесного тела на высоте~$h$ над~поверхностью тела радиуса $r_0$:
%\begin{equation}
%	v_h=\sqrt{\frac{GM}{r_0+h}}=\sqrt{\frac{g r_0^2}{r_0+h}}.
%\end{equation}
%
%\begin{wrapfigure}[10]{r}{.5\tw}
%	\centering
%	\vspace{-1pc}
%	\caption{Движение по окружности \label{pic:orb-vel}}
%\end{wrapfigure}
%\change{Вывод:
%\\
%Рассмотрим 2 точки из траектории ($А$ и $В$) и скорости объекта в этих точках ($v_A$ и $v_B$) соответственно.
%\begin{equation}
%	\vec{\Delta v} = \vec{v_B} - \vec{v_A}
%\end{equation}
%При устремлении промежутка времени между пунктами к 0 можно считать, что длина пути становится равна длине хорды $AB$.
%\begin{equation}
%	\alpha = \frac{\Delta v}{v_A} = \frac{AB}{R}.
%\end{equation}
%Посчитаем модуль среднего ускорения:
%\begin{equation}
%	\langle a \rangle = \frac{\Delta v}{\Delta t} = \frac{v \Delta l}{R \Delta t}.
%\end{equation}
%Перейдя к пределу, получим мгновенное ускорение
%\begin{equation}
%	a = \lim_{\Delta t \rightarrow 0} \langle a \rangle = \lim_{\Delta t \rightarrow 0} \frac{v \Delta l}{R \Delta t} = \frac{v}{R} \lim_{\Delta t \rightarrow 0} \frac{\Delta l}{\Delta t} = \frac{v^2}{R}.
%\end{equation}
%Отсюда получаем значение для круговой скорости:
%\begin{equation}
%	v = \sqrt{a R} = \sqrt{\frac{G M}{R}}.
%\end{equation}
%}
%
%\term{Параболическая} или \term{вторая космическая скорость} ---
%минимальная скорость, необходимая для того, чтобы маломассивное тело преодолело гравитационное притяжение центрального тела, стартуя с расстояния $r$ от его центра масс, и покинуло замкнутую орбиту вокруг
%последнего. Выражение для нее имеет следующий вид:
%\begin{equation}
%	v_{2}=\sqrt{\frac{2GM}{r}}.
%\end{equation}

%%%%%%%%%%%%%%%%%%%%%%%%%%%%%%%%%%%%%%%%%
%                                       %
%       Не удалять, еще пригодится      %
%                                       %
%%%%%%%%%%%%%%%%%%%%%%%%%%%%%%%%%%%%%%%%%
