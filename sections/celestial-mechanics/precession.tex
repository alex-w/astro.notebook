\subsection{Прецессия}
\begin{wrapfigure}[15]{l}{0.41\tw}
    \vspace{-1pc}
    \centering
    \includegraphics[width = .42\tw]{precession_bw}
    \caption{Прецессионное движение северного полюса мира}
    \label{fig:precession-path}
\end{wrapfigure}
Под действием возмущающих сил ось вращения Земли совершает прецессионное движение: описывает вокруг оси эклиптики конус с углом раствора $23.5^\circ$ с периодом около  25\,765~лет. Из-за этого меняется положение полюса мира. Например, сейчас полюс мира практически совпадает с Полярной звездой ($\alpha$\,UMi), а 15\,000~лет назад роль полярной звезды играла Вега ($\alpha$\,Lyr). Если считать, что величина прецессии постоянна, то полюсы мира описывают вокруг полюсов эклиптики малые круги с радиусом $23.5^\circ$. В~действительности~же величина прецессии меняется, поэтому путь полюсов мира представляет собой не~окружность, а~спираль.

Поворот оси Земли имеет различные последствия. Во-первых, меняется продолжительность тропического года, он становится примерно на $20$~минут короче звёздного, во-вторых, меняется вид звёздного неба  (см.~Рис.\,\ref{fig:precession-path}).
