\subsection{Собственное движение звёзд}
\term{Собственным движением} $(\mu)$ называется изменение координат звёзд на небесной сфере, вызванное относительным движением звёзд и Солнца.
\begin{equation}
	\mu = \sqrt{\mu_\delta^2 + \mu_\alpha^2 \cos^2 \delta} = \frac{V_\tau}{D},
\end{equation}
где $V_\tau$~--- тангенциальная относительная скорость звезды, $D$~--- расстояние до неё, $\mu_\delta$~--- собственное движение по склонению, $\mu_\alpha$~--- собственное движение по прямому восхождению, определяются по следующим формулам:
\begin{equation}
  \mu_\delta = \frac{\delta(t_2) - \delta(t_1)}{t_2 - t_1}, \quad \quad \mu_\alpha = \frac{\alpha(t_2) - \alpha(t_1)}{t_2 - t_1}.
\end{equation}
